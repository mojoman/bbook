\chapter{Математическое приложение}

\section{Векторы и векторные
пространства\protect\footnote{\cite{Braverman:1976}.}}\label{app:vectors}

\dw{Вектором}\index{вектор} называется упорядоченный (записанный в
определенном порядке) набор $n$ чисел. Каждое из этих чисел
называется \dw{компонентой} вектора. Векторы в этой книге
обозначаются полужирным шрифтом для того, чтобы было легко отличать
их от компонент.

\remrk{?Добавить запись вектора}

Если все компоненты вектора равны нулю, то такой вектор называется
\dw{нулевым вектором} и обозначается $\vc{0}$.

\input{pics/app_vector.TpX}

Вектор, имеющий $n$ компонент, называется $\mathbf{n}$\dw{-мерным
вектором} или \dw{вектором размерности} $\mathbf{n}$. Вектор
размерности 2 изображают на плоскости в виде точки, имеющей первую
координату, равную первой компоненте вектора, а вторую координату
--- второй компоненте. Иногда также вектор изображают в виде
стрелки, идущей из начала координат в эту точку.

В связи с этим, даже когда речь идет об $n$-мерных векторах, часто
наряду с термином <<вектор>> используют термин <<точка>> точно в том
же смысле.


\section{Элементы топологии\protect\footnote{\cite{Takayama:1985}.}}\label{app:topology}

\remrk{Понятие шара и эпсилон окрестности}

Пусть $\st{X}$ --- некоторое множество в $\R^n$. Точка $\vc{\bar x}$
называется \dw{граничной} точкой множества $\st{X}$, если для любого
$\varepsilon > 0$ соответствующая
$\varepsilon$\nobreakdash-\hspace{0pt}окрестность точки $\vc{\bar
x}$ содержит как точки, принадлежащие множеству $\st{X}$, так и
точки, ему не принадлежащие.

Точки множества $\st{X}$, не являющиеся граничными, называются
\dw{внутренними точками} этого множества.

На Рис.~\ref{fig:int_point} точка $\vc{\bar x}$ является граничной
точкой множества $\st{X}$, а точка $\vc{\tilde x}$ --- внутренней.

\input{pics/app_intpoint.TpX}

Совокупность внутренних точек множества $\st{X}$ образует
\dw{внутренность} этого множества (иногда называется \dw{открытым
ядром} и обозначается $\st{X}^o$).

Совокупность граничных точек множества $\st{X}$ образует
\dw{границу} этого множества (обозначается $\st{\bar X}$).

Очевидно, что $\st{X}^o \cap \st{\bar X} = \st{X}$.





%---End of file
