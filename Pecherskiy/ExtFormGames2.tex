




1. {\bf Дуополия по Штакельбергу}.\\
Напомним, что дуолполия по Штакельбергу~--- модификация дуополии по Курно,
рассмотренной нами в .... . Теперь мы считаем, что есть лидер, например,
фирма 1, который делает ход первым. Затем, \emph{зная этот выбор},
другой игрок (фирма 2) делает свой ход.

Итак, игра протекает следующим образом:

1) фирма 1 выбирает $q_1\ge 0$;

2) фирме 2 становится известно  это $q_1$,  и после этого она выбирает
$q_2\ge 0$;

3) выигрыш фирмы $i$  есть
$$
\pi_i(q_i,q_j)=q_i(P(Q)-c),
$$
где $P(Q)=a-Q;$\, $Q=q_1+q_2;$\, $c$~---  постоянные предельные затраты.

Для нахождения равновесия здесь мы воспользуемся обратной индукцией.
Вычислим вначале функцию реагирования фирмы 2, решая задачу
$$
\max_{q_2\ge 0}\pi_2(q_1,q_2)=\max_{q_2\ge 0}q_2[a-q_1-q_2-c].
$$
Легко видеть, что
$$
R_2(q_1)={{a-q_1-c}\over 2}.
$$
То же самое было и в случае дуополии Курно. Но здесь разница в том, что это
\emph{действительная}, а не гипотетическая функция реагирования фирмы  2.

Фирма 1, естественно, также может вычислить эту функцию реагирования, а
следовательно, задача фирмы 1 на первом шаге становится следующей:
$$
\max_{q_1}\pi_1(q_1,R_2(q_1))=\max_{q_1}q_1[a-q_1-R_2(q_1)-c]=\max_{q_1}q_1{{a-q_1-
c}\over 2},
$$
что дает
$$
q_1^*={{a-c}\over 2}\quad{\mbox{\rm  и}}\quad R_2(q^*_1)={{a-c}\over 4}.
$$

Прибыль в случае дуополии по Штакельбергу:
$$
\pi_1={{a-c}\over 2}\biggl[{1\over 4}(a-c)\biggr]=
{{(a-c)^2}\over 8};\,\, \pi_2={{(a-c)^2}\over 16}.
$$

Заметим, что прибыль в случае дуополии по Курно:\\
${1\over 9}(a-c)^2$.


Этот пример лишний раз подчеркивает существенное различие между принятием
единоличного решения и решения при нескольких участниках. Здесь
лишняя информация для игрока и знание того, что другие имеют больше
информации, могут ухудшить положение игрока. В данном примере это
особенно заметно, поскольку вторая фирма знает, что первая фирма
может точно предвидеть оптимальное поведение конкурента после
выбора своего объема производства.
\smallskip

\noindent
2. {\bf Последовательный торг} (Rubinstein, 1982).\\
Рассмотрим следующую ставшую уже классическим примером использования
обратной индукции игру. Игроки 1 и 2 торгуются о разделе 1
доллара: 1-й предлагает некоторый способ деления, 2-й либо
принимает это предложение, либо нет; если нет, то он предлагает свой
способ деления, а 1-й принимает, либо нет, и т./.д. Каждое предложение
занимает один период, но при этом есть дисконтирующий множитель.
Итак, формально рассмотрим следующую трехпериодную игру.
\begin{itemize}
\item[(1a)] В начале 1-го периода игрок 1 предлагает
свою долю $s_1$ доллара, оставляя $1-s_1$ игроку 2.

\item[(1b)] Игрок 2 принимает предложение, тогда игра заканчивается.
Если же он отклоняет его, то в этом случае игра переходит ко 2-му периоду.

\item[(2a)] В начале 2-го периода игрок 2 предлагает долю $s_2$,
которую получает игрок 1, оставляя себе $1-s_2$.

\item[(2b)] Игрок 1 либо принимает предложение, либо нет. В последнем
случае игра переходит к 3-му периоду.

\item[(3)] Игроки в 3-м периоде получают доли ($s$, $1-s$), $0<s<1$,
причем $s$ задана экзогенно и известно каждому из игроков.
\end{itemize}

Поскольку есть дисконтрующий множитель $0< \delta <1$, то во
втором периоде стоимость доллара становится $\delta$, а в третьем - уже $\delta^2$.

Интуитивно, по-видимому, достаточно ясно, что наиболее разумно ожидать,
что на первом же этапе первый игрок предложит такую долю,
чтобы второй игрок это предложение принял, поскольку каждый
лишний этап приводит к денежным потерям.

Мы будем решать задачу с помощью обратной индукции. Вначале
вычисляем, что происходит, если дело доходит до 2-го периода. Игрок
1 может получить $s$, если отклонит $s_2$, но стоимость при этом
будет $\delta s$ (в сравнении с предыдущим (вторым) периодом).
Следовательно, игрок 1 примет $s_2$ тогда и только тогда, когда
$s_2\ge \delta s$ (для простоты считаем, что принимает, если равенство). Значит,
задача игрока 2 состоит в выборе между получением $1-\delta s$
(предлагая первому игроку  $s_2=\delta s$) и получением $1-s$ в
следующем периоде (предлагая $s_2<\delta s$).  Дисконтированная
стоимость последнего действия есть $\delta(1-s)$, что меньше,
чем $1-\delta s$, а потому 2-й игрок во 2-м периоде предлагает
$s_2^*=\delta s$.

Таким образом, если игра доходит до 2-го периода, то 2-й игрок
предложит $s^*_2,$ и 1-й игрок примет это предложение.

Однако 1-й игрок может предвидеть, что игрок 2 может получить
$1-s^*_2$ во втором периоде,  отклоняя предложение $s_1,$ но
стоить это будет только $\delta(1-s^*_2)$ в следующем периоде. Значит 2-й
игрок принимает $1-s_1$ тогда и только тогда, когда
$1-s_1\ge\delta(1-s^*_2),$ или $s_1\le 1-\delta(1-s^*_2)$.

Поэтому задача 1-го игрока в периоде 1 состоит в выборе между
получением $1-\delta(1-s^*_2)$ в этом периоде (предлагая
$1-s_1=\delta(1-s^*_2)$ игроку 2) и получением $s^*_2$ в следующем
периоде (предлагая $1-s_1<\delta(1-s^*_2)$ игроку 2).
Дисконтированная стоимость последнего есть $\delta s^*_2=\delta^2 s,$
что меньше, чем $1-\delta(1-s^*_2)=1-\delta(1-\delta s).$ Значит,
оптимальное предложение в первом периоде есть
$s^*_1=1-\delta(1-s^*_2)=1-\delta(1-\delta s).$ Следовательно, на
первом ходу 1-й игрок предлагает $s^*_1$, 2-й принимает это
предложение и получает $1-s^*_1.$  Таким образом, выигрыши игроков
есть $1-\delta+\delta^2s$ и $\delta-\delta^2s$ соответственно.

З а д а ч а.\,\, Докажите, что если бы игра продолжалась бесконечно
(здесь уже нет экзогенно заданного $s$), то игрок 1 на 1-м шаге
предложил бы $s^*=1/(1+\delta)$, оставляя второму
$1-s^*=\delta/(1+\delta)$, и второй игрок принял бы это предложение.
\smallskip

\noindent

Следующий пример, который может показаться достаточно примитивным, тем не менее
достаточно любопытен и может рассматриваться как попытка
проиллюстрировать на простом примере, как "из ничего"\, может возникать банковский кризис.

3. {\bf Инвесторы и банк} (Diamond, Dybvig, 1983).\\
Представим себе следующую стилизованную ситуацию. Два инвестора вкладывают
по $D$  долларов в банк. Банк инвестировал эти депозиты в долгосрочный проект. Если
форс-мажорные обстоятельства заставляют банк
ликвидировать свои инвестиции до того, как проект
созревает, то он может покрыть сумму $2r$, где $D>r>D/2$.
Если банк позволяет проекту созреть, то проект принесет
$2R$, где $R>D$.

Есть 2 даты, когда вкладчики могут забрать свой вклад: дата 1~--- до
созревания, дата 2~--- после. Для простоты считаем, что нет
дисконтирования. Если оба вкладчика забирают вклады в момент 1, то
оба получают по $r$, и игра заканчивается. Если только один вкладчик
забирает в момент 1, то он получает $D$, а второй~--- $2r-D$. Hаконец,
если ни один вкладчик не забирает вклад в момент 1, то проект созревает и
оба вкладчика забирают в момент 2, при этом каждый получает по $R$.
Если только один вкладчик забирает в момент 2, то он получает $2R-D$,
другой получает $D$. Если, наконец, ни один не забирает в момент 2,
то банк возвращает по $R$ каждому.

Неформально для момента 1 игру можно попытаться
изобразить следующим образом (см. рис.\,16):

\begin{center}
\begin{tabular}{cccc}
&&{забирать}&нет\\
$\begin{array}{c}
забирать\\
\\
нет \end{array}$&
\multicolumn{3}{r}{$(\begin{array}{cccc}
r,r&&&D,2r-D\\
\\
2r-D,D&&& ({\mbox{\rm шаг 2}}) \end{array})$}\\
\multicolumn{4}{c}{}\\
\multicolumn{4}{c}{Рис. 16.}\\
\end{tabular}
\end{center}

Для момента 2 игра изображена на рис.\,17:

\begin{center}
\begin{tabular}{cccc}
&&забирать&нет\\
$\begin{array}{c}
забирать\\
\\
нет \end{array}$&
\multicolumn{3}{c}{$(\begin{array}{ccc}
R,R&&2R-D,D\\
\\
D,2R-D&& R,R \end{array})$}\\
\multicolumn{4}{c}{}\\
\multicolumn{4}{c}{Рис. 17.}\\
\end{tabular}
\end{center}

Дерево этой игры изображено на рис.\,??????.



ExtForm1.




Hачнем с момента 2: так как $R>D$ (и $2R-D>R$), то
"забирать"\, строго доминирует "нет". Значит,
единственное равновесие по Hэшу~--- это (забрать, забрать), давая выигрыши $(R,R)$.
Поскольку нет дисконта, то можно просто подставить в первую
игру (см. рис.\,19):

\begin{center}
\begin{tabular}{ccccc}
&&&\quad З.&\,\,\,H.\\
$\begin{array}{c}
З.\\
\\
H. \end{array}$&
\multicolumn{4}{c}{$(\begin{array}{ccc}
r,r&&D,2r-D\\
\\
2r-D,D&& R,R \end{array})$}\\
\multicolumn{5}{c}{}\\
\multicolumn{5}{c}{Рис. 19.}\\
\end{tabular}
\end{center}

Так как $r<D$, то $2r-D<r,$ и мы имеем два равновесия по Hэшу, дающие выигрыши
$(r,r)$ и $(R,R),$ а именно:

1) оба вкладчика "бегут"\, в банк в момент 1;

2) оба забирают в момент 2.

Первое можно интерпретировать как панику, которую мы иногда мы наблюдаем:
если вкладчик верит в то, что другой побежит, то ему тоже надо бежать, хотя,
конечно, обоим лучше подождать.


\section{Повторяющиеся игры}

Рассмотрим следующий вариант Дилеммы Заключенного (рис.\,20).
Будем считать, что игра повторяется дважды, причем игроки узнают
исход первого розыгрыша до того, как начинается второй. Считаем пока,
что нет дисконта и, поэтому выигрыши есть просто сумма выигрышей в
первом и втором разыгрывании, т.\,е. мы имеем дело с двухпериодной
или двухшаговой Дилеммой Заключенного.

\begin{center}
\begin{tabular}{ccc}
&$\begin{array}{cc}\quad L_2&\quad R_2\end{array}$ \\
\multicolumn{1}{c}{\begin{tabular}{c} $L_1$\\ $R_1$\\ \end{tabular}}&
\multicolumn{2}{c}{$ \begin{tabular}{cc}
(1,1)&(5,0)\\ (0,5)&(4,4)\\ \end{tabular})$}\\
\multicolumn{3}{c}{}\\
\multicolumn{3}{c}{Рис. 20.}\\
\end{tabular}
\end{center}

Следуя той логике СПРH,  которая у нас была ранее, посмотрим, что
происходит на втором шаге игры. Ясно, что на втором шаг игроки
должны играть равновесные по Нэшу стратегии (этого требует совершенство),
т.\,е. $(L_1,L_2)$.  А это значит, что, учитывая то, что выигрыши на первом
и втором щаге игры складываются, то можно считать что к каждому
элементу исходной матрицы нужно добавить выигрыши второго шага, т.\,е. $(1,1)$.

Таким образом, матрица становится
$$
(\begin{array}{cc}
(2,2)&(6,1)\\
(1,6)&(5,5) \end{array})
$$
а в ней равновесие по Нэшу единственно~--- $(L_1,L_2)$, значит СПРH в этой
двухшаговой Дилемме Заключенного~--- это $(L_1,L_2)$ на
первом шаге и $(L_1,L_2)$~--- на втором.

Теперь отвлечемся на время от двукратного повторения игры. Пусть
$G=(I, A_1,\ldots,A_n;u_1,\ldots,u_n)$~--- статическая игра с полной
информацией, в которой игроки одновременно выбирают ходы $a_i$ из
своих пространств стратегий $A_i$ и соответствующие выигрыши есть
$u_i(a_1,\ldots,a_n)$.  Будем называть $G$ базовой игрой.

\begin{definition}
Конечной повторяющейся игрой $G(T)$ базовой игры $G$ называется игра,
в которой $G$ разыгрывается $T$ раз и перед началом каждого
очередного розыгрыша игрокам известны исходы всех предыдущих
розыгрышей, т.\,е. известны стратегии, избранные игроками, и
полученные выигрыши. Выигрыши в игре $G(T)$ определяются как сумма
(или дисконтированная сумма) выигрышей на каждом шаге.
\end{definition}

Рассмотренная выше ситуация на самом деле характерна и для общего
случая, а именно, имеет место следующее предложение, которое
очевидным образом доказывается с помощью обратной индукции.

\begin{proposition}
Если базовая игра $G$ имеет единственное
равновесие по Hэшу, то для любого конечного $T$ повторяющаяся игра
$G(T)$ имеет единственное СПРH: на каждом шаге разыгрывается
равновесие по Нэшу.
\end{proposition}

Рассмотрим теперь ситуацию, когда базовая игра $G$ имеет несколько
равновесий (Gibbons):

\begin{center}
\begin{tabular}{cc}
&$\begin{array}{ccc} L_2\quad &M_2&\quad R_2\end{array}$\\
$\begin{array} {c}L_1 \\M_1\\R_1\end{array}$&
$( \begin{array}{ccc} (1,1)&(5,0)&(0,0)\\
(0,5)&(4,4)&(0,0)\\
(0,0)&(0,0)&(3,3) \end{array})$\\
\end{tabular}
\end{center}


Здесь два равновесия по Hэшу в чистых стратегиях $(L_1,L_2)$ и $(R_1,R_2)$.
Мы будем говорить о первом как о плохом, а о втором как хорошем равновесиях.

Предположим теперь, что эта игра повторяется дважды, причем исход первой
игры известен до того, как разыгрывается вторая. Для нас важно,
что \emph{может существовать} СПРH, в котором на первом шаге играется
$(M_1,M_2)$, то есть набор стратегий, не являющийся равновесным.
Это тот самый нюанс, который важен для нас, поскольку
он, так сказать, разделяет дух того, что происходит в случае
бесконечного разыгрывания игры $G$.

Как и раньше, предполагаем (поскольку речь идет о СПРH), что игроки
считают, что исход второго розыгрыша~--- это равновесие по Нэшу базовой
игры. Теперь представим себе, что игроки будут следовать хорошему или
плохому равновесию в зависимости от того, \emph{что} произошло
на первом шаге. Иными словами, можно предположить, что игроки {\it могут} ожидать,
что различным исходам 1-го этапа будут соответствовать разные исходы
2-го этапа.

Предположим, например, что игроки ожидают, что
$(R_1,R_2)$ будет исходом, если первый исход был $(M_1,M_2),$ но
$(L_1,L_2),$ если исходом 1-го этапа  был один из 8 оставшихся. В этом
случае игра на 1-м шаге сводится к игре

\begin{center}
\begin{tabular}{cc}
&$\begin{array}{ccc}L_2\quad&M_2&\quad R_2\end{array}$\\
$\begin{array}{c} L_1\\M_1\\R_1 \end{array}$&
$(\begin{array}{ccc}
(2,2)&(6,1)&(1,1)\\
(1,6)&(7,7)&(1,1)\\
(1,1)&(1,1)&(4,4)\end{array})$\\
\end{tabular}
\end{center}

Здесь к выигрышам, соответствующим $(M_1,M_2)$, добавлено $(3,\,3)$, а
к 8 остальным элементам исходной матрицы добавлено $(1,\,1)$ .

В этой игре уже есть 3 равновесия по Нэшу: $(L_1,L_2),$\, $(M_1,M_2),$\,
$(R_1,R_2).$ Эти три равновесия по Нэшу
соответствуют СПРH в первоначальной повторяющейся игре.
Обозначим $((w,x),(y,z)),$~--- исходы в повторяющейся игре:
$(w,x)$~--- на 1-м шаге, $(y,z)$~--- на 2-м. Равновесие $(L_1,L_2)$
соответствует совершенному под-игровому исходу
$((L_1,L_2),(L_1,L_2))$ в повторяющейся игре. Аналогично
равновесие по Нэшу $(R_1,R_2)$ этой игры соответствует
совершенному под-игровому исходу $((R_1,R_2),(L_1,L_2))$ в
повторяющейся игре. Эти два исхода просто наследуют равновесия
по Нэшу базовой игры. Hо третий исход~--- {\it качественно
другой}: $(M_1,M_2)$~--- соответствует совершенному под-игровому
(СП) исходу $((M_1,M_2),(R_1,R_2))$ в повторяющейся игре, т.\,к.
ожидаемый исход 2-го шага~--- это $(R_1,R_2)$ вслед за
$(M_1,M_2)$.

Иными словами, кооперацию можно достичь на 1-м шаге СП-исхода
повторяющейся игры.  А это уже дает пример более общей природы: если
$G$~--- статическая игра с полной информацией и множественными
равновесиями по Нэшу, то \emph{может существовать} СП исход в игре
$G(T)$, в которой на любом шаге $t$, кроме последнего, $t<T$, исход шага
$t$~---  \emph{не является} равновесием по Нэшу.

Основной вывод здесь такой: угрозы или обещания, которым можно
верить в будущем, {\it могут} влиять на текущее поведение.
Второй вывод, однако, состоит в том, что под-игровое совершенство
может не воплощать достаточно сильные определения правдоподобия.
Говоря, например, о СП исходе $((M_1,M_2),(R_1,R_2)),$ мы
предполагали, что игроки предвидят, что $(R_1,R_2)$ будет
исходом на втором шаге, если исход первого шага был
$(M_1,M_2)$, а $(L_1,L_2)$~--- исходом второго шага игры,
если любой другой из 8 оставшихся исходов возникает на первом шаге.
Однако игра $(L_1,L_2)$ на втором шаге может показаться достаточно
глупой, если $(R_1,R_2)$ с выигрышем (3,\,3) также возможно в
равновесии на втором шаге игры. Далее можно рассуждать следующим
образом. Если $(M_1,M_2)$ не стало исходом первого шага, так как
$(L_1,L_2)$ предположительно будет играться на втором шаге, то каждый
игрок может считать, что "что прошло, то прошло", и
предпочтительная для обоих игроков ситуация $(R_1,R_2)$
должна разыгрываться на 2-м шаге. Hо если $(R_1,R_2)$ будет
исходом 2-го шага после {\it любого} исхода розыгрыша,
то пропадают стимулы играть $(M_1,M_2)$ на 1-м шаге:
розыгрыш 1-го шага сводится просто к добавлению к каждому
исходу (3,\,3). А тогда $L_i$ есть лучший ответ игрока $i$ на
$M_j$ игрока $j$.

Прежде чем перейти к бесконечным повторяющимся играм, вернемся к
нашему определению и введем коэффициент дисконтирования, без
которого при бесконечном разыгрывании игры уже не обойтись. Считаем
теперь, что игроки дисконтируют будущие выигрыши с общим дисконтом
$\delta$. Иногда бывает удобно рассматривать не просто суммарный выигрыш
$$
\sum^T_{t=1}\delta^{t-1}u_i(a^t),
$$
а нормировать его для того, чтобы рассматривать среднюю полезность
за период, т.\,е.
$$
{{1-\delta}\over{1-\delta^T}}\sum^T_{t=1}\delta^{t-1}u_i(a^t)~\mbox{---}
$$
средний дисконтированный выигрыш (за период).
Он показывает, сколько нужно платить игроку $i$ в {\it каждом}
периоде, чтобы он получил тот же суммарный выигрыш. Разумеется,
нужно помнить, что в каждом последующем периоде, соответствующий
выигрыш умножается (по сравнению с предыдущим шагом) на $\delta$.

Например, для $T=2$ средний дисконтированный выигрыш есть:
$$\overline{u}_i={{1-\delta}\over{1-\delta^2}}(u_i(a^1)+u_i(a^2))=
{{1}\over{1+\delta}}(u_i(a^1)+\delta u_i(a^2)).$$

Тогда, если в каждом из периодов игрок $i$ будет получать выигрыш,
равный $\overline{u}_i$, то с учетом того, что во втором периоде
$\overline{u}_i$ будет стоить уже $\delta\overline{u}_i$, то мы получаем:
$$\overline{u}_i+\delta\overline{u}_i=(1+\delta)\overline{u}_i=
(u_i(a^1)+\delta u_i(a^2)).$$

Если теперь $T=3$, то средний дисконтированный выигрыш есть:
$$\overline{u}_i={{1-\delta}\over{1-\delta^3}}(u_i(a^1)+u_i(a^2)+u_i(a^3))=
{{1}\over{1+\delta+\delta^2}}(u_i(a^1)+u_i(a^2)+u_i(a^3)).$$
Тогда, если в каждом из периодов игрок $i$ будет получать по $\overline{u}_i$,
то во втором периоде это будет стоить $\delta\overline{u}_i$, а в третьем --
$\delta^2\overline{u}_i$. Следовательно,
$$\overline{u}_i+\delta\overline{u}_i+\delta u_i(a^2)=(1+\delta+\delta^2)\overline{u}_i=
u_i(a^1)+u_i(a^2)+u_i(a^3).$$

Если Дилемма Заключенного разыгрывается один раз, то  нужно
сознаваться (не кооперироваться). Если разыгрывается конечное число раз,
то под-игровое совершенство требует в последний раз
сознаться, а обратная индукция говорит, что единственное
СПРH~--- это сознаваться (не кооперироваться) всегда. Если игра разыгрывается
бесконечное число раз (или если существует отличная от нуля вероятность
того, что игра может быть разыграна бесконечное число раз), то сознаться
остается СПРH.  Более
того~--- это единственное равновесие такое, что игра на каждом шаге
{\it не меняется} в зависимости от того, \emph{что} игралось на предыдущих
шагах. Hо если горизонт бесконечен и $\delta \ge \frac{1}{4}$, то, как мы увидим
ниже, следующий набор стратегий оказывается тоже СПРH: молчать
(кооперироваться) на 1-м шаге и продолжать молчать
(кооперироваться) до тех пор, пока никто не предал. Если только кто-то
предал, то далее предавать всегда.
\smallskip

П р и м е р.% (Gibbons).

\begin{center}
\begin{tabular}{cc}
&$\begin{array}{ccc} L\quad&M&\quad R\end{array}$\\
$\begin{array}{c} U\\M\\D\end{array}$&$(\begin{array}{ccc}
(0,0)&(3,4)&(6,0)\\
(4,3)&(0,0)&(0,0)\\
(0,6)&(0,0)&(5,5)\end{array})$\\
\end{tabular}
\end{center}

Считаем, что эта игра разыгрывается дважды и что выигрыши~---
дисконтированная сумма выигрышей.

Если эта игра разыгрывается один раз, то здесь 3 равновесия:
$(M,L),(U,M)$ и $\biggl({3\over 7}U,{4\over 7}M\biggr),\biggl({3\over
7}L,{4\over 7}M\biggr)$ с выигрышами $(4,3),(3,4)$ и
$\biggl({{12}\over 7},{{12}\over 7}\biggr)$ соответственно. Здесь
запись $\biggl({3\over 7}U,{4\over 7}M\biggr)$ означает, что с
вероятностью ${3\over 7}$ играется $U$ и с вероятностью ${4\over
7}$~--- играется $M$.  Эффективный набор выигрышей $(5,5)$ не
достижим. Однако в двухшаговой игре с $\delta>7/9$ следующий набор
стратегий является СПРH: Играть $(D,R)$ на первом шаге.  Если исход
первого шага $(D,R)$, то играть $(M,L)$ во втором шаге; если исход
первого шага~--- не $(D,R)$, то играть $((3/7U,4/7M),(3/7L,4/7M))$ на
втором шаге.

По построению эти стратегии используют р.H. на 2-м шаге.
Отклонение этой стратегии на 1-м шаге увеличивает текущий выигрыш на
1 и уменьшает следующие выигрыши игроков 1 и 2 соответственно с 4
или 3, до $12/7.$  Поэтому игрок 1 не будет отклоняться, если
$1<\biggl(4-{{12}\over 7}\biggr)\delta$ или $\delta>7/16$, а второй
не будет отклоняться, если $1<\biggl(3-{{12}\over 7}\biggr)\delta$
или $\delta>7/9$.

Итак, как мы отмечали, имеет место следующее уточнение: если в базовой
игре $G$ есть более одного равновесия по Нэшу, то \emph{может существовать} СПРH в
повторяющейся игре $G(T)$ такое, что для любого $t<T$ исход шага $t$
не является равновесием по Нэшу. В бесконечно повторяющихся играх
справедлив более сильный результат:  даже если в базовой игре
$G$ есть единственное  равновесие по Нэшу, то может существовать СПРH
бесконечно повторяющейся игры, в которой никакой по-шаговый
исход не будет равновесием по Нэшу.

Итак, рассмотрим вариант Дилеммы Заключенного, повторяющейся
бесконечно, причем для любого $t$ исходы $t-1$ предыдущего шага игры
известны до начала разыгрывания шага $t$:

\begin{center}
\begin{tabular}{cc}
&$\begin{array}{cc} L_2&R_2\end{array}$\\
$\begin{array}{c} L_1\\ R_1\end{array}$&$(\begin{array}{cc}
(1,1)& (5,0)\\
(0,5)&(4,4)\end{array})$\\
\end{tabular}
\end{center}

Как мы уже отмечали, в бесконечном случае уже без дисконтирующего
множителя не обойтись.

\begin{definition}
Если $\delta$~--- коэффициент дисконтирования, то приведенная стоимость
бесконечной последовательности выигрышей $\pi_1,\pi_2,\ldots$ есть
$$
\sum^\infty_{t=1}\delta^{t-1}\pi_t.
$$
\end{definition}

Мы покажем, что в нашем варианте Дилеммы Заключенного
"кооперация" $(R_1,R_2)$ на каждом шаге  может быть  СПРH
бесконечно повторяющейся игры (хотя единственный равновесный исход в
базовой игре~--- это $(L_1,L_2)$). А именно, если игроки кооперируются
сегодня, то они кооперируются и завтра, и т.\,д., а в противном случае
они играют плохое равновесие.

Предположим, что игрок $i$ начинает игру, кооперируясь, и продолжает
так только и если только оба игрока кооперировались на любом
предыдущем шаге.  Формально его стратегия описывается следующим
образом:
\begin{itemize}
\item[]
Играть $R_i$ на 1-м шаге. На шаге $t$, если все предыдущие $t-1$
исхода были $(R_1,R_2)$, играть $R_i$; в противном случае играть
$L_i$. \end{itemize}

Это так называемая {\it триггерная} стратегия (стратегия
переключения). Если игроки придерживаются этой стратегии, то в
бесконечно повторяющейся игре равновесным набором будет $(R_1,R_2)$
на каждом шаге\footnote{ Вообще говоря, триггерные стратегии
допускают неоднократные переключения с одного хода на другой.
Указанная стратегия называется иногда жесткой (или жестокой)
стратегией (grim strategy).}.

Мы вначале покажем, что если $\delta$ достаточно близко к 1, то это есть
равновесие по Нэшу в бесконечно повторяющейся игре для обоих игроков,
придерживающихся этой стратегии. А затем покажем, что это СПРH.

Чтобы показать, что это есть равновесие по Нэшу в бесконечно
повторяющейся игре, предположим, что $i$-й игрок использует
триггерную стратегию, и покажем, что если $\delta$ достаточно
близко к 1, то для $j$-го игрока лучшим ответом будет тоже
применять такую стратегию. Так как игрок $i$ будет играть $L_i$
всегда, как только на каком-то шаге исход отличается от
$(R_1,R_2)$, то лучшим ответом $j$-го будет тоже играть $L_j$
всегда после нарушения $(R_1,R_2)$. Т.\,е. осталось определить
лучший ответ $j$-го игрока на 1-м шаге и на всех шагах таких, что
все предыдущие были $(R_1,R_2)$.  Игра $L_j$ даст $5$ на этом
шаге, но переключит на некооперативное поведение игрока $i$ (а
значит и $j$) навсегда.  Следовательно, на любом будущем шаге
выигрыш будет 1; так как
$1+\delta+\delta^2+\cdots+\cdots=1/(1-\delta)$, то приведенная
стоимость последовательности выигрышей есть
$5+\delta+\delta^2+\cdots=5+{{\delta}\over{1- \delta}}$.

С другой стороны, ответ $R_j$ дает выигрыши 4 и аналогичный выбор между
$L_j$ и $R_j$ на следующем шаге. Пусть  $V$~--- приведенная стоимость
выигрыша $j$-го игрока, если он играет оптимально. Если игра $R_j$
оптимальна, то $V=4+\delta V$. Следовательно,
$$
V={4\over{1-\delta}}.
$$

Если $L_j$ оптимальна, то $V=5+{\delta\over{1-\delta}}$, следовательно,
$R_j$ оптимальна в том и только в том случае, если
$$
{4\over{1-\delta}}\ge{5+{\delta\over{1-\delta}}}\quad{\mbox{\rm
или}}\quad\delta\ge{1\over 4}.
$$

Пусть теперь $G$~--- игра с полной информацией, в которой игроки
одновременно выбирают ходы. Если дана базовая игра $G$, то
$G(\infty,\delta)$~--- это {\it бесконечно повторяющаяся игра}, в которой
$G$ повторяется всегда и у игроков общий коэффициент дисконтирования
$\delta$. Для любого $t$ исходы предыдущих $t-1$ шагов наблюдаются до начала
шага $t$. Выигрыш каждого игрока~--- приведенная стоимость его выигрышей.

Как хорошо известно, в любой игре стратегия~---  {\it полный} план
действия. В статической игре с полной информацией~--- это просто
ходы. В динамике, разумеется, все сложнее.  Скажем, в двухшаговой
Дилемме Заключенного стратегию можно записать как {\it пятерку}
$(v,w,x,y,z)$:

$v$~---  на 1-м шаге;

$w$~--- ходить $w$, если исход был $(L_1,L_2)$;

$x$~--- ходить $x$, если~--- $(L_1,R_2)$;

$y$~--- ходить $y$, если~--- $(R_1,L_2)$;

$z$~--- ходить $z$, если~--- $(R_1,R_2)$.

Это можно представить себе, как набор команд агентам: 1-й ходит на первом
шаге, 2-й~--- на втором и т.\,д.

В повторяющейся игре $G(T)$ или $G(\infty,\delta)$ {\it история
игры до шага $t$}~--- это запись ходов игроков до шага $t$. В
конечно повторяющейся игре $G(T)$ или бесконечно повторяющейся игре
$G(\infty,\delta)$ {\it стратегия игрока} описывает действие игрока,
которые он предпринимает на каждом шаге, для любой возможной истории.
(В этом смысле история соответствует информационному множеству:
каждая история приводит к вполне определенному информационному
множеству (одноточечному), а каждому информационному множеству
(одноточечному) соответствует вполне определенный путь (история),
который  приводит именно к этому информационному множеству.)

{\it Для конечно повторяющейся игры $G(T)$ {\bf под-игра},
начинающаяся на шаге $t+1$,~--- это конечно повторяющаяся игра, в
которой $G$ разыгрывается $T-t$ раз и которая обозначается $G(T-
t)$}.

{\it В $G(\infty,\delta)$ каждая под-игра, начиная с шага $t+1,$
идентична $G(\infty,\delta).$ Игр, начинающихся с $t+1,$ столько же,
сколько историй.  Разумеется, каждая под-игра осмысленна вместе с
предысторией}.

Таким образом, здесь, как и ранее, равновесие по Hэшу является СПРH,
если соответствующие стратегии игроков образуют равновесие по Нэшу в
любой под-игре.

СПРH является уточнением равновесия по Hэшу в том смысле, что
стратегии игроков должны, во-первых, образовывать равновесие по Hэшу,
а кроме того, выдерживать дополнительный тест~--- образовывать равновесие
в под-играх.

Вернемся к Дилемме Заключенного и к триггерной стратегии,
рассмотренной выше. Здесь все под-игры можно разбить на 2 группы:

(1) под-игры, в которых на всех предыдущих шагах игрались
$(R_1,R_2),$ и

(2) под-игры, в которых хотя бы один раз было сыграно не
$(R_1,R_2).$

Если игроки используют триггерную стратегию во всей игре, то 1)
стратегии игроков в под-игре первой группы тоже оказываются
триггерными стратегиями, которые формируют равновесие по Hэшу во
всей игре; 2) стратегии игроков в под-игре второй группы просто
навечно повторяют пошаговое равновесие $(L_1,L_2)$, которое также
является равновесием во всей игре. Поэтому равновесие по Hэшу в
триггерных стратегиях является СПРH.

Прежде, чем сформулировать общую теорему о СПРН для бесконечно
повторяющихся игр, определим некоторые понятия.
Набор выигрышей $(x_1,\ldots,x_n)$ называется {\it достижимым} в
базовой игре $G$, если он является выпуклой комбинацией выигрышей в
ситуациях в чистых стратегиях игры $G$. Hа рис.\,21 изображено
множество достижимых выигрышей в Дилемме Заключенного~--- это
параллелограмм.


ExtForm2.


{\it Средний выигрыш (за период)} бесконечной последовательности
выигрышей $\pi_1,\pi_2,\pi_3,\ldots$ при данном коэффициенте
дисконтирования $\delta$ есть
$$
(1- \delta)\sum^\infty_{t=1}\delta^{t-1}\pi_t.
$$

Легко видеть, что это аналог среднего дисконтированного выигрыша, рассмотренного выше.
Преимущество среднего выигрыша по сравнению с приведенной стоимостью
в том, что средний выигрыш непосредственно можно сравнивать с
пошаговыми выигрышами. В рассмотренном нами варианте Дилеммы
Заключенного оба игрока могут получать выигрыш 4 в каждом периоде.
Такая последовательность выигрышей дает средний выигрыш 4 и
приведенную стоимость $4/(1-\delta)$, поскольку в каждом последующем
периоде выигрыш становится в $\delta)$ раз меньше по сравнению
с предыдущим периодом. С другой стороны, средний
выигрыш~--- это просто приведенная стоимость с некоторым множителем;
максимизация среднего выигрыша эквивалентна максимизации приведенной
стоимости.

Мы можем сформулировать теперь знаменитую теорему, которая носит
название народной (фольклорной)~--- Folk Theorem: она столь
хорошо известна специалистам, что ее авторство считается народным,
хотя, по-видимому, первым ее доказал Джеймс Фридман.

\begin{theorem} {\rm (Friedman, 1971)}. Пусть $G$ конечная,
статическая игра с полной информацией. Пусть $(e_1,\ldots,e_n),$
выигрыши в состоянии равновесия по Hэшу, и пусть
$(x_1,\ldots,x_n)$~--- любой достижимый вектор выигрышей в $G$. Если
$x_i>e_i$ для любого $i$ и $\delta$ достаточно близко к 1, то
существует СПРH в игре $G(\infty,\delta)$, дающее $(x_1,\ldots,x_n)$
в качестве среднего выигрыша.
\end{theorem}

Доказательство этой теоремы можно найти, например, в учебнике Gibbons
(1992). В идейном плане оно близко рассмотренному нами рассуждению
о СПРН в триггерных стратегиях для Дилеммы заключенного.
Hа рис. ExtForm2 множество возможных исходов, соответствующих СПРH,
заштриховано.
\smallskip

П р и м е р.\,\, Сговор в дуополии по Курно.

Вспомним статическую дуополию по Курно. Cпрос на рынке $P(Q)=a-Q$, где
$Q=q_1+q_2$, $Q<a$, у фирм постоянные предельные затраты $c$, и нет
фиксированных затрат.  В единственном  равновесии по Нэшу каждая
фирма производит $q_c=(a-c)/3$.  Поскольку суммарный объем в
равновесии $2(a-c)/3$ превышает монопольный объем $q_m=(a-c)/2$,
обеим фирмам было бы лучше, если бы каждый производил половину
монопольного выпуска $q_i=q_m/2$.

Рассмотрим бесконечно повторяющуюся игру, в которой базовая
игра~--- это рассматриваемая дуополия по Курно, причем у обеих
фирм  общий коэффициент дисконтирования $\delta$. Мы сейчас
вычислим значение $\delta$, для которых в совершенном под-игровом
равновесии по Нэшу этой бесконечно повторяющейся игры играется
(обеими фирмами) следующая стратегия:

Производить половину монопольного объема, $q_m/2$, в первом периоде. В
периоде $t$ играть $q_m/2$, если обе фирмы производили $q_m/2$ в каждом из
предыдущих $t-1$ периодов; в противном случае производить $q_c$.

Прибыль фирмы, когда обе фирмы производят $q_m/2$, есть $(a-c)^2/8$,
которую мы обозначим через $\pi_m/2$. Прибыль фирмы, когда обе
производят $q_c$, есть $(a-c)^2/9$, которую мы обозначим $\pi_c$.
Далее, если фирма $i$ собирается производить $q_m/2$ в этом периоде,
то объем, максимизирующий прибыль фирмы $j$, решает задачу $$
\max_{q_j}(a-q_j-{1\over 2}q_m-c)q_j.
$$
Решением этой задачи является $q_j={{3(a-c)}\over 8}$ с
соответствующей прибылью $\pi_d={{9(a-c)^2}\over 64}$. Таким образом,
ситуации, в которых фирмы играют триггерную стратегию, приведенную
выше, являются равновесием по Hэшу, если
$$
{1\over{1-\delta}}{1\over 2}\pi_m\ge \pi_d+{\delta\over{1-\delta}}\pi_c.
$$
Подставляя
$\pi_m$, $\pi_c$, $\pi_d$, получаем  $\delta\ge{9\over 17}$.


\section{Дуополия по Курно с неполной информацией}

Мы начинали рассмотрение теоретико-игровых моделей с дуополии по Курно,
закончим эти рассмотрения мы также дуополией по Курно, но теперь в более
сложной постановке. До сих пор мы все время говорили об играх с полной
информацией: мы считали, что у игроков была вся необходимая
информация друг о друге, включая выигрыши игроков. В реальных
ситуациях, конечно, все далеко не так, и фирмы, например, могут не
знать затраты других фирм, а значит, не знать функции выигрыша
друг друга.  Поэтому здесь возникает то, что называется Байесовыми (Байесовскими)
играми, введенныи Дж.\,Харшаньи (Harsanyi, 1967).

На примере дуополии по Курно мы покажем, каким образом это предположение
о полноте информации может быть ослаблено. Это позволит дать представление о том, каким
образом моделируются ситуации, когда игроки обладают неполной информацией,
и что именно понимается под неполной информацией в теоретико-игровых моделях.

Итак, рассмотрим следующую модификацию рассмотренной нами дуополии по Курно
с обратной функцией спроса $P(Q)=a-Q$, $Q=q_1+q_2$.  Предположим
теперь, что функция затрат фирмы 1 есть $C_1(q_1)=cq_1,$ и будем
считать, что функция затрат второй фирмы есть $C_2(q_2)=C_Hq_2$ с
вероятностью $\Theta$ и $C_2(q_2)=C_Lq_2$ с вероятностью  $1-\Theta$,
причем $C_L<C_H$.

Будем предполагать, что фирма 2 знает свою функцию затрат и функцию
затрат фирмы 1, но фирма 1 знает \emph{только
свою} функцию затрат и вероятности $\Theta$  и $1-\Theta$
того, что предельные затраты второй фирмы есть $C_H$  и $C_L$
соответственно.  При этом все это  общеизвестно: фирма 1 знает, что 2
имеет "больше"\, информации, фирма 2 знает, что 1 знает это,
и т.\,д.

Разумеется, естественно было бы ожидать, что фирма 2 будет принимать
различные решения в зависимости от своего типа, т.\,е. от уровня
своих предельных затрат.  То есть в данном случае под стратегией
следовало бы понимать отображение, которое ставит в соответствие каждому
из двух возможных уровней предельных затрат $C_H$ и $C_L$ некоторый
объем выпуска, который определялся бы фирмой 2 в случае, если бы ее
предельные затраты были высокими~--- $C_H$ или низкими~--- $C_L$.
Хотя на первый взгляд такое определение стратегии излишне, тем не менее,
как мы сейчас увидим, оно вполне соответствует пониманию стратегии в
играх в позиционной форме.

Все дело в том, что нашу модификацию дуополии по Курно можно трактовать,
следуя Харшаньи, следующим образом. Будем считать, что наша игра
протекает как бы следующим образом:
вначале Природа выбирает с вероятностью $\Theta$ и $1-\Theta$
соответствующий уровень предельных затрат и "сообщает"\, выбранный
уровень \emph{только} фирме 2, а затем уже фирмы принимают свои
решения об объемах выпускаемой продукции. Если на минуту представить
себе, что у каждой из фирм имеется только два возможных уровня выпуска
(а не континуум), скажем, $r_i$ и $R_i$, то тогда позиционная форма этой
игры может быть представлена следующим образом (см. рис.\,??????).


ExtForm3.


Заметим сразу же, что это позиционная форма отражает определение неполноты
информации, данное в начале обсуждения игр в позиционной форме: Природа
делает ход первой, и этот ход ненаблюдаем по крайней мере для одного игрока
(в данном случае это первая фирма). Как мы видим, у второй фирмы два
информационных множества, соответствующих случаю высоких и низких предельных
затрат: на
этом рисунке $2_H$ и $2_L$ соответствуют тому, что вторая фирма является
фирмой с высокими и, соответственно, низкими затратами. Поэтому в полном
соответствии с определением стратегии для игр в
позиционной форме стратегий второй фирмы будет указание объема выпуска для
случая, когда у нее высокие предельные затраты, и указание объема выпуска для
случая, когда у нее низкие предельные затраты. Мы уже обсуждали, почему
собственно необходимо указывать свои ходы даже в тех информационных
множествах, которые не будут достигаться во время розыгрыша (в данном случае
фирма, будучи, скажем, фирмой с низкими предельными затратами, должна указать, как
она сыграла бы, если бы оказалась фирмой с высокими предельными затратами). Это
станет понятным и в данном случае.

Пусть $q_2^*(C_H)$ и $q_2^*(C_L)$ соответственно выбор фирмы 2, а
$q_1^*$~--- выбор фирмы 1 .

Если предельные затраты высоки, то $q_2^*(C_H)$ (в равновесии по Нэшу) решает
задачу
$$
\max_{q_2}((a-q_1^*-q_2)-C_H)q_2.
$$
Аналогично $q_2^*(C_L)$ решает задачу
$$
\max_{q_2}((a-q_1^*-q_2)-C_L)q_2.
$$
Для фирмы 1 $q_1^*$ решает задачу
$$
\max_{q_1}\Theta[(a-q_1-q_2^*(C_H))-c]q_1+(1-\Theta)[a-q_1-q^*_2(C_L))-c]q_1.
$$
Иными словами, объем выпуска $q_1^*$ максимизирует ожидаемую
прибыль первой фирмы. Мы видим, что в этом выражении
появляется и $q_2^*(C_H)$, и $q_2^*(C_L)$. Содержательно это
соответствует тому, что вторая фирма, даже при условии, что у нее
вполне определенный тип затрат, должна указать выпуск, соответствующий
другому возможному типу затрат, поскольку ее конкурент ориентируется в своих
рассуждениях на обе возможности. А так как вторая фирма ориентируется
на действия первой фирмы, а первая, определяя свой оптимальный объем выпуска,
исходит из двух возможных типов затрат, то вторая фирма, принимая это во внимание
должна определить свой оптимальный выпуск и для того типа затрат, который реально
не соответствует действительности, но возможен с точки зрения конкурента.


Условие первого порядка дает нам

\begin{center}
\begin{eqnarray}
q_2^*(C_H)&=&{{a-q_1^*-C_H}\over 2},\nonumber\\
q_2^*(C_L)&=&{{a-q_1^*-C_L}\over 2},\nonumber\\
q^*_1&=&{\frac{\Theta[a-q^*_2(C_H)-c]+(1-\Theta)[a-q_2^*(C_L)-c]}{2}}.\nonumber
\end{eqnarray}
\end{center}

\noindent
Следовательно, мы имеем
\begin{center}
\begin{eqnarray}
q_2^*(C_H)&=&{{a-2{C_H}+c}\over 3}+{1-\Theta\over 6}(C_H-C_L),\nonumber\\
q_2^*(C_L)&=&{{a-2{C_L}+c}\over 3}-{\Theta\over 6}(C_H-C_L),\nonumber\\
q_1^*&=&{\frac{a-2c+\Theta C_H+(1-\Theta)C_L}{3}}.\nonumber
\end{eqnarray}
\end{center}
\noindent
Если бы у нас была полная информация с затратами $c_1$ и $c_2$
соответственно, то было бы $$ q^*_i={\frac{a-2c_i+c_j}{3}}.
$$
Заметим, что
$$
q^*_2(C_H)>{\frac{a-2C_H+c}{3}},
$$
а
$$
q^*_2(C_L)<{\frac{a-2C_L+c}{3}}.
$$
Это происходит потому, что при высоких затратах фирмы 2 конкурент (фирма 1)
"недопроизводит", а при низких затратах~--- "перепроизводит".
Связано это с тем, что фирма 1 не знает
точно структуру затрат фирмы 2, а знает, что они {\it могут быть}
(с соответствующими вероятностями) либо высокими, либо низкими.
Поэтому, принимая решение об объеме выпуска своей продукции, фирма
должна учитывать потенциально {\it обе} возможности, при этом, если
ее выпуск при высоких затратах конкурента (фирмы 2) был бы $q^H_1$,
то, учитывая возможность низких затрат у конкурента (а в этом случае
конкурент будет производить больше), фирма 1 должна
уменьшить этот объем выпуска. Именно в этом смысле фирма 1
недопроизводит продукцию при высоких затратах конкурента. Аналогично
при низких затратах конкурента фирма производит "лишнюю"\,
продукцию.

Напомним, что для игр с полной информацией нормальная форма игры ~--- это
$G=\{S_1,\ldots,S_n,u_1,\ldots,u_n\}$, где $S_i$~--- пространство
стратегий игрока $i$, а $u_i(s_1,\ldots,s_n)$~--- выигрыш игрока $i$ в
ситуации $(s_1,\ldots,s_n)$. (Мы опускаем здесь фиксированное множество
игроков $I$.) Если мы рассматриваем игру с одновременными ходами,
то $S_i=A_i$~--- множество ходов. Игра с полной информацией проходила так:

(1)~--- игроки одновременно выбирали ходы;

(2)~--- игроки получали свои выигрыши $u_i(a_1,\ldots,a_n)$, $i\in I$.

Теперь мы хотим описать ситуацию с неполной информацией. Мы
должны для начала учесть как-то тот факт, что игрок знает свою
функцию выигрыша, но может не знать функций выигрыша остальных
игроков. Пусть возможная функция выигрыша игрока $i$ имеет вид
$$
u_i(a_1,\ldots,a_n;t_i),
$$
где $t_i$~--- тип игрока $i$, $t_i\in T_i$~--- множество (пространство)
возможных типов игрока $i$.

В приведенном выше примере с дуополией по Курно: $T_1=\{c\}$,
$T_2=\{C_L,C_H\}$.  Сказать, что игрок $i$ знает свою функцию
выигрыша, означает, что он знает свой тип.  Аналогично, если игрок не
знает функций выигрышей других игроков, то соответственно он не
знает их тип, т.\,е. он не знает
$$
t_{-i}=(t_1,\ldots,t_{i-1},t_{i+1}\ldots,t_n)\in T_{-i},
$$
где $T_{-i}$~--- множество возможных значений $t_{-i}$.

Введем теперь понятие представлений в том смысле, в котором оно появляется
в играх с неполной информацией. {\it Представления (или система
представлений)}\footnote{ Belief  или system of beliefs.} игрока
$i$ о типах остальных игроков~--- это вероятность $p_i(t_{-i}|t_i)$
того, что типы остальных игроков описываются вектором $t_{-i}=(t_1,
\ldots,t_{i-1},t_{i+1},\ldots,t_n)$, при условии, что $i$-й игрок имеет (и
знает свой) тип $t_i$.  В большинстве случаев обычно предполагается, что
эта вероятность не зависит от типа самого игрока, т.\,е. мы можем писать не
условную вероятность, а просто $p_i(t_{-i})$.

\begin{definition}
Байесова (байесовская) игра $n$-лиц в нормальной форме определяется:

набором множеств (пространств) ходов $A_1,\ldots,A_n$;

набором множеств (пространств) типов $T_1,\ldots,T_n$ игроков;

представлениями $p_1,\ldots,p_n$ игроков;

функциями выигрышей $u_1,\ldots,u_n$.

Тип $t_i\in T_i$ игрока $i$ известен игроку $i$ и определяет функцию
выигрышей $u_i(a_1,\ldots,a_n;t_i)$. Представления $p_i(t_{-i}|t_i)$
игрока $i$ описывают неопределенность относительно типов $t_{-i}$
оставшихся $n-1$ игроков, при данном типе $t_i$ игрока $i$. Эту игру
будем обозначать $G=\{A,T,p,u\}$, где $A=A_1\times\cdots\times A_n$,
$T=T_1\times\cdots\times T_n$, $p=(p_1,\ldots,p_n)$,
$u=(u_1,\ldots,u_n)$.
\end{definition}

Следуя Харшаньи, можно предположить, что игра протекает следующим образом:
\begin{itemize}
\item[(1)] Природа выбирает вектор типов $t=(t_1,\ldots,t_n)\in
T=T_1\times\cdots\times T_n$;
\item[(2)] Природа сообщает каждому игроку $i$ {\it его} тип
$t_i$ ({\it и никому другому});
\item[(3)] Игроки одновременно выбирают свои ходы (соответственно, $i$-й
игрок из~--- $A_i$);
\item[(4)] Игроки получают выигрыши $u_i(a_1,\ldots,a_n;t_i)$, $i\in I$.
\end{itemize}

Введение этапов (1) и (2) сводит нашу игру к игре с несовершенной
информацией.

Вообще говоря, можно рассматривать и случай, когда функция $u_i$
зависит  от типов и других игроков, т.\,е. функции выигрышей имеют
вид $u_i(a_1,\ldots,a_n;t_1,\ldots,t_n)$.

Обычно предполагается, что типы игроков $t$ выбираются в соответствии с априорным
вероятностным распределением $p(t)$,  и это общеизвестно.

Когда Природа "объявляет"\, игроку $i$ его тип $t_i$, то он может
вычислить представление $p(t_{-i}|t_i)$, используя формулу Байеса
$$
p_i(t_{-i}|t_i)={\frac{p(t_{-i},t_i)}{p(t_i)}}={\frac{p(t_{-
i},t_i)}{\sum_{t_{-i}\in T_{-i}}p(t_{-i},t_i)}}.
$$
Далее, другие игроки могут также вычислить различные представления,
которые игрок $i$ может иметь, в зависимости от типа $t_i$. Часто
делается предположение о том, что типы игроков независимы, т.\,е. $p_i(t_{-i})$ не
зависит от $t_i$.

В заключение мы приведем определение стратегии в статической Байесовой
(Байесовской) игре.

\begin{definition}
В\,статической\,Байесовой\,игре
$G=\{A_1,\ldots,A_n$; $T_1,\ldots,T_n$, $p_1,\ldots,p_n$, $u_1,\ldots,u_n\}$
стратегия игрока $i$~---  это функция $s_i:T_i\to A_i$, которая для
каждого типа $t_i\in T_i$ определяет ход из $A_i$, который был бы выбран
игроком $i$, если бы Природой был выбран его тип $t_i$. Символически
$S_i=A_i^{T_i}.$
\end{definition}


В нашей модели дуополии по Курно, как уже отмечалось, стратегия
игрока 2~--- это пара $(q_2(C_H),q_2(C_L))$.

Теперь мы можем дать определение равновесия по Байесу--Нэшу в
статической Байесовой игре. Центральная идея все та же: стратегия
каждого игрока должна быть лучшим ответом на стратегии других
игроков, т.\,е. равновесие по Байесу-Нэшу~--- это просто
равновесие по Нэшу в Байесовой игре. Существование равновесия
немедленно следует из теоремы существования равновесия по Нэшу.
