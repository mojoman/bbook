\section{Общее равновесие}

Все это время мы рассматривали один рынок, а соответствующее равновесие называлось
частичным равновесием, при этом все цены, кроме цены исследуемого товара, предполагались
фиксированными. В модели общего равновесия \underline{все} цены переменны, а
равновесие требует, чтобы \underline{все} рынки были в состоянии равновесия.
Таким образом, теория общего равновесия учитывает взаимодействие между
всеми рынками, а также функционирование каждого индивидуального рынка.

\subsection{Простейшая модель обмена с двумя товарами и логарифмическими функциями полезностей}

Представим себе следующую ситуацию. Есть два потребителя A и B, у каждого из
которых есть начальный запас двух товаров $\omega^1_A,\, \omega^2_A$ и
$\omega^1_B,\, \omega^2_B$ соответственно. Предположим также, что функции
полезностей этих агентов имеют логарифмический вид, то есть
$$u_A(x^1_A,\,x^2_A)=a\ln(x^1_A)+(1-a)\ln(x^2_A)$$ и
$$u_B(x^1_B,\,x^2_B)=b\ln(x^1_B)+(1-b)\ln(x^2_B)$$
соответственно. (То есть предпочтения потребителей описываются
соответствующими функциями Кобба-Дугласа). Мы считаем также, что
производство отсутствует, поэтому потребители хотят (в общем случае)
совершить каким-то образом обмен с целью увеличения своей
полезности. Представим себе теперь, что товары имеют некоторую
(заданную извне) цену $p_1$ и $p_2$ соответственно. (Мы остановимся
ниже на обсуждении того, почему мы считаем цены заданными экзогенно).
Обмен и наличие
цен можно интерпретировать следующим образом. Оба потребителя
вначале продают имеющиеся у них запасы товаров некоему посреднику, а
затем на вырученные деньги $m_A$ и $m_B$ соответственно покупают
(хотят купить) тот набор товаров, который максимизируют их
полезности. (Наличие цен в описанной ситуации обмена с двумя
агентами может показаться несколько натянутым, поэтому ниже мы
остановимся на этом несколько подробнее).

Мы уже знаем, что в этом случае спрос (по Маршаллу), то есть
желательный для потребителя набор товаров при ценах на товары $p_1$
и $p_2$, соответственно, будет определяться следующим образом:
$$x^1_A(p_1,\,p_2,\,m_A)=a\frac{m_A}{p_1},$$
$$x^2_A(p_1,\,p_2,\,m_A)=(1-a)\frac{m_A}{p_2},$$
$$x^1_B(p_1,\,p_2,\,m_B)=b\frac{m_B}{p_1}$$ и
$$x^2_B(p_1,\,p_2,\,m_B)=(1-b)\frac{m_B}{p_2},$$
где $m_A$ и $m_B$ -- доходы потребителей A и B соответственно,
которые определяются их начальными запасами, то есть
$$m_A=p_1\omega^1_A+p_2\omega^2_A$$ и
$$m_B=p_1\omega^1_B+p_2\omega^2_B$$.

Заметим, что потребители будут стремиться
потратить все имеющиеся у них деньги, так как если потрачено не все,
то, купив еще немного какого-то товара они увеличат полезность. Тем
самым бюджетные ограничения должны выполняться как равенства.

Разумеется, в общем случае (при произвольных ценах $p_1$ и $p_2$)
агрегированный спрос на соответствующий товар может превышать
суммарное предложение этого товара (то есть имеющееся суммарное
количество данного товара), значит, будет иметь место дефицит
данного товара, либо, напротив, суммарное предложение товара
может превышать агрегированный спрос на соответствующий товар
(то есть будет иметь место избыток товара). Понятно, что если
говорить о равновесии, то в первом случае следует повысить цену
этого товара, уменьшая тем самым
спрос на него, тогда как во втором случае, наоборот, цену
следует понизить. Поступая таким образом, можно надеяться на то,
что найдутся такие цены $p_1$ и $p_2$, что агрегированный спрос
на каждый из имеющихся товаров будет равен его предложению (в данном
случае, поскольку нет производства, суммарному количеству каждого из товаров,
имеющихся у агентов).

Определим \emph{функции избыточного спроса} на эти два товара:
$$z_1(p_1,\,p_2)=a\frac{m_A}{p_1}+b\frac{m_B}{p_1}-\omega^1_A-\omega^1_B=$$
$$a\frac{p_1\omega^1_A+p_2\omega^2_A}{p_1}+b\frac{p_1\omega^1_B+p_2\omega^2_B}{p_1}-
\omega^1_A-\omega^1_B$$ и
$$z_2(p_1,\,p_2)=(1-a)\frac{m_A}{p_2}+(1-b(\frac{m_B}{p_2}-\omega^2_A-\omega^2_B=$$
$$(1-a)\frac{p_1\omega^1_A+p_2\omega^2_A}{p_2}+(1-b)\frac{p_1\omega^1_B+p_2\omega^2_B}{p_2}-
\omega^2_A-\omega^2_B.$$

Легко видеть, что агрегированный спрос обладает следующим свойством
(поскольку бюджетные ограничения выполняются как равенства):
$$p_1z_1(p_1,\,p_2)+p_2z_2(p_1,\,p_2)\equiv0$$
Здесь знак тождества обозначает, что равенство нулю имеет место
всегда (то есть для любых цен $p_1$ и $p_2$). Как мы увидим ниже,
это тождество -- оно называется законом Вальраса -- справедливо и
в общем случае.

Мы уже отмечали, что одновременное увеличение (или уменьшение) всех
цен в одно и то же число раз не меняет спрос на товары, поэтому мы
можем считать, что второй товар -- это деньги ($p_2=1$). В этом
случае мы получаем

$$z_1(p_1,\,p_2)=a\frac{p_1\omega^1_A+\omega^2_A}{p_1}+b\frac{p_1\omega^1_B+\omega^2_B}{p_1}-
\omega^1_A-\omega^1_B$$ и
$$z_2(p_1,\,p_2)=(1-a){p_1\omega^1_A+p_2\omega^2_A}+
(1-b){p_1\omega^1_B+p_2\omega^2_B}-\omega^2_A-\omega^2_B.$$

Для того, чтобы найти \emph{равновесные цены} мы должны найти такую
цену $p_1$, что избыточный спрос на первый и на второй товары должны
быть равны нулю. Из закона Вальраса следует, что для этого нам достаточно
решить одно (любое) из получившихся уравнений.

Легко видеть, что равновесная цена есть
$$p^*_1=\frac{a\omega^2_A+b\omega^2_B}{(1-a)\omega^1_A+(1-b)\omega^1_B}.$$
