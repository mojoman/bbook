\section{Дуополия по Курно}

В предыдущей главе мы рассмотрели ситуацию, когда на рынке (или в
отрасли) действует монополист, и то, каким образом монополист,
максимизирующий прибыль, определяет цену и объем
производства своего продукта.

Напомним, что в этом случае монополист, технология которого
описывается функцией затрат $C_i(q_i)=cq_i$ $(c<a)$, а спрос линеен
$P(q)=a-q$, назначит цену $p=(a+c)/2$, определит объем выпускаемой
продукции равным $q=(a-c)/2$ и получит прибыль $\pi = (a-c)^2/4$.

Представим себе теперь, что ситуация на рынке меняется: появляется
еще один производитель того же продукта (в этом случае говорят, что
фирмы производят однородный продукт). Совершенно понятно, что если
новый производитель "выбросит" \, на рынок дополнительно какой-то объем
продукции, то это приведет к снижению рыночной цены продукции, а
значит экс-монополист, продолжая выпускать тот же объем продукции
$q=(a-c)/2$ получит меньшую прибыль. Возникает естественный вопрос,
каким образом строится взаимодействие на рынке, на котором действуют
два производителя. Ситуация такого типа называется дуополией.

Формально модель дуополии впервые предложил французский экономист
Антуан Огюстен Курно (1801-1877) в 1838 г. Мы рассмотрим простейший
вариант модели дуополии по Курно, предполагая, что обе действующие
на рынке фирмы симметричны (одинаковы) и спрос линеен.

Итак, предположим, что две фирмы $i=1,2$ производят однородный
продукт и $q_1,q_2$~---  объемы их производства этого продукта.
Предположим далее, что обратная функция спроса линейна и имеет вид
$P(Q)=a-Q$, причем теперь уже $Q=q_1+q_2$,
$(P(Q)=a-Q$, при $Q<a$, и $P(Q)=0$, при $Q\ge a$). Функции затрат
обеих фирм $C_i(q_i)=cq_i$ $(c<a)$ (иными словами, нет фиксированных
затрат и предельные затраты постоянны).

Предположим далее, что фирмы выбирают объемы производства $q_i$ {\it
одновременно и независимо}. Разумеется, совершенно не обязательно
предполагать, что выбор осуществляется одновременно. Достаточно
считать, что фирмы не знают выбора конкурента. Легко видеть, что
ни одна фирма не будет производить $q_i>a$. Считаем, что фирмы
максимизируют свои прибыли:
$$
\pi_1(q_1,q_2)=q_1(P(q_1+q_2)-c)=q_1[a-(q_1+q_2)-c] \text{и}
$$
$$
\pi_2(q_1,q_2)=q_2(P(q_1+q_2)-c)=q_2[a-(q_1+q_2)-c].
$$
Здесь следует подчеркнуть особо, что теперь прибыли фирм зависят не
только от их собственного выбора объема производства (как это было в
случае с монополией), но и от выбора объема производства
конкурентом.

Теперь каждая из фирм должна считаться с тем, что в отрасли действует
конкурент, а потому должна пытаться предсказать выбор конкурента, и
исходя из своего прогноза, выбрать оптимальный объем производства. В
такого рода модели достаточно естественно считать, что равновесной
(о точном смысле понятия равновесия в данном случае мы будем говорить
чуть ниже) окажется ситуация, когда прогнозы каждой из фирм подтвердятся.

Рассмотрим, что происходит, если фирма 1 предполагает, что фирма 2
будет производить $q^e_2$ единиц готовой продукции. (Здесь значок $e$
символизирует то, что речь идет об ожидаемом объеме выпуска). Тогда, выбирая,
сколько ей производить, $q_1$, фирма 1 должна исходить из того, что
будет продано $q_1 + q^e_2$ единиц продукции по цене $p=a-(q_1 +
q^e_2)$. Это значит, что задача максимизации прибыли первой фирмой
принимает вид
$$
\max_{q_1}q_1(a-(q_1 + q^e_2))-cq_1.
$$
Легко видеть, что в этом случае максимизирующий прибыль объем
производства первой фирмы есть
$$
q_1(q^e_2)=(a-q^e_2-c)/2.
$$

Таким образом, для каждого прогнозируемого объема выпуска второй
фирмы  $q^e_2$ определен оптимальный выбор объема производства
первой фирмы, который мы обозначим через $R_1(q^e_2)$. (Все
сказанное верно и в случае более общего вида и функции спроса, и
функций затрат). Функцию $R_1$ называют функцией (или кривой)
реагирования фирмы 1.

Аналогично можно определить функцию реагирования $R_2(q^e_1)$ фирмы
2, определяющую оптимальный объем выпуска второй фирмы при условии,
что она ожидает, что фирма 1 будет выпускать $q^e_1$.

Разумеется, поскольку оптимальные объемы производства выбираются
фирмами на основе их \emph{представлений} о том, что конкурент
производит $q^e_1$ или $q^e_2$ соответственно, то оптимальный выбор
фирмой 1 объема производства $q^*_1$, вообще говоря, не будет
совпадать с прогнозом $q^e_1$ фирмы 2 относительно объема производства
фирмы 1.

В этом смысле естественно считать, что разумным исходом будет такая
пара объемов выпуска $(\overline{q}_1, \overline{q}_2)$, для которой
ожидания оправдываются, то есть
$$
\overline{q}_1=R_1(\overline{q}_2) \text{и}
\overline{q}_2=R_2(\overline{q}_1). $$ Такая пара называется
\emph{равновесием по Курно}. В равновесии по Курно каждая фирма
максимизирует свою прибыль при ее заданном представлении о том,
какой объем производства выбирает конкурент, и более того, каждая
фирма оптимально выбирает тот объем производства, который
прогнозирует ее конкурент. В равновесии по Курно ни одной из фирм не
выгодно отклониться от выбранного объема производства, если
становится известным выбор конкурента.

Мы вернемся к более подробному обсуждению дуополии по Курно ниже,
когда познакомимся с аппаратом теории игр, без которой анализ
дуополии (олигополии, а в общем случае и бесчисленных ситуаций
стратегического взаимодействия) в настоящее время уже немыслим.
Сейчас мы ограничимся лишь упоминанием еще двух вариантов дуополии.
Первый из них - это дуополия по Бертрану. Эта модель отличается от
описанной выше тем, что фирмы, производящие однородную продукцию,
одновременно и независимо выбирают цены своей продукции. Формально
эта модель описывается следующим образом.

Предположим, что на рынке есть спрос, определяемый функцией спроса
$D(p)$. Фирмы одновременно и независимо выбирают цены $p_1$ и  $p_2$
соответственно. Естественно предполагать, что потребители покупают
продукцию той фирмы, которая назначила меньшую цену. (Считаем, что в
случае равенства цен, потребители равновероятно распределяются по
фирмам). Тогда спрос, с которым сталкивается фирма 1, есть
$$
D_1(p_1,p_2)={D(p_1),\quad{\mbox{\rm если}}\quad p_1<p_2,
D(p_1)/2, \quad{\mbox{\rm если}}\quad p_1=p_2, 0,\quad{\mbox{\rm
если}}\quad p_1>p_2.}
$$
Аналогично определяется спрос, с которым сталкивается фирма 2. Иными
словами, фирма, назначившая меньшую цену, получает весь спрос, а
если цены одинаковы, то потребители покупают продукцию фирм
равновероятно.

Наконец еще одна модель дуополии - модель дуополии по Штакельбергу -
это вариант модели Курно, единственное отличие которой от последней
состоит в том, что вначале объем производства выбирает одна из фирм
(скажем, первая), а уже затем вторая фирма, \emph{зная} этот выбор,
определяет свой объем выпуска (также максимизируя свою прибыль). В
этой модели выбор второй фирмой объема производства - это просто
задача максимизации прибыли, которая приводит к выбору $R_2(q^e_1)$,
как и в случае модели Курно, но, что очень важно, в данном случае
$q^e_1$ - это не ожидаемый выбор, а \emph{реальный} выбор объема
производства фирмой 1.

Подробнее мы рассмотрим эти модели ниже, а теперь
перейдем к изложению основных понятий теории игр.


\section{Основные понятия теории бескоалиционных игр.}

В последние три-четыре десятилетия наблюдается стремительное повышение
интереса к теории игр и значительное возрастание ее роли. Во многом
это объясняется тем, что без нее в настоящее время уже немыслима
современная экономическая теория, причем область применения теории
игр постоянно расширяется. Теория игр прошла путь от весьма
формализованной теории, представлявшей интерес в первую очередь для
математиков и ставшей источником целого ряда работ чрезвычайно
глубокого математического содержания, до одного из важнейших
инструментов анализа огромного многообразия задач, возникающих в
экономике, политике, социальных науках и т.\,д. (разумеется, не
утратив при этом своего математического содержания).
\smallskip

Первыми исследованиями игр в экономической литературе, по-видимому,
следует считать статьи Курно (Cournot, 1838), Бертрана (Bertrand,
1883) (о моделях дуополии по Курно и Бертрана мы начали разговор в
предыдущем параграфе) и Эджворта (Edgeworth, 1897), в которых
рассматривались проблемы производства и ценообразования в
олигополии. Правда, они рассматривались тогда как весьма
специфические модели и в некотором смысле существенно опередили свое
время.

Анализ различных салонных игр проводился еще в Древнем Китае, но,
видимо, первые работы, в которых нахождение оптимальных стратегий в
играх формулировалось как математическая задача, появились только в
XVII веке (Bachet de Mezirak, Lyon, 1612). Первым серьезным
математическим результатом в этом направлении явилась работа
Э.\,Цермело 1912\,г.  "О применении теории множеств к шахматной
игре"\, (см.: Матричные игры. Под. ред. Н.\,Н.\,Воробьева, М., 1961.
С.\,137--153). В ней он доказал, что в каждой позиции шахматной
партии один из игроков может форсированно выиграть или обеспечить
себе ничью, выбирая "правильные"\, ответы на любой ход противника.
Заметим, что  к счастью для любителей шахмат существование
"правильных ответов" \, вовсе не гарантирует возможность их
вычисления. (И в этой связи интересно отметить, что общее число
возможных ходов в шахматах оценивается как $10^{120}$. Компьютеру,
делающему миллиард операций в секунду, понадобилось бы $3\cdot10^{103}$
лет, чтобы просчитать все эти ходы. Недавно физики предложили
мысленный эксперимент, при котором вся вселенная рассматривается,
как гигантский компьютер. Если бы вся энергия и порядок во вселенной
были бы посвящены вычислениям, то сколько вычислений можно было бы
сделать с момента происхождения вселенной -- Большого Взрыва?
Ответ -- примерно $10^{120}$)\footnote{The Physics of Information//
\emph{The Economist, } June 6, 2002.}

Хотя именно эта работа Цермело считается первой работой по теории игр,
общепризнанным годом рождения\, теории игр стал 1944~г.

В 1944\,г. вышла в свет основополагающая монография Джона фон
Неймана и Оскара Моргенштерна "Теория игр и экономическое
поведение\, (von Neumann, Morgenstern, 1944), которая, по существу,
заложила фундамент общей теории игр и обосновала возможность анализа
огромного массива экономических вопросов с помощью теоретико-игровых
моделей.  А в 1950~г. Джон Нэш (будущий Нобелевский лауреат по
экономике 1994~г.) ввел понятие ситуации равновесия, названной
впоследствии его именем, как метода решений бескоалиционных игр
(т.\,е. игр, в которых не допускается возможность создания
коалиций).  Ситуация, образующаяся в результате выбора всеми
игроками некоторых своих стратегий, называется равновесной, если ни
одному из игроков невыгодно изменять свою стратегию при условии, что
остальные игроки придерживаются равновесных стратегий. Именно
равновесие по Нэшу и его модификации признаются наиболее подходящими
концепциями решения для таких игр.

За прошедшие с момента появления книги Дж.\,фон\,Неймана и
О.\,Моргенштерна немногим более шестидесяти лет теория игр прошла
различные этапы своего развития и пережила несколько волн интереса к
ней.  Примерно 50--55 лет назад казалось, что теория игр представляет
огромные возможности экономике, однако в то время эти возможности
оказались переоценены во многом из-за того, что это было лишь начало
развития теории игр, и аппарат ее еще не был достаточно развит.
Хотя в то же время нельзя не отметить, что был
получен целый ряд очень глубоких математических результатов,
представляющих значительный интерес даже вне экономических
приложений. 35--40 лет назад элементы теории игр\, можно было найти разве
лишь в некоторых учебниках по теории
организации промышленности как краткое введение, предшествовавшее
изложению дуополии (олигополии) по
Курно, по Бертрану или по Штакельбергу. Однако за последние 35--40
лет произошел огромный шаг вперед, и теперь вряд ли можно найти
область экономики или дисциплины, связанной с экономикой, такой,
скажем, как финансы, маркетинг и т.\,д., в которых основные
концепции теории игр не были бы просто необходимыми для понимания
современной литературы.

Среди многочисленных определений того, что есть теория игр и каковы
ее задачи, которые можно найти в различных статьях, учебниках и
монографиях, упомянем лишь четыре. Первые два~--- это определения
теории игр, которые с некоторыми вариациями, по-видимому, наиболее
часто встречаются в литературе и достаточно точно характеризуют
общую проблематику, охватываемую теорией игр: Теория игр~--- это
теория рационального поведения людей с несовпадающими интересами\,
(Aumann, 1989), и Теория игр~--- наука о стратегическом мышлении,
(Di\-xit, Na\-le\-buff, 1991). Третье определение подчеркивает математическую
природу теории игр: Теория игр~--- это теория математических моделей
принятия оптимальных решений в условиях конфликтов, (Во\-ро\-бьев,
1984). Наконец, четвертое определение выделяет роль теории игр
именно в экономическом моделировании:  Суть теории игр в том, чтобы
помочь экономистам понимать и предсказывать то, что будет
происходить в экономическом контексте, (Kreps, 1990).  В настоящий
момент, если говорить об экономическом контексте, речь идет уже не
только о применении теоретико-игровых методов к ставшим достаточно
традиционными проблемам организации промышленности, но и, по сути
дела, ко всему многообразию экономической проблематики. Так,
например, на микроуровне~--- это модели процесса торговли (модели
торга, модели аукционов). На промежуточном уровне агрегации
изучаются теоретико-игровые модели поведения фирм на рынках факторов
производства (а не только на рынке готовой продукции, как в
олигополии).  Теоретико-игровые модели возникают в связи с
различными проблемами внутри фирмы.  Наконец, на высоком уровне
агрегации, с международной экономикой связаны модели конкуренции
стран по поводу тарифов и торговой политики, а макроэкономика
включает модели, в которых, в частности, стратегическое
взаимодействие рассматривается в контексте монетарной политики.
"Аппарат теории равновесия и теории игр послужил основой для создания
современных теорий международной торговли, налогообложения,
общественных благ, монетарной экономики, теории производственных
организаций" (Полтерович, 1997, с.\,11).

Разумеется, следует иметь в виду, что к настоящему моменту область
применения теории игр значительно расширилась и не ограничивается
уже только экономическим контекстом (который для нас представляет,
естественно, особый
интерес). Это\,политический\,и\,социальный контексты, биология и
военное дело, и многое другое (см., например, Дюбин, Суздаль, 1981;
Shubik, 1984; Moulin, 1983, 1986; Ordeshook, 1986; Rawls, 1971;
Maynard, Smith 1974 и др.).
Скажем,\,теоретико-игровой\,подход\,к\,изучению\,фор\-ми\-ро\-ва\-ния
коалиций~---
это\,уже\,своего\,рода\,традиция\,в\,социальных\,и\,по\-ли\-тических
науках (см.,\,например,\,Riker,\,1962;\,Riker,\,Or\-des\-hook, 1973;
De Swan, 1973; Or\-des\-hook, 1978, 1992; Van Deemen, 1997). Здесь
же следует упомянуть, например, книгу Game Theory and the Law,
(D.\,Baird, R.\,Gertner, C.\,Picker, 1994), в которой аппарат теории
игр впервые применяется к анализу того, как законы влияют на
поведение людей, партий и т.\,д.
\smallskip

Теория игр делится на две составные части: одна~--- это теория
бескоалиционных (некооперативных) игр, а вторая~--- теория
кооперативных игр. Это базовое деление, хотя подчас оно достаточно
расплывчато, основано на том, что в бескоалиционной теории основной
единицей анализа является (рациональный) индивидуальный участник,
который старается сделать максимально хорошо себе в соответствии с
четко определенными правилами и возможностями.

В противоположность этому, в теории кооперативных игр основная
единица анализа - это, как правило, группа участников, или коалиция:
если игра определена, то частью этого определения является описание
того, \emph{что} каждая коалиция может получить, без указания того,
\emph{как} исходы или результаты будут влиять на конкретную
ситуацию.

Мы рассматриваем здесь только бескоалиционные игры, которые имеют
ярко выраженную стратегическую направленность в том смысле, что
центральным вопросом здесь является нахождение оптимальных
стратегий агентов (игроков). В кооперативных же играх центральной
проблемой становитсяпроблема распределения (скажем, суммарной прибыли,
полученной агентами, или, наоборот, общих затрат).

\subsection{Способы задания бескоалиционных игр}


Теория бескоалиционных игр~--- это способ моделирования и анализа
ситуаций, в которых оптимальные решения каждого участника (в теории
игр участников принято называть  игроками) зависят от его
представлений (или ожиданий) об игре оппонентов. При этом чрезвычайно
существенным является то, что теория игр исходит из того, что
игроки не должны придерживаться
{\it произвольных} представлений об игре своих оппонентов. Напротив,
каждый игрок должен пытаться предсказать, как будут играть его оппоненты,
используя свои знания правил игры и исходя из предположений, что его
оппоненты сами рациональны, а потому пытаются сами также предсказать
действия своих оппонентов и максимизировать свои собственные выигрыши.

Есть два способа задания игры. Первый~--- это {\it позиционная
форма} игры. Позиционная форма описывает: (1) порядок ходов игроков;
(2) альтернативы (выбор), доступные игроку тогда, когда наступает
очередь его хода;
(3) информацию, которую игрок имеет на каждом из своих ходов; (4)
выигрыши (всех) игроков как функцию выбранных ходов;(5)
вероятностные распределения на множестве ходов Природы.

Позиционная форма представляется {\it деревом игры}, которое можно
рассматривать как обобщение дерева принятия решений, используемое в
теории принятия решений, на случай нескольких игроков. Формальное
определение мы приведем в п.      Древесная структура описывает,
какая вершина следует за какой, какой игрок ходит в соответствующей
вершине. Информация, которую имеют игроки, описывается с помощью
информационных множеств. На данный момент для нас
важно лишь дать первое представление о том, что представлет из
себя позиционная форма игр, поскольку мы начинаем изложение с
рассмотрения так называемых статических игр, то есть игр, в которых
игроки делают свои ходы одновременно, независимо, и каждый из игроков
делает лишь по одному ходу. Формальные определения мы приведем позднее,
когда перейдем к рассмотрению игр, в которых игроки делают свои ходы
последовательно и, быть может, не по одному разу.

GT1.

Развитие игры, изображенной на рисунке GT1, следует понимать следующим
образом. Цифры около вершин дерева указывают на то, чья очередь хода.
Буквы, соответствующие дугам дерева, обозначают альтернативы, доступные
игрокам в тот момент, когда наступает их очередь хода. Наконец,
пунктирные линии выделяют информационные множества. Первым делает ход
игрок 1, при этом он может выбрать альтернативу
l или альтернативу r (какое именно действие скрывается за этими обозначениями
для нас не существенно; так, например, первая из этих альтернатив может
обозначать для фирмы, принимающей решение о том, начинать ли производство
нового продукта или нет, "начинать производство",\, а вторая -- "не начинать").
Далее наступает очередь хода второго игрока, который может выбрать
одну из двух альтернатив L или R, если первый игрок выбрал l, и одну из двух
альтернатив M или N, если первый игрок выбрал r. Информационные множества
второго игрока (в данном случае их два -- это обе вершины с номерами 2,
каждая из которых обведена пунктиром) содержит каждое по одной вершине,
и это означает, что игрок 2 знает, какую из альтернатив до него выбрал
первый игрок. Далее наступает очередь хода. Здесь у игрока 3 два
информационных множества. Это означает, что третий игрок знает, что второй
игрок выбрал L или R (если речь идет о левом информационном множестве
игрока 3), но не знает, какую \emph{именно} из них тот выбрал, либо M или
N (если речь идет о правом информационном множестве
игрока 3), но не знает, какую \emph{именно} из них тот выбрал. После хода
третьего игрока игра заканчивается. Здесь очень важно отметить, что
дугам, исходящим из вершин одного информационного множества, соответствуют
одни и те же альтернативы (соответственно, одинаковые обозначения):
поскольку игрок не знает, какой именно ход выбрал предшествующий игрок, то
различие имеющихся в его распоряжении альтернатив означало бы, что он
в состоянии различить вершины информационного множества, а значит, тем самым,
определить, какой ход до этого был сделан. Концам дуг u, d, U, D соответствуют
так называемые терминальные (окончательные, концевые) вершины дерева.
Они символизируют окончание игры. В них указаны выигрыши, которые получают
игроки: первый игрок получает $a_i$, второй -- $b_i$, третий -- $c_i$.


Hа рис.~2 и 3 изображены недопустимые информационные множества.
Информационные множества не могут пересекаться: не различая вершины
одного информационного множества в том смысле, что игрок не знает,
в какой именно вершине этого информационного множества он находится
(в нашем случае, скажем, левая и средняя вершины для второго игрока)
и вершины другого информационного множества
(в данном случае средняя и правая вершины), которое пересекается с первым,
игрок тем самым не различает вершины, лежащие в объединении этих информационных
множеств (то есть не различает все три вершины). В вершинах одного
информационного множества множества
доступных игроку альтернатив должны совпадать (иначе игрок сможет
различать вершины информационного множества, а стало быть, различать
действия, предшествовавшие его ходу).


РИС2 и РИС3



Далее, если информационное множество игрока одноточечно, то мы
это информационное множество отмечать пунктиром не будем, как, например,
мы не отметили информационные множества в начальных вершинах рассмотренных
выше игр.

Приведем пример чрезвычайно простой игры и несколько ее модификаций
для того, чтобы несколько прояснить понятие информационных множеств.
Исходная игра такова: каждый из двух
игроков одновременно и независимо выбирает одну из трех цифр~--- 1, 2 или 3.
Если сумма выбранных цифр четна, то первый игрок выигрывает
у второго один рубль (доллар, фунт и пр.). Если сумма~--- нечетная,
то наоборот~--- выигрывает второй. Далее мы рассмотрим
варианты возможных последовательностей ходов игроков и их информированности,
а также покажем соответствующие этим вариантам деревья.


Начнем с случая, когда игроки ходят одновременно и независимо (например,
они записывают свой выбор и отдают соответствующую запись арбитру.
Дерево соответствующей игры изображено на рис.\,4. Обратим внимание на то, что
в данном случае порядок ходов, указанный на дереве можно изменить, при этом оно
будет соответствовать, по сути, той же самой игре: ни один из игроков не
знает выбора другого. Цифры у дуг соответствуют выбору игроком одной из трех
цифр.

На рис.5 изображена модификация этой игры, в которой второму игроку
становится известен выбор конкурента.

Hа рис.~6 изображена модификация этой игры, в которой второму игроку
становится известно либо, что первый игрок выбрал цифру 3, либо,
напротив, что цифру 3 он не выбрал.

Очень важно иметь в виду, что
дерево игры описывает \emph{все} возможности игроков в игре и \emph{не вводит
никаких ограничений} на осмысленность или целесообразность тех или иных ходов.


РИС5 и РИС6


Мы вернемся к позиционной форме в п.     (поскольку сейчас нас будут
интересовать статические игры с полной информацией, для которых
позиционная форма~--- это некоторое излишество, поскольку отсутствует,
например, необходимость указания порядка ходов, так как игроки, как мы
уже отмечали, делают ходы один раз, независимо и одновременно, после чего
игра закинчивается, и игроки получают свои выигрыши, возможно, отрицательные).
Теперь же мы перейдем ко второй возможной форме представления игры~---
{\it  нормальной} или {\it стратегической форме}, которая описывает игру
с помощью следующих элементов:  множества игроков $I$, множеств стратегий
каждого из игроков и функций выигрышей игроков, каждая из которых ставит
в соответствие каждому набору стратегий всех игроков соответствующие выигрыши
игроков.


\subsection{Игры в нормальной форме}

Итак, {\it игра в нормальной (или стратегической) форме}~--- это тройка
$\{I,S=\prod_i\{S_i\}_{i\in I},\,\,u=(u_1,\ldots,u_n)\}$, где
$I=\{1,\ldots,n\}$~--- множество игроков; $S_i$~--- множество
стратегий (ходов)\footnote{В настоящей главе, в которой
рассматриваются статические игры, то есть игры, в которых игроки
ходят один раз, одновременно и независимо, стратегия игрока и его
ход~--- это одно  и то же. Принципиальная разница, как мы увидим
ниже, возникает в динамическом случае, когда игроки делают ходы
в определенном порядке, возможно, несколько раз, и возникает необходимость
указания имеющейся в распоряжении игрков информации.}, доступных игроку
$i=1,\ldots,n$, $u_i:S=\prod_{i\in I} S_i\to R^1$~--- функция
выигрышей игрока $i$, ставящая в соответствие каждому набору
стратегий $s=(s_1,\ldots,s_n)$, называемому также ситуацией, выигрыш
этого игрока\footnote{ Разумеется, в общем случае, мы не должны
исключать случаи $S\varsubsetneqq\prod S_i,$ соответствующие
так называемым играм с запрещенными ситуациями.  Однако мы здесь
такие игры не рассматриваем.}.

Стандартные примеры здесь~--- дуополия по Бертрану и по Курно, о
которых мы говорили выше. В обеих моделях $I=\{1,2\}$,
в первой стратегия -- это выбор цены, во второй -- выбор объема выпуска,
а выигрыши~--- это прибыль.

Формально в дуополии по Курно $S_i=[0,+\infty)$, если нет
ограничений по мощности, и $S_i=[0,Q_i]$, если такие ограничения
есть (в данном случае $Q_i$ -- максимальный объем продукции,
который может выпустить фирма $i$). Отметим, что в рассмотренном
нами случае линейного спроса $P(Q)=a-Q$ фирмам бессмысленно
производить больше $a$, тем не менее это формально не
накладывает ограничений на множество возможных стратегий фирм. Функции
выигрышей в этой модели -- это функции $\pi_i(q_1,q_2), i=1,2$. В
дуополии по Бертрану тоже $S_i=[0,+\infty)$, однако теперь уже
выбираются не объемы производства, а цены.

Важным предположением, которое играет ключевую роль в теории,
является предположение о том, что все игроки {\it {рациональны}}, в
том смысле, что каждый игрок рассматривает имеющиеся в его
распоряжении альтернативы, формирует представления относительно
неизвестных параметров, имеет четко определенные предпочтения и
выбирает свои действия в результате некоторого процесса оптимизации
(максимизации своей целевой функции).  Более того, не менее
существенным является факт общеизвестности (общего знания)\footnote{
Common knowledge.} рациональности игроков, т.\,е. все игроки не
только рациональны, но и знают, что другие игроки рациональны,
знают, что все игроки знают,  что они рациональны и т.\,д.
Формальное определение общеизвестности см. Aumann (1976).

Остановимся на этом подробнее. Как мы только что отметили, ключевое
предположение теории игр -- это \emph{рациональное поведение}
игроков. Это означает, что у каждого агента есть четко определенное
ранжирование всех логически возможных исходов, и он вычисляет
стратегию наилучшим образом отвечающую его интересам. Рациональность
агента включает в себя два важных ингредиента: полное знание своих
интересов и способность безупречно вычислять то, какое действие
действие будет наилучшим образом служить этим интересам.

При этом не менее важно отдавать себе отчет в том, \emph{что не
включается} в такое понимание концепции рациональности. Это не
означает, в частности. что игроки эгоистичны;, так игрок может
высоко оценивать "бытие"\, другого и включать эту высокую оценку в
свои выигрыши. Это не означает, что игрок исходит из короткого
горизонта планирования; в действительности вычисление будущих
последствий является важной частью стратегического мышления, и
действия, кажущиеся не очень осмысленными с точки зрения ближайшей
перспективы, могут играть весьма существенную стратегическую роль в
долгосрочной перспективе. Очень важно также то, что рациональность
вовсе не означает, что у игрока такая же система оценок, как и у
других игроков; рациональность просто означает согласованное
следование своей системе оценок. Поэтому, когда игрок анализирует,
как другие игроки будут отвечать (в игре с последовательными ходами)
или анализируют последовательные этапы размышления о размышлениях
других (в игре с одновременными ходами), он должен отдавать себе
отчет в том, что другие игроки вычисляют последствия своих действий,
исходя из своих собственных систем оценок. Игрок не должен
приписывть другим свою систему оценок и предполагать, что они будут
действовать так же как действовал бы в этой ситуации он сам.

В общем случае игрок может не знать действительных систем оценок
других игроков. Поэтому в реальности многие игры характеризуются
неполнотой и асимметричностью информации.

Обратимся к тому случаю, когда $I=\{1,2\}$ и множества стратегий
каждого из двух игроков конечны.  В этом случае игру можно
изобразить с помощью матрицы (см. рис.\,7), где $M=|S_1|$~---
число возможных стратегий игрока 1, $K=|S_2|$~--- число возможных
стратегий игрока 2, $a_{mk}=u_1(s^{(m)}_1,s^{(k)}_2)$,
$b_{mk}=u_2(s^{(m)}_1,s^{(k)}_2)$, $k=1,\ldots,K$, $m=1,\ldots,M$.
Интерпретация здесь такова: если первый игрок выбирает стратегию
$m$ (отметим вновь, что не принципиально, что именно стоит за
этим номером стратегии), а второй игрок -- стратегию $k$, то они
выигрывают соответственно $a_{mk}$ и $b_{mk}$

РИС7.

Эту же игру можно представить в виде двух матриц (поэтому такие игры
называются часто  биматричными), элементами которых являются
элементы $a_{mk}$ и $b_{mk}$, соответственно.

Для конечной {\it антагонистической} игры, т.\,е. игры двух лиц
такой, что $u_1(s_1,s_2)=-u_2(s_1,s_2)$ для всех $s_i\in S_i$,
$i=1,2$, справедливо равенство $a_{mk}=-b_{mk}$ для всех $m$ и $k$,
а поэтому такая игра может быть  задна только одной матрицей
$(a_{mk})_{m=1,\ldots,M\atop{k=1,\ldots,K}}$, и поэтому конечные
антагонистические игры называются матричными. Поскольку в
экономических приложениях матричные игры встречаются относительно
редко, хотя они представляют собой класс весьма подробно
исследованных игр, мы практически не будем их рассматривать.


В дальнейшем нам придется рассматривать стратегии игроков более
общего вида. Как мы увидим, например, при рассмотрении равновесия
по Нэшу, вопрос о его существовании решается положительно при
условии, что допускаются так называемые смешанные стратегии,
представляющие собой в некотором смысле расширение исходных
стратегий. Чтобы различать стратегии, о которых шла речь до сих
пор, и смешанные стратегии, мы будем называть первые (исходные)
\emph{чистыми стратегиями}.

{\it Смешанная стратегия}\footnote{Mixed strategy} $\sigma_i$~---
это вероятностное распределение на множестве {\it чистых} стратегий
$S_i$. (Мотивацию введения смешанных стратегий и их интерпретацию мы
оставляем на будущее). Рандомизация каждым  игроком своих стратегий
статистически независима от рандомизаций его оппонентов, а выигрыши,
соответствующие профилю (набору) смешанных стратегий~--- это
ожидаемое значение выигрышей соответствующих чистых стратегий
(т.\,е. речь здесь идет об ожидаемой полезности). Одна из причин, по
которой мы сосредотачиваемся (по крайней мере пока) на конечном
случае~--- стремление избежать осложнений, связанных с теорией меры.

Будем обозначать пространство (множество) смешанных стратегий
$i$-ого игрока через $\sum_i$, а $\sigma_i(s_i)$~--- веростность
того, что выбирается стратегия $s_i$. Пространство наборов
смешанных стратегий~--- $\sum=\prod_{i\in I}\sum_i$, элементы
которого мы будем обозначать через $\sigma$. {\it Носитель}
смешанной стратегии $\sigma_i$~--- это множество тех чистых
стратегий, которым приписана положительная вероятность.

Формальное определение смешанной стратегии следующее.

\begin{definition}
Если $S_i$~--- конечное множество чистых стратегий игрока $i$, то
{\it смешанная стратегия} $\sigma_i:S_i\to [0,1]$ ставит в
соответствие каждой чистой стратегии $s_i\in S_i$ вероятность
$\sigma_i(s_i)\ge 0$ того, что игрок выберет чистую стратегию
$s_i$, причем $\sum_{s_i\in S_i}\sigma_i(s_i)=1$.
\end{definition}

Обратим внимание на то, что в приведенных обозначениях индекс $i$
означает, что речь идет о стратегии игрока $i$. Поэтому, если мы
будем говорить о различных стратегиях игрока $i$, то мы будем
обозначать их $s_i,s'_i,s''_i,\ldots$. Нетрудно заметить, что
множество смешанных стратегий игрока $i$~--- это $(k_i-1)$-мерный
симплекс, где $k_i$ - число чистых стратегий $i$-ого игрока, то
есть

$$\Sigma_i=\{t\in\mathbb{^{k_i}}_+:\sum^{k_i}_i=1\}.$$

Так, например, если  у игрока есть всего две чистые стратегии, то
любая его смешанная стратегия имеет вид $(p,1-p)$, где
$0\leq{p}\leq1$. При этом если $p=1$, то смешанная стратегия
совпадает с первой чистой стратегией игрока, а если $p=0$, то --
со второй. Если стратегий у игрока три, то любая смешанная
стратегия будет иметь вид $(p,q,1-p-q)$, где все компоненты
неотрицательны. Заметим также, что чистые стратегии -- это вершины
симплекса смешанных стратегий.

Выигрыш игрока $i$, соответствующий профилю (набору) стратегий
$\sigma$, есть \begin{equation}\label{P1} u_i(\sigma)=\sum_{s\in
S}\biggl (\prod^n_{j=1}\sigma_j(s_j)\biggr)u_i(s). \end{equation}

В этом выражении сумма берется по всем возможным наборам чистых
стратегий $s=(s_1,s_2,...,s_n)$. Тогда произведение, стоящее в
скобках, представляет собой вероятность возникновения ситуации
$s$, если игроки придерживаются смешанных стратегий
$\sigma_1,\sigma_2, ... ,\sigma_n$. Напомним, что $\sigma_j(s_j$
-- это вероятность того, что игрок $j$, выбирающий смешанную
стратегию $\sigma_j$, играет свою чистую стратегию $s_j$.
Следовательно, выражение (1) представляет собой математическое
ожидание выигрыша игрока $j$ в ситуации $\sigma $.

Поскольку на наборах чистых стратегий значения этой функции
совпадают со значениями исходной функции выигрышей $u_i$, мы
сохраняем то же обозначение.

Важно отметить, что выигрыш $i$-ого игрока есть линейная функция
от вероятностей $\sigma_i$, а также является полиномом от профиля,
а потому непрерывен. Наконец, подчеткнем это еще раз, чистые
стратегии являются вырожденными смешанными стратегиями,
приписывающими вероятность $1$ данной чистой стратегии и
вероятность $0$ -- остальным.

\begin{definition}
Смешанным расширением игры $\Gamma=\{I,S,u\}$ называется игра
$\bar\Gamma=\{I,\sum,u\}$,  где $\sum=\prod_{i\in I}\sum_i$, а
$u(\sigma),$ где $\sigma\in\sum$ определяется равенством
(\ref{P1}).
\end{definition}

П р и м е р.\,\,  Рассмотрим игру, изображенную на рис.8.

\begin{center}
\begin{tabular}{cc}
&$\begin{array}{ccc} L\quad &M& \quad R \end{array}$\\
$\begin{array}{c} u\\m\\d \end{array}$& $\left(
\begin{array}{ccc}
(3,3)&(4,1)&(5,2)\\
(1,1)&(7,4)&(2,6)\\
(2,0)&(8,6)&(1,8)\end{array}\right)$\\
\multicolumn{2}{c}{}\\
\multicolumn{2}{c}{Рис. 8.}\\
\end{tabular}
\end{center}

Пусть $\sigma_1=\biggl({1\over  3},{1\over 3},{1\over 3}\biggr)$
(это означает, что смешанная стратегия игрока 1 приписывает ему
играть стратегии $u$, $m$  и $d$  c вероятностями  1/3),
$\sigma_2=\biggl({1\over  4}, {1\over  4},{1\over 2}\biggr)$ (эта смешанная
стратегия игрока 2 предписывает играть стратегии $L$ и $M$ с равными
вероятностями ${1\over  4}$ и играть стратегию $R$ с вероятностью ${1\over 2}$).

В данном случае мы получаем
\begin{eqnarray}\nonumber
u_1(\sigma)&=&{1\over 3}\biggl({1\over  4}\cdot 3+{1\over 4}\cdot 4+ {1\over
2}\cdot 5\biggr)+\cr &+&{1\over 3}\biggl({1\over  4}\cdot 1+{1\over 4}\cdot 7+
{1\over 2}\cdot 2\biggr)+\cr &+&{1\over 3}\biggl({1\over  4}\cdot 2+{1\over
4}\cdot 8+ {1\over 2}\cdot 1\biggr)={41\over 12}\cr
u_2(\sigma)&=&{47\over 12}.
\end{eqnarray}

\subsection{Доминирующие и доминируемые стратегии}

Рассмотрим несколько примеров. Мы начнем со знаменитой {\it Дилеммы
Заключенного}~--- в некотором смысле чрезвычайно простой игры,
которая в разных формулировках встречается в большинстве учебников
по теории игр, которая приводится едва ли не в самом начале каждого
курса и которую многие сразу же вспоминают, когда слышат
словосочетание теория игр.

{\it Дилемма Заключенного}. Ставший почти хрестоматийным сюжет
этой стилизованной истории таков. Двое подозреваемых в совершении
тяжкого преступления арестованы и помещены в одиночные камеры,
причем они не имеют возможности передавать друг другу какие-либо
сообщения. Их допрашивают поодиночке. Если оба признаются в
совершении преступления, то им грозит, с учетом их признания,
тюремное заключение сроком по 6 лет каждому. Если оба будут
молчать, то они будут наказаны за совершение какого-то
незначительного преступления (скажем, незаконное хранение оружия
или что-нибудь в этом духе) и получат в этом случае по 1 году
тюремного заключения (улик для серьезного наказания не хватает).
Если же один из них сознается, а другой -- нет, то первый, за
содействие следствию, будет вовсе освобожден от наказания, тогда
как второй будет приговорен к максимально возможному за данное
преступление наказанию~--- 10-летнему тюремному заключению.
(Приведенные в данном примере сроки, разумеется могут
варьироваться. Важно здесь также то, что каждому заключенному это
известно).

Описанная история может быть представлена следующей игрой
(рис.\,9), в которой у каждого из игроков по две чистых стратегии
-- молчать (М) и сознаться (С).

\begin{center}
\begin{tabular}{cc}
&$\begin{array}{cc} M\quad &\quad C \end{array}$\\
$\begin{array}{c} M\\  C\end{array}$& $\left(\begin{array}{cc}
(-1,-1)&(-10,0)\\
(0,-10)&(-6,-6) \end{array}\right)$\\
\multicolumn{2}{c}{}\\
\multicolumn{2}{c}{Рис.9.}\\
\end{tabular}
\end{center}

Представим себе теперь следующие рассуждения первого игрока
(заключенного). Важно помнить, что каждый из игроков рационален.
Напомним, что каждый из игроков знает всю матрицу, поэтому, если
первый игрок предположит, что второй выбирает молчать, то ему
выгоднее сознаться, так как тогда вместо года тюрьмы он будет
освобожден. Если же первый игрок предположит, что второй игрок
сознается, то ему тоже выгоднее сознаться, так как вместо 10 лет
тюрьмы он получит шесть. Совершенно аналогично может рассуждать
второй игрок (заключенный). Это означает, что и у первого игрока,
и у второго стратегия "сознаться" является лучшей стратегией. В
этом случае говорят, что стратегия "сознаться" \emph{строго
доминирует} стратегию "молчать",\, поскольку стратегия
"сознаться"\, дает больший выигрыш (меньший проигрыш) независимо
от того, какую стратегию выбирает оппонент. В этом смысле
стратегия "молчать" является \emph{строго доминируемой} стратегией
каждого из игроков. А следовательно, исход становится практически
очевидным: каждый из игроков выбирает "сознаться".

При всей кажущейся простоте этой игры, она порождает целый ряд
вопросов и оказывается исходным пунктом многочисленных
исследований, призванных каким-то образом ответить на возникающие
вопросы. Первое, что бросается в глаза - это то, что получающийся
исход (по 6 лет тюрьмы каждому) очень плохой: он дает, например,
максимальный суммарный срок заключенным и здесь об оптимальности
по Парето приходится забыть (разумеется, и мы подчеркиваем это еще
раз, не следует забывать о нашем предположении {\it
рациональности} игроков, поскольку здесь исключаются из
рассмотрения проблемы предательства, и т.\,д.). Это послужило
толчком к многочисленным исследованиям этой игры, поскольку,
например, естественным желанием было бы получить в качестве исхода
этой игры (или ее модификаций) ситуацию ("молчать", "молчать"),
дающую каждому заключенному лишь по одному году заключения. Мы
будем говорить подробнее об этом позднее, когда речь пойдет о
повторяющемся взаимодействии.

Следующая игра имеет уже ярко выраженный экономический подтекст,
хотя разделяет с дилеммой заключенного, упомянутую выше специфику,
поэтому мы позволим себе сохранить то же название:

{\it Дилемма Заключенного --- 2}. Рассмотрим две нефтедобывающие
страны, которые мы назовем, скажем, А и В.  Эти две страны могут
кооперироваться (К), договариваясь об объемах ежедневной добычи
нефти, ограничиваясь, к примеру, добычей 2 млн. баррелей нефти в
день для каждой страны. С другой стороны, страны могут действовать
некооперативно (Н), добывая, скажем, по 4 млн. баррелей в день.
Такая ситуация может быть представлена следующей игрой, в которой
указаны прибыли стран в зависимости от их объемов добычи нефти
(рис.\,10).

\begin{center}
\begin{tabular}{ccc}
&&$\begin{array}{c} B\end{array}$\\
&&$\begin{array}{cc}  K\quad &\quad H \end{array}$\\
$\begin{array}{c} \\ A \\ \end{array}$& $\begin{array}{c} K\\ \\ H
\end{array}$&
$\left(\begin{array}{cc} (48,40)& (26,44)\\
\\
(54,26)&(36,28)\end{array}\right)$\\
\multicolumn{3}{c}{}\\
\multicolumn{3}{c}{Рис. 10.}\\
\end{tabular}
\end{center}

Эта картина достаточно типична для картеля, когда у каждого из
членов картеля есть стимул отклониться от договора, чтобы за счет
увеличения объемов производства (хотя и при некотором снижении
цен) получить дополнительную прибыль.

Легко видеть, что и здесь у каждого из игроков есть доминирующая
стратегия~--- (Н)~--- не кооперироваться. В результате страны
получают прибыль 36 и 28 (млн. долларов в день), что гораздо меньше,
нежели в ситуации кооперативного поведения. Как мы увидим ниже,
кооперативное поведение здесь может быть достигнуто, если
предполагать, что эта игра разыгрывается не однократно, а
потенциально может разыгрываться бесконечно.

Приведем еще один пример, который понадобится нам в дальнейшем и
который также разделяет специфику указанных выше примеров, а
именно, у каждого из игроков есть строго доминирующая стратегия, а
получающийся при этом исход нельзя считать удачным.

{\it Дилемма заключенного~-- 3}. Подобного рода игра возникает, как
считается, в ситуациях, когда два производителя, скажем, согласуют
производство каких-то деталей (кооперируются), либо, напротив,
производят детали, не согласующиеся с производством конкурента (рис. 11).

\begin{center}
\begin{tabular}{cc}
&$\begin{array}{cc} к\quad& \quad н \end{array}$\\
$\begin{array}{c} к\\ н \end{array}$& $\left(\begin{array}{cc}
(4,4)&(0,5)\\
(5,0)&(1,1) \end{array}\right)$\\
\multicolumn{2}{c}{}\\
\multicolumn{2}{c}{Рис. 11.}\\
\end{tabular}
\end{center}

Легко видеть, что здесь, как и в приведенных выше примерах стратегии
"кооперироваться" являются строго доминируемыми для каждого из
игроков.

Теперь мы готовы дать формальное определение доминирующих и
доминируемых стратегий.

Введем следующие обозначения: пусть $i\in I$, тогда через $s_{-i}\in
S_{-i}$~--- будем обозначать набор стратегий игроков из
$I\setminus\{i\}$, $(s'_i,s_{-i})$ обозначает набор стратегий
$(s_1,\cdots,s_{i-1},s'_i,s_{i+1},s_n)$.

(Заметим, что в этих обозначениях $s=(s_i,s_{-i})$).


Чистая стратегия $s_i$ игрока $i$ в игре $\Gamma$ \emph{строго
доминируема (строго доминируется)}, если существует другая чистая
стратегия $s'_i$ такая, что
$$ u_i(s'_i,s_{-i})>u_i(s_i,s_{-i}) $$ для всех
$s_{-i}\in S_{-i}$.


В этом случае говорят, что стратегия $s'_i$ \emph{строго
доминирует} стратегию $s_i$.  Стратегия $s_i$ {\it слабо
доминируется}, если существует такая $s'_i$, что указанное выше
неравенство выполняется как нестрогое неравенство, но хотя бы для
одного набора $s_{-i}$~--- неравенство строгое. Соответствующий
пример мы приведем ниже.

Совершенно естественно, что строго доминируемые стратегии надо
удалять, поскольку их использование заведомо приносит игроку меньший
выигрыш. Если игра разрешима в смысле последовательного удаления
строго доминируемых стратегий, т.\,е. каждый игрок остается с
единственной стратегией, как в нашем первом примере, то,
получившаяся ситуация будет хорошим кандидатом для предсказания
того, как будет проходить игра.

Вернемся к игре, изображенной на рис.\,8. Нетрудно убедиться в том,
что здесь в результате последовательного удаления строго
доминируемых стратегий остается пара стратегий $(u,L)$. На первом
шаге удаляется стратегия $M$ (она доминируется стратегией $R$).
Затем удаляется стратегия $m$ (доминируемая стратегией $u$).На
третьем шаге удаляется стратегия $d$ (доминируется стратегией $u$).
Наконец, на последнем шаге удаляется $R$.

Обратим особое внимание на то, что каждый из игроков может
поставить себя на место другого и провести соответствующую оценку
стратегий конкурента его (конкурента) глазами. Это крайне важное
замечание, и мы еще вернемся к нему.

Аналогично определение и для смешанных стратегий:

\begin{definition}
Смешанная стратегия  $\sigma_i$ строго доминируется в игре
$\bar\Gamma$, если существует другая стратегия $\sigma'_i$ такая,
что для всех $\sigma_{-i}\in\sum_{-i}$
$$
u_i(\sigma'_i,\sigma_{-i})>u_i(\sigma_i,\sigma_{-i}).
$$
\end{definition}

Стратегия $\sigma_i$ называется строго доминирующей стратегией для
игрока $i$ в игре $\bar\Gamma$, если она строго доминирует любую
другую стратегию из $\sum_i$.

Заметим, что для проверки того, что смешанная стратегия $\sigma'_i$
строго доминирует смешанную стратегию $\sigma_i$ стратегией , нам
достаточно посмотреть на то, как эти две стратеги ведут себя против
{\it чистых} стратегий оппонентов игрока $i$.

Сформулируем это формально:
$$
(A)\quad u_i(\sigma'_i,\sigma_{-i})>u_i
(\sigma_i,\sigma_{-i})\quad\forall\sigma_{-i}
$$
тогда и только тогда, когда
$$
(B)\quad u_i(\sigma'_i,s_{-i})>u_i (\sigma_i,s_{-i})\quad\forall
s_{-i}.
$$
Действительно: рассмотрим разность
$$
u_i(\sigma'_i,\sigma_{-i})-u_i
(\sigma_i,\sigma_{-i})=\sum_{s_{-i}\in S_{-i}} (\prod_{k\ne
i}\sigma_k(s_k))[u_i (\sigma'_i,s_{-i})-u_i(\sigma_i,s_{-i})].
$$
Тогда если (B), то (A), т.\,к. все
$[u_i(\sigma'_i,s_{-i})-u_i(\sigma_i,s_{-i})]>0$.  (B) следует из
(A), т.\,к. $s_{-i}$~--- вырожденный случай $\sigma_{-i}$.

З а д а ч а.\, Докажите, что если чистая стратегия $s_i$ является
строго доминируемой, то таковой же является и любая стратегия,
использующая $s_i$ с положительной вероятностью.

Однако смешанная стратегия может быть строго доминируемой, даже если
она использует с положительной вероятностью чистые стратегии,
которые даже не слабо доминируемы. Действительно, рассмотрим
следующую игру (рис.\,12).

\begin{center}
\begin{tabular}{cc}
&$\begin{array}{cc} L\quad &\quad R \end{array}$\\
$\begin{array}{c} u\\ M\\ D\end{array}$& $\left(\begin{array}{cc}
(1,2)&(-2,0)\\
(-2,0)&(1,2)\\
(0,1)&(0,1) \end{array}\right)$\\
\multicolumn{2}{c}{}\\
\multicolumn{2}{c}{Рис. 12.}\\
\end{tabular}
\end{center}

Стратегия первого игрока $\left({1\over 2}, {1\over 2},0\right)$
дает ожидаемый выигрыш $-{1\over 2}$ вне зависимости от того, что
играет игрок 2, а следовательно, строго доминируется стратегией $D$.


\section {Последовательное удаление слабо доминируемых стратегий}

Рассмотрим следующую известную игру Море Бисмарка. Предыстория
события такова:  1943 г. Адмирал Imamura получил приказ доставить
подкрепление по морю Бисмарка на Новую Гвинею.  В свою очередь
адмирал Kenney должен был восприпятствовать этому. Imamura должен
был выбрать между Северным (более коротким) и Южным маршрутами, а
Kenney~--- решить, куда посылать самолеты, чтобы разбомбить
конвой. Причем в течение одного дня самолеты могут бомбить лишь на
одном из двух направлений~--- либо на Северном, либо на Южном
маршрутах (но не на двух). Поэтому, если Kenney посылает самолеты
в сторону неправильного маршрута, то они могут вернуться, но число
дней, когда возможна бомбежка, уменьшается. Описываемая ситуация
моделируется следующей игрой, в которой выигрыши~--- это число
дней, когда возможна бомбежка конвоя (естественно, со знаком + для
Kenney и -- для Imamur'ы).  Считаем, что Северный маршрут занимает
2 дня, а Южный~--- 3. (См. рис.\,13).

\begin{center}
\begin{tabular}{cccc}
&&\multicolumn{2}{c}{Imamura}\\
&&\quad\,\, Север& Юг\\
$\begin{array}{c} \\Kenney\\ \end{array}$& $\begin{array}{c} Север\\ \\
Юг \end{array}$& \multicolumn{2}{c}{$\left( \begin{array}{cc}
(2,-2)&(2,-2)\\
\\
(1,-1)&(3,-3)\end{array}\right)$}\\
\multicolumn{4}{c}{}\\
\multicolumn{4}{c}{Рис. 13.}\\
\end{tabular}
\end{center}

Вообще говоря~--- это матричная игра, т.\,е. антагонистическая игра
с конечным множеством стратегий у каждого игрока.  Ни один игрок не
имеет доминирующей стратегии. Но здесь можно говорить о слабом
доминировании: для Imamur'ы стратегия Ю слабо доминируема, так как
для любой стратегии Kenney проигрыш Imamur'ы (число дней, когда
конвой будет подвергаться бомбардировкам) не меньше для Ю, чем для
С, но для стратегии Kenney Ю~--- проигрыш при С строго меньше, чем
при Ю.


Последовательное (итерированное) удаление слабо доминируемых
стратегий проходит следующим образом: исключается одна из слабо
доминируемых стратегий одного из игроков, затем из оставшихся
стратегий исключается одна из слабо доминируемых стартегий и т.\,д.

Представим себе, что Kenney понимает это и считает, что Imamura
выбирет Север.  В этой новой ситуации Kenney имеет уже
доминирующую стратегию~--- Север. Это и дает нам равновесие при
последовательном удалении доминируемых стратегий. (В
действительности так и случилось: 2--5\,марта 1943\,г. ВВС США и
Австралии атаковали японский конвой, который шел по Северному пути
и потопили все транспортные корабли и 4 эсминца: из 7000 чел. до
Новой Гвинеи добрались 1000. Мы, конечно, не имеем в виду, что
командующие принимали свои решения, рассмотрев именно указанную
игру, однако приведенная игра может рассматриваться как достаточно
удачная модель).

Процедура последовательного удаления слабо доминируемых стратегий
аналогична удалению строго доминируемых стратегий. Однако здесь есть
одно весьма значительное отличие. А именно, множество стратегий,
которые выдерживают последовательное удаление слабо доминируемых
стратегий (то есть остаются), может зависеть от порядка удаления
стратегий.

Действительно, рассмотрим следующую игру (рис.\,14).

\begin{center}
\begin{tabular}{cc}
&$\begin{array}{cc} L\quad &\quad R \end{array}$\\
$\begin{array}{c} u\\ M\\ D\end{array}$& $\left(\begin{array}{cc}
(2,2)&(1,1)\\
(2,2)&(3,2)\\
(1,1)&(3,2) \end{array}\right)$\\
\multicolumn{2}{c}{}\\
\multicolumn{2}{c}{Рис. 14.}\\
\end{tabular}
\end{center}

Если вначале удаляется $U$ (слабо доминируется $M$), а затем $L$
(слабо доминируется $R$), то мы приходим к исходу (3,2) (второй
игрок выбирает $R$). Если же вначале удаляется $D$ (слабо
доминируется $M$), а затем  $R$ (слабо доминируется $L$), то мы
приходим к исходу (2,2).



{\it Аукцион второй цены}. У продавца есть одна единица неделимого
товара. Есть $n$ потенциальных покупателей, которые оценивают
товар соответственно в $0\le v_1\le\cdots\le v_n$, и эти оценки
являются общеизвестными.  Покупатели одновременно делают свои
заявки (назначают цену) $s_i\in[0,+\infty)$. Назначивший
максимальную заявку получает товар и платит вторую цену, т.\,е.
если игрок $i$ выигрывает ($s_i>\max_{j\ne i}s_j$), то его
полезность есть $u_i=v_i-\max_{j\ne i}s_j$, а остальные ничего не
получают и ничего не платят (т.\,е. $u_j=0$). Если несколько
покупателей назначают высшую цену, то товар
 распределяется случайным образом (например, равновероятно).

Легко убедиться в том, что стратегия назначения своей оценки
($s_i=v_i$) слабо доминирует все остальные. Действительно, пусть
$r_i\equiv\max_{j\ne i}s_j$.  Пусть $s_i>v_i$. Тогда, если $r_i\ge
s_i$, то $i$-ый участник получает 0, что он получил бы и при
$s_i=v_i$. Если $r_i\le v_i$, то он получает $v_i-r_i$, что он опять
же получает, назначив $v_i$. Если теперь $v_i<r_i<s_i$, то его
полезность  $v_i-r_i<0$, а если бы он назвал $v_i$, то он бы получил
0. Аналогично и для $s_i<v_i$: если $r_i\le s_i$  или $r_i\ge v_i$,
то он получает ту же полезность, назвав $v_i$ вместо $s_i$. Если же
$s_i<r_i<v_i$, то он упускает возможность получить положительную
полезность.

Полезно в данном случае заметить, что поскольку назначение
собственной оценки есть доминирующая стратегия, то не играет роль,
имеют ли покупатели информацию об оценках других.


\section{Никогда не лучшие ответы. Рационализуемость стратегий}

Мы обсуждали исключение строго доминируемых стратегий, исходя из
того, что рациональный игрок никогда не выбрал бы такую стратегию,
вне зависимости от того, как играют его оппоненты. Однако общее
знание структуры игры и того, что игроки рациональны, позволяет
исключить {\it больше}, нежели просто последовательно удалить строго
доминируемые стратегии, причем здесь опять же важную роль играет
общее знание.  Далее мы рассматриваем смешанное расширение
$\bar\Gamma$ игры $\Gamma$.

\begin{definition}
Стратегия $\sigma_i$ является лучшим ответом игрока $i$ на набор
стратегий оппонентов $\sigma_{-i}$, если
$u_i(\sigma_i,\sigma_{-i})\ge u_i(\sigma'_i,\sigma_{-i})$ при любых
$\sigma'_i\in\sum_i$. Стратегия $\sigma_i$ является никогда не
лучшим ответом\footnote{ Never a best response} (далее НЛО), если не
существует такого набора стратегий оппонентов  $\sigma_{-i}$, для
которого она была бы лучшим ответом.
\end{definition}

Естественно, что рациональный игрок не будет играть стратегию,
которая является никогда не лучшим ответом.

Совершенно очевидно, что строго доминируемая стратегия является
никогда не лучшей. Важно однако то, что может случиться так, что
стратегия окажется никогда не лучшим ответом, даже если она не
является строго доминируемой (см. пример ниже). Таким образом,
удаляя стратегии, являющиеся никогда не лучшими ответам, мы удаляем
тем самым и все стратегии, являющиеся строго домиинируемыми
стратегиями. Более того, поскольку мы предполагаем общее знание, мы
можем итерировать удаление никогда не лучших ответов: рациональный
игрок не должен играть НЛО, как только он исключает возможность
того, что его противники могут играть НЛО и т.\,д.

Стратегии, остающиеся после такого итеративного удаления,~--- это те
стратегии, которые рациональный игрок может оправдать, или {\it
рационализовать}, разумеется, при некоторых разумных предположениях
о выборе своих противников.

\begin{definition}
Стратегии в $\sum_i$, которые выдерживают последовательное удаление
никогда не лучших ответов назывются рационализуемыми стратегиями.
\end{definition}

Понятие рационализуемых стартегий было введено независимо Бернхеймом
и Пирсом (Bernheim, 1984; Pearce, 1984).

Можно показать, что так же, как и при последовательном удалении
строго доминируемых стратегий, порядок удаления не существенен.
Подчеркнем еще раз, что множество рационализируемых стратегий не
может быть шире, чем множество стратегий, выживающих, при
последовательном удалении строго доминируемых стратегий, поскольку
на каждом шаге процесса, определяющего множество рационализируемых
стратегий, все стратегии, строго доминируемые на данном шаге,
удаляются.

П р и м е р\,\, (Osborne, Rubinstein) Рассмотрим следующую игру
(см.\,рис.\,16)

\begin{center}
\begin{tabular}{cc}
&$\begin{array}{cccc} b_1\qquad& b_2\quad\,& b_3&\qquad b_4 \end{array}$\\
$\begin{array}{c} a_1\\ a_2\\ a_3\\ a_4 \end{array}$&
$\left(\begin{array}{cccc} (0,7)&(2,5)&(7,0)&(0,1)\\
(5,2)&(3,3)&(5,2)&(0,1)\\
(7,0)&(2,5)&(0,7)&(0,1)\\
(0,0)&(0,-2)&(0,0)&(10,-1)\end{array}\right)$\\
\multicolumn{2}{c}{}\\
\multicolumn{2}{c}{Рис. 15.}\\
\end{tabular}
\end{center}

На первом шаге исключения удаляется стратегия $b_4$, т.\,к. она
является НЛО, поскольку она строго доминируется, например, смешанной
стратегией $\sigma_2=({1\over 2},0,{1\over 2},0)$ или
${\sigma^1}_2({2\over 3},{1\over 3},0,0)$. Как только исключена
$b_4$, можно исключить $a_4$, т.\,к. она строго доминируется $a_2$
(поскольку $b_4$ удалена). Но дальше мы уже не можем удалить ни одну
стратегию, т.\,к. $a_1$~--- лучший ответ на $b_3$, $a_2$~--- на
$b_2$ и $a_3$~--- на  $b_1$. Аналогично остаются $b_1$, $b_2$,
$b_3$. Таким образом, множество рационализуемых чистых стратегий
есть $\{a_1,a_2,a_3\}$ для игрока 1 и $(b_1,b_2,b_3)$~--- для игрока
2.

Для каждой рационализуемой стратегии, игрок может построить
последовательность оправданий своего выбора без ссылок на убеждение
в том, что другой игрок не будет играть НЛО-стратегию. Например, в
этой игре игрок 1 может оправдать выбор $a_2$ убеждением, что  игрок
2 будет играть $b_2$, которое игрок 1 может оправдать убеждением,
что игрок 2 будет думать, что он собирается играть $a_2$, что
осмысленно, если игрок 1 убежден, что игрок 2 думает, что он, игрок
1, думает, что игрок 2 будет играть $b_2$  и т.\,д.

Мы отметили, что множество рационализуемых стратегий не больше, чем
множество стратегий, остающихся после последовательного удаления
строго доминируемых стратегий. Однако в случае двух игроков ($n=2$)
эти два множества совпадают, так как  в игре 2-х лиц (смешанная)
стратегия $\sigma_i$ является лучшим ответом на некоторую стратегию
противника, если $\sigma_i$ не является  строго доминируемой. Если
чистая стратегия $s_i$  игрока $i$  является НЛО для любой смешанной
стратегии оппонента, тогда $s_i$ строго доминируется некоторой
смешанной стратегией $\sigma_i\in\Sigma_i$.

Посмотрим это на примере (Mas-Colell, Whinston, Green)
(см.\,рис.\,16).

\begin{center}
\begin{tabular}{cc}
&$\begin{array}{cc} L\quad &\quad R \end{array}$\\
$\begin{array}{c} U\\ M\\ D\end{array}$& $\left(\begin{array}{cc}
(10,1)&(0,4)\\
(4,2)&(4,3)\\
(0,5)&(10,2) \end{array}\right)$\\
\multicolumn{2}{c}{}\\
\multicolumn{2}{c}{Рис. 16.}\\
\end{tabular}
\end{center}

У игрока 1~--- три стратегии $U$, $M$ и $D$. $U$~--- лучшая против
$L$, но худшая против $R$, $D$ лучшая против $R$, и худшая~---
против $L$. С другой стороны, $M$ относительно неплоха и против $L$
и против $R$. Ни одна из этих трех стратегий не доминируется никакой
другой. Но если разрешить игроку 1 рандомизацию, то игра $U$ и $D$ с
вероятностями $1/2$ каждая дает игроку 1 ожидаемый выигрыш 5, вне
зависимости от стратегии второго игрока, тем самым строго доминируя
$M$.

Предположим теперь, что выигрыши от использования стратегии $M$
изменены так, что $M$ не является строго доминируемой. Тогда
выигрыши от $M$ лежат где-то выше, чем линия, соединяющая точки,
соответствующие стратегиям $U$ и $D$. Здесь оси соответствуют
ожидаемым выигрышам игрока 1 в случае, если игрок 2 играет $R$ (ось
$u_R$) и $L$ (ось $u_L$) (см. рис.\,18). Линия $ab$ -- это множество
$$
\{(u_R,u_L):{1\over 2}u_R+{1\over 2}u_L= {1\over 2}u_1(M,R)+{1\over
2}u_1(M,L)\}
$$
Является ли $M$ здесь лучшим ответом? ДА


РИС18.


Действительно, земетим, что если игрок 2 играет $R$ с вероятностью
$\sigma_2(R)$, тогда ожидаемый выигрыш игрока 1 от выбора стратегии
с выигрышами ($u_R,u_L$) есть $\sigma_2(R)u_R+(1-\sigma_2(R))u_L$.
Легко видеть, что $M$~--- это лучший ответ на $\sigma_2(R)=1/2$; он
дает ожидаемый выигрыш, строго больший, чем ожидаемый выигрыш,
достижимый с помощью стратегий $U$ и/или $D$. (В случае $n>2$ это
уже не так: могут быть стратегии, являющиеся HЛО, но не являющиеся
строго доминируемыми; это связано с тем, что  рандомизация
осуществляется независимо).


\section{Равновесие по Нэшу}

Теперь мы переходим к ключевому для нас понятию -- понятию
равновесия по Нэшу. Вернемся к рассмотренным нами выше вариантам
Дилеммы Заключенного. Во всех трех случаях мы видели, что у каждого
из игроков была (строго) доминирующая стратегия, однозначно
определявшая исход игр. В общем случае, конечно же, таких
доминирующих стратегий у игроков может не быть. Однако, обратим
внимание вот на какую особенность этих доминирующих стратегий. Если
мы рассмотрим пару стратегий, определяющую исход в указанных
примерах, то мы увидим, что ни одному из игроков не выгодно
отклоняться от этой своей стратегии, при условии, что его конкурент
(партнер) придерживется выбранной стратегии. То есть, если в первом
варианте Дилеммы заключенного оба игрока выбирают сознаться, то ни
первому, ни второму игркоу не будет выгодно \emph{в одиночку}
отклониться от выбранной стратегии: если первый игрок (гипотетичски)
решит отклониться и сыграть молчать, а второй при этом будет
придерживаться выбранной им стратегии сознаться, то первому игроку
станет только хуже. Аналогично и второму игроку.

Именно такое рассуждение и лежит в основе определения равновесия по
Нэшу. Равновесием по Нэшу называется такая ситуация (то есть такой
набор стратегий каждого из игроков), что ни одному из игроков не
выгодно отклоняться от этой ситуации в одиночку. Необходимо особенно
подчеркнуть невыгодность именно \emph{одиночного отклонения}. (В
принципе, разумеется, гораздо более сильным являлось бы рассмотрение
ситуаций, от которых не выгодно отклоняться целым группам игроков,
однако такое определение равновесия уже сталкивается с достаточно
серьезными проблемами существования).

Итак, определим теперь равновесие по Нэшу формально.

Мы начнем со случая, когда рассматривается исходная игра $\Gamma,$ а
к смешанному расширению обратимся несколько позже.

\begin{definition}
Набор стратегий $s=(s_1,\ldots,s_n)$ образует  равновесие по Нэшу
(или ситуация $s=(s_1,\ldots,s_n)$ является равновесной по Нэшу) в
игре $\Gamma=\{I,\{S_i\},\{u_i\}\}$, если для любого игрока
$i=1,\ldots n$
$$
u_i(s_i,s_{-i})\ge u_i(s'_i,s_{-i})\quad\forall\quad s'_i\in S_i.
$$
\end{definition}

Иными словами (подчеркнем это еще раз), если игрок в одиночку решает
отклониться от выбранной стратегии, то он разве лишь ухудшит свое
положение.

Здесь следует иметь в виду то, что поскольку мы сейчас рассматриваем
игры, в которых игроки ходят всего лишь один раз делают ходы
независимо (не зная ходов других игроков), то отклоняться игроки
могут только гипотетически, рассуждая приблизительно следующим
образом: если мои партнеры выберут стратегии $s_{-i}$, то мне нужно
выбрать стратегию $s_i$, так как иначе я получу меньше, получил бы в
этом случае. Но почему мои партнеры сыграют $s_{-i}$? Но ведь они
могут рассуждать также как я, то есть они будут знать, что если они
выбирают $s_{-i}$, то я, будучи рациональным игроком, должен выбрать
$s_i$ и т.\,д. (мы вернемся к этому рассуждению чуть позже и обсудим
его более подробно).

В ситуации равновесия по Нэшу выбранная каждым игроком стратегия
является лучшим ответом на стратегии, {\it  действительно} играемые
соперниками. В этом принципиальное отличие от рационализируемости,
которая следует из общего знания о рациональности друг друга и
структуры игры и требует только, чтобы стратегия игроков была лучшим
ответом на некоторую разумную гипотезу о том, \emph{что} его
противник будет играть, причем под разумностью понимается, что
гипотетическая игра его противников может быть также оправдана.
Равновесность по Нэшу добавляет к этому требование того, чтобы
игроки были {\it правы}  в своих гипотезах. (Далее мы для краткости
будем писать р.Н. для обозначения равновесия по Нэшу.)

Как мы уже отмечали, полученные нами ситуации в рассмотренной ранее
Дилемме Заключенного, (во всех ее вариантах) являются равновесными
по Нэшу.
\smallskip

П р и м е р.\,\,  Семейный спор{. Этот пример также относится к
числу традиционных примеров, различные вариации которого встречаются
в большинстве учебников. История примерно такова. Он и Она
независимо (мы оставляем в стороне вопрос о разумности или
неразумности подобной постановки вопроса) решают, куда пойти~--- на
балет (Б) или  футбол (Ф). (Другие варианты здесь -- это бокс или
фильм, концерт из произведений Баха или Стравинского и т.\,д.) Если
они вместе пойдут на футбол, то Он получит больше удовольствия, чем
Она; если они вместе пойдут на балет, то~--- наоборот. Наконец, если
они окажутся в разных местах, то это будет им не в радость.
Рассматриваемая ситуация моделируется следующей игрой (см.
рис.\,19):

\begin{center}
\begin{tabular}{ccc}
&&ОHА\\
&&$\begin{array}{cc} Ф\quad&\quad Б\end{array}$\\
$\begin{array}{c} \\ ОH\\ \end{array}$& $\begin{array}{c} Ф\\ \\
Б\end{array}$& $\left(\begin{array}{cc}
(2,1)&(0,0)\\
\\
(0,0)&(1,2)\end{array}\right) $\\
\multicolumn{3}{c}{}\\
\multicolumn{3}{c}{Рис. 19.}\\
\end{tabular}
\end{center}

Легко видеть, что здесь есть 2 равновесия по Нэшу в чистых
стратегиях~--- (Ф,Ф) и (Б,Б). Мы увидим ниже, что в этой игре есть
еще одно равновесие по Нэшу~--- в смешанных стратегиях.
\smallskip

Рассмотрим еще два примера, которые близки рассмотренному только что
семейному спору, но в то же время имеют некоторые отличия.

П р и м е р.\,\, Представим себе теперь , что ситуация в предыдущем
примере несколько изменилась. Теперь речь идет просто о том, чтобы
встретиться, скажем, в одном из двух кафе, которые назовем для
простоты кафе A и B. Представим себе, например, что они достаточно
часто встречаются (к примеру, после работы) в одном из них, но
забыли (второпях) договориться в каком именно (забудем ненадолго о
мобильных телефонах и будем считать, что расстояние между кафе
достаточно велико, чтобы можно было быстро перебраться из одного в
другое). Тогда соответствующую игру можно представить себе следующим
образом (рис. 20).

\begin{center}
\begin{tabular}{ccc}
&&ОHА\\
&&$\begin{array}{cc} A\quad&\quad B\end{array}$\\
$\begin{array}{c} \\ ОH\\ \end{array}$& $\begin{array}{c} A\\ \\
B\end{array}$& $\left(\begin{array}{cc}
(1,1)&(0,0)\\
\\
(0,0)&(1,1)\end{array}\right) $\\
\multicolumn{3}{c}{}\\
\multicolumn{3}{c}{Рис. 20.}\\
\end{tabular}
\end{center}


Рассмотрим еще одну модификацию игры "Семейный спор", которая
отличается от только что тем, что одно из кафе, скажем B, по
каким-то соображениям (ну например, есть опасения встретить в кафе А
кого-то) более привлекателено для обоих наших героев.

Тогда соответствующая игра может быть представлена следующим образом
(рис. 20').

\begin{center}
\begin{tabular}{ccc}
&&ОHА\\
&&$\begin{array}{cc} A\quad&\quad B\end{array}$\\
$\begin{array}{c} \\ ОH\\ \end{array}$& $\begin{array}{c} A\\ \\
B\end{array}$& $\left(\begin{array}{cc}
(1,1)&(0,0)\\
\\
(0,0)&(2,2)\end{array}\right) $\\
\multicolumn{3}{c}{}\\
\multicolumn{3}{c}{Рис. 20'.}\\
\end{tabular}
\end{center}

В каждой из этих трех игр есть два равновесия по Нэшу в чистых
стратегиях - и Он, и Она выбирают одинаковые стратегии.

Все три приведенных игры относятся к числу так называемых
координационных игр, поскольку в таких играх у игроков общие,
хотя и не обязательно совпадающие интересы. В то же время, в силу
того, что они осуществляют свой выбор независимо, достижение
предпочтительного исхода становится проблематичным. Поэтому, если
есть множественность равновесий, то для осуществления успешного
выбора должна быть какая-то координация представлений игроков
относительно действий других игроков.

Эта ситуация близка ситуации, в которой есть то, что называется
\emph{фокальной точкой}. Хороший пример на эту тему можно найти
в книге Dixit, Skeath (2004). Вкратце он состоит в следующем.

Два студента-химика одного из университетов прекрасно учились на
протяжении всего семестра, выполняя все промежуточные лабораторные
и другие задания, так что не сомневались в успехе на итоговом экзамене.
Они решили на уикенд поехать за город к своим друзьям,
и проведя довольно бурно выходные, вернулись домой под утро понедельника,
на который был назначен экзамен. Чувствуя себя не вполне готовыми к экзамену,
они решили попросить профессора разрешить им сдать экзамен во вторник,
сославшись на то, что они всю ночь не спали, так как, возвращаясь накануне
в город, прокололи колесо, а поскольку у них не было запасного, то они большую
часть ночи провели в ожидании помощи и слишком устали, чтобы успешно
сдавать экзамен. Профессор пошел им навстречу. В понедельник ребята готовились,
и во вторник были готовы сдать экзамен. Профессор посадил их
в разные аудитории и дал задания. Вопросы на первой странице, вес которых
составлял одну десятую, были очень простые. На второй странице был всего один
вопрос (весил он 9/10): "Какое колесо?".

Если студенты предвидели подобную ситуацию, то, конечно, они могли заранее
договориться, и такой вопрос не вызвал бы никаких затруднений. (На самом
деле, идея предвидения будущих ходов является крайне важной и представляет
собой общий принцип выбора стратегий, и мы будем говорить об этом ниже).
Но если они не предвидели такого рода подвох со стороны профессора, то
могут ли они рассчитывать на согласованный обман, отвечая независимо? Если
будут отвечать случайным образом, то вероятность того, что они выберут одно
и то же колесо, будет 1/4. Могут ли они достичь чего-то большего?

Вы можете считать, что наиболее вероятным (поскольку чаще всего осколки стекла
или мусор оказывается ближе к обочине) является прокол правого переднего колеса.
Вы можете считать, что это логика хороша, однако это вовсе не гарантирует, что
это будет хорошим выбором, поскольку не только Ваша логика определяет исход,
но еще нужно, чтобы ваш друг исходил бы из той же логики. Таким образом Вы
должны подумать о том, чтобы ваш друг использовал ту же логику и считал
этот выбор столь же очевидным. Будет ли ваш друг считать, что этот выбор
столь же очевиден для вас? И так далее. Важно здесь не то, что выбор
очевиден или логичен, а то, очевидно ли другому, что это очевидно для вас,
что это очевидно для другого и т.\,д. Иными словами, нужно, чтобы
представления о том, что нужно выбирать в подобных обстоятельствах,
"сходились". Такая взаимно ожидаемая стратегия называется \emph{фокальной
точкой}.

В общем случае, конечно, фокальная точка в координационной игре может быть
не найдена, и возможность ее нахождения будет существенным образом зависеть
от наличия некоторой общей установки, детерминируемой социо-культурным
контекстом.

Хотя, как мы уже отмечали, в третьем из приведенных примеров оба
предпочитают кафе В, однако это, вообще говоря не гарантирует встречи в нем,
даже если оба они, как мы это предполагаем, знают всю матрицу выигрышей.
Такое детальное знание может возникать, если, к примеру, оба игрока обсуждали
достоинства того или иного кафе, но в итоге забыли договориться встретиться
именно в В. Или Он может думать, что Она может по каким-то причинам выбрать
А, или думать, что Она думает, что Он думает таким образом. Без "сходимости
представлений"\, они могут выбрать худшее равновесие, или, что еще хуже,
получить по 0. В реальной жизни в подобного рода ситуации гарантировать
хорошее равновесие бывает достаточно просто. Интересы игроков совпадают,
поэтому, если, скажем, один говорит другому, что он собирается пойти в
кафе В, то у другого нет оснований сомневаться в этом.

П р и м е р.\,\, Рассмотрим следующую игру (рис.\,21):

\begin{center}
\begin{tabular}{cc}
&$\begin{array}{ccc} l\qquad& m &\qquad r \end{array}$\\
\begin{tabular}{c} U\\ M\\ D\\ \end{tabular}&
$\left( \begin{array}{ccc} (4,2)&(0,3)&(2,4)\\
(3,1)&(4,4)&(3,0)\\
(2,4)&(1,6)&(4,2) \end{array}\right)$\\
\multicolumn{2}{c}{}\\
\multicolumn{2}{c}{Рис. 21.}\\
\end{tabular}
\end{center}

Ясно, что здесь набор стратегий $(M,m)$ образует равновесие по Нэшу.
Если игрок 1 выбирает $M$, то у 2-го лучший ответ~--- $m$ и
наоборот.
\smallskip

П р и м е р.\,\, Вернемся к примеру, касавшемуся рационализуемости
(рис.\,15).  В нем существует единственная (даже если разрешены
смешанные стратегии) ситуация равновесия по Нэшу~--- ($a_2$, $b_2$).

Этот пример иллюстрирует общее взаимоотношение между p.H. и
рационализуемыми стратегиями. {\it Каждая стратегия, являющаяся
частью p.H., рационализуема}, поскольку каждая стратегия игрока в
ситуации p.H. может быть оправдана равновесными стратегиями других
игроков.  Таким образом, равновесие по Нэшу предсказывает как
минимум не хуже, чем рационализуемость, впрочем, очень часто эти
предсказания оказываются значительно более четкими.

Очень удобно следующее переопределение равновесия по Нэшу. Введем
следующее многозначное \emph{отображение лучших ответов} $b_i:S_{-i}\to
S_i$ (в игре $\Gamma$):
$$
b_i(s_{-i})=\{s_i\in S_i:u_i(s_i,s_{-i})\ge
u_i(s_i',s_{-i})\quad\forall\quad s'_i\in S_i\}.
$$
Тогда нетрудно видеть, что ситуация $(s_1,\ldots,s_n)$ является
равновесием по Нэшу в игре $\Gamma$, если $s_i\in b_i(s_{-i})$
$\forall \,\, i=1,\ldots,n$.

Иными словами, равновесие по Нэшу -- это в некотором смысле
"неподвижная точка"\, отображения лучших ответов. Но как можно
ответить на что-то, что еще не произошло? Как можно находить
равновесия? Один метод нахождения -- наблюдать и экспериментировать.
Он, безусловно, хорош, если игра разыгрывается многократно, и каждый
новый розыгрыш дает вам новую информацию о том, \emph{что} делали игроки.

Второй метод -- это размышлять о том, \emph{что} думают другие: вы
ставите себя на место других участников игры и думаете о том, о чем
думают они. Так возникает то, что в теории игр принято называть
\emph{представлениями} (о том, что будут делать другие игроки в игре
с одновременными ходами). Если игрок не знает действительного выбора
других игроков, но имеет представление о нем, то в равновесии по
Нэшу эти представления должны быть правильными, то есть
действительный выбор должен совпадать с вашими представлениями о
том, каким он (выбор) должен быть. В этом смысле равновесие по Нэшу
можно переопределить следующим образом.

Равновесие по Нэшу -- это такой набор стратегий игроков и
представлений, что: (1) у каждого игрока правильные представления
относительно стратегий других игроков; (2) стратегия каждого из
игроков является лучшей при данных его представлениях о стратегиях
остальных игроков.

В таком определении равновесия по Нэшу есть следующие преимущества.
Во-первых, становится понятным смысл понятия "лучший ответ": каждый
игрок выбирает свой лучший ответ не на еще ненаблюдаемые действия
других игроков, а на уже сформированные его собственные
представления об их действиях. Во-вторых, когда мы будем говорить о
смешанных стратегиях, это позволит нам интерпретировать случайность
в выборе игроком своих стратегий как неопределенность в
представлениях остальных игроков относительно действий этого игрока.


Что же можно сказать по поводу того, а почему собственно нам нужно
заниматься pавновесием по Hэшу? На самом деле это один из проблемных вопросов
теории игр, несмотря на очень широкое использование равновесия по Нэшу.
\begin{itemize}
\item[(1)] {\it Равновесие по Нэшу как последовательность рациональных
выводов (умозаключений)}. Хотя это часто используется в качестве
довода, тем не менее мы видим, что следствие общего знания~--- это
необходимость играть рационализируемые стратегии.  Рациональность не
обязательно ведет к правильности предсказания.

\item[(2)] {{\it Равновесие по Нэшу как необходимое условие, если есть
единственный предсказуемый исход игры}. Если игроки думают и
разделяют представления о том, что существует {\it очевидный} (в
частности, единственный) способ играть игру, то это должно быть p.H.
Разумеется, этот аргумент подходит, если существует  единственное
предсказание, как игроки будут играть. Однако, вспомнив
рационализуемость, мы придем к выводу, что этого недостаточно.
Поэтому этот аргумент полезен, если есть действительно повод считать
некоторый набор стратегий очевидным способом сыграть в игру.}

\item[(3)] {\it Фокальные точки}. Иногда случается так, что  определенный
исход является тем, что Шеллинг (1960) называет {\it фокальным
исходом}, о котором мы говорили выше. То есть речь идет о некотором
\emph{напрашивающемся} исходе. Это, конечно, явный кандидат, но
только если он pавновесие по Hэшу.

\item[(4)] {\it Равновесие по Нэшу как самофорсирующее соглашение}.
Если игроки перед игрой имеют возможность предварительных
необязывающих переговоров, и если они соглашаются на какой-то исход,
то это, конечно, очевидный кандидат. Чтобы он стал самофорсирующим,
нужно, чтобы он был p.H. Хотя даже, если они договорились играть
p.H., они все равно могут отклониться, если ожидают, что другие
могут тоже уклониться.

\item[(5)] {\it Рановесие по Нэшу  как устойчивое социальное
соглашение}.  Определенный способ играть в игру может возникнуть во
времени, если игра разыгрывается повторно и появляется некоторое
устойчивое социальное соглашение. Если это так, то для игроков может
быть очевидным, что это соглашение будет поддерживаться. Это
соглашение становится, так сказать, фокальным.

Более подробное обсуждение этой проблематики можно найти, например,
в учебнике Mas-Colell, Whinston, Green. \end{itemize}
\smallskip

П р и м е р.\,\, {\it Аукцион второй цены.} Мы видели, что стратегия
назначения своей цены $(s_i=v_i)$ слабо доминирует все остальные
стратегии. Оказывается (см., например, Moulin, 1986), что в этой
игре {\it много} равновесий по Нэшу. Точнее, для любого игрока $i$ и
для любой заявки $s$ такой, что $0<s\leq v_i$ существует по крайней
мере одно равновесие по Нэшу, при котором игрок $i$ платит $s$ и
получает товар.

Действительно, выберем $i\in I$, зафиксируем $s\in(0,v_i]$ и
положим:
$$
x^*_i=\max_{1\leq j\leq n} v_j+1,
$$
$$
x^*_j=s\qquad \forall\, j\ne i.
$$
Тогда $x^*=(x^*_1,\ldots,x^*_n)$~--- равновесие по Нэшу.

В этом случае единственная возможность отклониться у игрока $i$~---
это уступить товар кому-либо другому, получая в этом случае нулевой
выигрыш. Единственная возможность игрока $j$, $j\ne i$~--- это
получить товар, заплатив цену $x^*_i$, превосходящую $v_j$, получая
тем самым отрицательный выигрыш.

\section{Равновесие по Нэшу в смешанных стратегиях}

Примеры, которые мы рассмотрели выше, продемонстрировали, что даже в
очень простых играх равновесие по Нэшу в чистых стратегиях может
быть не единственным.  Однако, как мы увидим сейчас, равновесия в
чистых стратегиях может не существовать вообще.
\smallskip

П р и м е р.\,\, Игра в орлянку, или Орел или решка. 2 игрока
одновременно, независимо выбирают либо решку, либо орла. Если их
выбор различен, то первый игрок платит второму 1 рубль (доллар, и
т.\,д.), если их выбор одинаков, то наоборот~--- второй платит
первому столько же. Соответствующая игра имеет следующий вид
(рис.\,22):

\begin{center}
\begin{tabular}{cc}
&$\begin{array}{cc} 0\qquad&\quad p\end{array}$\\
$\begin{array}{c} 0\\p\end{array}$& $\left(\begin{array}{cc}
(1,-1)&(-1,1)\\
(-1,1)&(1,-1) \end{array}\right)$\\
\multicolumn{2}{c}{}\\
\multicolumn{2}{c}{Рис. 22.}\\
\end{tabular}
\end{center}

Легко видеть, что в этой игре нет равновесия по Hэшу в чистых
стратегиях, так как в любой ситуации одному из игроков выгодно
отклониться от выбранной стратегии. Однако, как мы увидим, пара
смешанных стратегий $\sigma_1=({1\over 2},{1\over 2}),$
$\sigma_2=({1\over 2},{1\over 2}),$ в которых каждый из игроков
играет свои чистые стратегии с равными вероятностями, образует
равновесие по Hэшу в смешанных стратегиях.

Определение равновесия по Нэшу в смешанных стратегиях в точности
повторяет определение для случая чистых стратегий с очевидной
заменой игры на ее смешанное расширение.

\begin{definition}
Ситуация (набор смешанных стратегий)
$\sigma=(\sigma_i,\ldots,\sigma_n)$ является равновесием по Нэшу в
игре $\bar\Gamma=\{I,\{\Sigma_i\},\{u_i\}\}$, если для любого
$i=1,\ldots,n$
$$
u_i(\sigma_i,\sigma_{-i})\ge u_i(\sigma'_i,\sigma_{-
i})\quad\forall\quad\sigma'_i\in\Sigma_i.
$$
\end{definition}

Весьма полезным для нахождения равновесий по Нэшу в смешанных
стратегиях оказывается следующей утверждение.

\begin{proposition}
Пусть $S^+_i\subset S_i$~---  множество чистых стратегий, которые
игрок $i$ играет с положительной вероятностью в ситуации
$\sigma=(\sigma_1,\ldots,\sigma_n)$.  Ситуация $\sigma$ является
p.H. в смешанном расширении $\bar\Gamma$ игры $\Gamma$ тогда и
только тогда, когда для всех $i=1,\ldots,n$
$$
\begin{array}{rcl} &(1)&\quad
u_i(s_i,\sigma_{-i})=u_i(s'_i,\sigma_{-i})\quad\forall\quad
s_i,s'_i\in
S_i^+\\[10pt]
&(2)&\quad u_i(s_i,\sigma_{-i})\ge u_i(s'_i,\sigma_{-i})\quad
\forall\quad s_i\in S^+_i,\,\,s'_i\notin S_i^+. \end{array}
$$
\end{proposition}

Д о к а з а т е л ь с т в о.\,\, Hеобходимость. Если бы одно из этих
условий не выполнялось для некоторого $i$, то нашлись бы две
стратегии $s_i\in S^+_i$ и $s'_i\in S_i:u_i(s'_i,\sigma_{-i})>
u_i(s_i, \sigma_{- i})$, а значит, это не p.H.

Достаточность. Предположим теперь, что (1) и (2) выполнены, но
$\sigma$~--- не  p.H. Тогда существуют игрок $i$ и стратегия
$\sigma'_i$ такие, что
$$
u_i(\sigma'_i,\sigma_{-i})>u_i(\sigma_i,\sigma_{-i}).
$$
Но если это так, то существует чистая стратегия $s'_i$, которая
играется с положительной вероятностью при $\sigma_i'$ и для которой
$u_i(s_i',\sigma_{-i})> u_i(\sigma_i,\sigma_{-i})$. Так как
$u_i(\sigma_i,\sigma_{-i})=u_i(s_i, \sigma_{-i})$ для любой $s_i\in
S^+_i$, это противоречит (1) и (2).

Таким  образом, необходимые и  достаточные условия того, что
ситуация $\sigma$~--- p.H., состоят в том, что: 1) каждый игрок при
данном распределении стратегий, которые играют его противники,
безразличен между чистыми стратегиями, которые он играет с
положительной вероятностью; 2) эти чистые стратегии  не хуже тех,
которые он играет с нулевой вероятностью.

Применим это свойство для нахождения равновесия по Hэшу в смешанных
стратегиях в следующей игре.
\smallskip
\clearpage П р и м е р.\,\, Рассмотрим следующую игру (рис.\,23):

\begin{center}
\begin{tabular}{cc}
&$\begin{array}{cc} A\qquad&\qquad\, B\end{array}$\\
$\begin{array}{c} A\\ B\end{array}$& $\left(\begin{array}{cc}
(100,100)& (0,0)\\
(0,0)&(10,10)\end{array}\right)$\\
\multicolumn{2}{c}{}\\
\multicolumn{2}{c}{Рис. 23.}\\
\end{tabular}
\end{center}

Очевидно, что ситуации (А,А) и (В,В) являются равновесными по Hэшу
(в чистых стратегиях). Hайдем равновесия по Hэшу в смешанных
стратегиях. Предположим, что в таком равновесии игрок 1 играет
смешанную стратегию $(p,1-p)$, а второй~--- $(q,1-q)$, причем
$0<p,q<1$.

Тогда, учитывая приведенное предложение, мы получаем, что ожидаемый
выигрыш игрока 2 от игры $A$ есть $100p+0(1-p)$, а от игры $B$ есть
$10\cdot(1-p)+0p$, а значит,
$$
100p+(1-p)\cdot 0=10\cdot (1-p)+0\cdot p.
$$
Отсюда $110p=10$ и, следовательно, $p=1/11$. Аналогично $q=1/11$.
Заметим, что в соответствии с предложением ???? у игроков в данном
примере нет предпочтений относительно вероятностей, которые они
приписывают своим стратегиям. Эти вероятности определяют равновесное
рассмотрение: необходимость сделать {\it другого} игрока
безразличным относительно {\it его} стратегий. Иными словами, игрок
выбирает вероятности так, чтобы оппоненту было безразлично (с точки
зрения ожидаемых выигрышей), какую чистую стратегию играть.
\smallskip

П р и м е р.\,\, Вернемся к игре Семейный спор. Поступая так же, как и в
предыдущем примере, мы получаем, что Она,  играя "Ф", получает
$1\cdot p+0(1-p)$, а играя "Б", получает $0\cdot p+2(1-p)$.
Следовательно, $2(1-p)=p$. Отсюда $3p=2$, а следовательно, $p=2/3$.
Аналогично получаем $2q+(1-q)\cdot 0=0\cdot q+(1-q)1$, а значит,
$3q=1$ и $q=1/3$. Таким образом, в смешанном равновесии Он играет
"Ф" с вероятностью $2/3$, а Она играет "Ф",  с вероятностью $1/3$.

Здесь уместно более подробно остановиться на одном из приведенных
выше примеров, скажем на модификации игры "Семейный спор",\, которая
описывается следующей матрицей:

\begin{center}
\begin{tabular}{ccc}
&&ОHА\\
&&$\begin{array}{cc} A\quad&\quad B\end{array}$\\
$\begin{array}{c} \\ ОH\\ \end{array}$& $\begin{array}{c} A\\ \\
B\end{array}$& $\left(\begin{array}{cc}
(1,1)&(0,0)\\
\\
(0,0)&(2,2)\end{array}\right) $\\
\multicolumn{3}{c}{}\\
\multicolumn{3}{c}{Рис. 24.}\\
\end{tabular}
\end{center}

На этом примере мы посмотрим, как можно использовать метод анализа лучших
ответов, а также проиллюстрируем "правило безразличия". Детальный анализ
общего случая биматричных игр $2\times2$ представлен в дополнении в конце главы.

Мы видели уже, что здесь есть два равновесия по Нэшу в чистых стратегиях.
Так, если Она уверена, что Он пойдет в кафе А, то она тоже пойдет в кафе А;
если Она уверена, что Он пойдет в кафе В, то Она пойдет туда же. Но если
у Нее нет такой уверенности, \emph{что} является ее лучшим ответом?
(Технический термин в теории вероятностей для такой "неуверенности"\, --
это субъективная неопределенность).

Она может считать, что вероятность того, что Он выберет А, есть
$p$, причем эта вероятность может принимать любое значение между нулем и единицей.
Тогда вероятность того, что Он выберет В (в ее расчетах), будет $1-p$.
Таким образом, субъективная неопределенность для Нее описывается следующим образом:
Она считает, что Он использует смешанную стратегию, "смешивая"\, чистые стратегии
А и В, соответственно с вероятностями $p$ и $1-p$. Для краткости мы будем называть
такую Его смешанную стратегию $p$-смесью (хотя эта смесь целиком в Ее расчетах).
Каков же Ее лучший ответ, если Она исходит из таких предпосылок?

Ее рассуждения можно представить себе следующим образом. "Если я пойду в кафе А,
то Он с вероятностью $p$ окажется там, и я получу выигрыш $1$; с вероятностью
$1-p$ Он будет в кафе В, и я получу выигрыш $0$. Ожидаемый выигрыш от такой развития
событий есть $p\times1+(1-p)\times0=p$. Аналогично, ожидаемый выигрыш от выбора
похода в кафе В против Его $p$-смеси есть $p\times0+(1-p)\times2=2(1-p).$

На рис. СЕМСПОР1 изображены Ее лучшие ответы в зависимости от Его $p$-смеси.
Возрастающая прямая, соответствующая Ее походу в кафе А, пересекает убывающую
линию, соответствующую Ее походу в кафе В, в точке $p=2/3$. Слева от точки пересечения
Она получает больший ожидаемый выигрыш от похода в кафе В, а справа -- от похода в кафе А.
Если $p=2/3$, то ожидаемые выигрыши от обеих Ее чистых стратегий одинаковы, а значит,
и от любой Ее $q$-смеси этих двух чистых стратегий.

СЕМСПОР1

На рис. СЕМСПОР2 изображена кривая Ее лучших ответов в зависимости от различных значений $p$.
Если $p<2/3$, то Ее лучший ответ -- это $А, (q=0)$. Если $p>2/3$, то Ее лучший ответ --
это $В, (q=1)$. Наконец, если $p=2/3$, то любые значения $q$ для Нее одинаково хороши.
Эта часть кривой лучших ответов представляет собой вертикальный отрезок от $q=0$ до $q=1$
при $p=2/3$.

СЕМСПОР2

Наличие вертикального отрезка  в Ее кривой лучших ответов имеет важное следствие. Если
$p=2/3$, то все Ее чистые и смешанные стратегии одинаково хороши для Нее, поэтому Она
может выбирать между двумя чистыми стратегиями или выбирать их случайным образом.
Ее субъективная неопределенность относительно того, что делает Он, может привести к
объективной неопределенности относительно Ее собственных действий. То же самое относится и к Нему.
Равновесие по Нэшу в этой игре будет ситуацией, в которой каждый игрок имеет как раз
"правильное"\, представление о неопределенности, касающейся действий другого, чтобы
оправдать свою рандомизацию.

Аналогично можно построить Его кривую лучших ответов. Обе кривых лучших ответов изображены
на рис. СЕМСПОР3. Они пересекаются в трех точках. Первой, слева внизу, соответствует точка
$p=0, q=0$, т.\,е.\, оба выбирают $B$ с правильными представлениями, что другой поступает так же.
Третьей, справа вверху, соответствует точка$p=0, q=0$, т.\,е.\, оба выбирают $А$ с
правильными представлениями, что другой поступает так же. Это как раз два равновесия в
чистых стратегиях, о которых мы говорили ранее.

СЕМСПОР3

Еще одна, вторая точка пересечения, это уже точка $p=2/3, q=2/3$. В этом равновесии оба игрока
используют свои смешанные стратегии. Каждый не уверен в том, что именно выберет другой, и здесь
мы имеем равновесный баланс их субъективных неопределенностей. Поскольку Она думает, что Он
"смешивает"\, кафе с вероятностью $2/3$, ей ничего не остается, как тоже "смешивать"\, с
$q=2/3$. То же самое верно и для Него.


Таким образом, вычисленные только что специфические значения $p$ и $q$
обладают очень важной характеристикой, уже отмеченной выше.
Равновесное значение $p$ для Него таково,
что Ей \emph{безразлично}, какую стратегию играть -- любую чистую или любую
смешанную. Ее равновесное значение $q$ таково, что Ему \emph{безразлично},
какую стратегию играть -- чистую ли, смешанную ли. Таким образом,
равновесная смесь \emph{каждого} игрока такова, что \emph{другому}
игроку безразлично, какую смесь выбирать.

Если обратиться к сформулированной выше предложению и приведенному
только что "принципу безразличия",\, то мы приходим к уже знакомому нам
принципу дополняющей нежесткости. А именно, если мы допускаем, что
некоторые чистые стратегии могут оказаться неиспользованными, то мы
должны несколько модифицировать или расширить упомянутый "принцип
безразличия"\,. Равновесная смесь каждого игрока должна быть такой,
что другой игрок безразличен по отношению к любой стратегии, которая
\emph{действительно используется в его равновесной смеси}. Другой игрок
\emph{не безразличен} между такими стратегиями и \emph{не используемыми}
стратегиями: он предпочитает используемую стратегию неиспользуемой.
Иными словами, против равновесной смеси соперника каждая из используемых
вами стратегий в вашей равновесной смеси должна давать один и тот же
ожидаемый выигрыш, который должен быть больше, чем вы получили бы любой
из ваших неиспользуемых стратегий. Подробное обсуждение принципа
дополняющей нежесткости в контексте равновесия по Нэшу в смешанных
стратегиях можно найти в книге Dixit, Skeath, 2004.

Рассмотрим еще один пример, который показывает, что в равновесии по
Нэшу в смешанных стратегиях возможны не вполне очевидные ситуации.

П р и м е р. (Kreps, 1993)\,\, Рассмотрим следующую игру.

$$\begin{array}{rcccl}
&t_1&t_2&t_3&t_4\\
s_1&(200,6)&(3,5)&(4,3)&(0,-1000)\\
s_2&(0,-10000)&(5,-1000)&(16,3)&(3,20)
\end{array}$$

Нетрудно заметить, что в этой игре есть два равновесия по Нэшу в
чистых стратегиях: это пары стратегий $s_1, t_1$  и $s_2, t_4$. Как
мы сейчас увидим в равновесии по Нэшу в смешанных стратегиях второй
игрок не играет стратегии $t_3$ и $t_4$. Рассмотрим следующую пару
стратегий: первый игрок играет стратегию $s_1$ с вероятностью
9000/9001 и стратегию $s_2$ с вероятностью 1/9001, а второй игрок
играет стратегию  $t_1$ с вероятностью 1/101 и стратегию $t_2$ с
вероятностью 100/101. Тогда при такой стратегии второго игрока
первый игрок получает, играя $s_1$:
$$200\cdot1/100+3\cdot100/101+4\cdot0+0\cdot0=500/10;$$
а играя $s_2$:
$$0\cdot1/101+5\cdot100/101+6\cdot0+3\cdot0=500/10.$$

Поэтому первый игрок "довольствуется"\, такой смешанной стратегией.
Второй игрок против стратегии первого игрока получает (это нетрудно
проверить): 44000/9001 от игры $t_1$, и от игры $t_2$, но 27003/9001
от игры $t_3$ и -8999980/9001 от игры $t_4$. Поэтому второй игрок
безразличен между стратегиями $t_1$ и $t_2$, но предпочитает их
стратегиям $t_3$ и $t_4$. Область лучших ответов второго игрока (на
указанную стратегию первого игрока) -- это любая смешанная
стратегия, использующая с нулевой вероятностью $t_3$ и $t_4$.

Таким образом в равновесии по Нэшу в смешанных стратегиях второй
игрок вовсе не использует стратегию $t_4$, которая была его чистой
равновесной стратегией.

Обратим внимание еще раз на то, что в такого типа равновесии
вероятности, используемые игроками для рандомизации, зависят не от
их собственных выигрышей, а от выигрышей других игроков. Суть
рандомизации здесь в том, чтобы сделать другого (других) игрока
безразличным между теми стратегиями, которые он рандомизирует.


В определении смешанного расширения или равновесия в смешанных
стратегиях мы предполагаем, что игроки осуществляют рандомизацию
своих чистых стратегий независимо. Иными словами, мы можем считать,
например, что Природа передает игрокам индивидуальные, независимо
распределенные сигналы $(\theta_1,\theta_2,\cdots,\theta_n)
\in[0,1]\times [0,1]\times\ldots\times [0,1]$, а каждый игрок
$i$принимает решение в зависимости от различных возможных реализаций
его сигнала $\Theta_i$.

Предположим, однако, что есть некий общий сигнал $\Theta\in [0,1]$,
который могут наблюдать все игроки. В этом случае появляются новые
возможности. Так, к примеру, в упомянутой только что игре "Семейный
спор" оба игрока могут, например, решить идти на футбол, если,
скажем, $\theta<{{1}\over{2}}$, и идти на балет, если $\theta\ge
{{1}\over{2}}$. (Можно, например, предположить, что таким
сигналом является погода: если идет дождь, то идти на балет,
а если дождя нет, то -- на футбол). Выбор стратегии каждым игроком остается {\it
случайным}, тем не менее здесь мы имеем дело со вполне
скоординированными действиями (Он и Она оказываются вместе), явно
имеющими равновесный характер, причем если один игрок решает
следовать этому правилу, то и для второго оптимально придерживаться
этого же правила. Это дает нам пример {\it коррелированного
равновесия} (совместного равновесия)\footnote{ Correlated
equilibrium.}, введенного Р.\,Ауманом (Aumann, 1974).


Смешанные стратегии и равновесие по Нэшу в смешанных стратегиях
заслуживают того, чтобы остановиться на них более подробно. Когда
ходы в игре делаются одновременно, игрок не может отвечать на
\emph{действительный} выбор других игроков. Вместо этого он
предпринимает свое наилучшее действие, исходя из того, \emph{что он
думает} относительно того \emph{что} другой (другие) могут выбрать в
этот момент. В п.   мы назвали такие размышления
\emph{представлениями} игроков относительно выбора стратегий другими
игроками. После этого мы интерпретировали равновесие по Нэшу как
конфигурацию, в которой эти представления правильны, так что каждый
игрок выбирает свой лучший ответ на действительные ходы других.
Однако раньше мы рассматривали только равновесия по Нэшу в чистых
стратегиях и тем самым негласно предполагали, что каждый игрок в
своих представлениях исходит из того, что остальные игроки выбирают
определенную чистую стратегию. Теперь же, поскольку мы рассматриваем
стратегии более общего вида, понятие представлений нуждается в
соответствующей реинтерпретации.

Игроки могут быть неуверенными относительно того, что могут делать
другие игроки. В координационной игре, в которой Он хотел встретить
Ее, Он может быть не уверен в том, в какое кафе Она пойдет и Его
представления могут быть, например, такими, что шансы, что Она
пойдет в кафе А, есть 50 на 50. Важно, однако, различать "быть
неуверенным"\, и "иметь неверные представления"\,. Скажем, в игре
"Семейный спор"\, Он может быть не уверен в том, \emph{что} Она
выберет. Однако у него могут правильные представления относительно
выбираемой Ею вероятностной смеси. "Иметь правильные
представления"\, относительно смешанных стратегий других игроков
означает знание, или вычисление, или догадка относительно
действительных ("правильных") вероятностей выбора другими игроками
своих чистых стратегий. Отсюда возникает эквивалентное определение
равновесия по Нэшу в смешанных стратегиях в терминах представлений:
каждый игрок формирует свои представления относительно вероятностей,
с которыми другие игроки выбирают свои чистые стратегии, и выбирает
свой лучший ответ. Равновесие по Нэшу в смешанных стратегиях
возникает тогда, когда представления правильны.

Итак, мы видим, что если один из игроков выбирает свою равновесную
смесь, то лучшим ответом другого может быть любая смесь, включая
крайний случай двух чистых стратегий. Это значит. что равновесие
по Нэшу в смешанных стратегиях является равновесием в слабом смысле:
когда один из игроков выбирает свою равновесную смесь, то у другого
нет мотивов отклониться от его собственной равновесной смеси, хотя
ему не станет хуже, если он выберет любую другую смесь или даже чистую
стратегию. На первый взгляд кажется, что это подрывает основы
равновесия по Нэшу в смешанных стратегиях, поскольку напрашивается
вопрос, а почему нужно выбирать соответствующую смесь, если
другой игрок выбирает свою? Почему не играть просто какую-то
чистую стратегию? Кроме всего прочего, ожидаемые выигрыши одинаковы.
Ответ здесь таков: сделать так значит уклониться от равновесия по Нэшу.
Это не будет устойчивым исходом, поскольку тогда другой игрок не стал
бы выбирать свою смесь.

Отметим еще раз тот момент, который представляется важным и с которым
мы уже сталкивались в разобранных примерах. Для этого рассмотрим
общую ситуацию, изображенную на рис. ????

\begin{center}
\begin{tabular}{cc}
&$\begin{array}{cc} U\qquad&\qquad\, D\end{array}$\\
$\begin{array}{c} L\\ D\end{array}$& $\left(\begin{array}{cc}
(x,X)& (y,Y)\\
(z,Z)&(t,T)\end{array}\right)$\\
\multicolumn{2}{c}{}\\
\multicolumn{2}{c}{Рис. 23.}\\
\end{tabular}
\end{center}

Предположим, что в этой игре есть равновесие по Нэшу в смешанных
стратегиях, в котором $U$ играется с вероятностью $p$, а $D$ --
с вероятностью $1-p$. Для того, чтобы гарантировать, что второй
игрок тоже будет играть смешанную стратегию, $p$ должно быть таким,
чтобы второму игроку было безразлично, какую стратегию играть --
$L$ или $R$, то есть должно выполняться равенство
$$pX+(1-p)Z=pY+(1-p)T$$
или
$$p=(T-Z)/[(X-Y)+T-Z)].$$
Удивительным здесь является не то, что содержится в выражении для
$p$, а то что в этом выражении \emph{не содержится}. Ни один из
собственных выигрышей первого игрока $x, y, z, t$ не появляется
в правой части этого выражения. Аналогично и для второго игрока.
Если вспомнить принцип безразличия, то, конечно, становится понятным,
что ничего удивительного в этом нет. Неприятность здесь в том, что,
по большому счету, только в антагонистической игре у игрока есть
причины делать оппонента безразличным, поскольку предпочтение
оппонента, отдаваемое им одной из его чистых стратегий, ухудшает
положение игрока (в силу антагонистичности игры). В неантагонистическом
же случае такое безразличие оппонента уже не имеет под собой столь же
очевидного обоснования. Это уже скорее логическое свойство
равновесия в смешанных стратегиях.

В заключение мы приведем важные результаты о существовании
равновесий по Hэшу.

\begin{proposition}
В смешанном расширении $\bar\Gamma$ любой игры $\Gamma$ с конечными
множествами стратегий $S_1,\ldots,S_n$ существует равновесие по Нэшу
в смешанных стратегиях.
\end{proposition}

Это предложение непосредственно следует из следующего более общего
результата, так как в игре $\bar\Gamma$ множества стратегий
игроков~--- это симплексы в соответствующем пространстве $\Re^M$.

Hапомним, что функция $f:\Re^K\to\Re$ называется квазивогнутой, если
для любого $a$ множество $\{x:f(x)\ge a\}$ выпукло.

\begin{theorem}
{\rm Debreu, 1952; Glicksberg, 1952; Fan Ky, 1952)}\footnote{
Русский перевод статьи Гликсберга опубликован в сб.: Бесконечные
антагонистические игры. Под ред. H.\,H.\,Воробьева. М.: Физматгиз,
1963. В русских переводах можно встретить две версии транскрипции
Fan Ky: Фань Цзи (см., например, упомянутый выше сборник) и Ки Фань
(см., например, Обен Ж.-П., Экланд\,И. Прикладной нелинейный анализ.
М.: Мир, 1988).}. Если для каждого $i=1,\ldots,n$ выполнены следующие
два условия:
\begin{itemize}
\item[(1)] $S_i$~--- непусто, выпукло и компактно (в некотором
пространстве $\Re^M$);
\item[(2)] $u_i(s_1,\ldots,s_n)$~--- непрерывна по $(s_1,\ldots,s_n)$ и
квазивогнута по $s_i$, \end{itemize} то в игре
$\Gamma=\{I,\{S_i\},\{u_i\}\}$ существует равновесие по Hэшу в
чистых стратегиях.
\end{theorem}


Доказательство этого предложения опирается на следующую лемму.

\begin{lemma} \label{???}
Если выполнены условия Теоремы 1.7.1, то отображение лучших ответов
$b_i$ непусто, выпукло-значно (т.\,е. множества $b_i(s_{-i})$~---
непусты и выпуклы) и полунепрерывно сверху\footnote{ Многозначное
отображение $F$ называется полунепрепрывным сверху (п.н.св.), если
из $x_n\to x$, $y_n\in F(x_n)$, $y_n\to y$ следует $y\in F(x)$.
Таковым, например, является отображение, которое ставит в соответствие
каждой точке поверхности (границы) выпуклого компакта множество нормалей
к этой поверхности в этой точке}.
\end{lemma}

Д о к а з а т е л ь с т в о\,\, леммы \ref{???}. Во-первых,
заметим, что $b_i(s_{-i})$~---  это множество тех стратегий $i$-го
игрока, которые максимизируют $u_i(\cdot,s_{-i})$ на компакте $S_i.$
Его непустота следует из непрерывности $u_i.$ Выпуклость множества
$b_i(s_{-i})$ следует из квазивогнутости функции
$u_i(\cdot,s_{-i}).$ Чтобы проверить полунепрерывность сверху, мы
должны показать, что для любой последовательности $(s^k_i,s^k_{- i})
\to(s_i,s_{-i})$ такой, что $s^k_i\in b_i(s^k_{-i})\forall k,$ мы
имеем $s_i\in b(s_{-i})$. Заметим, что $\forall k$
$u_i(s^k_i,s^k_{-i})\ge u_i(s'_i,s^k_{-i})$ $\forall\quad s'_i \in
S_i$. В силу непрерывности $u_i(\cdot),$ $u_i(s_i,s_{-i})\ge
u_i(s'_i, s_{-i})$.

Д о к а з а т е л ь с т в о\,\,  Теоремы. Определим отображение
$b:S\to S$ формулой
$$
b(s_1,\ldots,s_n)=b_1(s_{-1})\times b_2(s_{-2})\times\cdots\times
b(s_{-n}).
$$
Ясно, что $b(\cdot)$~--- многозначное отображение
$S=S_1\times\cdots\times S_n$ в себя. По лемме $b(\cdot)$ непусто,
выпукло-значно, полунепрерывно сверху. Следовательно, по
Т.\,Какутани о неподвижной точке, существует неподвижная точка
этого отображения, т.\,е. такой набор стратегий $s\in S$, что $s\in b(s)$. Э
тот набор стратегий является равновесием по Нэшу, т.\,к. по построению
$$
s_i\in b_i(s_{-i})\quad\forall\quad i=1,\ldots,n.
$$
\smallskip

П р и м е р.\,\, "Голосование". Рассмотрим следующую ситуацию~---
три игрока 1, 2, 3 и три альтернативы~--- $A$, $D$, $C$.

Игроки голосуют одновременно за одну из альтернатив, воздержаться
невозможно. Таким образом, пространство стратегий $S_i=\{A,B,C\}$.
Альтернатива, получившая большинство, побеждает.  Если ни одна из
альтернатив не получает большинства, то выбирается альтернатива $A$.
Функции выигрышей таковы:
$$
u_1(A)=u_2(B)=u_3(c)=2,
$$
$$
u_1(B)=u_2(C)=u_3(A)=1,
$$
$$
u_1(C)=u_2(A)=u_3(B)=0.
$$
В этой игре три равновесных исхода\footnote{ Вообще говоря, под
исходом следует понимать полное описание результата игры: и
выбранные стратегии, и соответствующие выигрыши игроков, и,
возможно, какие-то другие атрибуты (например, объявление о том, что
победил такой-то игрок $X$). В данном случае мы имеем в виду
победившую альтернативу.} (в чистых стратегиях): $A$, $B$  и $C$.
Теперь посмотрим на равновесия (их больше трех): например, если игроки 1 и 3
голосуют за $A$, то игрок 2 не изменит исход, как бы он ни
голосовал, и игроку 3 безразлично, как он голосует.

$(A,A,A)$ и $(A,B,A)$~--- p.H.,  но $(A,A,B)$~--- не p.H., т.\,к.
второму лучше голосовать за $B$.

Теперь мы можем вернуться к рассмотрению описанных в начале этой
главы примеров и посмотреть на них с точки зрения введенного
равновесия по Нэшу.

\subsection{Модель дуополии по Курно}


Теперь мы можем посмотреть на рассмотренную в начале модель
дуополии по Курно с точки зрения равновесия по Нэшу.
Итак две фирмы $i=1,2$ производят однородный продукт и $q_1,q_2$~---
объемы производства этого продукта. Обратная функция спроса имеет
вид (для простоты) $P(Q)=a-Q$, где $Q=q_1+q_2$, $(P(Q)=a-Q$, при
$Q<a$, и $P(Q)=0$, при $Q\ge a$).  Функции затрат $C_i(q_i)=cq_i$
$(c<a)$ (нет фиксированных затрат и предельные затраты постоянны).
(Разумеется, и спрос и функции затрат могут иметь значительно более
обший вид, однако в данном случае нас интересует в первую очередь
структура модели).

Фирмы выбирают $q_i$ {\it одновременно} и независимо. Здесь два
игрока, стратегии $S_i=[0,+\infty)$. (Мы уже отмечали, что, конечно,
ни одна фирма не будет производить $q_i>a$.) Фирмы максимизируют
свои прибыли:
$$
\pi_i(q_1,q_2)=q_i(P(q_1+q_2)-c)=q_i[a-(q_1+q_2)-c].
$$
Нетрудно убедиться, что в той постановке, о которой шла речь в
начале главы, нас интересует равновесие по Нэшу этой модели. Если
пара $(q^*_1,q^*_2)$~--- p.H., то $q^*_i$ решает задачу
$$
\max\pi_i(q_i,q^*_j).
$$
(В данных обозначениях имеется в виду, что игрок $i$ выбирает $q_i$,
а игрок $j$ выбирает $q^*_j$).


Предположим $q^*_j<a-c$ (можно доказать, что это действительно так),
тогда условие 1 порядка дает нам
$q_i={1\over 2}(a-q^*_j-c)$. Тогда
$$
{q^*_1={1\over 2}(q-q^*_2-c) q^*_2={1\over
2}(a-q^*_1-c)}\Rightarrow q^*_1=q^*_2={{a-c}\over 3}\quad (<a-c).
$$
Напомним, что монопольный выпуск был бы $(a-c)/2$.

Как мы уже видели, при исследовании дуополии по Курно важную роль
играют функции лучших ответов (кривые реагирования)~--- это функции
вида
$$
\begin{array}{rcl}
R_2(q_1)&=&{1\over 2}(a-q_1-c),\\[10pt]
R_1(q_2)&=&{1\over 2}(a-q_2-c). \end{array}
$$

Как мы отмечали ранее, $R_i(q_j)$~--- это объем выпуска $i$-й фирмы,
максимизирующий ее прибыль, при условии, что $j$-я фирма производит
$q_j$, то есть это лучший ответ фирмы $i$ на (гипотетический) ход фирмы $j$.
Кривые реагирования изображены на рис. Дуопол1


Рис. Дуопол1


Точка пересечения кривых реагирования определяет равновесие по
Курно, т.\,е.  равновесие по Нэшу в модели дуополии по Курно.

\subsection{Равновесие по Нэшу в дуополии по Курно как результат обучения}

Мы сейчас посмотрим на нашу модель насколько иначе, а именно, мы
будем предполагать теперь, что игроки пытаются предсказывать игру
своих оппонентов, используя свой предыдущий опыт. Эта идея восходит
еще к Курно, который рассматривал своеобразный динамический вариант
нахождения равновесия. При этом игроки выбирали объем выпуска по
очереди как лучший ответ,  исходя из выбора оппонента на предыдущем
шаге, предполагая (гипотеза Курно), что  оппонент оставит свой объем
выпуска без изменения.

Точнее, если игрок 1 делает ход в период $0$ и выбирает $q^0_1$, то
выпуск игрока 2 в период 1 есть $q^1_2=r_2(q^0_1)$, где
$r_2(\cdot)$~--- функция реагирования второго игрока. Затем
$$
q^2_1=r_1(q^1_2)=r_1(r_2(q^0_1)).
$$

Если этот процесс сходится к $(q^*_1,q^*_2)$, то $q^*_2=r_2(q^*_1)$
и $q^*_1=r_1(q^*_2)$, т.\,е. $(q^*_1,q^*_2)$~--- p.H. Если процесс
сходится к некоторому состоянию $(\bar q_1,\bar q_2)$ для любого
начального состояния, достаточно близкого к $(\bar q_1,\bar q_2)$,
то говорят, что
состояние $(\bar q_1,\bar q_2)$~--- асимптотически устойчиво, а сам
процесс называется процессом нащупывания (см. рис.\,???)\footnote{
Вообще говоря, этот процесс можно рассматривать и без чередования
ходов, когда каждая фирма на следующем шаге выбирает объем выпуска
как лучший ответ на предыдущий выбор конкурента (см., например,
Fudenberg, Levine, 1998).}.


Рис. Дуопол2


В общем случае картина может быть более сложной (см. рис.\,???):

Рис. Дуопол3

$B$, $D$~--- неустойчивы  (к ним процесс не сходится, если
только не начинается   в них самих), $A$ и $С$ ~--- устойчивы.

Вообще говоря, достаточное условие устойчивости выглядит следующим
образом:
$$
\biggl |{dr_1\over dq_2}\biggr |\cdot\biggl |{dr_2\over dq_1}\biggr
|<1.
$$
Заметим, что если функции выигрышей дважды непрерывно
дифференцируемы, то наклон функции реагирования $i$-й фирмы есть
$$
{dr_i\over dq_j}=-{\partial^2 u_i\over{\partial\pi_i\partial
q_j}}\Big /{{\partial^2\pi_i}\over{\partial q^2_i}}.
$$

\subsection{Дуополия по Бертрану}

{\bf 1. Парадокс Бертрана.} Возвращаемся к ситуации, когда две фирмы
производят однородный продукт, но теперь мы предполагаем, что фирмы
одновременно и независимо объявляют цену, по которой они готовы
продавать свою продукцию. Тогда, как мы говорили, спрос, с которым
сталкивается каждая фирма, определяется следующим образом:
$$
D_i(p_i,p_j)={D(p_i),\quad{\mbox{\rm если}}\quad p_i<p_j,
D(p_i)/2, \quad{\mbox{\rm если}}\quad p_i=p_j, 0,\quad{\mbox{\rm
если}}\quad p_i>p_j.}
$$
Иными словами, фирма, назначившая меньшую цену, получает, весь
спрос, а если цены одинаковы, то потребители покупают продукцию фирм
равновероятно. В этом случае $D(p_i)/2$~--- ожидаемый спрос.

Предположим, что цены ($p^*_1,p^*_2$) образуют равновесие по Нэшу.
Во-первых, очевидно, что $p^*_i\ge c$, так как назначение цены ниже
предельных затрат приведет к отрицательной прибыли, чего не может
быть в равновесии, т.\,к. цена, равная предельным затратам,
обеспечивает нулевую прибыль. Далее, ни одна из цен $p^*_i$ не может
быть выше $c$.  Действительно, предположим для определенности, что
$p^*_1>c$, тогда, если $p^*_2\ge p^*_1$, то фирма 2, сталкивающаяся
в этом варианте в лучшем случае с половинным спросом, может
перехватить весь спрос, назначив цену $p'_2=p^*_1-\varepsilon$ для
достаточно малого $\varepsilon>0$ и тем самым улучшив свое
положение. Если же $p^*_1>p^*_2>c$, то фирма 1 аналогично может
назначить цену $p^*_2-\varepsilon$, перехватывая весь спрос.

Таким образом, в равновесии по Бертрану (или в равновесии по Нэшу в
дуополии по Бертрану) $p^*_1=p^*_2=c$ и фирмы получают нулевую
прибыль. Это и есть парадокс Бертрана.

Как можно избежать этой парадоксальной ситуации? Во-первых, можно
ввести условие ограничения мощности фирм, то есть считать, что есть
цены, при которых фирмы не могут обеспечить весь спрос. Во-вторых,
можно снять условие однократности этой игры, и это, как мы увидим
позднее, существенно меняет ситуацию. Наконец, можно избавиться от
предположения об однородности продукции, которую мы сейчас кратко
рассмотрим.
\smallskip


{\bf 2. Рассмотрим ситуацию с дифференцированной продукцией.} Фирмы
1 и  2 выбирают цены $p_1$ и $p_2$ одновременно и независимо. Спрос,
с которым сталкивается фирма $i$, $q_i(p_i,p_j)=a-p_i+bp_j$, где
$b>0$ отражает степень заменяемости $i$-го продукта $j$-м.  (Мы не
останавливаемся на обсуждении того, насколько реалистична такая
функция спроса. Для нас в данном случае важно то, что, как мы увидим,
ситуация, по сравнению с только что рассмотренной, существенно
меняется.) Предельные
затраты  есть $c$, $c<a$. Пространство стратегий~--- это
$S_i=[0,\infty)$~---  фирмы выбирают цены.  Тогда прибыль  $i$-й
фирмы определяется равенством
$$
\pi_i(p_i,p_j)=q_i(p_i,p_j)[p_i-c]=[a-p_i+bp_j][p_i-c].
$$
Пара $(p^*_1,p^*_2)$ образует p.H., если $\forall i$ $p^*_i$ решает
задачу
$$
\max_{0\le p_i<\infty}\pi_i(p_i,p^*_j)=\max[a-p_i+bp^*_j][p_i-c].
$$
Решение задачи для $i$-й фирмы есть
$$
p^*_i={1\over 2}(a+bp^*_j+c),
$$
то есть
\begin{eqnarray}
p^*_1&=&{1\over 2}(a+bp^*_2+c),\nonumber\\
p^*_2&=&{1\over 2}(a+bp^*_1+c).\nonumber
\end{eqnarray}
Следовательно, $p^*_1=p^*_2=(a+c)/(2-b)$
\smallskip

В этом случае каждая из фирм получает прибыль
$$(\frac{a-c_bc}{2-b})^2$$.

\subsection{Проблема общего}

Рассмотрим следующую стилизованную модель (Hardin, 1968), которая
в различных вариантах приводится для иллюстрации проблемы безбилетника
(free rider problem).
Представим себе, что есть $n$ фермеров. Летом их козы (коровы)
пасутся на зеленом поле. Обозначим через $g_i$ число коз у $i$-го
фермера, тогда численность всего стада $G=g_1+\cdots+g_n$.  Затраты
на покупку и содержание козы равны $c$ (независимо от числа коз у
фермера). Ценность (стоимость) одной козы при общем числе коз $G$
есть $v(G)$.

Предполагая, что козе необходим определенный уровень минимального
пропитания (для выживания), считаем, что есть некоторое максимальное
число коз, которое может прокормиться на этом поле,  $G_{\max}$: $v(G)>0$ для
$G<G_{\max}$, но $v(G)=0$ для $G\ge G_{\max}$. Можно предположить,
что если есть одна коза, то  она спокойно прокормится; можно
добавить еще одну и т.\,д., но с ростом числа коз естественно
считать, что ценность падает, т.\,е. $v'(G)<0$, $(G<G_{\max})$,  и
$v''(G)<0$.

Весной фермеры выбирают (одновременно и независимо), сколько
заводить коз ($g_i$ для $i$-го фермера). Выигрыш фермера $i$ есть
$$
g_iv(g_1+\cdots+g_n)-cg_i. \eqno(*)
$$
Следовательно, если $(g^*_1,\ldots,g^*_n)$~--- p.H., то $g^*_i$
должно максимизировать $(\ast)$ при данных $(g^*_1,\ldots,g^*_{i-1},
g^*_{i+1},\ldots,g^*_n)$.  Условие первого порядка есть
$$
v(g_i+G^*_{-i})+g_iv'(g_i+G^*_{-i})-c=0,
$$
где
$$
G^*_{-i}=\sum_{k\ne i}g^*_k.
$$
Подставим в это равенство $g^*_i$ и, просуммировав по $i$, получаем
(разделив на $n$)
$$
v(G^*)+{1\over n}G^*v'(G^*)-c=0.
$$

Рассмотрим теперь, что произойдет, если "социальный плановик"\, будет
искать социальный оптимум, то есть решать задачу нахождения
$$
\max_{0\le G<\infty}Gv(G)-Gc.
$$
Здесь условие $I$ порядка есть
$$
v(G^{**})+G^{**}v'(G^{**})-c=0.
$$
Нетрудно проверить, что при сделанных предположениях относительно
функции $v$ имеет неравенство $G^*>G^{**}$, т.\,е. коз слишком много!
Иными словами, общие ресурсы используются слишком интенсивно.

\section{Равновесие "дрожащей руки"}

Мы уже обсуждали ранее слабо доминируемые стратегии и сейчас вновь
обратимся к ним. Рассмотрим игру, изображенную на рис.\,28.

\begin{center}
\begin{tabular}{cc}
&$\begin{array}{cc} L\quad &\quad R \end{array}$\\
$\begin{array}{c} U\\  D\end{array}$& $\left(\begin{array}{cc}
(2,2)&(0,-2)\\
(-2,0)&(0,0) \end{array}\right)$\\
\multicolumn{2}{c}{}\\
\multicolumn{2}{c}{Рис. 28.}\\
\end{tabular}
\end{center}

Легко видеть, что  в этой игре два р.H. в чистых стратегиях $(U,L)$
и $(D,R)$, причем второе равновесие характерно тем, что оба игрока
выбирают свои слабо доминируемые стратегии.

Мы очень кратко остановимся сейчас на понятии совершенного
равновесия (по Hэшу) дрожащей руки игры в нормальной форме\footnote{
Normal form trembling hand perfect Nash equilibrium.}, определение
которого восходит к работе Рейнхарда Зельтена\footnote{
Р.\,Зельтен~--- лауреат Hобелевской премии по экономике 1994 года.}
(Selten, 1975). Такое равновесие выдерживает возможность того, что с
некоторой очень небольшой вероятностью игроки делают ошибки (грубо
говоря, дрожащей рукой не попадая на нужные кнопки).

Для произвольной игры в нормальной форме
$\Gamma=\{I,(\Sigma_i),(u_i)\}$ можно определить "возмущенную"\, игру
$\Gamma_\varepsilon=\{I,(\Sigma^\varepsilon_i), (u_i)\}$, выбирая
для каждого игрока $i$ и каждой чистой стратегии $s_i\in S_i$ числа
 $\varepsilon_i(s_i)\in (0,1)$ так, что $\sum_{s_i\in
 S_i}\varepsilon_i(s_i)<1$, и затем, определяя множество
"возмущенных"\, стратегий как
$$
\Sigma_i^\varepsilon=\{\sigma_i\in\Sigma_i:\sigma_i(s_i)\ge\varepsilon_i(s_i)
\quad{\rm для \,\,\,всех}\quad s_i\in S_i\,\,{\rm и}
$$
$$
\sum_{s_i\in S_i}\sigma_i(s_i)=1\}.
$$

Иными словами, в игре $\Gamma_\varepsilon$ каждый игрок $i$ играет
\emph{каждую} свою стратегию $s_i$ с вероятностью не меньшей, чем некоторая
минимальная вероятность $\varepsilon_i(s_i)$, которая
интерпретируется как неизбежная вероятность сыграть $s_i$ по ошибке.

\begin{definition}
Равновесие по Hэшу $\sigma$ в игре (в нормальной форме)
$\Gamma=\{I,(\Sigma_i),(u_i)\}$ называется равновесием дрожащей
руки, если существует такая последовательность возмущенных игр
$\left\{\Gamma_{\varepsilon_k}\right\}^\infty_{k=1}$, сходящихся к
$\Gamma$ (в том смысле, что $\lim\varepsilon^k_i(s_i)=0$ для любых
$i\in I$ и $s_i\in S_i$), что существует последовательность
равновесий (в соответствующих играх $\Gamma_{\varepsilon_k}$)
$\{\sigma^k\}^\infty_{k=1}$, сходящаяся к $\sigma$, т.\,е.
$\lim\sigma^k=\sigma$.
\end{definition}

Таким образом, рассматриваемые равновесия --- это те равновесия по
Hэшу, которые "выживают"\, при возможных ошибках.

Заметим, что в определении требуется лишь {\it существование}
возмущенных игр\footnote{ Следует иметь в виду, что под
"возмущенной" игрой, в отличие от использованного здесь определения,
часто понимают игру, в которой, по сравнению с исходной игрой,
изменены {\it только} функции выигрышей.}, имеющих равновесия,
близкие к $\sigma$. Более сильным было бы требование выживания при
{\it всех} возмущениях исходной игры.

\begin{proposition} {\rm (Selten, 1975)}. В
смешанном расширении любой игры $\Gamma=\{I,\{S_i\},(u_i)\}$ с
конечными множествами стратегий $S_1,\ldots,S_n$ существует
равновесие дрожащей руки.
\end{proposition}

\begin{proposition} {\rm
(Selten, 1975)}. Равновесие по Hэшу $\sigma$ в игре в нормальной
форме $\Gamma=\{I,(\Sigma_i),(u_i)\}$ является совершенным
равновесием дрожащей руки (в игре в нормальной форме) тогда и только
тогда, когда существует последовательность таких вполне смешанных
стратегий $\sigma^k$ (т.\,е. стратегий, в которых все чистые
стратегии играются с положительными вероятностями), что
$\sigma^k\rightarrow\sigma$ и $\sigma_i$ является лучшим ответом на
любой элемент последовательности $\{\sigma^k_{-i}\}^\infty_{k=1}$
для любого $i=1,\ldots,n$.
\end{proposition}

\begin{proposition}
{\rm (Selten, 1975)}. Если $\sigma=(\sigma_i,\ldots,\sigma_n)$~---
совершенное равновесие дрожащей руки (в игре в нормальной форме), то
$\sigma_i$ не является слабо доминируемой ни для какого
$i=1,\ldots,n$.
\end{proposition}

Легко видеть, что набор стратегий $(D,R)$ в приведенном в начале
этого раздела примере, не является совершенным равновесием дрожащей
руки.

\section{Дополнение.  Решение биматричных игр 2\,x\,2}

В этом параграфе мы подробно остановимся на анализе решений
биматричных игр, в которых у каждого из игроков есть только две
стратегии. Этот анализ основан на изучении отображения лучших
ответов и здесь отчетливо проявляется то, что мы называли принципов
безразличия. Разумеется, эти игры представляют собой частный случай
рассмотренных ранее игр, но здесь появляется возможность дать
наглядную графическую интерпретацию поиска равновесных ситуаций в
игре.  (Наше изложение здесь следует книге Воробьева, 1985.)

Рассмотрим биматричную игру $2\times 2$ c матрицей
$$
\left(\begin{array}{cc}
(a_{11},b_{11})&(a_{12},b_{12})\\
(a_{21},b_{21})&(a_{22},b_{22})
\end{array} \right)
$$
или, как мы уже отмечали, игру, выигрыши в которой можно задать с
помощью двух матриц:
$$
\left(\begin{array}{cc}
a_{11},&a_{12}\\
a_{21},&a_{22}
\end{array} \right),
\left(\begin{array}{cc}
b_{11},b_{12}\\
b_{21},b_{22}
\end{array} \right),
$$
первая из которых описывает выигрыши игрока 1, а вторая~--- выигрыши
игрока 2.

Очевидно, что смешанные стратегии игроков в случае игр $2\times 2$
полностью описываются вероятностями $p$ и $q$ выбора игроками своих
первых чистых стратегий.  (Вторые чистые стратегии выбираются,
соответственно, с вероятностями $1-p$ и $1-q$.) Поэтому, поскольку
$0\leq p$, $q\leq 1$, каждая ситуация в смешанных стратегиях в
биматричной игре $2\times 2$ представляется как точка на единичном
квадрате.

Напомним, что пара смешанных стратегий $\sigma_1^*=(p^*,1-p^*)$ и
$\sigma_2^*=(q^*,1-q^*)$ является равновесием по Нэшу, если смешанная
стратегия $\sigma_i^*$ одного игрока является лучшим ответом на
смешанную стратегию $\sigma_j^*$ другого игрока, т.\,е. выполняются
следующие неравенства:
$$
u_1(\sigma_1, \sigma_2^*)\leq U_1(\sigma_1^*, \sigma_2^*)~~~ \forall \sigma _1 ~
{\rm и}
$$
$$
u_2(\sigma_1^*, \sigma_2)\leq U_1(\sigma_1^*, \sigma_2^*)~~~ \forall \sigma _2 .
$$

Рассмотрим уже знакомый нам пример Орел или Решка (см.
рис.\,23, п.\,1.8). Пусть игрок $1$ считает, что игрок $2$ будет
выбирать Орла\, с вероятностью $q$ и Решку\, с
вероятностью $1-q$.  Ожидаемый выигрыш игрока $1$ от разыгрывания
Орла\, будет $(-1)q+1\cdot (1-q)=1-2q$, а от разыгрывания
Решки\, $1\cdot q+(-1)\cdot (1-q)=2q-1 $. Если $1-2q \le 2q-1$
, т.\,е. $q\le\frac{1}{2}$, то лучшей чистой стратегией игрока $1$
будет Орел, а если $q\ge\frac{1}{2}$, то Решка, и игроку $1$ будет все
равно, что разыгрывать, если $q=\frac{1}{2}$.  Рассмотрим возможные
смешанные стратегии игрока $1$. Пусть $(p, 1-p)$ обозначает смешанную
стратегию, в которой игрок $1$ разыгрывает Орла\, с
вероятностью $p$.  Для каждого значения $q$ мы можем вычислить
значения $p=p^*(q)$, такие что $(p, 1-p)$ будет являться лучшим
ответом игрока $1$ на $(q, 1-q)$ игрока $2$.

Ожидаемый выигрыш игрока $1$ от разыгрывания $(p, 1-p)$, когда игрок $2$
разыгрывает $(q, 1-q),$ будет
$$
(-1)p\cdot q + 1\cdot p\cdot (1-q) +1\cdot  (1-p) \cdot q + (-1) \cdot
(1-p)(1-q)=
$$
$$
=(2q-1) + p\cdot (2-4q).
$$

 Ожидаемый выигрыш игрока $1$ повышается (в зависимости от $p$),
если $2-4q\ge0$ и уменьшается, если $2-4q\le0$, поэтому лучший ответ
 игрока $1$ (среди всех стратегий, как чистых, так и смешанных) есть
$p=1$ (т.\,е. Орел), если $q\le\frac{1}{2}$, но $p=0$ (т.\,е. Решка),
если $q\ge\frac{1}{2}.$ Этим значениям $p$ соответствуют два
горизонтальных отрезка на рис.\,29.

Так~как~при~$q=\frac{1}{2}$ ожи\-да\-е\-мый вы\-иг\-рыш игрока $1$ не зависит от его
стратегии, мы получаем, что игроку $1$ безразлично, выбрать ли одну
из своих чистых стратегий, или же выбрать какую-нибудь смешанную
стратегию $(p,1-p)$.

БиматрРис29

Это означает, что если $q=\frac{1}{2}$, то
смешанная стратегия $(p,1-p)$ является лучшим ответом на смешанную
стратегию $(q,1-q)$ при любом значении $p$ от $0$ до $1$.
Поэтому
$p^*(\frac{1}{2})$ представляет собой вертикальный отрезок,
изображенный на рис.\,29.
Таким образом, ломаная линия на рис.\,29
представляет собой многознач-}
\vskip14pt
\noindent
ное отображение (поскольку при $q={\frac{1}{2}}$ мы имеем целый отрезок)
лучших ответов (в зависимости от $q$). Это и есть то, о чем мы неоднократно
говорили, когда обсуждали принцип безразличия. Первому игроку безразлично,
какую стратегию играть~-- любую чистую или любую смешанную.

Похожими рассуждениями и в силу симметрии матрицы выигрышей игрока $2$
получаем аналогичное отображение лучших ответов игрока $2$.
На рис.\,30 это ломаная $q^*(p)$.


БиматрРис30


Рис.\,30\,показывает, что
равновесие по Нэшу в игре Орел или Решка\, возникает, если игрок 1
разыгрывает~сме\-шан\-ную стра\-те\-гию $(\frac{1}{2},\frac{1}{2})$ и игрок
2 разыгрывает такую же стратегию, что, по-видимому, было
естественно ожи\-дать в силу симметричности игры.

\bigskip

Важно заметить, что
этот пример иллюстрирует,
что неслучайно, ес\-ли один из игроков
выбирает свои стра\-те\-гии равновероятно (т.\,е. придерживается своей
равновесной стратегии),
то второму игроку при этом абсолютно
безразлично, как играть.
Это следует из свойства, доказанного ранее
(см. п.\,1.7) в общем случае:
$$
\begin{array}{lcr} U_1(s_1,\sigma
_2^*)&=& U_1(\sigma_1^*,\sigma _2^*),\\ [10pt]
U_1(\sigma _1^*,s_2)&=&
U_1(\sigma_1^*,\sigma _2^*) \end{array}
\eqno(11.1)
$$
для тех $s_i$, которые входят в равновесную ситуацию с ненулевыми
вероятностями.  Для тех же $s^{'}_i$, которые входят в равновесную
ситуацию с нулевой вероятностью, верны неравенства:
$$
\begin{array}{lcr}
U_1(s^{'}_i,\sigma _2^*)&\le& U_1(\sigma_1^*,\sigma _2^*),\\[10pt]
U_1(\sigma _1^*,s^{'}_i)&\le& U_1(\sigma_1^*,\sigma _2^*).
\end{array}
\eqno(11.2)
$$

Формулы (11.1), (11.2) дают действенный способ определения равновесных
ситуаций в произвольных биматричных играх $2\times 2$.

Ожидаемый выигрыш игрока $1$ от разыгрывания $\sigma_1=(p,1-p)$,
когда игрок $2$ разыгрывает $\sigma_2=(q,1-q)$:

$$
\begin{array}{l}
U_1(\sigma_1,\sigma_2)=p(a_{11}q+(1-q)a_{22}+
(1-p)qa_{21}+(1-p)(1-q)a_{22},\\[6pt]
U_1(\sigma_1,\sigma_2)-U_1(s_1,\sigma_2)=
(a_{12}-a_{22}+q(a_{11}-a_{12}-a_{21}+a_{22}))p.
\end{array}
$$

Введем обозначения: $C=a_{11}-a_{12}-a_{21}+a_{22},~~~~
\alpha =a_{22}-a_{12}$.

Лучший ответ  игрока $1$ на произвольную стратегию $\sigma_2$ игрока
$2$ можно получить из условий неотрицательности:
$$
U_1(\sigma_1,\sigma_2)-U_1(s_1,\sigma_2)\geq 0;
$$
$$
U_1(\sigma_1,\sigma_2)-U_1(s_2,\sigma_2)\geq 0.
$$

C учетом введенных обозначений они выглядят следующим образом:
$$
\begin{array}{l}
(p-1)(Cq-\alpha )\geq 0,\\[10pt]
p(Cq-\alpha )\geq 0.\\
\end{array}
\eqno(11.3)
$$

Аналогично можно поступить для нахождения лучшего ответа игрока $2$.

Ожидаемый выигрыш игрока $2$ от игры $\sigma_2=(q,1-q)$, когда игрок $1$
играет $\sigma_1=(p,1-p)$:
$$
U_2(\sigma_1,\sigma_2)=q(b_{11}p+(1-p)b_{21})+(1-q)(b_{12}p+(1-p)b_{22})=
$$
$$
=b_{22}+(b_{12}-b_{22})p+(b_{21}-b_{22}+p(b_{11}-b_{12}-b_{21}+b_{22}))q.~~~~
$$

Из условий (неотрицательности)
$$
U_2(\sigma_1,\sigma_2)-U_1(\sigma_1,s_1)\geq 0,
$$
$$
U_1(\sigma_1,\sigma_2)-U_1(\sigma_1,s_2)\geq 0,
$$
обозначив $D=b_{11}-b_{12}-b_{21}+b_{22},~~~\beta =b_{22}-b_{21}$, получаем
аналогичные неравенства для нахождения лучшего ответа игрока $2$ на
произвольную стратегию $\sigma =(p,1-p)$ игрока $1$:
$$
\begin{array}{l}
(q-1)(Dp-\beta )\geq 0,\\[10pt]
q(Dp-\beta )\geq 0.\\
\end{array}
\eqno(11.4)
$$

Тогда, для того чтобы пара $\sigma_1=(p,1-p),~~\sigma_2=(q,1-q)$ определяла
равновесную ситуацию, необходимо и достаточно одновременное выполнение
систем неравенств $(11.3),~(11.4)$, а также
$0\leq p\leq 1,~~0\leq q\leq 1$.

Рассмотрим лучшие ответы каждого игрока, которые, разумеется, зависят
от того, как устроены матрицы выигрышей игрока $1$ и игрока $2$.
Начнем с неравенств (11.3).

Возможны три случая:
$$
\begin{array}{cr}
1)&p=1,\,\, Cq\geq \alpha;\\[10pt]
2)&0<p<1,\,\, Cq= \alpha;\\[10pt]
3)&p=0, Cq\leq \alpha.
\end{array}
\eqno(11.5)
$$

В свою очередь, в зависимости от соотношений между $C$ и $ \alpha $ возможны
следующие случаи и соответствующие лучшие ответы игрока $1$ в каждом из них.

I. Если $C>0$, $\alpha >0$, то лучшие ответы изображены на
рис.\,31--33:


БиматрРис31-33


II. Если $C<0,~~ \alpha <0$, то лучшие ответы изображены на
рис.\,34--36:


БиматрРис34-36


III. При  $C>0,~~ \alpha <0$ лучший ответ игрока $1$ изображен на
рис.\,37, если, наоборот, $C<0,~~ \alpha >0$, то этому случаю
соответствует рис.\,38:

БиматрРис37-38

Рассмотрение случаев, когда либо $C$, либо $\alpha$, либо они вместе
равны нулю, мы оставляем читателю.

Аналогично можно построить лучшие ответы игрока 2.


Равновесным ситуациям графически соответствуют точки пересечений множеств
лучших ответов игроков. Совмещая графики лучших ответов игрока $1$
с любым графиком наилучших ответов игрока $2$, можно получить
всевозможные варианты множеств равновесных
ситуаций биматричной игры $2\times 2$.

Рассмотрим геометрический смысл условий (11.3) и (11.4) на примере описанной
выше игры Дилемма Заключенного.
Напомним, что ситуация, сложившаяся в этой игре, задается матрицей
$$
\begin{array}{c}
{~~~}\\
{\rm Молчать}\\
{\rm Сознаться}\\
\end{array}~~
\left(\begin{array}{cc}
{\rm молчать}&{\rm сознаться}\\
(-1,-1)&(-10,0)\\
(0,-10)&(-6-6)\\
\end{array} \right)
$$

Имеем $C=-1-(-10)-0+(-6)=3,~~~\alpha =-6-(-10)=4,$
$D=-1-0-(-10)+(-6)=3,~~~\beta =-6=(-10)=4.$


Тогда условия (11.3), (11.4) выглядят следующим образом:
$$
(p-1)(3q-4)\geq 0,~~~(q-1)(3p-4)\geq 0,
$$
$$
p(3q-4)\geq 0,~~~q(3p-4)\geq 0.
$$
Отсюда получаем, что при $p=1 ~~q\geq \frac{4}{3}$, при $0<p<1 ~~q =
\frac{4}{3}$, при $p=0 ~~~q\leq \frac{4}{3}$; при $q=1 ~~p\geq
\frac{4}{3}$, при $0<q<1 ~~p= \frac{4}{3}$, при $q=0 ~~q\leq
\frac{4}{3}$. Полученные лучшие ответы изображены на рис.\,39.


БиматрРис39


Из рисунка видно, что
существует единственная ситуация равновесия $p=0,~q=0$.  Это
ситуация, в которой каждый из игроков выбирает вторую чистую
стратегию~--- Сознаться.

Отметим несколько возможных качественных особенностей существующих
равновесий.

1) Существует единственное равновесие по Нэшу в чистых стратегиях~---
например, рис.\,40 дает равновесную точку $p=1,$\, $q=0.$ Дилемма
Заключенного\, относится к такому случаю (здесь $p=1,$\, $q=1$).


БиматрРис40


2) Существует единственное равновесие по Нэшу в смешанных стратегиях.
Игра Орел\, или Решка\, является примером такого случая (см. рис.\,30).

3) Существует три равновесия по Нэшу~--- два в чистых стратегиях и одно~--- в
смешанных.  Такого типа ситуация возникает в игре Семейный спор\,
(см. рис.\,41).


БиматрРис41



4) Существует два равновесия по Нэшу в чистых стратегиях.
Такая ситуация возникает, в частности, когда в матрице $A$
выигрышей игрока 1 $a_{12}=a_{22}$,
а в матрице $B $ выигрышей игрока 2 $ b_{12}=b_{22}$ (см. рис.\,42).


БиматрРис42


5) Существует континуум равновесий по Нэшу (в смешанных стратегиях).
Примеров такого рода равновесий можно предложить очень много
(см.  рис.\,43).


БиматрРис43



Следует отметить интересное свойство,
присущее некоторым типам равновесий.  А именно, в случаях 1), 2), 3),
когда равновесных ситуаций в игре  нечетное число, при достаточно
малых изменениях элементов матриц выигрышей зигзаги также слегка
"пошевелятся", но их общая форма и характер их взаимного расположения
не изменятся, а значит, не изменится и число равновесных ситуаций.

Этого нельзя сказать о случаях четного и бесконечного числа
равновесных ситуаций. В этих случаях малейшее изменение элементов
матриц выигрышей может приводить к совершенно иным качественным
ситуациям. Например, ситуация, изображенная на рис.\,42,  может
перейти в ситуацию либо с одним чистым равновесием, если $\alpha
=0,~\beta <0~(b_{22}<b_{21})$ (см. рис.\,44), либо в ситуацию с
тремя равновесиями, если $\alpha >0,~\beta >0~(\alpha _{22}<\alpha
_{21})$ (см. рис.\,45), либо в ситуацию с континуумом равновесий,
если $\alpha >0,$\, $\beta =0 $ (см. рис.\,46).


БиматрРис44-46



Рассмотрим наиболее интересный случай, когда
$C $ и $D $ не равны нулю.  Тогда равновесная ситуация определяется
двумя формулами $ p=\frac{\beta}{D},$ \, $q=\frac{\alpha }{C}$, из которых
следует, что в равновесной ситуации выбор игрока $ 1$ определяется
элементами матрицы выигрышей игрока $ 2$ и не зависит от элементов
собственной матрицы, а выбор игрока $ 2$ в равновесной ситуации
полностью определяется элементами матрицы игрока $1 $ и не зависит от
элементов собственной матрицы. Иными словами, равновесная стратегия
обоих игроков определяется не столько стремлением увеличить
собственный выигрыш, сколько держать под контролем выигрыш другого
игрока (минимизировать его). Таким образом, в биматричной игре мы
сталкиваемся не с антагонизмом интересов, а с антагонизмом поведения.


