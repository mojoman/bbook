






























\subsection{Индекс цен}


    Как измерять уровень цен? Как быстро цены растут или падают?
    Иными словами, какова инфляция?
    Если бы производители производили только
    один продукт, а потребители только его и потребляли, то ответы
    на эти вопросы были бы тривиальными, ибо
    они сводились бы к вопросам о том, какова цена этого одного
    продукта и о том, как увеличивается или уменьшается его цена.
    Если же мы рассматриваем экономику, в которой производятся и
    потребляется хотя бы два различных вида продукции, то ситуация
    становится несколько более сложной. Если, например, двумя
    производимыми и потребляемыми продуктами являются яблоки и
    апельсины, цены на которые выросли на $5\%$ и $20\%$
    соответственно, то сказать однозначно, на сколько  цены
    выросли в целом, непросто.

    Один из распространенных способов измерения общего уровня цен
    состоит в том, что задается некоторая условная потребительская
    корзина, а общий уровень цен, который принято называть \emph{индексом цен},
    определяется как стоимость этой
    корзины, а темп роста цен в целом --- как темп роста индекс цен.
    Например, если наша условная корзина состоит из
    2 кг. яблок и 1 кг. апельсинов, а их цены равны 40 руб./кг. и 50
    руб./кг. соответственно, то значение интересующего нас индекса цен, т.е.
    цена рассматриваемой корзины составляет $130 руб.$ После
    увеличения цен на $5\%$ и $20\%$  индекс будет равен $144
    руб.$ Тем самым рост цен в целом, измеренный с помощью нашего индекса,
    составит $\frac{140}{13}\%$.

    В теоретических моделях с $n$ различных видов благ, когда говорят об
    индексе цен, имеют в
    виду виду некоторую функцию $P:\R_{+}^{n}\rightarrow \R_{+}$,
    которая каждому вектору (набору) цен $\vc{p}=(p_{1},\ldots,p_{n})$
    ставит в соответствие величину $P(\vc{p})$, которая
    интерпретируется как общий уровень цен. По поводу этой функции
    обычно предполагают, что она непрерывна, положительно однородна
    первой степени:
    \[P(\lambda\vc{p})=\lambda P(\vc{p}), \ \vc{p}\in\R_{+}^{n}, \lambda\geqslant0,\]
    а также монотонно возрастает в следующем смысле:
    \[\vc{p}\gg\vc{q} \Rightarrow P(\vc{p})>P(\vc{q}).\]
    Требование, чтобы индекс цен удовлетворял этим свойствам,
    представляется вполне обоснованным. Действительно, если
    цены на все блага увеличатся в два раза, то, несомненно, это
    означает, что и общий уровень цен возрастет тоже в два раза. Что
    касается монотонности, то она означает, что если все цены стали
    выше, то и общий уровень цен тоже вырос.


    В макроэкономических моделях индекс цен иногда задают с помощью некоторой
    условной функции полезности $U(\vc{x})=U(x_{1},\ldots,x_{n})$,
    заданной на $\R_{+}^{n}$ и удовлетворяющая стандартным предположениям (???????).
    А именно, в качестве индекса цен берется функция $P:\R_{+}^{n}\rightarrow
    \R_{+}$, определяемая равенством
    \[P(\vc{p})=\inf\{\vc{p}\vc{x} \ | \ U(\vc{x})\geqslant1\},\]
    где $\gamma$ (??????) --- это некоторое заданное число. Таким образом определяемая величина
    $P(\vc{p})$ имеет естественную интерпретацию. Она показывает,
    сколько при том или ином векторе цен $\vc{p}$ <<стоит>> 1
    единица полезности.

    Если функция полезности $U(\vc{x})$ является леонтьевской, т.е. задается равенством
    \[U(\vc{x})=\min\{x_{1}/\hat{x}_{1},\ldots,x_{n}/\hat{x}_{n}\},\]
    где $\vc{\hat{x}}=(\hat{x}_{1},\ldots,\hat{x}_{n})$ --- это
    фиксированный неотрицательный вектор, задающий некоторую потребительскую корзину,
    а $\gamma=1$, то индекс цен --- это в точности стоимость
    потребительской корзины, задаваемой вектором $\vc{\hat{x}}$.
    Действительно, как легко проверить,
    \[P(\vc{p})=\vc{p}\vc{\hat{x}} \ (=p_{1}\hat{x}_{1}+\ldots+p_{n}\hat{x}_{n}),\]
    поскольку вне зависимости от $\vc{p}\geqq\vc{0}$ хотя бы одним из решений задачи
    \begin{equation}\label{index}
    \vc{p}\vc{x}\rightarrow\min, \ U(\vc{x})\geqslant1
    \end{equation}
        является вектор $\vc{\hat{x}}$ (здесь важно то, что $\vc{p}$ является неотрицательным
    вектором). Этот факт является специфическим свойством именно
    леонтьевской функции полезности. Для других функций полезности
    решение задачи (\ref{index}) будет зависеть от вектора $\vc{p}$.
    Обозначим это решение, если оно существует и единственно, через
    $\vc{x}(\vc{p})=(x_{1}(\vc{p}),\ldots,x_{n}(\vc{p}))$. Очевидно, что
    \[P(\vc{p})=\vc{p}\vc{x}(\vc{p}).\]




    \begin{exer}
    Пусть
    \[U(\vc{x})=n\left(\frac{1}{n}\sum_{i=1}^{n}x_{i}^{\rho}\right)^{1/\rho}, \ 0<\rho<1.\]
    Положим
    \[\sigma=\frac{1}{1-\rho}.\]
    Покажите, что при $\vc{p}\gg\vc{0}$ решение задачи (\ref{index})
    единственно и этим решением является вектор
    \[\vc{x}(\vc{p})=(x_{1}(\vc{p}),\ldots,x_{n}(\vc{p})),\]
    задаваемый равенствами
    \[x_{i}(\vc{p})=\frac{1}{n}\left(\frac{p_{i}}{\vc{p}\vc{x}(\vc{p})}\right)^{-\sigma},
    \ i=1,\ldots,n.\]
    Покажите, что вектор $\vc{x}(\vc{p})$, задаваемый этими
    равенствами, является решением задачи (\ref{index}) при любом
    неотрицательном  ненулевым векторе $\vc{p}$ (хотя и не
    единственным).
    Покажите, что
    \[P(\vc{p})=\left(\frac{1}{n}\sum_{i=1}^{n}p_{i}^{1-\sigma}\right)^{\frac{1}{1-\sigma}}.\]
    \end{exer}


     \begin{exer}
    Постройте индекс цен $P(\vc{p})$ при
    \[U(\vc{x})=n\left(\frac{1}{n}\sum_{i=1}^{n}x_{i}^{\rho}\right)^{1/\rho}, \ \rho<0,\]
    и при
    \[U(\vc{x})=a\prod_{i=1}^{n}x_{i}^{\alpha_{i}}, \ a>0, \ \alpha_{i}>0, \ i=1,\ldots,n.\]

     \end{exer}

    \








\





\chapter{Приложение}


    Здесь приводятся часто используемые сведения из математического
    анализа и линейной алгебры, а также основные используемые
    общематематические обозначения. Мы предполагаем самые простейшие
    математические понятия и основные сведения из
    математического анализа и линейной алгебры читателю
    известны.

    $x\in\st{X}$ --- элемент $x$ принадлежит множеству $\st{X}$.


    $x\notin\st{X}$ --- элемент $x$ не принадлежит множеству $\st{X}$.

    $\st{X}\cup\st{Y}, \ \bigcup_{j\in J}\st{X}_{j}$ --- объединение множеств.

     $\st{X}\cap\st{Y}, \ \bigcap_{j\in J}\st{X}_{j}$ --- пересечение множеств.

     $\emptyset$ --- пустое множество.

     $\{x\mid P\}$, \ $\{x\in\st{X}\mid P\}$ --- множество элементов и
     подмножество элементов множества $\st{X}$, обладающих свойством
     $P$.


     $\{x,y,z\}$ --- множество, состоящее из элементов $x$, $y$ и $z$.

     $\Box$ --- символ конца доказательства или конца формулировки,
     если утверждение следует из предшествующих рассуждений или
     приводится без доказательств.

     $\R$ --- числовая прямая, множество всех вещественных чисел.

     $\R_{+}$ --- множество неотрицательных чисел.
     \begin{equation}
\label{interval}
    (a,b)=\{x\in\R \ | \ a<x<b\};
\end{equation}
\begin{equation}
\label{interval-1}
    (a,b]=\{x\in\R \ | \ a<x\leqslant b\};
\end{equation}
\begin{equation}
\label{interval-2}
    [a,b)=\{x\in\R \ | \ a\leqslant x<b\};
\end{equation}
\begin{equation}
\label{otrezok}
    [a,b]=\{x\in\R \ | \ a\leqslant x\leqslant b\}.
\end{equation}
    При этом в (\ref{interval}) и (\ref{interval-1}) допускается, что $a=-\infty$,
    а в (\ref{interval}) и (\ref{interval-2}) --- что $b=+\infty$. Например,
    \[(-\infty,+\infty)=\R.\]
    Множество вида (\ref{interval})
    называется интервалом, а множество (\ref{otrezok}) --- отрезком.
    Тот факт, что $\st{X}$ совпадает с
    одним из множеств вида (\ref{interval})---(\ref{otrezok}) мы
    будем записывать так:
    \[\st{X}=\langle a,b\rangle,\]
    при этом мы будем говорить, что множество $\st{X}$ является
    \emph{\textbf{промежутком}}. Заметим, что внутренностью промежутка $\langle
    a,b\rangle$ является интервал $(a,b)$.


    Монотонная функция


    Обратная функция




     $\R^{n}$ --- $n$-мерное координатное пространство. Элементы
     этого пространства называются точками или ($n$-мерными) векторами. Вектор из $\R^{n}$
     представляет собой упорядоченный набор из $n$ вещественных чисел и
     записывается либо как в виде (вектор-) строки:
     $\vc{x}=(x_{1},\ldots,x_{n})$, либо в виде (вектор-) столбца: $\vc{x}=\left(
                                                                            \begin{array}{c}
                                                                              x_{1} \\
                                                                              \vdots\\
                                                                              x_{n} \\
                                                                            \end{array}
                                                                          \right)$.
     В некоторых случаях для на существенно, а в некоторых
     несущественно, как записывается вектор: в виде строки или в
     виде столбца. Иногда используется обозначение
     $(x_{1},\ldots,x_{n})^{T}=\left(
                                                                            \begin{array}{c}
                                                                              x_{1} \\
                                                                              \vdots\\
                                                                              x_{n} \\
                                                                            \end{array}
                                                                          \right)$.

     $\vc{0}=(0,\ldots,0)$ --- нулевой вектор из $\R^{n}$.

     $\vc{x}+\vc{y}=(x_{1}+y_{1},\ldots,x_{n}+y_{n})$ сумма векторов
     $\vc{x}=(x_{1},\ldots,x_{n})$ и $\vc{y}=(y_{1},\ldots,y_{n})$
     из $\R^{n}$. Если мы различаем вектор-столбцы и вектор-строки,
     то складывать можно только вектора одного типа ---
     вектор-столбец с вектор-столбцом и вектор-строку с
     вектор-строкой.

     $\lambda\vc{x}=(\lambda x_{1},\ldots,\lambda x_{n})$ ---
     произведение вектора $\vc{x}=(x_{1},\ldots,x_{n})$ на число $\lambda$.

     $\vc{x}\vc{y}=x_{1}y_{1},+\ldots+x_{n}y_{n}$ --- скалярное
     произведение векторов $\vc{x}=(x_{1},\ldots,x_{n})$ и
     $\vc{y}=(y_{1},\ldots,y_{n})$ из $\R^{n}$.

     Для векторов
      $\vc{x}=(x_{1},\ldots,x_{n})$ и $\vc{y}=(y_{1},\ldots,y_{n})$
      из $\R^{n}$ используются следующие обозначения:
      $\vc{y} \geqq \vc{x} \Leftrightarrow y_i \geqslant x_i$, $i=1, \ldots, n$;
    $\vc{y} \geq \vc{x} \Leftrightarrow y_i \geqslant x_i$, $i=1, \ldots, n$, и хотя бы одно из
неравенств выполняется как строгое;
    $\vc{y} \gg \vc{x} \Leftrightarrow y_i > x_i$, $i=1,\ldots,n$.

    Если $\vc{x}=(x_{1},\ldots,x_{n})\in\R^{n}$ и
    $\vc{y}=(y_{1},\ldots,y_{n})\in\R^{m}$, то
    $(\vc{x},\vc{y})=(x_{1},\ldots,x_{n},y_{1},\ldots,y_{n})\in\R^{n+m}$.


    $\R^{n}_{+}=\{\vc{x}\in\R^{n}\mid \vc{x}\geqq \vc{0}\}$ ---
    неотрицательный ортант пространства $\R^{n}$, его элементы
    называются неотрицательными векторами или точками.

    $\|\vc{x}\|=\sqrt{x_{1}^{2}+\ldots+x_{n}^{2}}$ --- норма вектора
    $\vc{x}=(x_{1},\ldots,x_{n})$.

    Если дана $m\times n$ матрица $\vc{A}$, т.е. матрица размера $m\times n$
    (с $m$ строками и $n$ столбцами), то, как правило, $\vc{a_{j}}$
    --- это ее $j$-й столбец, $\vc{a^{i}}$ --- ее $i$-я строка, а
    $a_{ij}$ --- элемент на пересечении $i$-я строки и $j$-го
    столбца.


    Если даны $m\times n$ матрица $\vc{A}$, $n$-мерный
    вектор-столбец $\vc{x}=\left(
                                                                            \begin{array}{c}
                                                                              x_{1} \\
                                                                              \vdots\\
                                                                              x_{n} \\
                                                                            \end{array}
                                                                          \right)$
    и $m$-мерная вектор-строка $\vc{y}=(y_{1},\ldots,y_{n})$, то
    $\vc{A}\vc{x}$ --- это $m$-мерный вектор-столбец, задаваемый
    равенством
$
    \vc{A}\vc{x}=\left(
                   \begin{array}{c}
                     \vc{a^{1}}\vc{x}\\
                     \vdots \\
                     \vc{a^{m}}\vc{x} \\
                   \end{array}
                 \right),
$
    а $\vc{y}\vc{A}$ --- $n$-мерная вектор-строка, задаваемая
    равенством
$
    \vc{y}\vc{A}=(\vc{y}\vc{a_{1}},\ldots,\vc{y}\vc{a_{m}}).
$

    Пусть $\vc{A}$ и $\vc{B}$ --- это $m\times n$ матрицы с соответственно элементами
    $a_{ij} \ \text{и} \ b_{ij}, \ i=1,\ldots,m, \ j=1,\ldots,n,$ , а $\lambda$ ---
    число. Произведением матрицы $\vc{A}$ на число $\lambda$
    является это $m\times n$ матрица
         $\vc{C}=\lambda\vc{A}$ с элементами
    $c_{ij}=\lambda a_{ij}, \ i=1,\ldots,m, \ j=1,\ldots,n$. Суммой матриц
    $\vc{A}$ и $\vc{B}$ является
    --- $m\times n$ матрица $\vc{C}=\vc{A+B}$ с элементами $c_{ij}=a_{ij}+b_{ij}, \ i=1,\ldots,m, \
    j=1,\ldots,n$.


    Если $\vc{A}$ --- это $m\times n$ матрица с элементами
    $a_{ij}, \ i=1,\ldots,m, \ j=1,\ldots,n,$ а $\vc{B}$ ---
    $n\times s$ матрица с элементами
    $b_{jk}, \ i=1,\ldots,n, \ k=1,\ldots,s$, то произведением
    матрицы $\vc{A}$ на матрицу $\vc{B}$ является $m\times s$ матрица
    $\vc{C}=\vc{A}\vc{B}$ с элементами
\[
    c_{ik}=\vc{a^{i}}\vc{b_{k}}=\sum_{j=1}^{n}a_{ij}b_{jk}, \ i=1,\ldots,m,
    \  k=1,\ldots,s.
\]
    Напомним, что, вообще говоря, $\vc{B}\vc{A}\neq\vc{A}\vc{B}$ даже в случае,
    когда матрицы $\vc{A}$ и $\vc{B}$ являются квадратными.
    Равенство $\vc{B}\vc{A}=\vc{A}\vc{B}$ выполняется только в
    некоторых специальных случаях.


    $\vc{E}$ --- \emph{единичная матрица}: на ее диагонали стоят единицы, остальные элементы ---
    нули (эта матрица является квадратной, т.е. имеет столько же столбцов, сколько и
    строк).


    $\vc{A}^{-1}$ --- матрица, \emph{обратная} к матрице
    $\vc{A}$, т.е. матрица, для которой выполняются равенства
    $\vc{A}^{-1}\vc{A}=\vc{A}\vc{A}^{-1}=\vc{E}$. Обратная матрица
    может существовать только у квадратной матрицы. Квадратная
    матрица $\vc{A}$ имеет обратную тогда и только тогда, кода ее
    определитель $\det\vc{A}\neq 0$.

    $\mathcal{U}_{\epsilon}(\vc{a})=\{\vc{x}\in\R^{n} \mid
    \|\vc{x}-\vc{a}\|<\epsilon\}$ --- открытый шар радиуса $\epsilon>0$ с
    центром в точке $\vc{a}$ или, иначе, $\epsilon$ --- \emph{окрестность}
    точки $\vc{a}$.


    $(\vc{x}(k))_{k=1}^{\infty}$ --- последовательность точек $\vc{x}(1),
    \vc{x}(2),\ldots.$

    Говорят, что последовательность $\{\vc{x}(k)\}_{k=1}^{\infty}$
    точек из пространства $\R^{n}$ \emph{сходится} к $\vc{a}\in\R^{n}$,
    или что $\vc{a}$ является \emph{пределом} последовательности
    $\{\vc{x}(k)\}_{k=1}^{\infty}$, если
    $\parallel \vc{x}(k)-\vc{a}\parallel \rightarrow 0$ при
    $k\rightarrow\infty$, при этом пишем:
    $\vc{x}(k)\rightarrow\vc{a}$ при
    $k\rightarrow\infty$.


    Функция $f(\vc{x})$, заданная на некотором множестве
    $\st{X}\subset\R^{n}$, называется \emph{непрерывной} в точке
    $\vc{\hat{x}}\in\st{X}$, если $f(\vc{x}(k))\rightarrow f(\vc{\hat{x}})$
    при $k\rightarrow\infty$ для любой последовательности
    $\{\vc{x}(k)\}_{k=1}^{\infty}$, состоящей из элементов $\st{X}$,
    сходящейся к $\vc{\hat{x}}$. Эта функция называется \emph{непрерывной
    на множестве} $\st{Y}\subset\st{X}$, если она непрерывна в любой
    точке из этого множества.


    Точка $\vc{x}\in\st{X}$ называется \emph{внутренней} точкой множества
    $\st{X}$, если $\mathcal{U}_{\epsilon}(\vc{x})\subset\st{X}$ при
    некотором $\epsilon>0$. Множество всех внутренних точек
    $\st{X}$ называется его \emph{внутренностью} множества $\st{X}$ и обозначается
    как $\text{int}\st{X}$. Множество называется \emph{открытым}, если оно
    совпадает со своей внутренностью.


    Точка $\vc{x}$ называется \emph{граничной} точкой множества
    $\st{X}$, если при любом $\epsilon>0$ ее $\epsilon$-окрестность
    $\mathcal{U}_{\epsilon}(\vc{x})\subset\st{X}$ содержит как
    точки, принадлежащие $\st{X}$, так и точки, этому множеству не
    принадлежащие. Множество всех граничных точек $\st{X}$
    называется \emph{границей} множества $\st{X}$. Множество называется
    \emph{замкнутым}, если оно содержит в качестве подмножества свою
    границу.

    Множество $\st{X}\subset\R^{n}$ называется \emph{ограниченным}, если
    существует число $R>0$, такое что $\|\vc{x}\|<R$ при всех
    $\vc{x}\in\st{X}$.



    Рассмотрим функцию $f(\vc{x})$, заданную на некотором множестве
    $\st{X}\subset\R^{n}$, и некоторую точку
    $\vc{\hat{x}}\in\text{int}\st{X}$. Обозначим
    $\vc{e^{1}}=(1,0,0,\ldots,0)$,  $\vc{e^{2}}=(0,1,0,\ldots,0)$,
    ..., $\vc{e^{n}}=(0,0,0,\ldots,1)$.
    Если для некоторого
    $i=1,\ldots,n$ существует предел
\[
    \frac{\partial f}{\partial x_{i}}(\vc{\hat{x}})
    =\lim _{\alpha\rightarrow 0}
    \frac{f(\vc{\hat{x}}+\alpha\vc{e^{i}})-f(\vc{\hat{x}})}{\alpha},
\]
    то он называется $i$-й \emph{частной производной} функции $f(\vc{x})$ в
    точке $\vc{\hat{x}}$.
    Вектор
    \[\nabla f(\vc{\hat{x}})=\left(\frac{\partial f}{\partial x_{1}}(\vc{\hat{x}}),\ldots,
    \frac{\partial f}{\partial x_{n}}(\vc{\hat{x}})\right)\] (если он существует) называется (\emph{вектор}-)
    градиентом (вектором частных производных) функции $f(\vc{x})$ в
    точке $\vc{\hat{x}}$.
    Функция $f(\vc{x})$ называется дифференцируемой в точке
    $\vc{\hat{x}}$, если для вектор-градиент $\nabla
    f(\vc{\hat{x}}))$ существует и при всех достаточно малых по
    норме векторов
    $\vc{h}\in\R^{n}$ справедливо равенство
\begin{equation}
    \label{opred-diff}
    f(\vc{\hat{x}}+\vc{h})=f(\vc{\hat{x}})
    +\nabla f(\vc{\hat{x}})\vc{h}
    + o(\|\vc{h})\|).
\end{equation}
    Здесь, как обычно, $o(\alpha)$ --- некоторая числовая функция
    числового аргумента, удовлетворяющая условию
    $\lim_{\alpha\rightarrow 0}\frac{o(\alpha)}{\alpha}=0$. Иногда
    равенство (\ref{opred-diff}) записывают следующим эквивалентным образом:
\[
    f(\vc{\hat{x}}+\vc{h})\approx f(\vc{\hat{x}})
    +\nabla f(\vc{\hat{x}})\vc{h}.
\]
    Функция $f(\vc{x})$ называется непрерывно дифференцируемой в точке
    $\vc{\hat{x}}$, если для всех $i=1,\ldots,n$ $i$-я частная
    производная определена в некоторой окрестности точки $\vc{\hat{x}}$ и
    непрерывна в ней. Известно, что функция, непрерывно
    дифференцируемая в некоторой точке, является в ней
    дифференцируемой.

    Говорят, что функция дифференцируема (непрерывно
    дифференцируема) на множестве $\st{X}\subset\R^{n}$, если она
    дифференцируема (непрерывно дифференцируема) в каждой точке
    этого множества. При этом, фактически функция $f(\vc{x})$ должна
    быть определена на некотором открытом множестве, содержащем
    $\st{X}$.



\

\section{Основные определения теории экстремальных задач}


    Экстремальные задачи, или задачи оптимизации, --- это задачи на максимум или
    минимум.
    Для того чтобы задать экстремальную задачу, нужно указать целевую
    функцию $f(\vc{x})$, определенную на некотором подмножестве
    какого-то пространства $\R^{n}$,
    уточнив, хотим ли мы максимизировать или
    минимизировать эту функцию, а также множество $\st{D}\subset\R^{n}$,
    на которой ищется максимум
    или минимум. Обычно предполагается, что множество $\st{D}$ содержится
    в том подмножестве пространства $\R^{n}$, на котором задана целевая функция.



    Само множество $\st{D}$ называется допустимым множеством, а его
    элементы --- допустимыми векторами или допустимыми планами. В случае, когда
    $\st{D}=\R^{n}$, рассматриваемую
    задачу называют задачей безусловной оптимизации. В противном
    случае речь идет о задаче условной оптимизации.

    Задачу на максимум функции $f(\vc{x})$ на множестве $\st{D}$
    мы будем записывать следующим образом:
\begin{equation}
\label{zad-max1}
    f(\vc{x})\rightarrow\max,\ \vc{x}\in\st{D}.
\end{equation}
           В задачах на максимум мы допускаем возможность того, что
    функция $f(\vc{x})$ принимает значение $-\infty$. Такое соглашение
    позволяет считать, что целевая функция задана на всем
    пространстве $\R^{n}$. Для этого достаточно договориться, что в
    тех точках, где она не определена, мы придаем ей значение $-\infty$.

    Решением, (или оптимальным решением??????) задачи (\ref{zad-max1}) называется
    такой допустимый вектор $\vc{x^{*}}$, что для любого другого допустимого
    вектора $\vc{x}$ выполняется неравенство
    \[f(\vc{x^{*}})\geqslant f(\vc{x}).\]







    Задачу на минимум функции $f(\vc{x})$ на множестве $\st{D}$
    записывают виде
\begin{equation}
\label{zad-min1}
    f(\vc{x})\rightarrow\min,\ \vc{x}\in\st{D}.
\end{equation}
        В задачах на минимум допускается возможность того, что
    целевая функция  принимает значение $+\infty$. Это позволяет
    нам считать, что $f(\vc{x})$ задана на всем $\R^{n}$.

    Решением, (или оптимальным решением????) задачи (\ref{zad-min1}) называется
    такой допустимый вектор $\vc{x^{*}}$, что для любого другого допустимого
    вектора $\vc{x}$ выполняется неравенство
    \[f(\vc{x^{*}})\leqslant f(\vc{x}).\]




    Любая задача
    на минимум превращается в задачу на максиму (и наоборот), если мы умножим
    целевую функцию на $-1$ . Например, задача
    \[-f(\vc{x})\rightarrow\max,\ \vc{x}\in\st{D}\]
    --- это, по-существу, задача (\ref{zad-min1}), только записанная
    в несколько другой форме. Во всяком случае, решения этих задач
    совпадают, и если мы знаем свойства одной из них, мы знаем и свойства
    другой.

    Значение $f(\vc{x^{*}})$ целевой функции в точке $\vc{x^{*}}$, представляющей собой
    решение задачи на максимум или минимум, называется \emph{значением}, или оптимальным
    значением этой \emph{задачи}. Значение задачи (\ref{zad-max1}) удобно
    записывать как
    \[\max\{f(\vc{x})\ |\ \vc{x}\in\st{D}\},\]
    а значение задачи (\ref{zad-min1}) как
    \[\min\{f(\vc{x})\ |\ \vc{x}\in\st{D}\}.\]

    Обычно в конкретных задачах на максимум и минимум множество
    $\st{D}$ очень удобно хотя бы частично задавать в виде
    набора некоторых равенств и неравенств,
    что и отражается в форме записи этих задач. Например, если
    \[\st{D}=\{\vc{x}\in \st{X}\ |\ g_{1}(\vc{x})\leqslant b_{1},
    \ldots,g_{n}(\vc{x})\leqslant b_{n}\}, \]
    где $\st{X}$ --- это некоторое подмножество пространства
    $\R^{n}$,  то задача (\ref{zad-max1}) будет записана как
\begin{equation} \label{zad-max-ogr}
    \left\{
    \begin{array}{ccc}
      f(\vc{x}) & \rightarrow & \max, \\
      g_{1}(\vc{x})& \leqslant & b_{1}, \\
      \ldots & \ldots & \ldots \\
      g_{n}(\vc{x})& \leqslant & b_{n},\\
      \vc{x} & \in & \st{X}\\
    \end{array}
    \right.
\end{equation}
    а ее значение как
    \[\max\{f(\vc{x})\ |\ g_{1}(\vc{x})\leqslant b_{1},
    \ldots,g_{n}(\vc{x})\leqslant b_{n}, \vc{x}\in\st{X}\}.\]

    Заметим, что довольно часто в экономико-математической
    литературе задачи на максимум записывают таким образом, что
    ограничения, задающие допустимое
    множество, за исключением, быть может, ограничений на неотрицательность переменных,
     записаны в виде неравенств типа <<меньше или равно>>. В
     ограничении такого типа в левой части неравенства стоит выражение, зависящее от
    вектора переменных
    $\vc{x}$, а в правой --- некоторая не зависящая от $\vc{x}$
    константа. Такая запись допустимого множества связана с тем,
    что содержательный смысл многих
    задач на максимум состоит в том, что требуется максимизировать
    некоторую величину, показывающую какой-нибудь доход или
    полезность в условиях, когда ресурсы ограничены. Безусловно,
    ограничения типа <<больше или равно>>, например, ограничения на
    неотрицательность переменных тоже естественно возникают в задачах на максимум.
    При этом, очевидно, любое неравенство типа <<меньше или равно>>
    превращается в неравенство типа <<больше или равно>> простым
    умножением его правой и левой части на $-1$. Действительно,
    задачу (\ref{zad-max-ogr}) можно записать и в таком виде:
\begin{equation*}
    \left\{
    \begin{array}{ccc}
      f(\vc{x}) & \rightarrow & \max, \\
      -g_{1}(\vc{x})& \geqslant & -b_{1}, \\
      \ldots & \ldots & \ldots \\
      -g_{n}(\vc{x})& \geqslant & -b_{n},\\
      \vc{x} & \in & \st{X}\\
    \end{array}
    \right.
\end{equation*}

Что касается задач на минимум, то ограничения в таких задачах часто
записывают в виде <<больше или равно>>, что тоже зачастую связано с
содержательным смыслом задачи.

    В оптимизационных задачах часто встречаются и ограничения в виде
    равенств. Но тут надо помнить, что каждое равенство можно
    записать как пару неравенств. Отсюда следует, что ограничение
    вида
    \[g(\vc{x})=b\]
    можно записать в виде двух неравенств:
    \[\left\{\begin{array}{ccc}
        g(\vc{x}) & \geqslant & b \\
        g(\vc{x}) & \leqslant & b
      \end{array}
    \right.\]


Существует ли решение задач (\ref{zad-max1}) и (\ref{zad-min1})?
    При некоторых
    предположениях ответ на этот вопрос является положительным.
    Например, согласно известной теореме Вейерштрасса, если непрерывная
    функция, задана на ограниченном и замкнутом множестве, то
    существует точка, в которой она
    достигает своего максимума на рассматриваемом множестве,
    а также точка, в которой она достигает своего минимума.
     Значит, если множество $\st{D}$ ограничено и замкнуто,
    а целевая функция непрерывна на этом множестве, то решение
    задач (\ref{zad-max1}) и (\ref{zad-min1}) существует.
    Здесь нужно только сделать одно уточняющее замечание. Говоря о
    непрерывности функции $f(\vc{x})$ в точках, где $f(\vc{x})$
    принимает значение $-\infty$ или $+\infty$, мы имеем в виду, что в этих точках
    она тоже непрерывна. Это значит, что если, например,
    $\lim_{k\rightarrow\infty}\vc{x}_{k}=\vc{x}$ и
    $\lim_{k\rightarrow\infty}f(\vc{x}_{k})=+\infty$ $(-\infty)$,
    то $f(\vc{x})=\infty$ $(-\infty)$.

    В то же время, если множество $\st{D}$ является неограниченным,
    то возможна ситуация, когда решения задачи (\ref{zad-max1}) или
    задачи (\ref{zad-min1}) не существует. Но даже в случае, когда
    решения задачи не существует,
    можно говорить о ее значении. А именно, под значением задачи
    (\ref{zad-max1}) мы будем понимать величину
\[\sup\{f(\vc{x})\ |\ \vc{x}\in\st{D}\},\]
    которая может принимать значение $+\infty$
    а под значением задачи (\ref{zad-min1}) --- величину
\[\inf\{f(\vc{x})\ |\ \vc{x}\in\st{D}\},\]
    которая может принимать значение $-\infty$

Очевидно, что в случае, когда решение задачи  на максимум или
    минимум существует, то такие определения значения задачи совпадают
    с данными выше и в этом смысле являются их обобщением.

    Для читателя, который не достаточно хорошо знаком с понятиями
    супремума ($\sup$) и инфимума ($\inf$), напомним, что $\sup\{f(\vc{x})\ |\
    \vc{x}\in\st{D}\}$ --- это, по определению, наименьшее из таких
    чисел $\gamma$, для которых при любом $\vc{x}\in\st{D}$ выполняется неравенство
    $\gamma\geqslant f(\vc{x})$:
\[\sup\{f(\vc{x})\ |\ \vc{x}\in\st{D}\}=\min\{\gamma |
    \gamma\geqslant f(\vc{x})\ \forall \vc{x}\in\st{D}\},\]
    а $\inf\{f(\vc{x})\ |\ \vc{x}\in\st{D}\}$ --- это наименьшее из таких
    чисел $\gamma$, для которых при любом $\vc{x}\in\st{D}$ выполняется неравенство
    $\gamma\leqslant f(\vc{x})$:
\[\inf\{f(\vc{x})\ |\ \vc{x}\in\st{D}\}=\max\{\gamma |
    \gamma\leqslant f(\vc{x})\ \forall \vc{x}\in\st{D}\}.\]

    Напомним также, что супремум и инфимум любой функции на любом
    множестве существуют всегда. В случае, когда решения у задачи (\ref{zad-max1}) не
    существует, ее значение  --- это такое
    число $v$, что для любого сколь угодно малого $\delta>0$
    найдется $\vc{x}\in\st{D}$, при котором выполняется неравенство
    $f(\vc{x})>v-\delta,$
    но ни при каком $\vc{x}\in\st{D}$ не может выполняться
    неравенство $f(\vc{x})>v$. Соответственно, значением задачи
    (\ref{zad-min1}) (в случае отсутствия у нее решения)
    является такое число $v$, что для любого $\delta>0$
    существует $\vc{x}\in\st{D}$, при котором
    $f(\vc{x})<v+\delta,$
    но ни при каком $\vc{x}\in\st{D}$ не может выполняться
    неравенство $f(\vc{x})<v$.




    Если как целевая функция оптимизационной задачи, так и
    ограничения, задающие множество $\st{D}$, являются линейными, то
    такую задачу называют задачей линейного программирования. Термин
    <<программирование>> взят из зарубежной литературы и означает не
    что иное как <<планирование>> (отсюда и возникла традиция называть
    допустимые и оптимальные векторы допустимыми и оптимальными
    планами). Если целевая функция или
    ограничения, задающие  $\st{D}$, могут быть нелинейными, то
    такая задача называется задачей нелинейного программирования.


    Нам сразу же нужно подчеркнуть, что при рассмотрении той или
    иной экстремальной задачи нас не будет интересовать
    отыскание ее решения в буквальном смысле. Для этого существуют
    соответствующие численные методы. Например, для решения задач
    линейного программирования обычно применяют симплекс-метод или
    какие-нибудь его модификации. Нас будут интересовать
    качественные свойства решений экстремальных задач. Если такие
    задачи имеют  экономический смысл, то можно надеяться на то, что
    изучение качественных свойств решений окажется небессмысленным с
    содержательной экономической точки зрения. Это касается, в
    первую очередь, необходимых и достаточных условий оптимальности.


\


  qqqqqqqqqqqqqqqqqqqqq



\chapter{Фигня}


    \subsection{Излишек потребителя и излишек производителя}

УНИФИЦИРОВАТЬ ОБОЗНАЧЕНИЯ ВЫПУСКА: X, Y, Q ??????


    Экономисты полагают, что рыночные механизмы в значительной мере
    обеспечивают эффективное функционирование экономики. В
    простейшей модели спроса и предложения и некоторых более общих
    моделей экономического равновесия этот тезис обычно
    выражается в утверждении, что состояние равновесия является в
    некотором смысле оптимальным. Соотношению между равновесием и
    оптимальностью мы в дальнейшем уделим значительное внимание. А
    сейчас  введем в рассмотрение важные инструменты анализа этой
    проблемы: излишек потребителя и излишек производителя.



    Начнем с излишка потребителя. Предположим, что функция спроса на некоторый товар,
    например, велосипед, задается следующей таблицей:
\[\begin{tabular}{|c|c|c|c|c|c|c|c|}
  \hline
  % after \\: \hline or \cline{col1-col2} \cline{col3-col4} ...
  Цена (руб./шт.) & 1000 & 900 & 800 & 700 & 600 & 500 & 400 \\
  Спрос (шт.) & 1 & 2 & 3 & 4 & 5 & 6 & 7 \\
  \hline
\end{tabular}\]
    Предположим далее, что на рынке сложилась цена $P^{*}=450$ руб./шт.
    Следовательно будет куплено $6$ велосипедов.
    Будем считать, что их купят 6 различных потребителей. Первый из
    них был готов купить велосипед за $1000$ руб., т.е.
    он был готов пожертвовать ради покупки велосипеда покупкой других
    товаров на $1000$ руб. Фактически же ему приходится платить только
    $450$ руб. и жертвовать покупкой товаров именно на эту сумму.
    Тем самым, его чистая выгода, называемая излишком потребителя
    (consumer's surplas, англ.), при покупке велосипеда составляет
    $550(=1000-450)$ руб.


    Рассуждая аналогичным образом, заключаем, что излишек, получаемый
    вторым потребителем, тем, который готов был купить велосипед за
    $900$ руб. равен $450$ руб. Излишек третьего потребителя --- $350$ руб.,
    четвертого --- $250$ руб., пятого --- $150$ руб., шестого --- $50$ руб.
    Суммарный излишек потребителей, обозначаемый как $CS$, получаемый \
    всеми шестью потребителями равен $550+450+350+250+150+50=1800$ руб.



    Чтобы иметь возможность ввести излишек потребителя несколько иначе, формально
    опишем функцию спроса $D(P)$, которая задается приведенной выше таблицей.
    Эта функция задается следующим образом:
    \[D(P)=0 \ \text{при} \ P>1000; \ D(P)=1 \ \text{при} \ 900<P\leq1000;\]
    \[D(P)=2 \ \text{при} \ 800<P\leq900; \ D(P)=3 \ \text{при} \ 700<P\leq800;\]
    \[D(P)=4 \ \text{при} \ 600<P\leq700; \ D(P)=5 \ \text{при} \ 500<P\leq600;\]
    \[D(P)=6 \ \text{при} \ 400<P\leq500; \ D(P)=7 \ \text{при} \ P\leq400.\]

    Быть может, кому-то это покажется удивительным, но излишек
    потребителя $CS$ может быть представлен следующим образом:
    \[CS=\int_{450}^{1000}D(P)dP\]
    \[=\int_{450}^{500}D(P)dP
    + \int_{500}^{600}D(P)dP+\int_{600}^{700}D(P)dP\]
    \[+\int_{700}^{800}D(P)dP +\int_{800}^{900}D(P)dP
    +\int_{900}^{1000}D(P)dP\]
    \[=300+500+400+300+200+100=1800.\]

    Для того, чтобы понять, в чем тут дело,
    предположим сначала, что в некоторый момент времени цена
    велосипеда была равна $1000$ руб./шт. В этом случае будет
    куплен только одним покупателем, причем его чистая выгода
    будет равна нулю. Если после этого цена понизится до $900$ руб./шт.,
    то будет куплено $2$ велосипеда. При этом чистая выгода второго
    продавца будет равна нулю, а вот чистая выгода первого продавца
    станет равна  $\int_{900}^{1000}D(P)dP=100$ руб.

    Предположим теперь, что цена еще раз снизилась, теперь до уровня
     $800$ руб./шт. Понижение цены с $900$ руб./шт. до $800$ руб./шт.
     приведет к дополнительному увеличению чистой выгоды первого
     покупателя на $100$ руб. (со $100$ руб. до $200$ руб.), увеличение
     же чистой выгоды второго покупателя тоже составит $100$ руб.
     При цене $800$ руб./шт. появится еще и третий покупатель, но
     его чистая выгода будет при этом равняться нулю. Суммарное
     же увеличение чистой выгоды всех потребителей при снижении цены
      с $900$ руб./шт. до $800$ руб./шт. будет равно
      $\int_{800}^{900}D(P)dP=2\times100=200$ руб.
      Аналогично рассуждая, делаем заключение, что увеличение чистой
      выгоды всех потребителей при снижении цены с $800$ руб./шт. до
      $700$ руб./шт. будет равно  $\int_{700}^{800}D(P)dP=3\times100=300$ руб.
      И так далее.

      Теперь мы можем определить излишек потребителя в общей ситуации.
      Пусть нам задана монотонно невозрастающая функция спроса $D(P)$ на
      некоторый продукт. Если эта функция определена и хотя бы кусочно-непрерывна
      на промежутке $(0,+\infty)$, а на рынке
      сложилась цена $P^{*}>0$, то излишком потребителя называется величина
\begin{equation}
\label{opr-iz-ptr1}
      CS=\int_{P^{*}}^{+\infty}D(P)dP .
\end{equation}
      Если функция спроса устроена таким образом, что $D(P)>0$ для
      всех $P>0$, возможна ситуация, когда рассматриваемый несобственный интеграл расходится
    и, следовательно, $CS=+\infty$.



    Если функция $D(P)$ монотонно убывает на интервале $(P^{*},+\infty)$
    (или на интервале $(P^{*},\tilde{P})$, где $\tilde{P}$ --- это наименьшая
    цена, при которой спрос равен нулю), у нее на этом интервале имеется
    обратная функция $P(X)$ и для излишка потребителя верна следующая
    формула:
\begin{equation}
  \label{izl-pot1}
    CS=\int_{0}^{X^{*}}P(X)dP-P^{*}X^{*},
\end{equation}
    где $X^{*}=D(P^{*})$. Читатель, видимо, уже заметил, по-существу, именно эти две
    эквивалентные формулы для излишка потребителя мы применяли для
    вычисления излишка потребителя в нашем примере с велосипедами.

\begin{exer}
      Предположим, что функция спроса на некоторый продукт имеет вид:
     \[D(P)=1/P^{\,\alpha}, \ \alpha>0. \]
      Чему равен излишек потребителя, если на рынке сложилась цена $P^{*}=8$?
\end{exer}





      Естественно, в случае, когда функция спроса $D(P)$ на рынке
    некоторого товара сумму функций спроса отдельных потребителей:
    \[D(P)=\sum_{i=1}^{n}D_{i}(P),\]
    где $n$ --- общее количество потребителей на рынке, а $D_{i}(P)$
    --- функция спроса $i$-го потребителя, то
\begin{equation}
\label{sum-izl}
    CS=\sum_{i=1}^{n}CS_{i},
\end{equation}
    где $CS_{i}$ --- излишек потребителя $i$:
    \[CS_{i}=\int_{P^{*}}^{+\infty}D_{i}(P)dP, \ i=1,...,n.\]


    Мы дали определение излишка потребителя в предположении, что
    выполняется равенство $X^{*}=D(P^{*})$. Однако, все наши
    рассуждения можно провести и для случая, когда объемы приобретения
    и потребления рассматриваемого продукта не совпадает по каким-то
    причинам со спросом:
    \[X^{*}\neq D(P^{*}).\]
    И в этом случае равенство (\ref{izl-pot1}) может выступать в
    качестве определения излишка потребителя. Здесь, правда, нужно
    проявлять осторожность, потому, например, что в этом случае формула
    (\ref{sum-izl}) для суммы излишков потребителей уже неверна.
    Следует также указать, что мы здесь используем несколько

\begin{exer}
    Как нужно модифицировать равенство (\ref{opr-iz-ptr1}) для того,
    чтобы определить излишек потребителя в случае, когда
    $X^{*}\neq D(P^{*})$? Объясните содержательный смысл требуемой
    модификации.
\end{exer}






        Излишек производителя(лей) (точнее было бы - излишек продавцов) определяется
        аналогично. Пусть нам задана монотонно неубывающая функция предложения
        $S(P)$ на некоторый продукт. Будем считать, что эта функция определена
        на промежутке $[0,+\infty)$ и хотя бы кусочно-непрерывна.
        Если на рынке сложилась цена
        $P^{*}\in[0,+\infty)$,
        то излишком производителя(лей) $PS$ называется величина
        \[PS=\int_{0}^{P^{*}}S(P)dP.\]

    Очевидно, что если
    \[S(P)=\sum_{j=1}^{m}S_{j}(P),\]
    то
    \[PS=\sum_{j=1}^{m}PS_{j},\]
    при естественно определяемых значениях $PS_{j}$.

    В случае, когда функция $S(P)$ монотонно возрастает на интервале
    $(\tilde{P},P^{*})$, где $\tilde{P}$ --- наибольшая цена,
    при которой предложение равно нулю, то у нее на этом интервале
    имеется обратная функция $P^{S}(X)$, а излишек
    производителя(лей) может быть представлен как
    \[PS=P^{*}X^{*}-\int_{0}^{X^{*}}P^{S}(X)dX.\]
    Заметим, что последнее равенство может служить определением
    излишка потребителя и в случае, когда $X^{*}\neq S(P^{*})$
    (опять-таки с определенными оговорками).







        Пример. Предположим, что функция предложения некоторого
        товара, например, велосипеда, задается следующей таблицей:
        \[
    \begin{tabular}{|c|c|c|c|c|c|c|c|}
      \hline
      % after \\: \hline or \cline{col1-col2} \cline{col3-col4} ...
      Цена (руб./шт.) & 100 & 200 & 300 & 400 & 500 & 600 & 700 \\
      Предложение (шт.) & 0 & 1 & 2 & 3 & 4 & 5 & 6 \\
      \hline
    \end{tabular}
    \]
    Более точно, пусть функция предложения задается следующими соотношениями:
    \[S(P)=0 \ \text{при} \ 0\leq P<200; \ S(P)=1 \ \text{при} \ 200\leq P<300;\]
    \[S(P)=2 \ \text{при} \ 300\leq P<400; \ S(P)=3 \ \text{при} \ 400\leq P<500,\]
    \[S(P)=4 \ \text{при} \ 500\leq P<600; \ S(P)=5 \ \text{при} \ 600\leq P<700,\]
    \[S(P)=8 \ \text{при} \ 700\leq P.\]
    Предположим, что на рынке сложилась цена $P^{*}=550$ руб./шт.,
    а объем производства и продаж совпадает с предложением. В
    этом случае суммарный излишек производителей равен
    \[PS=\int_{0}^{550}S(P)dP=800.\]


        Зачастую излишек потребителя или излишек производителя может
        интересовать нас не сам по себе, а для того чтобы с его помощью
         можно было оценивать влияние на благосостояние потребителей
         некоторого продукта изменения цены этого продукта, вызванного
          теми или иными обстоятельствами, например, экономической
          политикой государства. Предположим, например, что цена
          некоторого продукта увеличилась с $P_{1}$ до $P_{2}$, а
          покупатели уменьшили потребление с $D(P_{1})$ единиц
    продукта до $D(P_{2})$ единиц. В этом случае излишек потребителя изменится на величину
     \[\Delta CS=-\int_{P_{1}}^{P_{2}}S(P)dP ,\]
      т.е. уменьшился на величину $\int_{P_{1}}^{P_{2}}S(P)dP$.
    Заметим, что она однозначно определена и в случае, когда и
    конечна и в том случае, когда сам излишек потребителя является
    бесконечным. Аналогичное рассуждение касается и излишка
    производителя.

\begin{exer}
    Предположим, что функция спроса на некоторый продукт имеет вид:
    \[D(P)=1/P.\]
    Предположим, далее, что функция предложения этого продукта сначала
    задавалась равенством
    \[S(P)=P+1,5,\]
    а затем предложение продукта увеличилось и функция предложения стала
    иметь вид
    \[S(P)=P+3,75.\]
    Насколько изменился излишек потребителя в предположении, что и до
    изменения функции предложения и после этого сдвига рынок продукта
    находился в состоянии равновесия (что такое равновесие???????)?
\end{exer}



\


\subsection{Еще раз об излишке потребителя и излишке производителя}

    Задачи о максимизации прибыли и полезности в том виде, как мы их
    сформулировали, позволяют лучше понять существо излишка
    потребителя и излишке производителя.

    Начнем со второго. Предположим, что производство некоторого
    продукта на некотором предприятии задается дифференцируемой
    выпуклой функцией затрат $C(Q)$. В этом случае обратная функция
    предложения $P^{S}(X)$ для данного предприятия
    задаются очень просто:
    \[P^{S}(X)=C\,'(X), \]
    а излишек производителя  при цене $P^{*}>C\,'(0)$ и объеме выпуска и продаж
    $X^{*}$, который может совпадать или не совпадать с $D(P^{*})$, равен
    \[PS^{*}=P^{*}X^{*}-\int_{0}^{X^{*}}C\,'(X).\]
    Вспомнив, как соотносятся между собой операции дифференцирования
    и интегрирования, заключаем, что
    \[PS^{*}=P^{*}X^{*}-(C(X^{*})-C(0)).\]
    Положив
    \[\pi^{*}=P^{*}X^{*}-C(X^{*}),\]
    мы можем записать:
    \[PS^{*}=\pi^{*}+C(0).\]
    Это означает, что в рассматриваемой ситуации излишек
    производителя представляет собой производителя за вычетом постоянных затрат.

    Излишек производителя можно представить и в следующем виде:
    \[PS^{*}=P^{*}X^{*}-C(X^{*})-(0-C(0)).\]
    Такая запись говорит, что излишек производителя показывает,
    насколько прибыль производителя при ценах $P^{*}$ и объемах
    выпуска $X^{*}$ больше, чем при нулевом выпуске (при любых ценах).

    Мы уже отмечали, что излишек производителю интересен нам, в первую очередь,
    не своим значением, а изменением своего значения при изменении
    цены на производимый продукт и объемов его производства и продаж.
    Пусть цена и объемы продаж после изменения стали равными $P^{**}$
    и $X^{**}$ соответственно, а излишек
    производителя ---
    \[PS^{**}=P^{**}X^{**}-C(X^{**})-(0-C(0)).\]
    Очевидно, что прирост излишка производителя
    $\Delta PS=PS^{**}-PS^{*}$ задается следующим образом:
    \[\Delta PS=P^{**}X^{**}-C(X^{**})-(P^{*}X^{*}-C(X^{*})).\]
    Это равенство говорит, что прирост излишка производителя,
    связанного с изменением цены выпускаемого продукта просто равен
    приросту прибыли данного производителя, вызванного этим
    изменением.


    Теперь перейдем к излишку потребителя. Предположим, что функция
    спроса $D(p)$ на некоторое потребительское благо задается
    равенством (\ref{f-sp-odn}), т.е.
    выводится из решения задачи (\ref{z-p}) в предположении, что функция
    $u(c)$ задана, дифференцируема, монотонно возрастает и строго вогнута на
    $\R_{+}$. В этом случае, практически повторив только что
    проведенные рассуждения рассуждения по поводу излишка
    производителя, заключаем, что при цене рассматриваемого блага,
    равной $p^{*}<u\,'(0)$, и
    размере потребления $c^{*}$, равного или неравного $D(p^{*})$,
    излишек нашего потребителя равен величине
\begin{equation*}
\label{iz-pot}
    CS=u(c^{*})-p^{*}c^{*}-u(0).
\end{equation*}
    Очевидно, что если мы просто предполагаем, что $u(0)=0$ (что, как мы
    уже указывали, вполне допустимо), то излишек потребителя
    представляет собой значение целевой функции задачи потребителя
    при цене $p^{*}$ и потреблении $c^{*}$. Если равенства $u(0)=0$ не
    предполагать, то излишек потребителя удобно записать следующим
    образом:
    \[CS=u(c^{*})-p^{*}c^{*}-(u(0)-0).\]
    Это равенство говорит, что излишек потребителя показывает,
    насколько выше благосостояние рассматриваемого потребителя,
    измеряемого с помощью функции $u(c)-p^{*}c$,
    при цене $p^{*}$ и потреблении $c^{*}$, чем при нулевом потреблении
    (и любых ценах).

    Как уже говорилось, по поводу функции $u(c)$ мы можем
    предполагать, что $u(0)=-\infty$ и $u\,'(0)=+\infty$. В этом
    предположении обратная функция спроса $p(x)=u\,'(x)$, вытекающая из решения задачи
    (\ref{z-p}) окажется (может оказаться????) такой, что интеграл
    $\int_{0}^{x^{*}}p(x)dx$ расходится при $x^{*}>0$ и, следовательно,
    излишек потребителя может оказаться равным бесконечности.
\begin{exer}
    Вычислите для различных $p^{*}$ и $c^{*}=D(p^{*})$ излишек
    потребителя в случае, когда функция спроса
    выводится из задачи (\ref{z-p}) при
    \[u(c)=\ln c .\]
    Для каких значениях параметра $\rho$ излишек потребителя конечен при
    \[u(c)=\frac{c^{\,\rho}}{\rho}, \ 0\neq\rho<1?\]
\end{exer}

    Излишек потребителя, как и излишек  производителя, интересен не
    своим значением, а возможными изменениями этого значения.
    вызванными изменением цены потребительского блага, которое
    фигурирует в задаче потребителя. Читатель без труда проверит,
    что изменение $\Delta CS$ излишка потребителя, вызванный
    изменением цены потребительского блага и объема его потребления
    просто равно изменению благосостояния
    рассматриваемого потребителя, вызванного этим изменением.

\begin{exer}
    Предположим, что спрос на некоторый продукт со стороны
    предприятия, для которого этот продукт выступает в качестве ресурса
    возникает из решения задачи (\ref{max-prib-3}). Как можно в этом
    случае можно проинтерпретировать излишек потребителя?
\end{exer}


\





\subsection{Сравнительная статика}

    ОБЩЕЕ ПРАВИЛО: если мы используем обозначение типа $f\,'(x)$, то
    мы имеем в виду, что функция $f(x)$ дифференцируема.

    \

    Предположим, что состояние равновесия в некоторой модели задается
    как решение уравнения или системы уравнений. Скорее всего это состояние
    равновесия зависит от тех или иных параметров. Изучение зависимости
    состояния равновесия от значения параметров называют в экономике
    анализом сравнительной статики.

    Пример. Рассмотрим модель частного равновесия в которой,
    напомним, спрос на некоторый продукт задается монотонно
    убывающей функцией спроса $D(P)$, предложение задается монотонно
    возрастающей функцией предложения $S(P)$, а равновесная цена
    $P^{*}$  определяется как решение уравнения
    \[D(P)=S(P). \ (1)\]
    Предположим, что функция спроса представима в виде
    \[D(P)=H(P)+X, \   (2)\]
    где $H(P)$ - спрос на продукт со стороны частных лиц, а $X$ - спрос со
    стороны государства (государственный заказ, не зависящий от цен).
    Исследуйте зависимость состояния равновесия на рынке в зависимости
    от $X$.

    Решение. Очевидно, что равновесная цена $P^{*}$ зависит от размера
     спроса со стороны государства:
    $P^{*}=P(X).$ В самом общем виде по поводу функции $P(X)$ можно сказать мало. В то
    же время следует заметить, что уравнение (2) задает функцию
    \[P(X)\] неявно. Тем самым, используя правило дифференцирования неявной
    функции, мы можем вывести производную
    \[dP^{*}/dX=P\,'(X)\]
     (если, конечно, нам известны производные функций спроса и предложения в
    точке равновесия).


    Определим
    \[G(P,X)=H(P)+X-S(P).\]
    Мы можем переписать (1) в следующем виде:
    \[G(P,X)=0\]
    и применить правило дифференцирования неявных функций:
    \[dP^{*}/dX=
    -\frac{\partial G(P^{*},X^{*})/\partial X}{\partial G(P^{*},X^{*})/\partial P}
    =\frac{1}{S\,'(P)-D\,'(P)}.\]

    Полученные выражения дают довольно много пищи для размышлений и
    достаточно полно описывают зависимость состояния равновесия
    от спроса на продукт со стороны государства. Во-первых, заметим, что
    \[dP^{*}/dX>0.\]
    Тем самым, с ростом спроса со стороны государства цена равновесия
    будет расти. Во-вторых, выполняются следующие неравенства
    \[0<dQ^{*}/dX<1.\]
    Эти неравенства говорят о том, что, с одной стороны, увеличение
    спроса со стороны государства ведет к росту продаж, а с другой - что
    этот рост продаж будет меньше, чем увеличение спроса. Это последнее
    свойство объясняет неравенство
    \[dH^{*}/dX<0,\]
    которое говорит о том, что с ростом спроса со стороны государства
    будет уменьшаться объем покупок со стороны частных лиц.

\begin{exer}
    Предположим, что
    \[D(P)=a-bP, \ a,b>0, \ S(P)=c+gP, \ g>0.\]
    Выведите с помощью правил дифференцирования неявных функций формулы
    для следующих следующих величин (ХОРОШИ ЛИ ОБОЗНАЧЕНИЯ?):
    \[dP^{*}/da, \ dP^{*}/db, \ dP^{*}/dc, \ dP^{*}/dg, \
    dQ^{*}/da,  \ dQ^{*}/db,  \ dQ^{*}/dc, \ dQ^{*}/dg,\]
    где, как и выше, $P^{*}$ - это равновесная цена, а $Q^{*}=S(P^{*})=D(P^{*})$
    - объем продаж на рынке в состоянии равновесия. Проверьте ответ,
    вычислив $P^{*}$ и $Q^{*}$ в явном виде. Проинтерпретируйте полученные результаты.
\end{exer}


\begin{exer}
    Рассмотрим  простейшую кейнсианскую модель. Пусть $Y^{*}$
    --- равновесный уровень национального дохода. Напомним, что он определяется
    как решение уравнения
    \[Y=C(Y)+I+G,\]
    где $C(Y)$ --- функция потребления, $I$ --- инвестиционный спрос,
    $G$ --- планируемый объем государственных закупок. Величина $C\,'(Y)$
    называется предельной склонностью к потреблению. Предполагается, что
    эта величина удовлетворяет неравенствам
    \[0<C\,'(Y)<1.\]
    Докажите, что справедливо следующее неравенство:
    \[dY^{*}/dI>1.\]
    Проверьте равенство
    \[dY^{*}/dG=1/(1-C\,'(Y^{*})). \  (3)\]
    Проинтерпретируйте эти соотношения.
    Предположите, что
    \[C(Y)=cY+a, 0<c<1, a>0,\]
    и покажите, что
    \[dY^{*}/dc>0.\]
    Проинтерпретируйте это неравенство.
\end{exer}


        Модель $IS-LM$.
    Модель $IS-LM$ является развитием простейшей кейнсианской
    модели. В ней предполагается, что инвестиционный спрос $I$
    не является заданным параметром, а зависит от ставки процента:
    \[I=I(r). \ (4)\]
    Функция $I(r)$ называется инвестиционной. Основное
    предположение по поводу этой функции состоит в том, что она является
    дифференцируемой и монотонно убывающей:
    \[I\,'(r)<0.\]
    Это предположение объясняется вполне естественными причинами. Чем
    выше ставка процента, т.е. чем дороже кредит, тем меньше будет
    прибыльных инвестиционных проектов и, следовательно, тем меньше
    будет спрос на инвестиционные товары.

    С учетом (4) уравнение, задающее равновесный уровень
    национального дохода в простейшей кейнсианской модели
    приобретает следующий вид:
    \[Y=C(Y)+I(r)+G.\]
    Решение этого уравнения относительно $Y$  задает в модели
    $IS-LM$ равновесие на рынке товаров и услуг. Оно зависит от r.
    Обозначим его как $Y(r)$.

\begin{exer}
    Поверьте, что $Y\,'(r)<0$ (здесь мы предполагаем, что все функции,
    которые фигурируют в этой модели, в достаточной степени
    гладкие).
\end{exer}


    В модели $IS-LM$ в явном виде присутствует рынок денег, на котором
    формируется спрос и предложение на деньги. Точнее, речь идет о
    спросе и предложении на деньги с учетом их покупательной способности,
    которая зависит от уровня цен $P$, т.е. на реальные денежные остатки.
    Реальные денежные остатки $m$ (реальные запасы денежных средств)
    определяется равенством
    \[m=M/P,\]
    где $M$ - количество денег в обращении (денежная масса), а $P$ - уровень
    цен. Эта величина показывает количество товаров и услуг, которое
    можно приобрести, имея деньги в количестве $M$ при уровне цен $P$.

    Предложение реальных денежных остатков $m^{s}$ задается в рамках
    модели $IS-LM$ равенством
    \[m^{s}=M^{s}/P,\]
    где $M^{s}$ - предложение денег (количество "напечатанных" денег), а $P$ -
    уровень цен. Обе этих величины считаются в рамках модели заданными
    параметрами. Что касается спроса на реальные денежные остатки $m^{d}$, то
    предполагается, что он зависит от размера национального дохода $Y$  и
    от ставки процента $r$:
    \[m^{d}=m^{d}(Y,r).\]
    Функция $m^{d}(Y,r)$ монотонно возрастающая в зависимости от $Y$, и
    монотонно убывает в зависимости от $r$:
    \[\partial m^{d}/\partial Y>0, \ \partial m^{d}/\partial r<0.\]

    Тот факт, что спрос на реальные денежные остатки растет с ростом
    национального дохода (в реальном исчислении) объясняется очень
    просто: деньги служат в первую очередь средством обращения, они
    нужны для совершения сделок. Тем самым, чем больше национальный
    доход, тем больше совершается сделок, и, следовательно, тем больше
    денег необходимо для обеспечения этих сделок.

    Что касается зависимости спроса на реальные денежные остатки, то
    она тоже объясняется довольно просто. Каждый экономический агент
    имеет возможность либо хранить деньги в виде наличности или на счету
    в банке, не приносящем доход, либо-каким то образом делать
    сбережения, приносящие доход по ставке процента $r$. Каждый лишний
    не сбереженный рубль означает что этот экономический агент несет некоторые
    потери от недополучения процентного дохода, при этом, чем выше будет
    ставка процента, тем больше будут потери. Отсюда естественно вытекает
    предположение о том, что с ростом процентной ставки будет расти желание
    деньги сберечь и уменьшится желание держать их "на руках", иными
    словами, уменьшится спрос на деньги со стороны данного экономического агента.

    Равновесие на рынке денег определяется равенством спроса и предложения
    на реальные денежные остатки:
    \[m^{s}=m^{d},\]
    или, иначе, равенством
    \[m^{d}(Y,r)= M^{s}/P.\]
    Это равенство задает неявную функцию $r(Y)$.

\begin{exer}
    Докажите, что $dr/dY>0$. Проинтерпретируйте это неравенство.
\end{exer}

    Состояние равновесия $(Y^{*},r^{*})$ в модели $IS-LM$ определяется как решение
    системы из двух уравнений с двумя неизвестными ($Y$ и $r$):
    \[Y=C(Y)+I(r)+G,\]
    \[m^{d}(Y,r)= M^{s}/P.\]
    Иными словами, состояние равновесия определяется одновременным
    равновесием на рынке товаров и услуг и рынке денег. Заметим, что
    вместе с равновесными значениями национального дохода $Y^{*}$ и ставки
    процента $r^{*}$, мы можем определить равновесные значения потребления
    $C^{*}$ и частных инвестиций $I^{*}$:
    \[C^{*}=C(Y^{*}), \ I^{*}=I(r^{*}).\]
    Модель $IS-LM$ является одной из основных моделей макроэкономики.
    Многие макроэкономические проблемы исследуются посредством анализа
    сравнительной статики в этой модели. Такой анализ позволяет отвечать
    на вопросы о том,  как повлияет изменение параметров модели на состояние равновесия.


    Пример. Исследовать зависимость равновесного уровня выпуска $Y^{*}$ в
    модели $IS-LM$ от объема государственных закупок $G$.
    Решение. Положим:
    \[\varphi (Y,r,G)=Y-C(Y)+I(r)+G,\]
    \[\psi (Y,r,G)=m^{d}(Y,r)-M^{s}/P.\]
    При каждом заданном $G$ состояние равновесия $(Y*,r*)$ задается как
    решение относительно $Y$ и $r$ системы из двух уравнений:
    \[\varphi (Y,r,G)=0,\]
    \[\psi (Y,r,G)=0.\]
    Тем самым, эти два уравнения задают неявную функцию:
    \[Y^{*}=\varpi(G),\]
    производную которой мы считать умеем:

    \[\frac{dY^{*}}{dG}=\varpi\,'(G)=\frac{\frac{\partial m^{d}}{\partial r}}
    {\frac{\partial m^{d}}{\partial r}\left(1-\frac{\partial C}{\partial Y}\right)
    +\frac{dI}{dr}\frac{\partial m^{d}}{\partial Y}}.\]


    Мы видим, что выполняются следующие неравенства:
    \[0<\frac{dY^{*}}{dG}<\frac{1}{1-C\,'(Y^{*})}.\]
    Эти неравенства имеют естественную интерпретацию. С одной стороны,
    увеличение государственных закупок, ведет к увеличению равновесного
    уровня национального дохода (первое неравенство), а с другой -
    это увеличение будет меньшим, чем в случае простейшей
    кейнсианской модели (второе неравенство, ср. с (3)).

\begin{exer}
    Докажите и проинтерпретируйте следующие неравенства:
    \[dr^{*}/dG>0, \   dI^{*}/dG<0.\]
\end{exer}

\begin{exer}
    Исследуйте зависимость равновесного уровня выпуска $Y^{*}$ в модели
    $IS-LM$ от предложения денег $M^{s}$. Проинтерпретируйте полученные результаты.
\end{exer}



\


Очевидно, что прибыль производителя при монопольных цене и выпуске
    выше, чем при конкурентных. С другой стороны, с точки зрения
    потребителей монопольная ситуация на рынке представляется менее
    предпочтительной, чем конкурентная. Можно ли сравнить конкурентное
    ($P^{C}, \ Y^{C}$) и монопольные ($P^{M}, \ Y^{M}$) равновесия
    с точки зрения общественного благосостояния?


    Чтобы нам было далее удобно провести такое сравнение,
    предположим, что произошел условный переход из первого равновесия
    во второе.

    Прирост прибыли производителя после такого перехода в точности
    совпадает с увеличением $\Delta PS$ его излишка производителя.
    Что касается потребителей, то их совокупные потери можно оценить
    с помощью изменения $\Delta CS$ излишка потребителя (точнее
    излишка потребителей). Очевидно, что в данном случае $\Delta CS<0$.
    Поскольку и излишек потребителей, и излишек производителя
    измеряются в денежном выражении, небессмысленной становится
    оценка перехода из конкурентного состояния рынка в монопольное
    посредством суммы
    \[\Delta PS+\Delta CS.\]
    Оказывается, эта сумма отрицательна. Действительно, мы имеем:
    \[\Delta CS=-\int_{P^{C}}^{P^{M}}D(P)dP,\]
    \[\Delta PS=(P^{M}-C\,'(Y^{M}))Y^{M}-\int_{C\,'(Y^{M})}^{P^{M}}(C\,')^{-1}(P)dP.\]
    Взглянув на рисунок ????, мы убедимся, что потери в излишка
    потребителей, равные $|\Delta CS|=\int_{P^{C}}^{P^{M}}D(P)dP$, превосходят увеличение
    излишка производителя $\Delta PS$.

    Величина $-(\Delta PS+\Delta CS)$
    называют безвозвратными потерями (deadweight loss). С ее помощью
    оценивают общественные потери, вызванные монополизацией рынка.
\begin{exer}
    Вычислите монопольные цену и выпуск, а также безвозвратные
    потери в случае, когда $D(P)=a-bP, \ a,b>0,$ и $C(Y)=cY, \
    0<c<a$. Проиллюстрируйте свои вычисления с помощью графиков.
\end{exer}

\begin{dfn}
Гиперплоскость $\st{H}$ называется \dw{опорной} гиперплоскостью к
множеству $\st{X}$, если эта гиперплоскость имеет общую точку с
$\st{X}$, и множество $\st{X}$ полностью лежит в одном из замкнутых
полупространств, определяемых этой гиперплоскостью.\end{dfn}

Иллюстрация данного определения представлена на Рис.
\ref{fig:sup-hplane}. Здесь гиперплоскость имеет с множеством
$\st{X}$ общую точку $\vc{\hat x}$. В таком случае говорят, что
гиперплоскость является \dw{опорной в точке $\vc{\hat x}$}.

\input{pics/sup_hplane.TpX}

\begin{exer}
Докажите, что:
\begin{itemize}
  \item замкнутое полупространство является выпуклым множеством;
  \item гиперплоскость является замкнутым и выпуклым множеством.
\end{itemize}
\end{exer}
т.д.


В классической математической экономике те задачи, которые сводятся
к максимизации или минимизации некоторых функций, обычно разрешаются
методами математического анализа. Основной прием, который при этом
используется, -- приравнивание производных нулю -- по существу
эквивалентен отысканию касательной, или опорной, гиперплоскости.
Суть дела здесь не в использовании аналитических методов, а в факте
существования указанных гиперплоскостей, на чем коренным образом
основывается решение многих таких задач.


\begin{teo}(Об отделяющей гиперплоскости, б/д)
\label{teo:sep-hplane}

Пусть $\st{X}$ --- непустое замкнутое выпуклое множество в $\R^n$.
Если точка $\vc{\hat x} \notin \st{X}$, то существует такой вектор
$\vc{p} \in \R^n$, $\vc{p} \neq \vc{0}$, и число $\gamma \in \R$,
что для любого $\vc{x} \in \st{X}$ выполняется:

\[\vc{p} \cdot \vc{x} \leq \gamma < \vc{p} \cdot \vc{\hat x}\]
\end{teo}

Фактически, речь идет о том, что множество  и точка $\vc{\hat x}$
лежат по разные стороны гиперплоскости $\vc{p} \cdot \vc{x} =
\gamma$ (см. Рис.~\ref{fig:sep-hplane} а)). Такая гиперплоскость
называется \dw{отделяющей}.

Тогда, утверждение теоремы можно переформулировать следующим
образом: \emph{для замкнутого выпуклого множества и любой не
принадлежащей ему точки существует отделяющая гиперплоскость}.

\begin{note}

Отметим, что на знаки компонентов вектора $\vc{p}$ ограничения не
накладываются, т.е. для вектора $\vc{p}$ может выполняться как
$\vc{p} \leq \vc{0}$, так и $\vc{p} \geq \vc{0}$, но не $\vc{p} =
\vc{0}$. Аналогично, $\gamma$ может быть больше, меньше, а кроме
того, и равно нулю.
\end{note}


\input{pics/sep_hplane.TpX}

В случае, когда $\vc{\hat x}$ является \emph{граничной} точкой
множества $\st{X}$ (см. Рис.~\ref{fig:sep-hplane} б)), можно
сформулировать следующее утверждение\footnote{См. также
раздел~\vref{app:topology} математического приложения.}.

\begin{teo}(Об опорной гиперплоскости, б/д)

Пусть $\vc{\hat x}$  --- граничная точка замкнутого выпуклого
множества $\st{X}$. Тогда существует опорная к $\st{X}$ в точке
$\vc{\hat x}$ гиперплоскость, т.е. существует такой вектор $\vc{p}
\in \R^n$, $\vc{p} \neq 0$, и число $\gamma \in \R$, что для любого
$x \in \st{X}$ выполняется $\vc{p} \cdot \vc{x} \leq \vc{p} \cdot
\vc{\hat x}$.
\end{teo}

\begin{note}
С другой стороны, это означает, что точка $\vc{\hat x}$ фактически
есть решение задачи:

\[
\left\{ \begin{array}{l}
 \vc{p} \cdot \vc{x} \rightarrow \max\\
 \vc{x} \in \st{X} \\
 \end{array} \right.
\]

Данное соображение понадобится нам ниже при обсуждении эффективных
точек множества.
\end{note}



\begin{teo}\label{teo:sep_hplane_Mink}(Минковского о
разделяющей гиперплоскости, б/д\/\footnote{\cite{Takayama:1985},
\cite{Braverman:1976}})

Пусть $\st{X}$ и $\st{Y}$ -- непустые, необязательно замкнутые,
выпуклые множества в пространстве $\R^n$, не имеющие общих точек
либо имеющие общими только граничные точки, и пусть хотя бы одно из
множеств, например, множество $\st{Y}$, имеет внутреннюю точку (т.е.
$\st{X} \cap \st{Y}^o = \emptyset$).

Тогда существует гиперплоскость, разделяющая эти два множества, т.е.
существует $\vc{p} \in \R^n$, $\vc{p} \neq \vc{0}$, и число $\gamma
\in \R$ такие, что для любого $\vc{x} \in \st{X}$ и любого $\vc{y}
\in \st{Y}$ выполняется соотношение:

\[
\vc{p} \cdot \vc{y} \leq \gamma \leq \vc{p} \cdot \vc{x}.
\]
\end{teo}

\input{pics/sep_sets.TpX}

Заметим, что в этом случае гиперплоскость называется
\dw{разделяющей}.

Утверждение данной теоремы проиллюстрировано на
Рис.~\ref{fig:sep-sets}. В случае \emph{б)} множества $\st{X}$ и
$\st{Y}$ имеют общую граничную точку.

\remrk{Единственность гиперплоскости в случае общей граничной
точки?}

Случай \emph{в)} демонстрирует существенность предположения о том,
что хотя бы одно из множеств должно иметь внутреннюю точку: в этом
случае множества $\st{X}$ и $\st{Y}$ не могут быть разделены.

\begin{exer}
Покажите на примерах, что для невыпуклых множеств это утверждения
трех теорем отделимости не имеют места.
\end{exer}


\begin{teo}(О строгой отделимости, б/д\/\footnote{\cite{Kapustin:2001} стр. 36})

Непустые, непересекающиеся множества $X$ и $Y$, по крайней мере одно
из которых ограничено, строго отделимы. \end{teo}

\section{Лемма Минковского-Фаркаша}

Одним из важнейших приложений теорем об отделимости является лемма
Минковского-Фаркаша, доказательство которой основывается на
соответствующих утверждениях. Однако предварительно мы сформулируем
еще одну лемму.

\begin{lem}

Пусть $\st{K}$ --- конус\index{конус} в пространстве $\R^n$ с
вершиной в нуле и пусть задан вектор $\vc{p} \in \R^n$. Если $\vc{p}
\cdot \vc{x}$ ограничено снизу для любого $\vc{x} \in \st{K}$, то
$\vc{p} \cdot \vc{x} \geq 0$ (для любого $\vc{x} \in \st{K}$).
\end{lem}

\remrk{В приложении дать определение конуса, многогранного выпуклого
конуса.}

\begin{proof}

По предположению об ограниченности $\vc{p} \cdot \vc{x}$, существует
число $\alpha \in \R$ такое, что $\vc{p} \cdot \vc{x} \geq \alpha$
для любого $\vc{x} \in \st{K}$. Поскольку $\st{K}$ является конусом
с вершиной в нуле, тот факт, что $\vc{x} \in \st{K}$ означает, что
$\theta \vc{x} \in \st{K}$ для любого $\theta \geq 0$. Тогда $\vc{p}
\cdot (\theta \vc{x}) \geq \alpha$ или $\vc{p} \cdot \vc{x} \geq
\alpha / \theta$ для любого $\vc{x} \in \st{K}$ и $\theta > 0$.
Рассматривая предел данного соотношения при $\theta \rightarrow
\infty$, получаем $\vc{p} \cdot \vc{x} \geq 0$.
\end{proof}


Сформулируем лемму Минковского-Фаркаша.

\begin{lem}(Минковского-Фаркаша)  \label{lem:MinkFarcas}

Пусть $\vc{a}^1, \vc{a}^2, \ldots, \vc{a}^l$ и $\vc{b}\neq0$ ---
векторы в пространстве $\R^n$. Предположим, что $\vc{b} \cdot \vc{x}
\geq 0$ для любых $\vc{x} \in \R^n$, таких, что $\vc{a}^k \cdot
\vc{x} \geq 0$, $k=1 \ldots l$. Тогда существуют неотрицательные
числа $\lambda_1, \lambda_2, \ldots, \lambda_l$, одновременно не
равные нулю, такие, что $\vc{b}=\sum\limits_{k=1}^l{\lambda_k
\vc{a}^k}$.
\end{lem}

\begin{proof}

Пусть $\st{K}$ --- многогранный выпуклый конус, порожденный
векторами $\vc{a}^1, \vc{a}^2, \ldots, \vc{a}^l$. Тогда $\st{K}$
является замкнутым множеством (\remrk{Почему?}). Необходимо
показать, что $\vc{b} \in \st{K}$.

Предположим, что $\vc{b} \notin \st{K}$. Тогда $\st{K}$ ---
непустое, замкнутое, выпуклое множество, не включающее точку
$\vc{b}$. Тогда, по Теореме \ref{teo:sep-hplane}, существует такой
вектор $\vc{p} \in \R^n$, $\vc{p} \neq 0$, и число $\alpha \in \R$,
что для любого $\vc{x} \in \st{K}$ выполняется\footnote{Знаки
неравенств в утверждении теоремы можно одновременно изменить на
противоположные}:

\[\vc{p} \cdot \vc{x} \geq \alpha > \vc{p} \cdot \vc{b}.\]

Таким образом, $\vc{p} \cdot \vc{x}$ ограничено снизу для любого
$\vc{x} \in \st{K}$. По предыдущей лемме получаем $\vc{p} \cdot
\vc{x} \geq 0$ для любого $\vc{x} \in \st{K}$. Кроме того, поскольку
$\vc{0} \in \st{K}$, $\vc{p} \cdot \vc{0} \geq \alpha$ или $\alpha
\leq 0$. Отсюда $\vc{p} \cdot \vc{b} < 0$.

Поскольку $\vc{a}^k \in \st{K}$, то $\vc{p} \cdot \vc{a}^k \geq 0$,
$k=1 \ldots l$. Тогда для найденного вектора $\vc{p}$ получаем
$\vc{b} \cdot \vc{p} < 0$ при $\vc{p} \cdot \vc{a}^k \geq 0$, $k=1
\ldots l$. Это противоречит условию теоремы. Значит, $\vc{b} \in
\st{K}$. По свойству многогранного выпуклого конуса, это означает,
что существуют неотрицательные числа $\lambda_1, \lambda_2, \ldots,
\lambda_l$, такие, что $\vc{b}=\sum\limits_{k=1}^l{\lambda_k
\vc{a}^k}$. Поскольку $\vc{b}\neq0$, числа $\lambda_1, \lambda_2,
\ldots, \lambda_l$ не могут быть одновременно равны нулю.


\end{proof}


\begin{note}(к лемме Минковского-Фаркаша)

\begin{enumerate}
\renewcommand{\theenumi}{(\alph{enumi})}

  \item Если $b=0$, то допускается одновременное равенство нулю всех
  $\lambda_k$, $k=1 \ldots l$.

  \item Верно и обратное утверждение относительно Леммы \ref{lem:MinkFarcas},
  а именно, если $\vc{a}^1, \vc{a}^2, \ldots, \vc{a}^l$ и $\vc{b}\neq0$
  ---  векторы в пространстве $\R^n$ и существуют неотрицательные
   коэффициенты $\lambda_1, \lambda_2, \ldots, \lambda_l$,
   одновременно не равные нулю, такие, что $\vc{b}=\sum\limits_{k=1}^l{\lambda_k
   \vc{a}^k}$,то $\vc{b} \cdot \vc{x} \geq 0$ для любых $\vc{x} \in \R^n$, таких,
   что $\vc{a}^k \cdot \vc{x} \geq 0$, $k=1 \ldots l$.

\end{enumerate}

\end{note}


\remrk{Можно добавить векторно-матричную и геометрическую
интерпре\-тацию леммы из \cite{Takayama:1985}, стр. 47.}

Лемма Минковского-Фаркаша играет важную роль в теории линейного
программирования, теории игр, теории нелинейного программирования и






\begin{lem}

Пусть $\st{X} \subset \R^n$ --- некоторое замкнутое множество. Если
$\vc{\hat x}$ является слабо эффективной точкой множества $\st{X}$,
то $\st{X} \cap \st{Z}^o=\emptyset$, где $\st{Z}^o=\{\vc{z} \in
\R^n: \vc{z} \gg \vc{\hat x}\}$.\end{lem}

\begin{proof}(От противного)

Предположим, что утверждение леммы не выполняется. Тогда существует
$\vc{z} \in \st{X}$ такой, что $\vc{z} \gg \vc{\hat x}$. Тогда
$\vc{\hat x}$ не является (слабо) эффективной точкой множества
$\st{X}$, что противоречит предположению леммы.

\end{proof}

\begin{exer}
Докажите, что если  $\vc{\hat x}$ является строго эффективной точкой
множества $\st{X}$, то $\st{X} \cap \st{Z}=\{\vc{\hat x}\}$, где
$\st{Z}=\{\vc{z}\in \R^n: \vc{z} \geqq \vc{\hat x}\}$.
\end{exer}

\begin{exer}
Докажите, что в обоих случаях множество $\st{Z}$ является выпуклым.
\end{exer}
