

\section*{Экономика чистого обмена}

Итак, мы рассматриваем специальный случай, когда все агенты -- потребители и нет производства.
Это экономика обмена. Простейший вариант этой модели с двумя агентами,
предпочтения каждого из которых описывались функциями полезности
типа Кобба-Дугласа, и двумя товарами мы рассматривали в начале этой главы.
В экономике обмена имеется несколько потребителей, каждый из которых
описывается набором своих предпочтений и набором товаров (начальным запасом),
которыми он обладает. Каждый из агентов
пытается совершить обмен с другими агентами, надеясь улучшить свое положение.
Здесь стоит отметить, что осуществление обмена в духе "я -- тебе, ты -- мне",
особенно, если агентов и товаров много, представляется не очень простой
задачей, поскольку может, вообще говоря, не существовать возможности обмена,
который устраивал бы всех агентов, или, напротив, таких вариантов обмена может оказаться
достаточно много (мы увидим это ниже на примере \emph{ядра} экономики обмена).
Поэтому можно условно предположить, как мы сделали это в начале главы, что есть
некий координатор, назначающий цены на имеющиеся продукты. Далее, каждый агент
может продать по этим ценам свой начальный запас и на вырученные деньги
купить желательный (максимизирующий полезность) набор товаров.
Каким же будет итог?


\underline{Агенты и товары}

Опишем рассматриваемую модель формально.
Каждый агент -- потребитель, причем агент $i$ полностью
описывается своей функцией
полезности $u_i, i=1,\ldots,n$ и начальным набором $w_i$ из $k$ товаров,
который называется начальным запасом.

Предположим далее, что задан вектор цен $p$, причем каждый агент
принимает эти цены как данные. Конечно, это предположение в случае
небольшого числа агентов может показаться странным:
здесь более естественным было бы считать, например, что агент сам
назначает цену товара (особенно, если считать, что у агента есть
весь начальный запас какого-то из товаров). Однако в подобного рода
ситуациях принято интерпретировать агента как представителя большого
числа идентичных агентов. В этом варианте лучше говорить не о потребителе,
а о \emph{типе} потребителя. Это уже делает предположение о том, что
агенты являются "ценополучателями",\, значительно белее естественным.
Если же к этому добавить предположение об одинаковом числе агентов каждого из
типов, то это не изменит анализ, в сравнении со случаем двух агентов.


Как мы уже говорили выше, для произвольного вектора цен $p$, вообще
говоря, спрос на какие-то товары будет превышать предложение, а на какие-то
товары -- наоборот, предложение будет превышать спрос. Естественно, что
в этом случае цена первых должна быть повышена, а цена вторых -- снижена.
Можно надеяться, что найдется такой вектор цен, что спрос на каждый из
товаров будет равен его предложению, и при этом каждый из агентов будет
максимизировать свою полезность.


Каждый агент пытается получить наиболее
предпочтительный для него набор. Будем далее обозначать через
$$
x_i=(x^1_i,x^2_i,\ldots,x^k_i)\in\R^k -{ \mbox{\rm потребительский
набор агента i}}.
$$
Обратим внимание на то, что в наших обозначениях верхний индекс
соответствует номеру продукта, а нижний -- номеру агента.
Распределение $x=(x_1,\ldots,x_n)$ - набор из $n$ потребительских наборов.
Распределение $x$ можно представлять себе либо как вектор (строку или столбец)
размерности $kn$, либо как матрицу размера $k\times n$ или $n\times k$. Для нас
это не принципиально, поскольку никаких действий типа умножения матрицы на
столбец и т.п. мы производить не будем. Для нас существенно то, что
распределение -- это перечень всех потребительских наборов, которые
имеются в распоряжении агентов, или которые агенты хотели бы получить.
Разумеется, не всякое распределение будет возможно получить в
рассматриваемой экономике обмена. Поэтому мы будем говорить, что $x$
является \emph{допустимым распределением}, если
$$
\sum^n_{i=1}x_i\leq\sum^n_{i=1}w_i.
$$

Иными словами, допустимым является распределение, при котором
потребление каждого из продуктов не превосходит
имеющихся в экономике (у агентов) запасов каждого из продуктов.

Часто бывает удобно в определении допустимости требовать выполнение равенства.
Мы обсудим различие в таких определениях допустимости ниже.

Обратимся к простейшему случаю $n=k=2$, когда возможна очень удобная
графическая интерпретация рассматриваемой ситуации, при этом допустимым
распределением мы будем считать распределение, удовлетворяющее именно равенству.
В этом случае мы можем изобразить так называемый \emph{ящик Эджворта}.

Пусть $w^1=w^1_1+w^1_2$ -- имеющийся в экономике запас первого продукта,
$w^2=w^2_1+w^2_2$ -- имеющийся в экономике запас второго продукта.
Заметим теперь, что если мы рассмотрим допустимое распределение $x$,
в котором первый агент располагает (или хотел бы иметь в своем распоряжении)
набором продуктов $(x^1_1,x^2_1)$, то потребительский набор второго агента
определяется из условий допустимости очевидным образом:
$(x^1_2,x^2_2)=(w^1-x^1_1,w^2-x^2_1)$.

Теперь мы можем изобразить ящик Эджворта (см. рис.   ).
На рисунке мы видим две системы координат: первая с началом в точке $0_1$
соответствует первому агенту. Вторая, оси которой
направлены влево и вниз, с началом в точке $0_2$, соответствует второму агенту.
Высота ящика, $w^2$, равна суммарному запасу второго товара, ширина $w^1$ --
суммарному запасу первого товара. Точка $0_2$ в первой системе координат
имеет координаты $(w^1,w^2)$ и соответствует ситуации, когда весь товар
(и первый и второй) полностью
сосредоточен в руках первого агента. Наоборот, точка $0_1$ во второй системе
координат имеет координаты $(w^1,w^2)$ и соответствует ситуации, когда весь товар
сосредоточен в руках второго агента.

Далее, некоторая точка $w$ соответствует начальному запасу, причем
в системе координат первого агента его начальный запас будет $(w^1_1,w^2_1)$,
а в системе координат второго агента его начальный запас будет $(w^1_2,w^2_2)$.
Наконец, произвольная точка $x$ в ящике Эджворта будет соответствовать
допустимому распределению.


РИСУНОК  Exchange 1


На рис.    изображены бюджетные множества каждого из агентов при данных ценах
$p_1$ и $p_2$ первого и второго товаров.


РИСУНОК  Exchange 2

Распределения в тех частях бюджетных множеств, которые лежат вне ящика Эджворта,
являются недопустимыми распределениями.

Далее в ящике Эджворта мы можем изобразить кривые безразличия каждого из агентов.
На рис.    в ящике Эджворта изображены два типа кривых безразличия:
кривые первого типа -- это кривые безразличия первого агента, причем чем выше,
тем выше соответствующий уровень полезности; кривые второго типа --
это кривые безразличия второго агента в его системе координат (стрелка,
указывающая направление возрастания полезности второго агента, направлена вниз).

РИСУНОК  Exchange 3

Мы будем использовать ящик Эджворта
для графической интерпретации соответствующих определений и результатов.
(Мы изображаем кривые безразличия, соответствующие выпуклым отношениям
предпочтения агентов. Разумеется, это не является обязательным требованием,
однако выпуклость играет ключевую роль в вопросе существования равновесия
(соответствующий пример мы приведем ниже), а также при рассмотрении
теорем экономики благосостояния).

\underline{Равновесие по Вальрасу}

Итак предположим, что на рынке есть вектор цен $p=(p_1,\ldots,p_k)$. Любой
потребитель принимает его как данный и выбирает наиболее предпочтительный набор,
доставляющий ему максимальную полезность при имеющихся бюджетных ограничениях,
то есть решает задачу:
$$
\begin{array}{c}
\max u_i(x)\\
px_i=pw_i.
\end{array}
$$



Возникает функция спроса потребителя $x_i(p,pw_i)$. При строго
выпуклом отношении предпочтении эта функция является "хорошей"\, непрерывной функцией.
На рис.    изображено решение соответствующей задачи для первого агента.

РИСУНОК  Exchange 4



Конечно, при произвольных ценах добиться желательной сделки может
быть и невозможно, так как может оказаться, например, что
$$
\sum_ix_i(p,pw_i)\ne\sum_iw_i.
$$
Так какие-то товары могут быть в избытке, а для каких-то, напротив,
спрос может превышать предложение.

На рис.     показана соответствующая ситуация. Под чистым предложением
агентом 2 продукта 2 понимается превышение (избыток) второго продукта
у второго агента по сравнению с оптимальным для него количеством
второго товара, то есть $w^2_2-x^2_2(p,pw_2)>0$. (Напомним, что в
ситуации на рисунке $w^2_2>x^2_2(p,pw_2)$, так как ось, соответствующая второму
продукту и второму агенту, направлена вниз). Аналогично, под чистым спросом
агента 1 на продукт 2 понимается превышение спроса первого агента на второй продукт
по сравнению с имеющимся у него начальным запасом этого продукта,
то есть $x^2_1(p,pw_1)-w^2_1>0$.

РИСУНОК  Exchange 5

В данном случае на рынке (чистый) спрос на продукт 2 превышает (чистое) предложение.
Читатель без труда сможет понять, что с первым товаром в данном случае
ситуация обратная -- предложение превышает спрос.
Естественно считать равновесными ценами такие цены, при которых
спрос на каждый товар равнялся бы его предложению. Понятно, что в
рассмотренном нами только что случае цену второго товара следовало бы увеличить,
а цену первого товара уменьшить с тем, чтобы получилась ситуация типа
той, которая изображена на рис.   .

РИСУНОК  Exchange 6

Максимизация полезности требует, чтобы предельные нормы замещения
равнялись относительным ценам (а значит и были бы равными у обоих
агентов), поэтому равновесное распределение -- это общая точка
касания двух кривых безразличия и бюджетной линии (напомним,
что бюджетная линия проходит через точку, соответствующую начальному запасу.

Итак, мы можем дать определение равновесия по Вальрасу для нашей экономики
обмена. Заметим сразу же, что позднее мы рассмотрим модель
Эрроу-Дебре, в которой появится производство и увидим, что модель экономики
обмена является специальным случаем экономики Эрроу-Дебре с весьма
специфическим технологическим множеством (понятие технологического
множества также будет введено ниже).

\textbf{Определение} Равновесием по Вальрасу (или конкурентным равновесием)
в экономике чистого обмена называется такая пара $(p^*,x^*)$,
состоящая из вектора цен $p^*$ и допустимого распределения $x^*$, что
(1) для каждого $i$ набор $x^*_i$ решает задачу максимизации полезности
агента $i$ при ценах $p^*$ и начальном запасе $w_i$, то есть решает задачу
$$
\begin{array}{c}
\max u_i(x)\\
p^*x_i=p^*w_i;
\end{array}
$$
(2) $ \sum^n_{i=1}x_i\leq\sum^n_{i=1}w_i.$


 Соответствующие вектор цен  и распределение называются \emph{равновесными}.

Таким образом, пара "вектор цен--распределение"\, образует равновесие по Вальрасу, если
потребительский набор $x^*_i$ каждого агента $i$  максимизирует его полезность при
данных ценах $p^*$ и спрос на каждый из товаров не превышает его предложения (напомним,
что предложение каждого из товаров определяется его начальным запасом). В действительности,
поскольку каждый потребитель, максимизируя свою полезность, тратит весь свой доход,
то есть $p^*x^*_i=p^*w*_i$ (поскольку мы всегда предполагаем предпочтения агентов
локально ненасыщаемыми), то, как мы увидим ниже, товарами, для которых имеет место
строгое неравенство, могут быть только товары с нулевой ценой.

Поскольку мы предполагаем, что предпочтения агентов возрастающие (неубывающие), то каждая
цена должна быть неотрицательной в любом равновесии. Если бы для какого-то товара
$l$ имело бы место $p_l<0$, то потребитель мог бы "покупать"\, товар $l$, не уменьшая
свою полезность и увеличивая свои ресурсы для покупки других товаров. Из локальной ненасыщаемости
следует, что если бы  цена некоего товара была отрицательной, то не существовало бы
решения задачи максимизации полезности, и мы не могли прийти к "очищению"\, рынка
этого товара.

Кроме того, не может существовать равновесия по Вальрасу, в котором все цены нулевые,
поскольку предпочтения всех потребителей (на самом деле достаточно даже одного потребителя)
локально ненасыщаемы.

Для доказательства существования равновесия нам будет удобно ввести эквивалентное
определение равновесия по Вальрасу, используя понятие функции избыточного спроса.
Мы продемонстрируем формальную эквивалентность, когда будем рассматривать модель
Эрроу-Дебре.

Как мы только что отметили предположение равенства спроса и предложения,
вообще говоря, чересчур сильно. Дело в том, что, например, \emph{нежелательный}
товар вполне может быть в избытке. Приведем формальные определения.


\underline{Равновесие по Вальрасу} -- это такая пара $(p^*,x^*)$,
состоящая из вектора цен $p^*$ и распределения  $x^*$,
что
$$
\sum x_i(p^*,p^*w_i)\le\sum w_i
$$
то есть ни один товар не в избытке (спрос не превышает предложения).

На самом деле, конечно же, для всех \emph{желательных} товаров будет
выполняться равенство (мы приведем формальное определение желательности товара
чуть ниже).


\underline{Существование равновесия по Вальрасу}

Так существует ли равновесный вектор цен? Напомним, что
функция спроса $x_i(p,pw_i)$ положительно однородна степени 0, то есть
$x_i(p,pw_i)=x_i(tp,tpw_i)$ для любого $t>0$.

Определим \emph{агрегированную функцию избыточного спроса}
$$
z(p)=\sum^n_{i=1}(x_i(p,pw_i)-w_i).
$$
Разумеется, функция избыточного спроса зависит и от $w$,
но мы будем считать начальный запас фиксированным, и поэтому указывать
его в качестве аргумента функции $z$ не будем.
Нетрудно понять, что в действительности для каждого $p$
$z(p)$ представляет собой вектор, $j$-ая компонента которого
представляет собой избыточный спрос на $j$-ый продукт. Отрицательность
соответствующей компоненты означает, что спрос меньше предложения
(напомним еще раз, что в нашей модели нет производства, а потому
предложение определяется начальным запасом).
Очевидно, что эта функция также положительно однородна степени 0.

Таким образом, с учетом сделанного замечания о нежелательных продуктах,
в равновесии по Вальрасу должно выполняться неравенство
 $z(p^*)\le 0$.

Если для каждого агента его индивидуальная функция спроса непрерывна,
то $z$ также будет непрерывной функцией.

Теперь мы можем доказать, что имеет место закон Вальраса, который
мы уже упоминали в начале этой главы.

\underline{Закон Вальраса}: для любого вектора цен $p$ имеет место
равенство $pz(p)=0$, то есть $pz(p)\equiv0$.

Доказательство очень просто:
$$
pz(p)=p\left [\sum^n_{i=1}x_i(p,pw_i)-w_i\right ]=\sum^n_{i=1}[px_i(p,pw_i)-p(w_i]=0,
$$
поскольку $x_i(p,pw_i)$ должно удовлетворять бюджетному ограничению.

Доказательство нижеследующего следствия не представляет труда,
поэтому мы оставляем его читателю.

\underline{Следствие}. Если спрос равен предложению на всех
рынках, кроме рынка $j$-ого товара, причем $p_j>0$, то на
 рынке $j$-ого товара спрос также равен предложению.

\underline{Бесплатные товары}. Если $p^*$ - вектор равновесных цен и
$z_j(p^*)<0$, то $p^*_j=0$.

Иными словами, если какой-то товар в равновесии находится в избытке, то
он должен быть бесплатным товаром.

Доказательство. Так как $p^*$ - вектор равновесных цен, то
$z(p^*)\le 0$/. Так как цены $\ge 0$, то
$p^*z(p^*)=\sum^k_{i=1}p^*_iz_i(p^*)\le 0$. Но если бы $z_j(p^*)<0$ и
$p^*_j>0$, то мы имели бы неравенство $p^*z^*(p^*)<0$, что
противоречило бы закону Вальраса.

Будем говорить, что товар \emph{желателен}, если из $p_i=0$ следует
$z_i(p)>0$ для любого $i=1,\ldots,k$.

\underline{Равенство спроса и предложения}. Если все товары желательны, а $p^*$ -
равновесные цены, то $z(p^*)=0$.

Доказательство. Предположим, что $z_i(p^*)<0$ для некоторого $i$. Тогда
из бесплатности следует $p^*_i=0$, а из
желательности мы получаем $z_i^*(p^*)>0$, что противоречит нашему предположению.

Таким образом, все, что требуется для равновесия, это
чтобы не было избыточного спроса ни на один продукт.

\underline{Существование равновесия}. Будем считать, что цены
нормированы (это возможно в силу того, что, как уже отмечалось ранее,
изменение всех цен
в $t>0$ раз не меняет бюджетные множества агентов) так, что
сумма всех цен равна единице. Если это не так, то можно просто
определить цены следующим образом:
$$
p_i={\frac{\hat p_i}{\sum^k_{j=1}\hat p_j}}.
$$
Обозначим через
$$T^{k-1}=\{p\in\R^k_+:\sum p_k=1\}$$
симплекс цен.

Далее нам понадобится следующая теорема о неподвижной точке,
доказательство которой мы приведем для самого простого случая $k=2$.

\underline{Теорема Брауэра о неподвижной точке}. Если функция $f:T^{k-1}\to
T^{k-1}$ непрерывна, то существует неподвижная точка $x\in T^{k-1}: x=f(x)$.

Доказательство (для $k=2$). Пусть $f:[0,1]\to[0,1]$ рассмотрим $g(x)=f(x)-x$.
Неподвижная точка функции $f$ - это такая точка $x^*$, что
$g(x^*)=0$.
Но
$$
\begin{array}{cc}
g(0)=f(0)-0\ge 0&\,\,{\mbox{\rm т.к.}}\,\, f(0)\in [0,1])\\
g(1)=f(1)-1\le 0&\,\,{\mbox{\rm (по тем же причинам)}}
\end{array}
$$
так как $g$  непрерывна, то существует такая точка $x:g(x)=f(x)-x=0$.

\underline{Теорема существования равновесия по Вальрасу}.
Пусть $z:T^{k-1}\to\R^k$ - непрерывная функция, удовлетворяющая
закону Вальраса, то есть $pz(p)\equiv 0$.
Тогда существует $p^*\in T^{k-1}:z(p^*)\le 0$.

Доказательство. Определим $g:T^{k-1}\to T^{k-1}$ следующим образом:
$$
g_i(p)={\frac{p_i+\max (0,z_i(p))}{1+\sum^k_{j=1}\max
(0,z_j(p))}}\quad i=1,\ldots,k.
$$
Ясно, что $g_i$ непрерывна (т.к. $z$ непрерывна и операция взятия
максимума сохраняет непрерывность). Далее $g(p)\in T^{k-1}$,  т.к.
$\sum g_i(p)=1$.

По теореме Брауэра существует неподвижная точка отображения $g$.
Пусть $p^*:p^*=g(p^*)$, тогда
$$
p^*_i={\frac{p^*_i+\max (0,z_i(p^*))}{1+\sum_j\max (0,z_j(p^*))}}\quad i=1,\ldots,k\eqno(1)
$$
Докажем, что это и есть вектор равновесных цен.
Освобождаясь от знаменателя, мы получаем
$$
p^*_i\sum^k_{j=1}\max(0,z_j(p^*))=\max(0,z_i(p^*))\quad i=1,\ldots,k.
$$
Домножим каждое из этих равенств на $z_i(p^*)$:
$$
z_i(p^*)p^*_i\left [\sum^k_{j=1}\max (0,z_j(p^*))\right ]=z_i(p^*)\max (0,z_i(p^*))\quad i=1,\ldots,k.
$$
Просуммируем эти равенства по всем  $i=1,\ldots,k$:
$$
\left [\sum^k_{j=1}\max (0,z_i(p^*))\right ]\sum^k_{i=1}p^*_iz_i(p^*)=\sum^k_{i=1}z_i(p^*)\max(0,z_i(p^*))
$$
Далее, так как по закону Вальраса $\sum^k_{i=1}p^*_iz_i(p^*)=0$,
мы имеем
$$
\sum^k_{i=1}z_i(p^*)\max(0,z_i(p^*))=0
$$

Рассмотрим теперь каждое из слагаемых, стоящих в левой части.
Нетрудно заметить, что каждое слагаемое неотрицательно,
поскольку оно либо равно нулю, если $z_i(p^*)\le 0$ ,
либо равно $(z_i(p^*))^2$, если $z_i(p^*)> 0$.
Поскольку сумма неотрицательных чисел может равняться нулю
только, если каждое слагаемое равно нулю, то мы получаем,
что $z_i(p^*)\le 0$ для всех $i$.
Это и означает, что $p^*$ является вектором равновесных цен.


\underline{Пример 1}. Пусть
\begin{eqnarray*}
u_1(x^1_1,x^2_1)&=&x^1_1-{\frac{1}{8}}(x^2_1)^{-8}\\
u_2(x^1_2,x^2_2)&=&-{\frac{1}{8}}(x^1_2)^{-8}+x^2_2
\end{eqnarray*}
Эти функции полезности квазилинейны, но относительно разных товаров. Начальные запасы
есть $w_1=(2,r)$, $w_2=(r,2)$, $r=2^{8/9}-2^{1/9}$. Кривая "цена-потребление" имеет вид
\begin{eqnarray*}
OC_1(p_1,p_2)&=&\left( 2+r\left( 9{\frac{p_2}{p_1}}\right
)-\left(\frac{p_2}{p_1}\right )^{8/9},
\left ({\frac{p_2}{p_1}}\right )^{-1/9}\right )>0\\
OC_2(p_1,p_2)&=&\left (\left ({\frac{p_1}{p_2}}\right
)^{-1/9},2+r\left ({\frac{p_1}{p_2}}\right )- \left
({\frac{p_1}{p_2}}\right )^{8/9}\right )>0.
\end{eqnarray*}
Для вычисления равновесия достаточно приравнять спрос на товар 2 и его предложение:
$$
\left ({\frac{p_2}{p_1}}\right )^{-1/9}+2+r\left
({\frac{p_1}{p_2}}\right )-\left ( {\frac{p_1}{p_2}}\right
)^{8/9}=2+r
$$
${\frac{p_1}{p_2}}=2$  или 1 и 1/2.

\section*{Две теоремы экономики благосостояния}

Говорят, что распределение $x$ - \emph{слабо  оптимально по Парето}, если
не существует другого допустимого распределения $x'$, которое все агенты
строго предпочитали бы $x$, то есть $x'\succ_i x$, или, в терминах
функций полезности, $u_i(x')>u_i(x)$.

Распределение $x$ - \emph{сильно оптимально по Парето}, если не существует
другого распределения $x'$, которое бы все агенты предпочитали $x'$,
то есть $x'\succcurlyeq_i x$, и при этом хотя бы для одного агента это
предпочтение было строгим, или, в терминах функций полезности,
для всех $i$ $u_i(x') \geq u_i(x)$, причем $u_i(x') \ge u_i(x)$ хотя бы для
одного $i$.

Содержательно это означает следующее. Распределение $x$ является
слабо оптимальным по Парето, если нет другого допустимого распределения,
которое улучшало бы положение сразу всех агентов.

Распределение $x$ является сильно оптимальным по Парето, если нет другого
допустимого распределения, которое бы не ухудшало положение ни одного из агентов,
при этом хотя бы одному агенту становилось бы лучше.

Из этих определений сразу же следует, что сильно оптимальное по Парето
распределение является и слабо оптимальным по Парето. Если же отношения
предпочтений агентов монотонны и непрерывны, то эти два определения
эквивалентны, то есть справедливо следующее утверждение.

Утверждение. Предположим, что предпочтения агентов монотонны и непрерывны.
Тогда  слабо оптимальное по Парето распределение, является сильно
оптимальным по Парето.

Идея доказательства очень проста (формальности читатель легко
может восстановить самостоятельно). Нам нужно проверить, что из
слабой оптимальности по Парето следует сильная оптимальность, или,
что эквивалентно, из того, что распределение $x$ не является
сильно оптимальным по Парето следует, что оно не является и слабо
оптимальным по Парето. Если мы предположим противное, то есть, что
$x$ является слабо оптимальным по Парето, то, значит, найдется
такое распределение $x'$, что какому-то агенту $i$ станет лучше, а
остальным не хуже (по сравнению с $x$). Но тогда мы можем
"отобрать"\, у агента $i$ чуть-чуть каждого из товаров, причем
так, чтобы ему по-прежнему было лучше, чем при распределении $x$.
Это возможно осуществить, в силу непрерывности отношения
предпочтения агента $i$. Теперь, если мы распределим
"отобранное"\, поровну между остальными агентами, то, в силу
монотонности предпочтений, каждому агенту станет лучше. И мы
приходим к противоречию.


Поэтому далее мы будем просто говорить об оптимальных по Парето
распределениях.

На рис.    изображено множество оптимальных по Парето распределений в ящике Эджворта.


EXCHANGE 7

Каждая точка кривой, изображающей множество оптимальных по Парето
распределений, представляет собой точку касания кривых безразличия
первого и второго агентов. Действительно, рассмотрим какую-либо
точку касания, например, $x$. Если мы выберем распределение,
лежащее выше кривой безразличия первого агента, то мы тем самым
ухудшим положение второго агента. Аналогично, выбирая
распределение, лежащее ниже кривой безразличия второго агента, мы
ухудшаем положение первого агента. Наконец, выбрав распределение,
лежащее между этими кривыми безразличия, мы ухудшаем положением
обоих агентов.

Часть оптимальной по Парето кривой, лежащая между кривыми
безразличия первого и второго агента, проходящими через точку $w$,
называется \emph{контрактной
кривой} (см. рис.    ). Такое название связано с тем, что это именно те
точки, в которых, во-первых, совпадают нормы предельного замещения
одного продукта другим для обоих агентов, а во-вторых, это те
точки, в которых полезность каждого из агентов не меньше, чем
первоначальная полезность. Следовательно, это именно те
распределения, которые потенциально могут стать результатом
обмена.

EXCHANGE 8.

С учетом всего сказанного выше мы можем изобразить равновесное по
Вальрасу распределение (см. рис. ... )

EXCHANGE 9.

Обратим внимание на то, что равновесное распределение в ящике Эджворта
оптимально по Парето. Соответствующую теорему для общего случая
экономики обмена мы докажем ниже.

Повторимся еще раз: равновесным распределением будет распределение,
лежащее на контрактной кривой, а значит, являющееся точкой касания
кривых безразличия агентов, причем такой, что их общая касательная
(соответствующая бюджетная линия) проходит через $w$. Стоит заметить,
что, вообще говоря, равновесие не обязано быть единственным;
соответствующая ситуация изображена на рис.   . Мы коснемся вопроса
единственности равновесия позже.

EXCHANGE 10.

На рис.     изображен случай углового равновесия.

EXCHANGE 10.1



На рис.    изображена ситуация, в которой равновесие отсутствует.

EXCHANGE 11.

Здесь у первого потребителя сосредоточен весь запас товара 2, а у второго --
весь запас товара 1. Наклон кривой безразличия первого агента в точке $w$
бесконечен. Предпочтения второго агента таковы, что он хочет только товар 1.
Тогда не существует равновесных цен $p^*$: действительно, если $p_2/p_1>0$,
то потребитель 2 оптимально хочет иметь
свой начальный запас $w_2$, тогда как $w_1$ никогда не оптимален для 1-ого.
С другой стороны, спрос 1 на товар 2 будет бесконечен при $p_2/p_1=0$.
Здесь отсутствие равновесия связано с тем, что предпочтения второго агента
не строго монотонны.


EXCHANGE 11.1

На этом рисунке отсутствие равновесие связано с невыпуклостью предпочтений
первого агента.

Теперь мы можем перейти к анализу благосостояния в модели
экономики обмена, то есть рассмотреть связь равновесия по Вальрасу
и оптимальности по Парето.

{\bf Первая теорема экономики благосостояния}. Если $(p^*,x^*)$ - равновесие по Вальрасу, то
распределение $x^*$ оптимально по Парето.

Доказательство. Предположим противное, и пусть $x'$ - допустимое
распределение, которое все агенты предпочитают распределению $x^*$.
Поскольку $x'$ - допустимое распределение, то
$$ \sum^n_{i=1}x'_i\leq\sum^n_{i=1}w_i.$$
Так как в равновесии цены неотрицательны, то
$$ \sum^n_{i=1}px'_i\leq\sum^n_{i=1}pw_i.$$
Поскольку предпочтения локально ненасыщаемы и имеет место закон
Вальраса, то
$$ \sum^n_{i=1}px^*_i=\sum^n_{i=1}pw_i.$$

По предположению, каждый агент $i$ предпочитает (не обязательно строго) $x'_i$.
Значит, $px'_i \geq px^*_i= pw_i$, так как если бы выполнялось неравенство
$px'_i \le px^*_i$, то, в силу локальной ненасыщаемости, агент $i$ мог бы
позволить себе что-то, строго лучшее, нежели $x'_i$, что, в свою
очередь, было бы строго лучше, чем $x^*_i$, что противоречило бы
оптимальности $x^*_i$ для агента $i$ при ценах $p$ .
Аналогично, для любого агента $i$, строго предпочитающего $x'_i$ набору
$x^*_i$, мы должны иметь $px'_i>px^*_i=pw_i$. Это неравенство должно
выполняться хотя бы для одного агента $i$, так как мы предположили, что
$x'$ доминирует (по Парето) $x^*$. Суммируя теперь эти неравенства, мы получаем
$$ \sum^n_{i=1}px'_i>\sum^n_{i=1}px^*_i.$$
И мы приходим к противоречию.

Если мы обратимся к ящику Эджворта, то можно заметить, что если мы возьмем
произвольное оптимальное по Парето распределение, то соответствующая
общая касательная вовсе не обязана проходить через $w$. Но если нам предварительно
удастся соответствующим образом переместить точку $w$ (например, перераспределив
начальный запас), то общая касательная
уже может пройти через точку, соответствующую "новому"\, начальному запасу.

EXCHANGE 11.2

На рис.   $x^*$ некоторое оптимальное по Парето распределение. Общая касательная
не проходит через начальный запас $w$. Если мы перераспределим запас первого продукта
(в данном случае, увеличив запас первого продукта у первого агента), то получим "новый"\,
начальный запас $w'$, через который общая касательная уже будет проходить. Тем самым
мы получаем, что $x^*$ становится равновесным распределением с начальным запасом $w'$.
Заметим, что если возможно перераспределение любого из товаров, то можно перераспределить
$w$ так, чтобы новый начальный запас стал именно $x^*$. В этом случае в равновесии
по Вальрасу обмен отсутствовал бы вовсе.

Отметим еще один важный момент. Общая касательная определяет
цены $p^*$, а тогда, скажем, налоговый орган может ввести налоги (или субсидии)
так, чтобы бюджетная линия агентов сдвинулась так, чтобы проходить как раз
через точку $x^*$. (См. рис.    .)

EXCHANGE 11.3

В этом случае говорят о \emph{равновесии с трансфертами}. Формально такое равновесие
определяется (для ящика Эджворта следующим образом).

Распределение $x^*$ в ящике Эджворта называется равновесным с
трансфертами, если существуют такие цены $p^*$ и такие трансферты
$T_1, T_2$, что $T_1+T_2=0$ и $x^*_i\succeq_ix_i$ для любых
$x_i \in \mathbb{R}^2_+$, таких что $p^*x_i\leq p^*w_i+T_i$.

На самом деле соответствующий результат имеет место и в общем случае.

{\bf Вторая теорема экономики благосостояния}. Предположим, что предпочтения агентов
выпуклы, непрерывны и локально ненасыщаемы. Пусть $x^*$ -- оптимальное по Парето
распределение начального запаса, причем все компоненты $x^*$ строго положительны.
Тогда $x^*$ является равновесным распределением, если начальный запас предварительно
соответствующим образом перераспределен между агентами.

Мы приведем здесь лишь схему доказательства. Предположим, что $x^*$ оптимальное по Парето
распределение начального запаса $w$, и мы хотим, чтобы $x^*$ было равновесным распределением.
Положим $w^*=\sum^n_{i=1}x^*_i$. Заметим, что $w^*\leq w$, причем строгое неравенство возможно
для некоторых компонент.
Определим множество $X^*\subset\mathbb{R}^{k}$ как множество таких наборов товаров, которые
могут быть распределены между всеми агентами так, чтобы получившиеся распределения
строго доминировали $x^*$ по Парето. Формально, положим
$$X_i=\{x_i\in\mathbb{R}{k}: x_i\succ_ix^*_i\},$$
то есть это множество тех потребительских наборов, которые агент $i$ предпочитает
набору $x^*_i$. Тогда
$$X^*=\sum^n_{i=1}X_i=\{z: z=\sum^n_{i=1}z_i, z_i\in X_i\}.$$

Можно доказать, что из выпуклости предпочтений
следует выпуклость множества $X^*$.

Пусть теперь $X\prime=\{x\in\mathbb{R}{k}: x\leq w$. Очевидно, что это множество тоже выпукло.

Нетрудно проверить, что множества $X^*$ и $X\prime$ не пересекаются. Тогда, по теореме отделимости
(см. раздел        ниже), существует такой вектор $p_1, \ldots , p_k$ и число $c$, что
$px\leq c$ для любого $x\in X\prime$ и $px\geq c$ для любого $x\in X^*$.

Можно показать, что, в силу локальной ненасыщаемости и непрерывности, $w^*$, который принадлежит
$X\prime$, лежит на границе $X^*$ и поэтому $pw^*=c$. Нетрудно проверить, что $p\geq0$ (это следует
из вида множества $X\prime$. Кроме того $pw=c$: поскольку предпочтения неубывающие, мы можем
взять каждый из товаров, оставшийся (от начального запаса) в распределении $x^*$, распределить
его между потребителями и снова воспользоваться локальной ненасыщаемостью и непрерывностью.
Таким образом, если $w^{*k}<w^k$, то $p_k=0$.

Пусть $w^*_i$ -- произвольное распределение начального запаса $w$, такое что
$pw^*_i=px^*_i$. Такое распределение существует, поскольку товары, не полностью
использованные в распределении $x^*$, должны иметь нулевую цену.

Оказывается, что пара $p, x^*$ образует равновесие по Вальрасу, если начальный
запас потребителей будет $w^*$. Чтобы убедиться в этом, предположим, что это не равновесие.
Разобьем множество потребителей на два множества: тех, для которых $x^*_i$ является лучшим,
чего они могут достичь при ценах $p$ и начальном запасе $w^*_i$, и тех, которые могут достичь
чего-то лучшего. Поскольку $p, x^*$, согласно нашему предположению, не является равновесием
по Вальрасу, то второе множество в нашем разбиении непусто.

Для каждого агента $i$ из второго множества обозначим через $x_i$ набор, который
агент $i$ строго предпочитает набору $x^*_i$ и который достижим для него при ценах
$p$ и начальном запас $w^*_i$. В силу непрерывности предпочтений существует
такое положительное $t_i<1$, что $t_ix_i$ строго предпочтительнее $x^*_i$ и
удовлетворяет неравенствам $p\cdot t_ix_i<px_i\leq pw^*_i$. (Здесь мы неявно предполагали,
что $px_i>0$. Если же  $px_i=0$, то, поскольку $x^*_i$ строго положителен,
$pw^*_i=px^*_i>0$. Но тогда $px_i<pw^*_i$, что собственно нам и нужно).

Определим $x\prime=t_ix_i$. Положим $y=\sum p\cdot(w^*_i-x\prime_i)$, где сумма берется по
второму множеству потребителей. По построению $y>0$. Используя этот "излишек запаса"\, и
локальную ненасыщаемость предпочтений, можно найти такие наборы $x\prime_i$ для потребителей
$i$ из первой группы, что $x\prime_i$ будет строго предпочитаться набору $x^*_i$ и при этом
$\sum^n_{i=1}px\prime_i<\sum^n_{i=1}px^*_i=c$. Но поскольку $x\prime$ строго доминирует по
Парето $x^*$, то $\sum^n_{i=1}px\prime_i \in X^*.$ Однако для всех $px\geq c$ для всех $x \in X^*$.
Полученное противоречие и доказывает теорему.

На рис.      изображена ситуация, демонстрирующая существенность
предположения выпуклости предпочтений. $x^*$ -- оптимальное по Парето
распределение. Но на бюджетной линии, на которой второй агент потребляет
$x^*_2$ первый предпочтет другую точку, скажем $x\prime_1$.

EXCHANGE 11.4

\section*{Оптимальность по Парето }

В предыдущем разделе мы убедились, что в любое равновесное распределение является
оптимальным по Парето и любое оптимальное по Парето распределение является
равновесным распределением для некоторого начального запаса. Теперь обратимся к
более подробному рассмотрению этих свойств с помощью условий первого порядка.

\underline{Предложение}. Если $(p^*,x^*)$ - равновесие, причем каждый потребитель имеет
положительное количество каждого товара, тогда существуют такие числа
$(\lambda_1,\ldots,\lambda_n)$, что имеет место равенство:
$$
\nabla u_i(x^*)=\lambda_ip^*,\,\,\,\quad i=1,\ldots,n
$$

Доказательство. Если мы имеем равновесие, тогда каждый агент
максимизирует свою функцию полезности на своем бюджетном  множестве,
а это как раз и есть условие
первого порядка такой максимизации полезности. Более того,
$\lambda_i$ - предельная полезность дохода. (Напомним, что когда мы доказывали
тождество Руа
$$
x_i(p,m)=-{\frac{\partial v(p,m)/\partial p_i}{\partial v(p,m)/\partial m}},
$$
где $v(p,m)=u(x(p,m))$, мы как раз и выяснили, что $\lambda={\frac{\partial v(p,m)}{\partial m}}$).

\underline{Характеризация оптимального по Парето  распределения}.
Допустимое распределение $x^*$ - оптимально по Парето  тогда и
только тогда, когда $x^*$ решает следующие $n$ задач максимизации:
$$
\begin{array}{c}
\max_{(x^l_i)}u_i(x_i)\\
\sum^n_{i=1}x^l_i\le w^l\quad l=1,\ldots,k\\
u_j(x^*_j)\le u_j(x_j)\quad j\ne i
\end{array}
$$
Доказательство. Предположим $x^*$ - решение, но $x^*$ -  не
оптимально по Парето . Это означает, что существует такое распределение $x'$,
которое лучше, чем $x^*$, для всех агентов. Но это значит, что
$x^*$ не решает ни одной из указанных задач.

Обратно, предположим,что $x^*$ оптимально по Парето, но
не решает одну из задач. Пусть $x'$ - решает эту частную задачу.
Тогда $x'$ улучшает положение одного из агентов, не ухудшая при этом
положения остальных агентов. Это противоречит оптимальности по Парето распределения $x^*$.

Посмотрим подробнее на Лагранжиан приведенной только что задачи максимизации для некоторого $i$:
$$
L=u_i(x_i)-\sum^k_{l=1}q^l\left [\sum^n_{i=1}x^l_i-w^l\right ]-\sum_{j\ne i}a_j(u_j(x^*_j)-u_j(x_j)).
$$
Продифференцируем $L$ по $x^l_j$, где $l=1, \ldots,k,\,j=1,\ldots,n$. Мы получаем условия первого порядка
\begin{eqnarray*}
{\frac{\partial u_i(x_i)}{\partial x^l_i}}-q^l=0&l=1,\ldots,k\\
a_j{\frac{\partial u_j(x^*_j)}{\partial x^l_j}}-q^l=0&j\ne i,\,\,l=1,\ldots,k.
\end{eqnarray*}
На первый взгляд они выглядят странно -- отсутствует симметричность:
выбирая разные $i$, мы получаем различные значения $q^l$ и $a_j$. Однако эта
странность разрешается достаточно просто: заметим, что относительные
значения $q$ независимы от выбора $i$, поскольку из приведенных условий следует, что
$$
{\frac{\partial u_i(x^*_l)/\partial x^l_i}{\partial u_i(x^*_i)/\partial x^h_i}}=
{\frac{q^l}{q^h}}\,\,{\mbox{\rm для}}\,\,i=1,\ldots,n\,\,{\mbox{\rm и}}\,\,
l,h=1,\ldots,k
$$
Так как распределение $x^*$ задано, то $q^l/q^h$ должно быть независимо от того,
какую задачу максимизации мы решаем. Аналогично мы получаем, что $a_i/a_j$ тоже не
зависит от того, какую задачу максимизации мы решаем. Таким образом,
если мы максимизируем полезность $i$-ого агента и используем
полезности других агентов, как ограничения, то это то же самое, как
если бы мы произвольно устанавливали множитель Куна-Таккера $i$-ого
агента $a_i=1$.
Теперь, не вдаваясь в особые подробности (это следует из первой
теоремы экономики благосостояния), мы получаем, что если $x^*$ - равновесное распределение, то
$$
\nabla u_i(x^*_i)=\lambda_ip,\,\,i=1,\ldots,n.
$$
Так как все равновесные распределения оптимальны по Парето, то
$$
a_i\nabla u_i(x^*_i)=q,\,\,i=1,\ldots,n.
$$
Следовательно, мы можем выбрать $p=q$  и $a_i=1/\lambda_i$. Таким образом, множители
Куна-Таккера ресурсных ограничений -- это конкурентные цены, а
множители Куна-Таккера полезностей агентов - это величины, обратные их
предельным полезностям дохода.

Если мы теперь исключим множители Куна-Таккера из условий первого порядка, то получим
$$
{\frac{\partial u_i(x^*_i)/\partial x^l_i}{\partial u_i(x^*_i)/\partial x^h_i}}=
{\frac{p_l}{p_h}}={\frac{q^l}{q^h}},\quad i=1,\ldots,n,\,\,l,h=1,\ldots,k.
$$
Иными словами, для любого оптимального по Парето распределения
предельная норма замещения между любыми парами
товаров одинакова для всех агентов. Эта предельная норма замещения есть отношение
конкурирентных цен. В этом нет ничего удивительного: если бы у двух агентов были
различные нормы замещения для какой-то пары товаров, то они могли бы "устроить небольшой обмен",\,
который бы улучшил положение обоих, а это противоречило бы оптимальности по Парето.

Здесь стоит заметить также, что условия первого порядка оптимальности по
Парето, те же самые, что и условия первого порядка для максимизации
взвешенной суммы полезностей агентов. Действительно, рассмотрим задачу
$$
\begin{array}{c}
\max\sum^n_{i=1}a_iu_i(x_i)\\
\sum^n_{i=1}x^l_i\le w^l,\quad l=1,\ldots,k
\end{array}
$$
Легко убедиться в том, что условия первого порядка для этой задачи
есть $a_i\nabla u_i(x^*_i)=q$.

В заключение мы приведем (без доказательства) теорему, в некотором смысле
подводящую итоги нашему обсуждению оптимальности по Парето.

\underline{Оптимальность по Парето и максимизация благосостояния}.
Пусть $x^*$ -- оптимальное по Парето  распределение и $x^*_i \ne 0$
для $i=1,\ldots,n$. Пусть $u_i$ - вогнута, непрерывна и монотонна,
тогда существуют такие веса $a^*_i$, что $x^*$ максимизирует cумму
$a^*_iu_i(x_i)$ при ресурсных ограничениях. Более того эти веса
таковы, что $a^*_i=1/\lambda^*_i$, где $\lambda^*_i$ - предельная
полезность дохода $i$-ого агента, то есть если $m_i$ - стоимость
начального распределения агента при равновесных ценах $p^*$, то
$$
\lambda^*_i={\frac{\partial v_i(p^*,m_i)}{\partial m_i}}
$$

\section*{Ядро экономики обмена}

Сейчас мы рассмотрим экономику обмена с совершенно других позиций и остановимся
на понятии \emph{ядра экономики обмена}. Как и раньше, мы считаем, что $i$-ый агент имеет
начальный запас $w_i$. Однако теперь представим себе, что вместо
механизма цен мы пытаемся использовать следующую схему: агенты "бродят"\,
по рынку и делают попытки договориться друг с другом об обмене товарами. Когда агенты
достигают наилучших (для себя) соглашений, они прекращают "торговлю".
Каким может быть исход, устраивающий всех агентов.

Предположим, что дано допустимое распределение $x$. Будем говорить, что группа
агентов $S$ (далее мы будем говорить о \emph{коалиции} $S$) может улучшить распределение $x$,
если существует такое распределение $x'$, что, во-первых,
$$\sum_{i\in S}x'_i=\sum_{i\in S}w_i$$
и, во-вторых,
$$
x'_i\succ_i x_i\quad\forall i\in S.
$$
В этом случае говорят также, что коалиция $S$ \emph{блокирует} распределение $x$.
Содержательно эти условия означают следующее: первое говорит о том, что
агенты из коалиции $S$ распределяют между собой
\emph{имеющийся у них} начальный запас; второе говорит, что при этом
положение каждого агента из этой коалиции станет лучше по сравнению с
тем, что ему "предлагает"\, распределение $x$.

Теперь мы можем определить ядро экономики. Будем говорить, что допустимое распределение $x$
принадлежит \emph{ядру экономики $\varepsilon$}  ${\cal C}(\varepsilon )$, если оно не
может быть блокировано ни одной коалицией.

Ясно, что если $x\in {\cal
C}(\varepsilon )$, то $x$ является оптимальным по Парето распределением
(поскольку оно не может быть блокировано большой коалицией, то есть коалицией,
состоящей из всех агентов). Столь же очевидно, что распнределение из ядра
является индивидуально рациональным, то есть для каждого агента оно не хуже, чем
его начальный запас (поскольку не блокируется коалицией, состоящей из одного агента).

Из сказанного сразу же следует, что для ящика Эджворта ядро совпадает с контрактной кривой
(см. рис.    ).

EXCHANGE 11.5

{\bf Предложение}. Пучть $(p^*,x^*)$ -  равновесие по Вальрасу с
начальным распределением $w$, тогда $x^*\in {\cal C}(\varepsilon )$.

Доказательство. Предположим, что это не так. Значит, существуют такая коалиция $S$
и $x'$, что $x'_i\succ_i x^*_i$ и
$$
\sum^n_{i\in S}x'_i=\sum_{i\in S}w_i.
$$
Но тогда $px'_i>pw_i$ для любых $x\in S$ и
$$
p\sum_{i\in S}x'_i>p\sum_{i\in S}w_i,
$$
что противоречит вышеприведенному равенству.

На самом деле ядро может быть достаточно большим, что видно, например, на рис.   .
Из этого же рисунка и рисунка EXCHANGE 9 видно также, что помимо равновесных
распределения ядро может содержать и другие распределения. Обратим внимание на то, что
в случае двух агентов (как в ящике Эджворта) число возможных коалиций очень мало -- их
всего три: две одноэлементные коалиции, содержащие первого и, соответственно, второго
агента, а также большая коалиция. Следовательно и ограничений на блокирование тоже всего три
(индивидуальная рациональность для каждого из агентов и оптимальность по Парето.
С ростом числа агентов число коалиций быстро растет, и, соответственно, растет число
ограничений на блокирование: в случае трех агентов возможных коалиций уже семь, в случае
четырех агентов -- пятнадцать и т.д. Поэтому с ростом числа агентов ядро экономики в некотором
смысле сужается (мы далее уточним, в каком именно смысле
что это означает).

Замечательный результат, который мы не доказываем (доказательство выходит за рамки
этой книги), но считаем необходимым
упомянуть, состоит в том, что если число игроков становится очень большим (в действительности
становится континуумом), то множество равновесных по Вальрасу распределений
совпадает с ядром экономики. Этот результат замечателен тем, что он показывает,
что два принципиально разных механизма, приводят к одному и тому же результату.
А именно, равновесие по Вальрасу -- это пример сугубо эгоистичного поведения агентов,
когда каждый агент заботится только о максимизации своей собственной полезности.
Напротив, ядро экономики -- это сугубо кооперативный механизм, учитывающий
возможности образования всевозможных коалиций агентов. В первом механизме
существенную роль играют цены, в то время, как во втором, они полностью исключены
из рассмотрения.

Рассмотрим специальный случай "роста"\, экономики. Будем говорить, что два агента
имеют один и тот же тип, если у них одинаковые предпочтения и одинаковые начальные
распределения. Будем говорить, что одна экономика является \emph{репликой} другой экономики, если в первой
экономике  в $r$ раз больше агентов каждого типа, чем в первой. Будем для простоты
считать, что у нас есть только 2 типа агентов -- агенты типа А и типа В, и рассмотрим
некоторую экономику с двумя агентами. $r$-ядром этой экономики будем
называть ядро $r$-реплики исходной экономики.

Достаточно, наверное, естественным было бы ожидать, что агенты одного типа должны
получать одинаковые наборы.

{\bf Предложение}. Предположим, что предпочтения агнетов строго выпуклы, строго монотонны и
непрерывны. Тогда, если $x$ -- распределение в $r$-ядре данной экономики, то
у двух агентов одного и того же типа должны быть одинаковые наборы товаров.

Доказательство. Пусть $x$ - распределение в $r$-ядре, и обозначим
$2r$ агентов соответствующими индексами $A1,A2,\ldots,Ar$  и
$B1,B2,\ldots,Br$. Если агенты одного и того же типа имеют
разные потребительские наборы, то среди них будут наиболее "бедные".
Назовем их, соответственно, "самый несчастный типа А"\, и "самый
несчастный типа В",\, (если их несколько, то выберем какого-то из них).
Рассмотрим средний набор агентов типа А и типа В: $\bar
x_A={\frac{1}{r}}\sum^r_{j=1}x_{A_j}$ и $\bar
x_B={\frac{1}{r}}\sum^r_{j=1}x_{B_j}$.

Так как распределение $x$ допустимо, то
${\frac{1}{r}}\sum^r_{j=1}x_{A_j}+{\frac{1}{r}}\sum^r_{j=1}x_{B_j}=
{\frac{1}{r}}\sum^r_{j=1}w_{A_j}+{\frac{1}{r}}\sum^r_{j=1}w_{B_j}=
{\frac{1}{r}}rw_A+{\frac{1}{r}}rw_B$ $\bar x_A+\bar x_B=w_A+w_B.$
Следовательно, распределение
$(\bar x_a,\bar x_B)$ допустимо для коалиции, состоящей из двух самых
несчастных типа А и В.
Мы предполагаем, что по крайней мере для одного типа, скажем, А, два агента этого типа
располагают различными наборами товаров. Это значит, что самый несчастный типа А, строго
предпочитает $\bar x_A$ имеющемуся у него набору (в силу строгой выпуклости предпочтений),
а самый несчастный типа B считает, что $\bar x_B$ не хуже имеющегося у него набора.

Строгая монотонность и непрерывность позволяет агенту A отдать немного из набора
$\bar x_A$ агенту B и тем самым образовать коалицию, которая может блокировать
исходное распределение.

Следующая теорема о сжимающемся ядре, которую мы приводим без доказательства,
показывает, что неравновесное распределение
не попадает в $r$-ядро экономики при достаточно большом $r$.

\textbf{Теорема о сжимающемся ядре.} Предположим, что предпочтения агентов строго выпуклы,
сильно монотонны и существует равновесное распределение $x^*$ с начальным запасом $w$.
Тогда, если $y$ не является равновесием, то найдется такое $r$, что $y$ не лежит в
$r$-ядре экономики.


\section{Технология}

Прежде чем перейти собственно к модели Эрроу-Дебре, нам придется
сделать отступление для того, чтобы при описании поведения фирм
мы могли использовать \emph{технологические множества}.

Заметим, что ранее, считая, что фирма выпускает один единственный
продукт, мы всегда имели дело с производственной функцией. Теперь
же мы должны учитывать возможность производства нескольких
продуктов. Поэтому сейчас мы введем понятие технологического
множества, рассмотрим некоторые свойства технологических множеств,
которые обычно считается разумными, а также рассмотрим задачу максимизации,
прибыли, поскольку в равновесии, о которым пойдет речь, потребители,
как всегда, будут максимизировать свою полезность, а фирмы -- максимизировать
прибыль.

Итак обратимся к рассмотрению поведения производителя (фирмы).
Здесь важно отметить очень существенный момент: нас не будет
интересовать фирма как организация с ее внутренней, подчас весьма
сложной структурой, и, возможно, очень большим числом работников,
каждый из которых может иметь свои интересы и цели. Это
самостоятельная, крайне интересная задача, но она выходит
за рамки этой книги. Нас же фирма будет
интересовать лишь как некий "черный ящик", в который попадают
факторы производства (ресурсы), а на выходе -- готовая продукция.
Поэтому "процесс производства"\, в таких предположениях принято
описывать с помощью \emph{призводственных планов} и \emph{технологических
множеств}, при этом в случае однопродуктового выпуска по
технологическому множеству тривиальным образом определяется
производственная функция, и обратно.


\emph{Производственный план}
--- это вектор $y=(y_1,\ldots,y_k)\in\R^k$, кратко описывающий некоторый
технологический процессе. При этом положительные компоненты
соответствуют выпуску, а отрицательные компоненты -- соответствуют
факторам производства (ресурсам). Так, скажем вектор (-7,4,-3,-1,6) в этом случае
интерпретируется как производственный план, при котором из 7 единиц
первого продукта, трех единиц третьего продукта и одной единицы четвертого
продукта производится (или может быть произведено) четыре единицы второго
и 6 единиц пятого продуктов.
Множество $Y\subset\R^k$ всех производственных планов, которые возможны для
данной фирмы, называется \emph{технологическим множеством} или
\emph{множеством производственных возможностей}.
Любой $y\in Y$ -- возможный производственный план,
$y\notin Y$ -- "невозможный"\, производственный план. Множество $Y$
определяется технологическими возможностями фирмы.

Разумеется, в реальных производственных процессах множество
продуктов, которые выпускаются, отличается от продуктов,
являющихся факторами производства. Поэтому часто бывает удобным
разделить выпускаемые продукты и факторы производства: например,
можно считать, что первые $m$ продуктов соответствуют выпуску, а остальные
являются факторами производства, то есть
$q=(q_1,\ldots,q_m)\ge 0$ -- уровни выпуска продуктов фирмы, а
$z=(z_1,\ldots,z_{k-m})\ge 0$ -- набор $k-m$ факторов
производства. В этом случае соответствующий производственный
план $y$ будет выглядеть следующим образом:
$y=(q,-z)$.

До сих пор мы рассматривали модели с однопродуктовым выпуском.
В этом случае технологию удобнее всего описывать с помощью
производственной функции $f(z)$, определяющей максимальное
количество $q$ (единственного) готового продукта, которое можно
произвести из набора факторов производства
$(z_1,\ldots,z_{k-1})\ge 0$. Например, если выпускается продукт
$k$, то производственная функция $f(\cdot )$ определяет
технологическое множество очевидным:
$$
Y=\{(-z_1,\ldots,-z_{k-1},q):q-f(z_1,\ldots,z_{k-1})\le 0\,\,{\rm
\text{и}}\,\, (z_1,\ldots,z_{k-1})\ge 0\}.
$$

\underline{Свойства технологических множеств}

Естественно, что не следует ожидать, что технологические множества
могут быть произвольными. Поэтому, как правило, предполагается,
что технологические множества обладают некоторыми свойствами,
которые мы перечислим ниже. При этом мы остановимся лишь на тех свойствах,
которые будут полезны нам в дальнейшем.
\begin{itemize}
\item[(1)] $Y\ne \emptyset$. Содержательно это свойство означает,
что фирма может что-то производить, то есть это свойство означает,
по сути дела, что фирма есть.
\item[(2)] $Y$ замкнуто. Это в основном техническое требование, без
которого, скажем, возникают проблемы решения задач максимизации
прибыли, о котором нам предстоит говорить далее.
\item[(3)] Невозможность "рога изобилия": если  $y\in Y$, $y\ge 0$, то $y=0$.
То есть из ничего нельзя произвести ничего.  Геометрически
$Y\cap\R^k_+\subset\{0\}$.

EXCHANGE 12

На рис.   (а) технологическое множество не удовлетворяет этому свойству,
поскольку, скажем, производственный план $y$ таков, что из ничего
производится положительное количество второго продукта. Технологическое
множество на рис.   (b) требуемым свойством обладает.

\item[(4)] Возможность беззатратной остановки производства, т.е. $0\in Y$.
Это означает, что фирма может ничего не производить, но и ничего
не тратить.

На рис.  (b) такая остановка производства возможна: поскольку $0\in Y$, то
это как раз и означает, что можно ничего ничего не производить и
ничего при этом не тратить.
На рис.   обозначена ситуация, когда беззатратная остановка производства
невозможна: даже если ничего не производится, необходимо затратить
$|y^0_1|$ единиц первого продукта (скажем, если фирма обязана
по контракту закупить такой объем первого продукта). Если, например,
считать, что первый продукт --
это деньги (что бывает часто весьма удобным,) то в ситуации на этом рисунке
остановка производства потребует затрат в размере $|y^0_1|$ .

EXCHANGE 13

\item[(5)] \underline{Технологическое множество обладает свойством свободного
расходования}. Это означает формально, что если $y\in Y$, $y'\le
y$, то $y'\in Y$, или $Y-\textbf{R}^k_+$. Поскольку факторы производства входят
в производственный
план со знаком минус, а выпуск -- со знаком плюс, то это означает, что из большего
объема факторов производства можно выпустить меньше готовой
продукции. То есть дополнительное количество факторов производства
(или выпуска) можно беззатратно уничтожить.

EXCHANGE 14

На рис.   в производственном плане $y'$ используется больше (по
сравнению с производственным планом $y$) первого фактора
производства и производится меньше готового второго продукта. Что
касается производственного плана $y''$, то здесь все координаты
отрицательны. Такой производственный план представляет собой
ситуацию, при которой нет выпуска, а есть только затраты. По сути
дела речь идет просто об уничтожении набора товаров. То есть
экономика может уничтожать любые количества товаров.

Безусловно, такое предположение очень сильно. Оно позволяет сразу
исключать нежелательные товары: загрязнение, отходы и т.п. как бы
просто уничтожаются. Это позволяет ограничиться неотрицательными
ценами. Отрицательные цены -- это как бы компенсация за вредность,
а если нет вредных товаров или их можно просто (даром) уничтожать,
то нет и оснований для компенсации.

Хотя такая интерпретация может показаться не вполне естественной,
она в первую очередь очень удобна технически, и именно так к ней и
можно относится. Дело в том, что мы всегда предполагаем цены
положительными (или уж во всяком случае неотрицательными). Поэтому
в ключевой для нас задаче -- задаче максимизации прибыли, которая
математически представляет собой задачу максимизации на
технологическом множестве $Y$ скалярного произведения $py$ при
заданном $p\geq0$, максимум не может достигаться на точках типа
$y'$ или $y''$. То, что $py=p_1y_1+p_2y_2+\ldots+p_ky_k$
представляет собой прибыль, которую приносит производственный план
$y$ при ценах $p$ следует из того, что в производственном плане
положительные компоненты соответствуют объемам выпускаемой
продукции, а значит, соответствующая часть этой суммы с
положительными $y_i$ показывает выручку, а часть суммы с
отрицательными $y_i$  соответствует затратам. Поэтому $py$ есть
разность выручки и затрат, то есть прибыль.


\item[(6)] \underline{Необратимость}. Если $y\in Y$ и $y\ne 0$, то $-y\notin Y$.
Производство необратимо. Из выпускаемых продуктов нельзя сделать
факторы производства и наоборот.
\item[(7)] Невозрастающая (убывающая) отдача от масштаба. Если для любого $y\in Y$, $\alpha y\in Y$
для всех $\alpha\in [0,1]$, то есть можно уменьшать масштабы производства.

EXCHANGE 15.


\item[(8)] Неубывающая (возрастающая) отдача от масштаба. Если для любого $y\in Y$,
$\alpha y\in Y$ для любого $\alpha\ge 1$. Иными словами, можно увеличивать
масштабы производства.

EXCHANGE 16.

На приведенном рисунке обе технологии обладают свойством неубывающей отдачи от
масштаба, но в первом случае возникают так называемые затраты на установку:
если считать, что первый продукт -- это деньги, то для начала производства
необходимо затратить определенную сумму.


\item[(9)] Постоянная отдача от масштаба. Если $y\in Y$, то $\alpha y\in Y$, для
любого $\alpha\ge 0$. В этом случае технологическое множество $Y$ является конусом.

EXCHANGE 17.

\end{itemize}

На самом деле получить из технологического множества  с
однопродуктовым выпуском производственную функцию очень просто.
Для этого достаточно технологическое множество отразить
симметрично относительно оси, соответствующей выпускаемому
(единственному) продукту.

EXCHANGE 18.

Пример. Рассмотрим производственную функцию типа Кобба-Дугласа.
Мы имеем

$$f(z_1,z_2)=z_1^\alpha z_2^\beta,\, f(tz_1,tz_2)=t^{{\alpha
+\beta}}z_1^\alpha z_2^\beta.$$

Тогда, если $\alpha +\beta=1$, то это технология с постоянной отдачей.
Если $\alpha +\beta>1$, то имеет место возрастающая отдача.
Наконец, если $\alpha +\beta<1$, то  отдача убывающая.

\begin{itemize}
\item[(10)] Выпуклость $Y$. Очень часто используемое свойство технологических множеств.

Если $y,y'\in Y$, $\alpha\in [0,1]$, то $\alpha y+(1-\alpha) y'\in
Y$. Выпуклость воплощает в себе сразу два важных момента:

а) невозрастающая отдача от масштаба. Действительно, если $0\in
Y$, то для любого $\alpha\in [0,1]$ $\alpha y=\alpha y+(1-\alpha
)0$. Следовательно из $y\in Y$, $0\in Y$ и выпуклости следует
$\alpha y\in Y$.

б) "несбалансированная"\, комбинация факторов производства не более
продуктивна, чем сбалансированная (или,
несбалансированный выпуск не менее затратен, чем
сбалансированный). В частности, если производственные планы $y$  и
$y'$ таковы, что производятся один и тот же выпуск, но используются разные комбинации
факторов производства, то производственный план, используюя
"средний"\, набор, можно произвести по крайней мере столько же.

Для однопродуктового выпуска выпуклость $Y$ означает, что
производственная функция $f(z)$ вогнута.

\end{itemize}

Следует особо подчеркнуть, что в дальнейшем практиче5ски всегда
речь будет идти о выпуклых (и замкнутых) технологиях, поскольку только в этом
случае можно гарантировать существование решения задачи максимизации
прибыли. Подробнее об этом мы будем говорить ниже.

Приведем еще несколько определений. Мы помним, что производственная
функция показывает максимальный объем (единственного) готового продукта,
который можно произвести из данного набора факторов производства (ресурсов).
Если же мы имеем дело с технологическим множеством, то
можно определить $n$-мерный аналог производственной функции в
терминах эффективности. А именно,
производственный план $y\in Y$ называется (технологически)
\emph{эффективным}, если не существует другого
производственного плана $y'\in Y$, такого что $y'\ge y$,
то есть нельзя, взяв меньше хотя бы одного фактора производства
произвести столько же готовой продукции, либо из тех же объемов
факторов производства произвести больше хотя бы одного из
выпускаемых продуктов, не сокращая при этом выпуск других
продуктов. Стоит обратить внимание на то, что эффективность является
технологическим аналогом оптимальности по Парето для распределений.


\section*{Отступление: выпуклые множества и опорные функции}

Прежде, чем перейти к рассмотрению задачи максимизации прибыли,
мы сделаем небольшое отступление для того, чтобы познакомить читателя
с очень удобным аппаратом изучения соответствующей задачи.
Соответствующие результаты мы приведем без доказательств.

Напомним, что множество $B$ называется выпуклым, если из
$x,y\in B$ следует $\alpha x+(1-\alpha)y\in B$ для любого
$\alpha\in[0,1]$. Иными словами, вместе с любыми двумя точками
множество содержит и весь отрезок, соединяющий эти две точки.

Читатель легко убедится в том, что пересечение любого числа
выпуклых множеств  также является выпуклым множеством, а объединение
даже двух выпуклых множеств выпуклым быть не обязано.

Точка $x\in B$ называется \emph{крайней (экстремальной) точкой} выпуклого
множества $B$, если она не может быть представлена как
$\alpha y+(1-\alpha)z$ ни для каких $y,z\in B$ и $\alpha\in(0,1)$.
Например, если множество $B$ -- это выпуклый многогранник, то
его крайними точками являются вершины, а если множество $B$ -- шар, то
все его граничные точки являются крайними.

Пусть $\beta\in\mathbb{R}^k, \beta\neq0$ и $c\in\mathbb{R}^1$. Тогда
гиперплоскость, порожденная $\beta$ и $c$ -- это множество
$H_{\beta,c}=\{z\in\mathbb{R}^k: \beta z=c\}$.
Множества $\{z\in\mathbb{R}^k: \beta z\geq c\}$ и
$\{z\in\mathbb{R}^k: \beta z\leq c\}$ называются, соответственно,
верхним и нижним полупространствами для $H_{\beta,c}$. При
этом $\beta$ называется нормалью (нормальным вектором) к
$H_{\beta,c}$.

Справедлива следующая теорема.

\textbf{Теорема отделимости.} Пусть $B\subset\mathbb{R}^k$ --
замкнутое выпуклое множество и $x\notin B$. Тогда существуют такие
$\beta\in\mathbb{R}^k, \beta\neq0$ и $c\in\mathbb{R}^1$, что $\beta x\geq c$ и
$\beta y<c$ для любого $y\in B$. Если $A,B$ два непересекающихся
выпуклых множества, то существуют такие
$\beta\in\mathbb{R}^k, \beta\neq0$ и $c\in\mathbb{R}^1$, что $\beta x\geq c$ для любого
$x\in A$ и
$\beta y \leq c$ для любого $y \in B$.

Первая часть теоремы означает, что точку, не лежащую в выпуклом множестве,
можно отделить от него гиперплоскостью (см. рис.  ), а вторая часть этой теоремы
означает, что два непересекающихся выпуклых
множества можно разделить гиперплоскостью так, что одно из множеств
будет лежать в нижнем полупространстве, а другое -- в верхнем (см. рис.   ).

EXCHANGE 18.1

EXCHANGE 18.2

Пусть $B$ -- замкнутое выпуклое множество и $x\in\partial B$. Тогда существует
$\beta\in\mathbb{R}^k, \beta\neq0$ так, что $\beta x\geq\beta y$ для любого $y\in B$.
Это означает, что для любой граничной точки замкнутого выпуклого множества существует
гиперплоскость, проходящая через эту точку и оставляющая все множество по одну сторону
от себя. Такая гиперплоскость называется опорной гиперплоскостью в направлении $\beta$.

Следствием перечисленных выше результатов является следующее важное свойство
замкнутых выпуклых множеств: каждое замкнутое выпуклое множество является
пересечением всех полупространств, содержащих данное множество (поскольку
любая точка, не лежащая в этом множестве, может быть отделена от него некоторой
гиперплоскостью).

Если же множество не является выпуклым, то пересечение всех полупространств,
его содержащих, является замкнутой выпуклой оболочкой этого множества, то
есть наименьшее (по включению) замкнутое выпуклое множество, содержащее исходное
множество.

\textbf{Определение} Для любого множества $B\subset\mathbb{R}^k$ (верхней) опорной
функцией множества $B$ называется функция $v_B$, определенная для
любого $p\in\mathbb{R}^k$ как
$v_B(p)=\sup\{px: x\in B\}.$

(Если множество $B$ компактно, то супремум в определении можно заменить на
максимум. Можно определить также нижнюю опорную функцию, если заменить $\sup$
на $\inf$. При этом, как правило, в первом случае говорят просто об опорной функции, а
во втором -- о нижней опорной функции. Нижние опорные функции удобны для
исследования множеств, не ограниченных сверху, типа, скажем $\mathbb{R}^k_+$.)

Функция $v_B(\cdot)$ может принимать значение $+\infty$. Например, если $B=-\mathbb{R}^2_+$ и
$p\notin\mathbb{R}^2_+$, то $v_B(p)=+\infty$. Геометрический смысл опорной функции
состоит в том, что если $p$ имеет единичную (евклидову) длину, то
$v_B(p)$ представляет собой расстояние от начала координат до опорной к
$B$ гиперплоскости в направлении $p$.

Опорная функция является выпуклой положительно однородной (степени 1) функцией переменной $p$.

Исключительно важное свойство, которое оказывается чрезвычайно полезным, состоит в следующем:
если $B$ -- замкнутое выпуклое множество, то
$$B=\{x\in\mathbb{R}^k: px\leq v_B(p)\, \forall p\in\mathbb{R}^k.$$
Иными словами, замкнутое выпуклое множество восстанавливается по своей опорной функции.

Без большого труда можно проверить, что опорная гиперплоскость множества $B$ в направлении
$p$ есть множество
$$\{z\in\mathbb{R}^k: pz=v_B(p).$$
Подмножество $\{z\in B: pz=v_B(p)$ называется множеством контакта $B$ и опорной гиперплоскости
в направлении $p$ (см. рис.   ) (или просто множеством контакта $B$ в направлении $p$).

EXCHANGE 18.3

Если замкнутое выпуклое множество является строго выпуклым, то есть каждая точка границы является
крайней (или, что то же самое, множество контакта в любом направлении одноточечно, или, эквивалентно,
граница этого множества не содержит отрезков), то можно говорить о точке контакта (в любом направлении).
Так, например, шар является строго выпуклым множеством. Выпуклый многогранник строго выпуклым
множеством не является. В случае, если в некотором направлении $p$ множество контакта одноточечно,
и значит, речь идет о точке контакта, то имеет место очень удобное представление, а именно, справедлива
следующая теорема.

\textbf{Теорема.} Пусть $B$ -- замкнутое выпуклое множество, а $v_B(\cdot)$ -- его
опорная функция. Тогда множество контакта $\{z\in B: \overline{p}z=v_B(\overline{p})$
в направлении $\overline{p}$ состоит из одной точки $\overline{x}$ тогда и только тогда,
когда $v_B(\cdot)$ дифференцируема в точке $\overline{p}$. В этом случае
$\overline{x}=\nabla V_B(\overline{p})$, где $\nabla V_B(\overline{p})$ -- градиент
опорной функции в точке $\overline{p}$.

(Заметим, что эта теорема может быть обобщена с использованием понятия субдифференциала
выпуклой функции, обобщающего понятие градиента на случай, когда функция не является дифференцируемой).

Пусть $A$ и $B$ -- два замкнутых выпуклых множества. Тогда (векторная) сумма множеств
$A$ и $B$ определяется следующим образом:
$$A+B=\{a+b: a\in A, b\in B\}.$$
Такая операция называется сложением по Минковскому. Получающееся при этом множество
также является замкнутым и выпуклым. Эта операция обладает очень важным свойством, а именно:
$v_{A+B}=v_A+v_B$, то есть опорная функция суммы двух выпуклых множеств есть сумма опорных
функций слагаемых.

EXCHANGE 18.4
EXCHANGE 18.5

На рис.   представлена сумма (по Минковскому) двух отрезков, а на рис.     сумма двух треугольников.
Общая идея здесь следующая. Во-первых, считаем для простоты, что хотя бы одно из множеств
содержит начало координат. Тогда для того, чтобы построить сумму множеств, нужно поступить
очень просто: взять то из множеств, которое содержит начало координат, и начало координат
вместе со всем этим множеством, как единое целое, поместить в каждую точку другого множества.
Множество, которое получится от такого "размазывания"\, начала координат по второму множеству
и есть сумма множеств. (В случае, если начало координат не содержится ни в одном из множеств,
нужно просто выделить какую-то точку в одном из множеств, а затем проделать с ней то же
самое, что и с началом координат в первом случае, не забыв после этого сдвинуть получившееся множество
соответствующим образом. См. рис.    )

EXCHANGE 18.6


Следствием этого является следующее утверждение/
\textbf{Предложение.} Пусть $A$ и $B$ -- два замкнутых выпуклых множества. Пусть далее
$Z_A(p)$ и $Z_B(p)$ -- множества контакта, соответственно, $A$ и $B$ в направлении $p$.
Тогда $Z_{A+B}(p)=Z_A(p)+Z_B(p)$.


EXCHANGE 18.7

На рис.       изображена ситуация, когда множества контакта в направлении $p$ одноточечны
и $Z_A(p)=x$, а $Z_B(p)=y$.

\section*{Максимизация прибыли. Свойства функции прибыли.}

Теперь мы можем перейти к рассмотрению задачи максимизации прибыли.
Рассмотрим конкурентную фирму, то есть
фирму, которая не может влиять на ценообразование на рынке и принимает
цены, сложившиеся на рынке, как данные. Задача максимизации прибыли
для фирмы, которая характеризуется технологией $Y$,
может быть сформулирована следующим образом: если заданы цены $p$, то
максимальная прибыль, которую может получить фирма есть
$$
\pi (p)=\max_{y\in Y} py.
$$
Напомним, что $py=p_1y_1+p_2y_2+\ldots+p_ky_k$ действительно представляет собой
прибыль, которую приносит производственный план $y$ при ценах
$p$, поскольку в производственном плане положительные компоненты соответствуют
объемам выпускаемой продукции, а значит, соответствующая часть этой суммы
с положительными $y_i$ показывает выручку, а часть суммы с отрицательными
$y_i$  соответствует затратам. Поэтому $py$ есть разность выручки и затрат,
то есть прибыль.

Функция $\pi (p)$ как функция от $p$ называется \emph{функцией прибыли}.
Для данного технологического множества функция прибыли $\pi (p)$ ставит
в соответствие каждому вектору цен $p$ максимальную прибыль, которую
может обеспечить себе фирма, характеризующаяся этим технологическим множеством.

EXCHANGE 19

Одновременно решение этой задачи определяет и предложение фирмы, то есть
множество $y(p)= \{y\in Y: py=\pi(p)\}$. Здесь, правда, стоит обратить внимание на то,
что отрицательные компоненты соответствуют спросу на факторы производства. Поэтому
лучше было бы говорить о \emph{чистой} функции предложения.
На рис. изображен случай, когда технологическое множество строго выпукло,
поэтому множество $y(p)$ одноточечно. Очевидно, что множество $y(p)$
представлает собой множество контакта технологического множества и
опорной гиперплоскости в направлении $p$, которое в общем случае
может быть неодноточечным (см. рис.    ).

EXCHANGE 20

Если фирма производит только один продукт, то проще всего иметь дело с
производственной функцией
$$
\pi (p,w)=\max_{x\ge 0}pf(x)-wx,
$$
где $p$ -- (скалярная) цена выпуска продукции, $w$~-- вектор цен
факторов производства  $x=x_1,\ldots,x_m)$.


Следует иметь в виду, что максимизирующего прибыль производственного плана
может не существовать. Обратите внимание, например, на рис. EXCHANGE 17.
Легко убедиться, что если цена выпуска намного больше цены фактора производства,
то задача максимизации решения не имеет. Формально говоря, в задаче максимизации
следовало бы максимум заменить на супремум. Однако, мы не будем подробно
останавливаться на этом, и, позволяя себе в этом случае некоторую вольность,
будем писать $\pi(p)=+\infty$. (Мы об этом уже говорили, когда
речь шла об опорных функциях).

В качестве простого примера здесь можно рассмотреть случай $k=2$. Предположим, что
технология обладает постоянной отдачей от масштаба и определяется производственной
функцией $f(x)=x$. Очевидно, что тогда, если $p_2\leq p_1$, то $\pi(p)=0$. Если же
$p_2>p_1$, то, произведя $y_2$ единиц готовой продукции, фирма получит прибыль,
равную $(p_2-p_1)y_2$. Ясно, что выбирая  $y_2$ сколь угодно большим, прибыль также
можно сделать сколь угодно большой, то есть при $p_2>p_1$ мы будем иметь
$\pi(p)=+\infty$.

Несложно доказать (и мы оставляем это читателю в качестве упражнения), что
если технологическое множество обладает свойством неубывающей отдачи, то либо
$\pi(p)=0$, либо $\pi(p)=+\infty$.

Итак, каковы же свойства функции прибыли?
Пусть $\pi(\circ )$ -- функция прибыли фирмы с технологическим
множеством $Y$, которое замкнуто и обладает свойством свободного
расходования. Предположим далее, что $y(\cdot)$ -- соответствующая функция
предложения (вообще говоря многозначная). Тогда функция прибыли обладает следующими свойствами.

\begin{itemize}
\item[1)] Функция прибыли является неубывающей функцией относительно
цен выпускаемых продуктов и невозрастающей функцией относительно
цен факторов производства. Формально, если $p'_i\ge p_i$ для выпускаемых продуктов и $p'_j\le
p_j$ для факторов производства, то $\pi(p')\ge\pi(p).$

Это свойство непосредственно следует из определения и неотрицательности вектора цен.
Функция может не возрастать строго, поскольку могут, например, возрасти цены тех продуктов, которые
фирма не выпускает, или, скажем, могут уменьшиться цены тех факторов производства,
которые фирмой не используются.

\item[2)] Функция прибыли однородна степени 1 по $p$, то есть $\pi(tp)=t\pi (p)$ для любого $t\ge 0.$

Это свойство немедленно следует из определения функции прибыли, поскольку
множитель $t$ можно вынести за знак максимума.

\item[3)] Функция прибыли выпукла по $p$, то есть $\pi (tp+(1-t)p')\le t\pi (p)+(1-t)\pi (p').$

Это свойство следует из наших обсуждений опорных функций, поскольку читатель без труда может заметить,
что функция прибыли представляет собой опорную функцию замкнутой выпуклой оболочки
технологического множества $Y$, а опорная функция, как мы отмечали, выпукла. (Разумеется,
выпуклость можно доказать непосредственно, воспользовавшись свойствами функции мкасимума).

\item[4)] Функция прибыли непрерыно зависит от $p$ для $p>0$.

Непрерывность представляет собой уже более сложное утверждение, и мы его не доказываем.
Отметим лишь, что это -- одно из важнейших свойств выпуклых функций.

\item[5)] Функция предложения $y(\cdot )$ положительно однородна степени $0$, то есть
$y(tp)=y(p)$ для любого $t>0$.

Это очевидно, поскольку умножение $p$ на положительный скаляр не меняет направления, а потому
множество контакта остается тем же самым.

\item[6)] Если $Y$~-- выпукло, то $y(p)$~-- выпуклое множество для любого $p$;\\
Если $Y$~-- строго выпукло, то $y(p)$ -- одноточечно (если
непусто).

Непосредственно следует из свойств выпуклых множеств.

\item[7)] Если $Y$ выпукло, то $Y=\{y\in\R^L:py\le\pi(p)\,\,
{\mbox{\rm для любых}}\,\,p\ge 0\}.$

Это одно из свойств опорных функций.

\end{itemize}

Легко видеть, что если мы имеем функцию $y(p)$, ставящую в
соответствие вектору цен $p$ оптимальное предложение, то вычисление прибыли
тривиально: $\pi (p)=py(p).$

Оказывается, что и "обратный вопрос"\, можно ли определить оптимальное предложение,
если известна функция прибыли, имеет простой ответ. А именно, справедливо следующее предложение.


\underline{Лемма Хотеллинга}. Если технологическое множество $Y$ выпукло
и $y(\bar{p})$ состоит из одной точки ($\bar{p}_i>0$, то $\pi (\cdot)$
дифференцируемо в точке $\bar{p}$ и
$$
y_i(\bar p)={\frac{\partial\pi(\bar p)}{\partial p_i}},\quad
i=1,\ldots,k.
$$

Доказательство немедленно следует из соответствующей теоремы раздела об опорных функций.

Итак, теперь мы можем, наконец, перейти собственно к модели Эрроу-Дебре.



\section*{Модель Эрроу-Дебре}


Выше мы рассматривали экономику чистого обмена, в которой производство отсутствовало.
Теперь мы можем рассмотреть более общую модель, в которой учитывается возможность
производства. Как и раньше мы рассматриваем потребителей, максимизирующих свои функции
полезностей. Далее мы предполагаем, что в экономике есть $m$ фирм, каждая из которых описывается
своим технологическим множеством $Y_j, j=1,\ldots,m$.

Мы считаем также, что начальный запас товаров в экономике описывается вектором
$w=(w^1,\ldots,w^k$.

Таким образом, элементы, определяющие модель Эрроу-Дебре -- это
набор, состоящий из $n$ отношений предпочтений (или функций полезностей),
$m$ технологических множеств и начального запаса товаров.

Экономика чистого обмена, которую мы рассматривали выше, представляет собой
специальный случай модели Эрроу-Дебре, в которой у всех фирм одна и та же
технология, а именно, технология свободного расходования, то есть $Y_j=-\mathbb{R}
^k_+$. Можно также считать, что есть одна фирма с технологией
свободного расходования.

Пара $(x,y)=(x(1,\ldots,x_n,y_1,\ldots,y_m)$, в которой
$x_i\in X_i$ является потребительским набором агента $i$, $y_j\in Y_j$ представляет собой
производственный план фирмы $j$, называется распределением.
Распределение $(x,y)$ называется \emph{допустимым}, если для любого  товара $k$
$$
\sum^n_{i=1}x^k_i=w^k+\sum^m_{j=1}y^k_j,
$$
или, что то же самое,
$$
\sum^n_{i=1}x_i=w+\sum^m_{j=1}y_j.
$$

Допустимое распределение $(x,y)$ называется \emph{оптимальным по Парето},
если не существует другого допустимого распределения $(x\prime,y\prime)$, которое бы
доминировало по Парето распределение $(x,y)$, то есть нет другого такого распределения
$(x\prime,y\prime)$, что $x\prime_i\succeq_ix_i$ для всех $i$, причем хотя бы для одного
$i$ предпочтение строгое, то есть $x\prime_i\succ_ix_i$.

Далее мы считаем, как и раньше, что наша экономика конкурентна, то есть
ни потребители, ни фирмы не влияют на цены. Кроме того, мы предполагаем, что фирмы
являются собственностью потребителей, причем агент $i$ владеет долей $t_{ij}\in[0,1]$ (акций)
фирмы $j$, а потому "претендует"\, на долю $t_{ij}$ прибыли фирмы $j$. Мы считаем также,
что начальный запас товаров также является собственностью агентов так, что агент $i$
обладает начальным запасов товаров $w_i$, причем $w=\sum^n_{i=1}w_i$.

Описанная нами экономика называется \emph{экономикой частной собственности}.

Теперь мы можем дать определение равновесия по Вальрасу для такой экономики.

\textbf{Определение} Распределение $x^*,y^*$ и вектор цен $p^*$ образуют \emph{равновесие
по Вальрасу}, если

\begin{itemize}
\item[1)] Для любого $j$ производственный план $y^*_j\in Y_j$ максимизирует прибыль:
$p^*y_j\leq p^*y^*_j$ для любого $y_j\in Y_j.$
\item[2)] Для любого $i$ потребительский набор $x^*_i$ максимизирует отношение предпочтения
$\succeq_i$ на бюджетном множестве
$$\{x_i\in X_i: p^*x_i\leq p^*w_i+\sum^m_{j=1}t_{ij}p^*y^*_j.$$
\item[3)] $\sum^n_{i=1}x^*_i=w+\sum^m_{j=1}y^*_j.$
\end{itemize}

Первое условие означает, что фирмы максимизируют свои прибыли. Второе -- это стандартное условие
максимизации полезности (если предпочтения описываются в терминах функции полезности), причем
бюджетное множество агента определяется его начальным запасом и его долями прибыли от прибылей,
которые получают фирмы. Наконец, последнее условие означает, что спрос равен предложению
на рынке каждого из товаров.

Сейчас мы вновь обратимся к экономике чистого обмена и покажем, в частности, что определение
равновесия по Вальрасу, данного нами  для экономики обмена, эквивалентно определению
равновесия с помощью функции избыточного спроса.

Как мы уже отмечали, экономика чистого обмена -- это экономика, в которой технологические множества
-- это технологии свободного расходования. Формально, положим $m=1$ и $Y_1=-\mathbb{R}^k_+$. Положим
$X_i=\mathbb{R}^k_+$. Будем также считать, что отношения предпочтения каждого из агентов непрерывны,
строго выпуклы и строго монотонны (в действительности, достаточно предположения о локальной
ненасыщаемости отношения предпочтения).

Нетрудно проверить, что для экономики чистого обмена приведенные выше условия,
определяющие равновесие по Вальрасу,
можно переформулировать следующим образом.
Распределение $x^*,y^*$  и вектор $p^*$ образуют равновесие по Вальрасу тогда и только тогда, когда
\begin{itemize}
\item[1')] $y^*_1\leq0, p^*y^*_1=0$ и $p^*\geq0.$
\item[2')] $x^*_i=x_i(p^*,p^*w_i)$ для любого $i$, где $x_i(\cdot)$ -- функция спроса по Маршаллу.
\item[3')] $\sum^n_{i=1}x^*_i-\sum^n_{i=1}w_i=y^*_i.$
\end{itemize}

Следующее предложение достаточно просто.
\textbf{Предложение} В
экономике чистого обмена, в которой отношения предпочтений агентов
строго выпуклы, непрерывны и строго монотонны, вектор $p^*$
является вектором равновесных цен тогда и только тогда, когда
$$\sum^n_{i=1}(x_i(p^*,p^*w_i)-w_i)\leq0.\,\,\,\,\,\,\,\,\,\,\,(*)$$

Доказательство. То, что приведенное неравенство выполняется для равновесия по Вальрасу,
конечно же, следует из условий 1') - 3'). Предположим теперь, что указанное неравенство
выполнено. Положим $y^*_1=\sum^n_{i=1}(x_i(p^*,p^*w_i)-w_i)$ и $x^*_i=x_i(p^*,p^*w_i).$
Тогда $(x^*_1,\ldots,x^*_n,y^*_1)$ удовлетворяют условиям 1') - 3'). В частности,
$p^*y^*_1=p^*\sum^n_{i=1}(x_i(p^*,p^*w_i)-w_i)=\sum^n_{i=1}(p^*x_i(p^*,p^*w_i)-p^*w_i)=0.$
Последнее равенство следует из строгой монотонности предпочтений, поскольку
$p^*x_i(p^*,p^*w_i)=p^*w_i$ для любого $i$.

Если мы вспомним теперь функцию избыточного спроса, которую мы определили, когда
определяли равновесие по Вальрасу в экономике чистого обмена, то станет понятна
эквивалентность, определений, о которой мы тогда говорили.
Действительно, если мы определим функции избыточного спроса агента $i$ как
$$z_i(p)=x_i(p,pw_i)-w_i,$$
то агрегированная функция избыточного спроса есть
$$z(p)=\sum^n_{i=1}z_i(p).$$

А это значит как раз, что $p^*$ является  вектором равновесных цен тогда и только тогда, когда
$z(p^*)\leq0$.

Обратим внимание также на то, что если $p^*$ является вектором равновесных цен, то
$p^*\geq0, z(p^*)\leq0$ и $p^*z(p^*)=\sum^n_{i=1}p^*z_i(p^*)=
\sum^n_{i=1}(p^*x_i(p^*,p^*w_i)-p^*w_i)=0.$ Следовательно,
для любого $l$ мы не только имеем $z^l(p^*)\leq0$, но и
$z^l(p^*)=0$, если $p^*_l>0.$ Таким образом в равновесии, как мы отмечали в начале,
товар $l$ может быть в избытке, то есть $z^l(p)$, но только если он является бесплатным
товаром, то есть $p_l=0$.


Аналогично тому, как мы это делали для экономики чистого обмена, в случае модели с
производством можно тоже использовать функцию избыточного спроса

Для того, чтобы определить функцию избыточного спроса, вначале определим агрегированный спрос.
\emph{Агрегированный спрос} естественно определяется как $X(p)=\sum^m_{i=1}x_i(p)$.

Поскольку у нас есть производство, предложение определяется не только начальным запасом,
но и производственными возможностями фирм, каждая из которых, напомним,
максимизирует свою прибыль. Мы считаем, что технологическое множество каждой из фирм
замкнуто и выпукло.

Итак, для каждого $j$ обозначим, как и раньше, через $y_j(p)$
функцию предложения фирмы $j$, ставящую в соответствие каждой фирме
решение ее задачи максимизации прибыли. (В случае, если технология не является строго выпуклой,
$y(p)$ может оказаться неодноточечным).
Мы можем также рассмотреть агрегированное множество производственных возможностей $Y$:
$$
Y=\sum^m_{j=1}Y_j.
$$
Определение суммы выпуклых множеств см. в разделе, посвященным выпуклым
множествам и опорным функциям.

Из теоремы, приведенной в том же разделе следует, что
агрегированный производственный план $y(p)=y_1(p)+\ldots+y_m(p)$ максимизирует
агрегированную прибыль при ценах $p$, то есть суммарную прибыль всех фирм,
тогда и только тогда, когда каждый
производственный план $y_j(p)$ максимизирует прибыль $j$-ой фирмы. Поэтому,
если мы имеем $m$ фирм, то агрегированная
(чистая) функция предложения есть $Y(p)=\sum^m_{j=1}y_j(p)$.

Таким образом, агрегированное
предложение представляет собой сумму агрегированного предложения
потребителей $w=\sum^n_{i=1}w_i$ и агрегированой (чистой) функции
предложения фирм $Y(p)$ (напомним, что отрицательные компоненты
в $Y(p)$ соответствуют спросу на факторы производства).
Следовательно, агрегированную функцию
избыточного спроса можно определить как функцию
$$
z(p)=X(p)-Y(p)-w.
$$

Так же, как и в случае экономики чистого обмена справдлив закон Вальраса

\underline{Закон Вальраса} $pz(p)=0$ для любых $p$.

Доказательство.
\begin{eqnarray*}
pz(p)&=&p(X(p)-Y(p)-w)=\\
&=&p\left (\sum^n_{i=1}x_i(p)-\sum^m_{j=1}y_j(p)-\sum^n_{i=1}w_i\right )=\\
&=&\sum^n{i=1}px_i(p)-\sum^m_{j=1}py_j(p)-\sum^n_{i=1}pw_i.
\end{eqnarray*}
Бюджетное ограничение есть $px_i=pw_i+\sum^n_{j=1}t_{ij}py_j(p)$
\begin{eqnarray*}
pz(p)&=&\sum^n_{i=1}pw_i+\sum^n_{i=1}\sum^m_{j=1}t_{ij}py_j(p)-\sum^n_{j=1}py_j(p)-\sum^n_{i=1}pw_i=\\
&=&\sum^m_{j=1}py_j(p)\sum^n_{i=1}t_{ij}-\sum^m_{j=1}py_j(p)=\\
&=&\sum^m_{j=1}py_j(p)-\sum^m_{j=1}py_j(p)=0.
\end{eqnarray*}

\textbf{Теорема Эрроу-Дебре}.  Пусть выполнены следующие условия:
\begin{itemize}
\item[(1)] Потребительские множества $X_i$ всех агентов замкнуты, ограничены снизу
(то есть для каждого$i$ существует такой $b_i\in\mathbb{R}^k$, что
$x_i\geq b_i\,\,\forall x_i \in X_i$) и выпуклы.
\item[(2)] Отношения предпочтения всех потребителей локально ненасыщаемы.
\item[(3)] Отношения предпочтения всех потребителей непрерывны.
\item[(4)] Для любого потребителя его начальный запас лежит внутри его потребительского
множества.
\item[(5)] Для любого потребителя $i$, если $x_i$ и $x'_i$ -- два
потребительских набора, то $x_i \ge _ix'_i$ влечет
$tx_i+(1-t)x'_i\succ_ix'_i$ для любых $0<t<1$.
\item[(6)] Для любой фирмы $j$, $0\in Y_j.$
\item[(7)] Каждое из множеств $Y_j$ замкнуто и выпукло.
\item[(8)] $Y\cap (-Y)\subset\{0\}$
\item[(9)]  $Y\supset (-\mathbb{R}^k_+).$
\end{itemize}
Тогда в экономике существует равновесие по Вальрасу.

Все эти условия уже обсуждались нами выше. Тем не менее, мы еще
раз прокомментируем их.

Условия (1) -- (3) нужны для существования набора,
максимизирующего полезность , (4)--(5) -- для непрерывности
потребительского спроса (который, вообще говоря, может быть
многозначным), свойство (5) означает выпуклость отношений
предпочтения; (6) гарантирует возможность остановки производства;
(7) обеспечивает непрерывность (чистой) функции предложения каждой
фирмы; (8) означает, что производство необратимо, и используется
также для обеспечения ограниченности множества допустимых
распределений; (9) говорит, что любой производственный план,
использующий все товары как факторы производства, является
допустимым, из чего следует, что $p^*\ge 0$.

Мы не приводим здесь доказательства теоремы, поскольку оно выходит
за рамки этой книги. Отметим, однако, следующий интересный факт.
Оригинальное доказательство Эрроу и Дебре этой теоремы заключалось
в доказательстве существования равновесия по Нэшу в следующей
игре, которую мы опишем, не вдаваясь в особые формальности.

В игре $n+m+1$ игроков: $n$ потребителей, $m$ фирм и "фиктивный"\,
("рыночный") игрок, который выбирает цены.
Каждая фирма $j=1,\ldots,m$ выбирает производственный план из своего
технологического множества $Y_j$ (причем этот выбор не ограничен действиями
других игроков)и получает выигрыш $py_j$.

Каждый потребитель $i=1,\ldots,n$  выбирает вектор $x_i$ из $X_i$
при ограничениях
$$px_i \leq pw_i+\max[0,\sum^m_{j=1}t_{ij}py_j]$$
и получает выигрыш $u_i(x_i)$.

Наконец, фиктивный игрок выбирает вектор цен $p$ (его выбор также не
ограничен действиями других игроков) и получает выигрыш $pz$, где
$z=x-y-w$. По-видимому, именно эта компонента игры требует некоторых комментариев.

Как мы видим, $z$ определяется набором $x_i$ и $y_j$. Предположим далее,
что рыночный игрок не  "мгновенно"\, максимизирует свой  выигрыш, а,
считая заданным выбор остальных игроков, осуществляет выбор цен так,
чтобы увеличить свой выигрыш. Для данного $z$ функция $pz$ является линейной
функцией от $p$. Поэтому она может быть увеличена за счет увеличения
$p_l$ для тех продуктов, для которых $z_l>0$, то есть спрос превышает предложение,
и, напротив, за счет уменьшения $p_l$ для тех продуктов, для которых $z_l<0$,
то есть предложение превышает спрос. А это как раз и есть классический
закон спроса и предложения, тем самым мотивация нашего фиктивного игрока
соответствует одному из элементов конкурентного равновесия. Разумеется, это
лишь интуитивное оправдание такого выбора функции выигрышей рыночного
участника.

Если говорить чуть более формально, то, во-первых, напомним, что
$\sum^m_{j=1}t_{ij}=1$. Далее из определения $z$ мы имеем $p^*z^*=0$.
Рассмотрим вектор (цен) $\delta^r$, в котором все координаты равны $0$, кроме
$r$-той координаты, которая равна 1. Поэтому из определения равновесия (по Нэшу)
$$0=p^*z^*\succeq\delta^rz^*=z^*_r,$$
или $z^*\leq0$.

\underline{Равновесие и благосостояние}

Напомним, что распределение $(x,y)$ допустимо, если
$$
\sum^n_{i=1}x_i\sum^m_{j=1}y_j-\sum^n_{i=1}w_i=0
$$
{\bf Первая теорема экономики благосостояния}. Если распределение $(x,y)$ и
вектор цен $p$ образуют равновесие по Вальрасу, то $(x,y)$ -- оптимально по Парето.

Доказательство. Предположим противное, и пусть $(x',y')$ - доминирует
$(x,y)$ по Парето. Так как потребители максимизируют свои
функции полезностей, то
$$
px'_i>pw_i+\sum^m_{j=1}t_{ij}py_j\,\,{\mbox{\rm для любого}}\,\,i=1,\ldots,n
$$
(На самом деле, достаточно, чтобы строгое неравенство было хотя для одного $i$.)
Суммируем по $i=1,\ldots,n$
$$
p\sum^n_{i=1}x'_i>\sum^w_{i=1}pw_i+\sum^m_{j=1}py_i\quad({\mbox{\rm т.к.}}\,\,\sum t_{ij}=1)
$$
Поскольку $x',y'$ допустимо, то заменим $\sum x'_i$ на $\sum^m_{j=1}y'_j+\sum^n_{i=1}w_i$:
$$
p\left (\sum^m_{j=1}y'_j+\sum^n_{i=1}w_i\right )>\sum^n_{i=1}pw_i+\sum^m_{j=1}py_j.
$$
Это дает нам
$$
\sum py'_j>\sum py_j.
$$
Но это значит, что агрегированнная прибыль для производственных планов $(y'_j)$
строго больше, чем агрегированная прибыль для производственных планов $(y_j)$, а
это противоречит предположению о максимизирующем поведении фирм.

{\bf Вторая теорема экономики благосостояния}. Пусть $(x^*,y^*)$ -- оптимальное
по Парето распределение, в котором каждый потребитель располагает строго
положительным количеством каждого продукта. Предположим, что предпочтения
потребителей выпуклы, непрерывны и строго монотонны, а технологические множества
$Y_j$ выпуклы для всех $j=1,\ldots,m$. Тогда существует такой вектор цен
$p\geq0$, что:
\begin{itemize}
\item[(1)] Если $x'_i\succ_i x^*_i$, то $px'_i>px^*_i$\quad $i=1,\ldots,n$
\item[(2)] Если $y'_j\in Y_j$, то $py^*_j\ge py'_j$ для любого $y'_j\in
Y_j$ $j=1,\ldots,n$.
\end{itemize}

Идея доказательства этой теоремы аналогична идее доказательства второй теорем
благосостояния в отсутствие производства и опирается на теорему отделимости.

Опять же, аналогично той же теореме, только что приведенная теорема
показывает, что любое оптимальное по Парето распределение
может быть получено с помощью соответствующего перераспределения
"богатства": мы берем распределение $x^*,y^*$ затем определяем
"подходящие"\, цены $p$, после чего даем потребителю доход, равный
$px^*_i$, получая который потребитель не будет пересматривать свой
потребительский набор.

Так же, как и при рассмотрении ящика Эджворта, мы можем интерпретировать
это несколькими способами. Во-первых, мы можем считать, что происходит
перераспределение начального запаса, которое было бы согласовано с
желаемым перераспределением доходов. Такое перераспределение может включать
перераспределение товаров или долей собственности. С другой стороны, мы можем
считать, что у потребителей сохраняются начальные запасы, но вводится
подоходный налог (или субсидии).


\section*{Немного о единственности равновесия}

Основная проблема, которая интересовала нас до сих пор, состояла в
том, чтобы выяснить, существует ли равновесие. Теперь обратимся к
естественно возникающему следующему вопросу: "Что можно сказать о
единственности равновесия?".

Как мы выяснили, при достаточно общих предположениях равновесие
существует, то есть существует такой вектор цен $p^*$, что
$z^*\leq0$. Мы отмечали также, что равновесие (например, в ящике
Эджворта) не обязано быть единственным. К сожалению, в общем случае
вопрос единственности становится уже достаточно сложной
математической задачей, для решения которой используются результаты
дифференциальной топологии, поэтому мы сможем в рамках данной книги
рассмотреть лишь один случай, который не требует сложного
математического аппарата.

Будем предполагать, что все товары являются желательными, то есть
избыточный спрос на любой товар положителен, если этот товар
является бесплатным. Следствием этого, как это подчеркивалось выше,
является положительность всех цен. Как и раньше, мы предполагаем,
что функция избыточного спроса $z$ непрерывна. Однако теперь, кроме
того, нам потребуется еще и непрерывная дифференцируемость: это
свойство исключает появление угловых точек на кривых безразличия,
что может приводить к существованию целого множества равновесных
цен.

При таких предположениях вопрос о единственности равновесия сводится
к следующему. В каком случае для гладкого (то есть непрерывно
дифференцируемого) отображения $z$ стандартного симплекса цен в
$\mathbb{R}^k$ существует единственный вектор цен $p$, для которого
$z(p)=0$?

Оказывается, что единственность можно гарантировать, если все товары
являются валово заменяемымы. Валовая заменяемость двух продутов
означает, что если увеличивается цена одного продукта, то
увеличивается спрос на другой. (Если быть более точным, то следовало
бы различать \emph{чистую заменяемость}, когда увеличение цены
одного продукта увеличивает спрос по Хиксу на другой продукт, и
\emph{валовую заменяемость}, когда спрос по Хиксу заменяется спросом
по Маршаллу, однако мы ограничимся приведенным выше определением).

Формально, товары $l$ и $h, h \neq l$ являются \emph{валово
заменяемыми} при ценах $p$, если
$\frac{\partial z_h(p)}{\partial p_l}\geq0.$

\textbf{Предложение} Предположим, что все товары явлются валово
заменяемыми при всех ценах. Тогда если $p^*$ является равновесным
вектором цен, то он единственен.

\emph{Доказательство} Предположим, что $p\prime$ -- другой вектор
равновесных цен . Поскольку $p^*>0$, то положим
$a=\max{p\prime_l/p^*_l\neq0}$. В силу положительной однородности и
рановесности $p^*$, мы имеем $z(p^*)=z(ap^*)=0$. Из определения $a$
следует, что для некоторого $h$ будет выполнено равенство
$ap^*_h=p\prime_h$. Уменьшим теперь каждую цену $ap^*_l$, отличную
от $p\prime_h$, до $p\prime_l$. Поскольку цена каждого продукта,
кроме $h$-ого, при этом уменьшается, то спрос на продукт $h$ должен
уменьшиться. Следовательно, $z_h(p\prime)<0$, а значит $p\prime$ не
может быть вектором равновесных цен.


\section*{Модель экономики Робинзона Крузо}

Рассмотрим в качестве примера модель, в которой есть один потребитель и один
производитель.

А именно, предположим, что есть два агента, которые принимают цены как данные:
первый агент -- это единственный потребитель, а второй агент --
единственная фирма.  В рассматриваемой экономике есть два продукта: труд
(на самом деле, конечно, нужно говорить о досуге и сейчас станет
понятным, почему) и продукт потребления, производимый фирмой.
Будем считать, что потребитель характеризуется непрерывным,
выпуклым и строго монотонным отношением предпочтения $\succeq$,
определенным на парах $x_1,x_2$, где $x_1$ -- потребление досуга,
а $x_2$ -- потребление продукта, производимого фирмой. Начальный запас
у потребителя следующий: у него есть в распоряжении
$L$ единиц досуга (24 часа в сутки, часть которого он может (и будет) выделять
на труд) и нет второго продукта.

Фирма использует труд для производства в соответствии с возрастающей
вогнутой производственной функцией $f(z)$, где $z$ - труд (фирма
нанимает потребителя, эффективно используя часть его досуга).

Далее, если $p$ -- цена выпускаемого продукта, а $w$ -- цена труда
(ставка заработной платы), то фирма решает задачу
$$
\max_{z\ge 0}pf(z)-wz.
$$
При ценах $(p,w)$ оптимальное потребление труда, то есть решение
приведенной задачи, есть $z(w,p)$, оптимальный объем выпуска
$q(w,p)$ и прибыль $\pi (w,p)$.

Так же, как мы считали ранее (фирмами владеют потребители)
потребитель владеет фирмой и получает ее прибыль $\pi(w,p)$
(поскольку он является единственным владельцем фирмы, то он получает
всю прибыль). Если $u(x_1,x_2)$ - функция полезности, представляющая
$\succeq$, то задача потребителя при данных ценах $(w,p)$ есть
$$
\max_{(x_1,x_2)\in\R^2_+}u(x_1,x_2)
$$
$$
px_2\le w(L-x_1)+\pi (w,p)
$$
Обозначим оптимальный спрос здесь через
$$
x_1(w,p),x_2(w,p)
$$
Выбирая потребление досуга $x_1$, потребитель тем самым выбирает
время, которое он будет работать $L-x_1$, а значит получит
заработную плату в размере $w(L-x_1)$ (если, конечно, фирма полностью
исполььзует предлагаемое время).
Разумеется, эта модель исключительно проста, но на нее можно смотреть как на
модель с репрезентативным потребителем и репрезентативной фирмой. (Правда, рассмотрение
репрезентативного потребителя требует некоторых дополнительных условий, например,
необходимо, чтобы экономика состояла из многих потребителей с
идентичными вогнутыми функциями полезностей и идентичными начальными запасами.)

Равновесие по Вальрасу в этой модели включает такую пару
$(w^*,p^*)$, что
$$
\begin{array}{c}
x_2(w^*,p^*)=q(w^*,p^*),\\
z(w^*,p^*)=L-x_1(w^*,p^*).
\end{array}
$$

Экономика, которую мы описали,  иногда называется экономикой
Робинзона Крузо. Жизнь в такой экономике устроена достаточно
своеобразно: потребитель (Робинзон Крузо) является собственником
фирмы производителя товара потребления (второго товара), для
производства которого используется единственный фактор производства
-- его собственный труд. Можно представить себе, что сегодня
Робинзон Крузо работает в фирме, владельцем которой является, и
получает зарплату, а завтра потребляет то, что произвел накануне.
Послезавтра снова работает, а затем потребляет и т.д.

На рис. EXCHANGE 21 изображено решение задачи фирмы. На следующем
рис. ( EXCHANGE 22) кроме того изображено решение задачи
потребителя.

EXCHANGE 22

Оси координат с началом в точке $0_R$ относятся к Робинзону Крузо
(потребителю), при этом горизонтальная ось направлена вправо. На ней
откладывается досуг, а вертикальная ось соответствует потреблению
готового продукта. Точка $x^*$ -- как всегда, точка касания кривой
безразличия потребителя и его бюджетной линии.

Оси координат с началом в точке $0_F$ относятся к фирме, при этом по
горизонтальной оси, направленной влево, откладываются отрицательные
количества фактора производства (труда), а расстояние между $0_R$ и
$0_F$ есть $L$, то есть начальный запас досуга (для потребителя) и,
одновременно, максимальное количество труда (фактора производства),
которое в принципе могла бы использовать фирма. На приведенном
рисунке оптимальное потребление готового продукта $x^*_2$ меньше,
чем оптимальный выпуск фирмы $q(w,p)$, а оптимальное предложение
труда $L-x^*_1$ меньше спроса на труд, то есть в изображенной
ситуации имеется избыточный спрос на труд и избыточное предложение
готового продута. Поэтому для достижения равновесия нужно увеличить
ставку $w$ и уменьшить цену $p$. Соответствующее равновесие по
Вальрасу изображено на рис.EXCHANGE 23.

EXCHANGE 23
