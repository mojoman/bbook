



\chapter{Предельные величины в экономике}

\section{ Максимумы и минимумы. Производные. Выпуклые и вогнутые функции}


\subsection{Простейшие задачи на максимум и минимум}


    Ключевая роль предельных величин в экономической науке определяется тем,
    что важнейшее предположение о поведении экономических агентов при построении
    тех или иных моделей  состоит в том, что ведут они себя рационально,  т.е.
    решают задачи на максимум или минимум. Понятия максимума и минимума
    объединены терминами <<оптимум>> и <<экстремум>>. Задачи на отыскание максимума
    и минимума называют оптимизационными или экстремальными
    задачами, а раздел математики, в котором
    изучают такие задачи, называется теорией оптимизации или теорией
    экстремальных задач.
    А анализ таких задач удобнее всего
    проводить именно на языке предельных величин, т.е. с помощью
    дифференциального исчисления. В этом пункте мы напомним
    читателю, как проводить этот анализ в самых простейших ситуациях.



    Напомним читателю основные определения.

    Пусть $\st{D}$ --- некоторое подмножество вещественной прямой
    $\R$, а функция $f$ определена на
    всей вещественной прямой или на некотором ее подмножестве,
    содержащем $\st{D}$ (быть может, только на самом $\st{D}$).
    Точка $x^{*}\in \st{D}$ называется точкой \emph{глобального
    максимума}, или просто точкой максимума, функции
     $f$ (на множестве $\st{D}$), если
    \[f(x^{*})\geqslant f(x) \ \forall x\in\st{D},\]
    и точкой \emph{глобального минимума}, или просто точкой минимума, если
     \[f(x^{*})\leqslant f(x) \ \forall x\in\st{D}.\]
     Если же существует  окрестность
     \[(\hat{x}-\epsilon,x^{*}-\epsilon), \  \epsilon>0,\]
     точки $x^{*}$, такая что
     \[f(x^{*})\geqslant f(x) \ \forall x\in\st{D}\cap(x^{*}-\epsilon,x^{*}-\epsilon)\]
     или
     \[f(x^{*})\leqslant f(x) \ \forall x\in\st{D}\cap(x^{*}-\epsilon,x^{*}-\epsilon),\]
     то эта точка называется точкой локального максимума или локального
     минимума соответственно.


     Точки (локального) максимума и минимума называются точками (локального) экстремума.
     Очевидно, что всякая точка максимума (минимума) является точкой
     локального максимума (минимума), но не наоборот.

     Задачу об отыскании максимума функции $f$ на множестве $\st{D}$
     записывают в виде
\begin{equation*}
\label{z-max-odnom}
     f(x)\rightarrow \max, \ x\in\st{D},
\end{equation*}
          а задачу об отыскании минимума --- в виде
\begin{equation}
\label{z-min-odnom}
     f(x)\rightarrow \min, \ x\in\st{D}.
\end{equation}


     При это зачастую не уточняют, идет ли речь о локальном или же
    о глобальном экстремуме. Из
     контекста всегда понятно, о чем идет речь. В экономических
     задачах обычно речь идет именно о глобальном экстремуме. Нас тоже будет
    интересовать глобальный максимум или минимум.
    Поэтому под
     решением задачи на максимум мы всегда будем понимать точку
     глобального максимума, а под решением задачи на минимум ---
     точку глобального минимума.

      Функция $f$,
     экстремум которой ищется, называется \emph{целевой функцией}, а
     множество $\st{D}$, на котором нужно найти этот экстремум,
     называют \emph{допустимым множеством}. Значение целевой функции в точке
    экстремума называется (оптимальным) \emph{значением задачи}.

     Первый вопрос, который заслуживает обсуждения после того, как
     сформулирована задача на экстремум, это вопрос о
     том, существует ли искомая точка максимума или минимума. Ответ
     на него зависит от устройства целевой функции и допустимого
     множества. Про целевую функцию почти всегда предполагают, что
     она непрерывна. Что касается допустимого множества, то чаще
     всего в содержательных задачах это множество является всей
     вещественной прямой $\R$ или замкнутым промежутком (например,
     отрезком $[\bar{x},\bar{\bar{x}}]$ или замкнутой полупрямой
     $[\bar{x},+\infty)$,
     первым примером которой является множество $\R_{+}$ всех
     неотрицательных чисел).

    На отрезке, представляющем собой компактное множество,
    непрерывная функция достигает своего максимума и минимума. Об
    этом говорит известная теорема Вейерштрасса (??????).

    Если же допустимое
    множество является хотя и замкнутым, но неограниченным, например всей вещественной
    прямой или полупрямой, то максимум или минимум непрерывной
    функции на этом множестве может и не достигаться. Например, на
    множестве $\R_{+}$ функция $f(x)=x^{1/2}$ достигает своего
    минимума (в нуле), но не достигает максимума, а функция
    $f(x)=x^{1/2}-x$ достигает своего максимума в некоторой положительной
    точке, но не достигает своего минимума.







\


    Как мы уже говорили, исследовать задачи на экстремум удобно с
    помощью дифференциального исчисления.
    Напомним, что функция $y=f(x)$, заданная в некоторой окрестности
    точки $\hat{x}\in\R$,  называется дифференцируемой в этой точке,
    если существует такая линейная функция $a\Delta x$
    (ее аргументом является $\Delta x$), что приращение
    \[\Delta y=f(x+\Delta x)-f(x)\]
    функции $f$ представимо в виде
\begin{equation}
\label{differents}
    \Delta y=a\Delta x+o(\Delta x) \ \text{при} \ \Delta x\rightarrow 0,
\end{equation}
    где, как обычно, символ $o(\cdot)$ используется для обозначения
    бесконечно малой.
    Иными словами, функция дифференцируема в точке $\hat{x}\in\R$,
    если изменение ее значения в окрестности исследуемой точки
    линейно с точностью до поправки, бесконечно малой по сравнению с
    величиной смещения от $\hat{x}\in\R$. Линейная функция $a\Delta x$,
    фигурирующая в (\ref{differents}), называется дифференциалом
    функции $f$ в точке $\hat{x}\in\R$ (ее аргументом является $\Delta x$).
    Читатель, несомненно помнит, что  дифференцируемость функции $f$ в
    точке $\hat{x}\in\R$ эквивалентна существованию конечной производной
    $f\,'(\hat{x})$, при этом последняя совпадает с числом $a$,
    задающим линейную функцию $a\Delta x$.

    Далее в этой главе мы иногда будем записывать соотношение (\ref{differents}) в
    виде
    \[\Delta y\approx a\Delta x=f\,'(\hat{x})\Delta x,\]
    а слова о том, что некоторое равенство выполняется с точностью
    до бесконечно малой, заменять словами о том, что равенство
    является примерным или приближенным. (\remrk{РИС ???????})
    При этом, мы зачастую будем
    проводить рассуждения, считая что что приращение  $\Delta x$
    задано, но что оно достаточно мало, а точность примерного
    равенства достаточно высока. Наши рассуждения будут
    не самыми строгими, но если читатель достаточно хорошо знаком с
    математическим анализом, он сможет перевести их на
    самый точный математический язык.

    В дальнейшем само по себе использование обозначения $f\,'(x)$
    будет автоматически подразумевать, что функция $f(x)$ дифференцируема в точке
    $x$, на всей области своего определения
    или на внутренности этой области. Соответственно,
    использование обозначения $f\,''(x)$ будет свидетельствовать о
    том, что речь идет о дважды дифференцируемой функции.

    Кроме
    того, мы будем допускать некоторую вольность и называть функцию
    дифференцируемой, если у нее имеется производная, быть может,
    даже бесконечная и, быть может, односторонняя. Например, функцию
    $f(x)=x^{1/2}$ мы будем считать дифференцируемой. Для нее
    $f'(0)=+\infty$.

    Теперь напомним, как производные применяются для анализа задач
    на экстремум.
    Далее в этом параграфе, если не оговорено противное, мы будем
    вести речь о задачах на максимум. С качественной точки зрения
    анализ задач на минимум ничем не отличается от задач на
    максимум, поскольку задачу на минимум
    (\ref{z-min-odnom}) можно переписать в виде
    \[-f(x)\rightarrow \max, \ x\in\st{X}.\]

    Рассмотрим следующую совсем простую задачу:
\begin{equation}
\label{monot-max}
    f(x)\rightarrow \max, \ x\in[\bar{x},\bar{\bar{x}}\,], \ -\infty<\bar{x}<\bar{\bar{x}}<+\infty,
\end{equation}
    где $f(x)=b+ax$.
    Очевидно, что ее решение зависит от того, чему равно $a$. Если
    $a>0$, то решением (точкой глобального максимума) является точка
    $\bar{\bar{x}}$, если $a<0$, то точка $\bar{x}$, а если $a=0$, то решением
    рассматриваемой задачи является любая точка отрезка $[\bar{x},\bar{\bar{x}}\,]$.
    Здесь ключевую роль играет то, что в случае $a>0$ функция $f$
    монотонно возрастает на всем отрезке $[\bar{x},\bar{\bar{x}}\,]$, а при $a<0$ она
    монотонно убывает. Для нас важно то, что в случае монотонно возрастающей
    функции никакая внутренняя точка отрезка не является точкой
    локального максимума.
    (\remrk{РИС ???????, a>0, a<0, a=0})

      Это простое рассуждение позволяет понять, почему при решении
    задачи на экстремум первое, что рекомендуется сделать, это взять
    производную и приравнять ее к нулю (хотя иногда эта
    рекомендация может оказаться неполной или даже неуместной).

    Напомним, что если $f\,'(\hat{x})>0$,
    то функция $f$ \emph{монотонно возрастает} в точке $\hat{x}$, т.е. при
    некотором достаточно малом $\epsilon>0$ выполняются неравенства
    \[f(x)<f(\hat{x}) \ \forall x\in(\hat{x}-\epsilon,\hat{x})\]
    и
    \[f(\hat{x})<f(x) \ \forall x\in(\hat{x},\hat{x}+\epsilon),\]
    поскольку при $b=f(\hat{x}), \ b=f\,'(\hat{x})>0$ и
    $x\in(\hat{x}-\epsilon,\hat{x}+\epsilon)$
    примерное  равенство
    \[f(x)\approx b+ax\]
    является достаточно точным (благодаря тому, что $\epsilon>0$ достаточно мало),
    а функция $b+ax$ монотонно возрастает.


    Если же $f\,'(\hat{x})<0$, то функция $f$ \emph{монотонно убывает}
    в точке $\hat{x}$, т.е. при
    некотором достаточно малом $\epsilon>0$ выполняются неравенства
    \[f(x)>f(\hat{x}) \ \forall x\in(\hat{x}-\epsilon,\hat{x})\]
    и
    \[f(\hat{x})>f(x) \ \forall x\in(\hat{x},\hat{x}+\epsilon).\]


     Заметим, что если $f\,'(\hat{x})=0$, то
    возможна как ситуация, когда функция $f$ не является монотонной
    в точке $\hat{x}$ (например, если $f(x)=x^{2}, \ \hat{x}=0$), так и ситуация,
    когда она монотонно возрастает ($f(x)=x^{3}, \ \hat{x}=0$) или
    убывает ($f(x)=-x^{3}, \ \hat{x}=0$).

    В случае, когда $f\,'(x)>0$ ($f\,'(x)<0$) для всех точек $x$ из
    некоторого интервала вещественной прямой, то функция $f$ возрастает
    (убывает) на всем интервале. (\remrk{Нужен ли этот абзац???????})
    (\remrk{РИС ??????? f'><=0})

    Читатель, несомненно знаком со следующей теоремой.


\begin{teo}
    Предположим, что функция $f$, заданная на некотором промежутке
    $\langle \bar{x},\bar{\bar{x}}\rangle$, дифференцируема в точке
    $\hat{x}\in(\bar{x},\bar{\bar{x}})$.
    Если $\hat{x}$ является точкой локального
    максимума (на $\R$), то
    \[f\,'(\hat{x})=0.\]
\end{teo}
    \textbf{Доказательство.} Достаточно заметить. что если $f'(\hat{x})>0$,
    то функция $f$ монотонно возрастает в точке $\hat{x}$,  а если
    $f\,'(\hat{x})<0$, то монотонно убывает. \ $\Box$


    Из этой теоремы следует, что в случае, когда множество $\st{D}$
    открыто, например, совпадает со всей вещественной прямой $\R$,
    а функция $f$ задана на этом множестве и
    дифференцируема на нем, то решение задачи (\ref{z-min-odnom})
    следует искать среди тех точек $x\in\st{D}$, для которых
    $f\,'(x)=0$. (\remrk{РИС ???????}) Такие точки называют \emph{подозрительными на
    экстремум}. При этом следует подчеркнуть, что среди этих точек
    находятся как все точки локального
    максимума, так и все точки локального минимума, а также точки,
    не являющиеся ни теми, не другими. Иными словами, равенство
    производной нулю не является достаточным условием максимума или
    минимума.

    Напомним, что необходимые и достаточные условия --- это не одно и то же.
    Например, равенство производной нулю в
    сформулированной только что теореме, является необходимым
    условием локального экстремума. Если некоторая точка из
    интервала  $\langle \bar{x},\bar{\bar{x}}\rangle$
    является точкой локального максимума или минимума, то она
    удовлетворяет этому условию. Когда речь идет о достаточном
    условии максимума или минимума, то это значит, что если некоторая точка
    удовлетворяет этому условию, то она обязательно является
    точкой максимума или минимума соответственно.

    Следует указать, что необходимые и/или достаточные условия
    экстремума, формулируемые в терминах первых производных,
    называют \emph{условиями первого порядка}.

    В случае, когда допустимое множество $\st{D}$ является отрезком
    $[\bar{x},\bar{\bar{x}}]$, замкнутой полупрямой (вида $[\bar{x},+\infty)$ или
    $(-\infty,\bar{\bar{x}}]$), точками, \emph{подозрительными на экстремум} являются
    также граничные точки $\st{D}$. А именно, если локальный
    максимум (минимум)
    функции $f$ на отрезке $[\bar{x},\bar{\bar{x}}]$ или
    полупрямой $[\bar{x},+\infty)$ достигается
    в точке $\bar{x}$, то $f\,'(\bar{x})\leqslant0$
    ($f\,'(\bar{x})\geqslant0$). Если же локальный
    максимум (минимум)
    функции $f$ на отрезке $[\bar{x},\bar{\bar{x}}]$ или
    полупрямой $(-\infty,\bar{\bar{x}}]$)
    достигается в точке $\bar{\bar{x}}$, то
    $f\,'(\bar{\bar{x}})\geqslant0$ ($f\,'(\bar{\bar{x}})\leqslant0$).
    (\remrk{РИС ???????})

    Более того, по  знаку производной на границе допустимого
    множества иногда можно с уверенностью сказать, являются ли они
    точкам локального максимума или минимума. А именно можно
    сформулировать достаточные условия максимума и минимума.


    Например, если $f\,'(\bar{x})<0$ ($f\,'(\bar{x})>0$), то
    точка $\bar{x}$ является точкой локального максимума (минимума) функции $f$ на
    отрезке $[\bar{x},\bar{\bar{x}}]$ или полупрямой $[\bar{x},+\infty)$.
    Если же $f\,'(\bar{\bar{x}})>0$ ($f\,'(\bar{\bar{x}})<0$), то
    точка $\bar{\bar{x}}$ является точкой локального максимума (минимума) функции $f$ на
    отрезке $[\bar{x},\bar{\bar{x}}]$ или полупрямой $(-\infty,\bar{\bar{x}}]$.
    (\remrk{РИС ???????})

    Здесь надо уточнить, что под производной на границе
    отрезка или полупрямой, например в точке $\bar{x}$,
    мы понимаем одностороннюю производную.
    В случае же, когда функция $f$ задана не только на самом отрезке
    $[\bar{x},\bar{\bar{x}}]$ или на полупрямой $[\bar{x},+\infty)$,
    но и в некоторой окрестности точки $\bar{x}$ и
    дифференцируема в этой точке, то читателю можно и не вспоминать, что такое
    односторонняя производная.


\subsection{Выпуклые и вогнутые функции}

    Выпуклые и вогнутые функции играют очень важную роль в
    экономическом анализе. Для случая функций нескольких переменных мы обсудим
    их свойства позднее, а здесь вкратце напомним, каковы основные свойства
    выпуклых и вогнутых функций одной переменной, предполагая, что
    эти свойства читателю уже знакомы из курса математического
    анализа.

    Пусть функция $f$ задана на некотором промежутке
    $\langle \bar{x},\bar{\bar{x}} \rangle$, где $\bar{x}$ может быть равным $-\infty$, а
    $\bar{\bar{x}}$ может быть равным $+\infty$.


    Эта функция называется \emph{выпуклой} (\remrk{РИС ???????}) (на промежутке
    $\langle \bar{x},\bar{\bar{x}} \rangle$), если для любых точек
    $x,y\in\langle \bar{x},\bar{\bar{x}} \rangle$ и любом числе
    $\alpha\in[0,1]$ имеет место неравенство
\begin{equation}
    \label{vipuk}
    f(\alpha x +(1-\alpha)y)\leqslant \alpha f(x)+(1-\alpha)f(y).
\end{equation}
    Если при $x\neq y$ и $\alpha\in(0,1)$ это неравенство является
    строгим, то функция $f$ \emph{строго выпуклой}.
    (\remrk{РИС ???????})

    Если же для любых точек
    $x,y\in\langle \bar{x},\bar{\bar{x}} \rangle$ и любом числе
    $\alpha\in[0,1]$ имеет место неравенство
\begin{equation}
    \label{vipuk}
    f(\alpha x +(1-\alpha)y)\geqslant\alpha f(x)+(1-\alpha)f(y),
\end{equation}
    то функция $f$ называется \emph{вогнутой}. Если при $x\neq y$ и
    $\alpha\in(0,1)$ последнее неравенство является
    строгим, то функция $f$ \emph{строго вогнутой}.
    (\remrk{РИС ???????})

    Иногда выпуклые функции называют выпуклыми вниз, а вогнутые ---
    выпуклыми вверх.

    Очевидно, что если функция $f(x)$ выпукла, то функция $-f(x)$
    вогнута. Отсюда следует, что функция $f(x)=b+ax$ является выпуклой
    и вогнутой одновременно. Конечно, эта функция не является строго выпуклой или строго
    выпуклой.
\begin{exer}
    Предположим, что функции $f(x)$ и $g(x)$ выпуклы.

    Докажите, что функция
    $h(x)=\max\{f(x),g(x)\}$ тоже выпукла. Может ли быть выпуклой
    функция $h(x)=\min\{f(x),g(x)\}$?

    Докажите, что если
    $\alpha\geqslant0$ и $\beta\geqslant0$, то функция $h(x)=\alpha f(x)+\beta g(x)$
    выпукла, а если $\alpha\leqslant0$ и $\beta\leqslant0$, то вогнута.
\end{exer}

\begin{exer}
    Докажите, что если функция выпукла, но не строго выпукла, то ее
    график содержит некоторый отрезок.
\end{exer}

\begin{exer}
    Докажите, что функция $f$ выпукла на промежутке
    $\langle \bar{x},\bar{\bar{x}}\rangle$
    тогда и только тогда, когда для каждого
    $t\in\langle \bar{x},\bar{\bar{x}}\rangle$ функция
    \[g_{t}(x)=\frac{f(x)-f(t)}{x-t},\]
    заданная на промежутке $\langle \bar{x},\bar{\bar{x}}\rangle$ за исключением точки
    $t$, монотонно не убывает.
\end{exer}


    Напомним, как характеризуются выпуклые и
    вогнутые функции в случае, когда они дифференцируемы и дважды дифференцируемы.
\begin{prop}
    Предположим, что функция $f$ непрерывна на промежутке
    $\langle \bar{x},\bar{\bar{x}} \rangle$
     и дифференцируема на внутренности этого промежутка, т.е. на
     интервале $(\bar{x},\bar{\bar{x}})$.

    Эта функция вогнута (выпукла) на $\langle \bar{x},\bar{\bar{x}} \rangle$
    тогда и только тогда, когда
    функция $f\,'$ монотонно не возрастает (не убывает) на $(\bar{x},\bar{\bar{x}})$.
    Функция $f$ строго вогнута (выпукла) на
    $\langle \bar{x},\bar{\bar{x}} \rangle$ тогда и только тогда, когда
     $f\,'$ монотонно убывает (возрастает) на $(\bar{x},\bar{\bar{x}})$.
\end{prop}                          (\remrk{здесь все правильно??????})

\begin{prop}
    \label{kasatel}
    Предположим, что функция $f$ дифференцируема на интервале
    $(\bar{x},\bar{\bar{x}})$.


    Эта функция выпукла (вогнута) на $(\bar{x},\bar{\bar{x}})$ тогда и
    только тогда, когда для любых точек $x,y\in(\bar{x},\bar{\bar{x}})$
    выполняется неравенство
    \[f(y)-f(x)\geqslant f\,'(x)(y-x)\]
    \[(f(y)-f(x)\leqslant f\,'(x)(y-x))\]
    и строго выпукла (строго вогнута) тогда и только тогда, когда
    для любых несовпадающих точек $x,y\in(\bar{x},\bar{\bar{x}})$
    выполняется неравенство
    \[f(y)-f(x)> f\,'(x)(y-x)\]
    \[(f(y)-f(x)< f\,'(x)(y-x)).\]
\end{prop}
    Геометрически это предложение говорит, что выпуклость
    (вогнутость) дифференцируемой на всем интервале
    $(\bar{x},\bar{\bar{x}})$ функции эквивалентна тому, что ее
    график лежит не ниже (выше) любой проведенной к нему касательной. При
    этом для строгой выпуклости (строгой вогнутости) функции
    необходимо и достаточно чтобы все точки графика, за исключением
    самой точки касания, лежали строго выше (строго ниже) этой
    касательной. (\remrk{РИС ???????})


\begin{prop}
    Предположим, что функция $f$ непрерывна на промежутке
    $\langle \bar{x},\bar{\bar{x}} \rangle$
     и дважды дифференцируема на внутренности этого промежутка, т.е. на
     интервале $(\bar{x},\bar{\bar{x}})$.

    Функция $f$ вогнута (выпукла) на
    $\langle \bar{x},\bar{\bar{x}} \rangle$ тогда и только тогда, когда
    \[f\,''(x)\leqslant0 \ \forall x\in(\bar{x},\bar{\bar{x}})\]
    \[(f\,''(x)\geqslant0 \ \forall x\in(\bar{x},\bar{\bar{x}})).\]
    Если
    \[f\,''(x)<0 \ \forall x\in(\bar{x},\bar{\bar{x}})\]
    \[(f\,''(x)>0 \ \forall x\in(\bar{x},\bar{\bar{x}})),\]
    то рассматриваемая функция строго вогнута (строго выпукла).
\end{prop}
    (\remrk{РИС ???????}) (\remrk{РИС ???????}) (\remrk{РИС ???????})

\begin{exer}
    Приведите пример строго выпуклой на $\R$ функции, вторая производная
    которой в некоторой точке обращается в ноль.
\end{exer}

\begin{exer}
    Проверьте следующие функции на (строгую) выпуклость и (строгую)
    вогнутость на множествах их определения при различных возможных
    значениях параметра $\alpha$:
    \begin{itemize}
      \item $f(x)=x^{\alpha}$;
      \item $f(x)=\alpha^{x}$;
      \item $f(x)=\log_{\alpha}x$.
    \end{itemize}
\end{exer}

\begin{exer}
    Докажите следующие соотношения:
    \begin{itemize}
      \item $e^{x}\geqslant1+x \ \forall x\in\R$, причем $e^{x}>1+x \  \forall   x\neq0$;
      \item $\ln x \leqslant x-1  \ \forall x\in\R$, причем $\ln x < x-1 \ \forall  x\neq1$.
    \end{itemize}

\end{exer}




    Перейдем к анализу задач на максимум и минимум для выпуклых и
    вогнутых функций.
    В следующем упражнении читателю предлагается доказать очень
    важное для экономической науки свойство выпуклых и вогнутых функций.


\begin{exer}
    Пусть $f$ представляет собой выпуклую (вогнутую) на интервале
    $\langle \bar{x},\bar{\bar{x}} \rangle$ функцию. Докажите, что любой локальный
    минимум (локальный максимум) функции $f$ на
    $\langle \bar{x},\bar{\bar{x}} \rangle$
    является глобальным минимумом (глобальным максимумом).
\end{exer}

    Другим важным свойством выпуклых и вогнутых функций является то,
    что для них равенство производной нулю в некоторой точке является достаточным
    условием минимума и максимума соответственно в случае, когда эта функция является
    внутренней. Более того, имеет
    место следующее предложение, для доказательства которого достаточно
    сослаться на предложение  \ref{kasatel}.
\begin{prop}
    Предположим, что $f$ --- это выпуклая (вогнутая) и дифференцируемая
    на промежутке $\langle \bar{x},\bar{\bar{x}} \rangle$ функция.

    Внутренняя точка $x^{*}$ промежутка  $\langle \bar{x},\bar{\bar{x}} \rangle$
    представляет собой точку локального и,
    следовательно, глобального минимума (максимума) функции $f$ на
    рассматриваемом промежутке тогда и только тогда, когда
    \[f\,'(x^{*})=0.\]

    В случае, когда промежуток $\langle \bar{x},\bar{\bar{x}} \rangle$ имеет
    вид $[\bar{x},\bar{\bar{x}})$ или $[\bar{x},\bar{\bar{x}}]$),
    точка $x^{*}=\bar{x}$ представляет собой точку локального и,
    следовательно, глобального минимума (максимума) функции $f$ на
    это промежутке
    тогда и только тогда, когда
    \[f\,'(\bar{x})\leqslant0 \ (f\,'(\bar{x})\geqslant0 ).\]

    В случае, когда промежуток $\langle \bar{x},\bar{\bar{x}} \rangle$
    имеет вид $(\bar{x},b]$ или
    $[\bar{x},\bar{\bar{x}}]$,
    точка $x^{*}=\bar{\bar{x}}$ представляет собой точку локального и,
    следовательно, глобального минимума (максимума) функции $f$ на
    рассматриваемом промежутке
    тогда и только тогда, когда
    \[f\,'(\bar{\bar{x}})\geqslant0 \ (f\,'(\bar{\bar{x}})\leqslant0).\]
\end{prop}

\begin{exer}
    Докажите, что если $g(x)$ --- монотонно возрастающая на
    $(-\infty,+\infty)$ функция, то функция $f(x)$ и $g(f(x))$ имеют
    одни и те же точки локального и глобального максимума и минимума.
\end{exer}

\begin{exer}
    Определите наибольшее значение произведения $m$-й и $n$-й
    степени ($m>0$, $n>0$) двух положительных чисел, сумма которых
    постоянна и равна $a$.
\end{exer}


\begin{exer}
    Из всех прямоугольников данной площади $S$ определите тот,
    периметр которого наименьший.
\end{exer}



\section{Спрос и предложение. Простейшие экономические задачи максимизации}

    В этом параграфе мы на некоторых важных примерах покажем, как
    можно применять только что приведенные математические
    конструкции к анализу экономических задач.

\subsection{Спрос и предложение. Цена равновесия}

    Рассмотрим рынок некоторого блага.
    Объем спроса на данное благо --- это то количество данного блага,
    которое один или несколько покупателей желают приобрести за некоторый единичный период времени.
    Этот объем зависит в первую очередь от цены данного блага, а также
    от многих иных факторов, включающих цены других благ, доходы
    потребителей и их вкусы. Объем предложения блага --- это количество блага,
    которое один или несколько продавцов желают продать за единичный период.
    Он тоже зависит от цены
    блага и других факторов, таких как имеющиеся в распоряжении производителей
    технологии по производству этого блага, а также от цен используемых в
    производстве ресурсов и их доступности. Далее мы будем все рассматриваемые блага
    считать товарами и считать понятия <<благо>> и <<товар>>
    взаимозаменяемыми синонимами. Зависимость спроса и предложения
    от различных факторов, которые их определяют, описывают с
    помощью функции спроса и предложения соответственно.

    В простейшей модели частичного равновесия, которую мы рассмотрим
    в этом пункте, предполагается,
    что все факторы, влияющие на спрос и предложение некоторого
    товара, за исключением его цены,  определяются экзогенно, т.е.
    вне рамок модели. В
    рамках этой модели исследуется  влияние  на спрос и предложение
    именно цены рассматриваемого товара.
    Зависимость между ценой товара $p$ и объемом спроса на этот товар
    задается функцией спроса $D(p)$, аргументом которой является
    только одна переменная --- его цена.
        Мы будем предполагать, что эта функция задана на всем множестве
    неотрицательных чисел $\R_{+}$ или некотором промежутке из этого множества,
    принимает неотрицательные значения и
    является убывающей в той области, где она принимает положительные значения,
    то есть что с ростом цены спрос падает (до тех пор пока не станет равной нулю).

    Функция предложения в $S(p)$ отражает зависимость предложения товара
    от его цены.
    Про нее мы тоже будем предполагать, что она задана на $\R_{+}$
    или некотором промежутке и
    принимает неотрицательные значения. Однако, в противоположность
    функции спроса, мы будем, как обычно, считать, что функция предложения
    является возрастающей в той области, где принимает положительные значения.

    В тех случаях, когда мы будем задавать функцию спроса или
    предложения с помощью некоторой формулы, то будем считать, что
    эта формула задает соответствующую функцию только при тех $p$,
    где значения функции, получаемые по формуле, неотрицательны или
    даже положительны, а
    при остальных $p$ значение функции спроса или предложения мы
    будем считать неопределенными либо равными нулю. Например, запись функции спроса в
    виде $D(p)=a-bp, \ a>0, \ b>0,$ означает, что этой формулой
    функция спроса задается этим равенством только при $0\leq p\leq a/b$
    и что при $p>a/b$ функция спроса либо не определена, либо  принимает
    нулевые значения.

    Под \emph{ценой равновесия} (равновесной ценой)
    понимается такая цена $p^{*}$, при которой объем суммарного спроса равен
    объему суммарного предложения. Иными словами, цена равновесия представляет
    собой решение уравнения
    \[S(p)=D(p),  \]
    где $D(p)$ и $S(p)$ --- функции суммарного спроса и предложения
    соответственно.
    (\remrk{РИС ???????})

    Если, например, функции суммарного спроса и предложения являются
    линейными и имеют вид
    \[D(p)=a-bp, \ a>0, \ b>0,\]
    \[S(p)=c+gp, \ 0<c<a, \ g>0,\]
    то ценой равновесия является величина
    \[p^{*}=\frac{a-c}{g+b}.\]

    В моделях частичного равновесия, когда идет речь о состоянии
    равновесия, обычно обычно имеется в виду либо просто сама по себе
    цена равновесия $p^{*}$, либо пара чисел $(p^{*},q^{*})$ где
    $q^{*}=D(p^{*})=S(p^{*})$ --- равновесное количество
    товара, продаваемого и покупаемого на рассматриваемом рынке.


    И еще одно замечание, объясняющее наше желание подчеркнуть то, что
    $D(p)$ и $S(p)$ --- это функции именно суммарного спроса и предложения
     Когда идет речь о функциях спроса и предложения на
    рынке какого-то товара, то имеют в виду, что спрос и предложение
    --- это суммарный спрос со стороны всех покупателей и суммарное
    предложение со стороны всех продавцов:
    \[D(p)=\sum_{i=1}^{n}D_{i}(p),\]
    \[S(p)=\sum_{j=1}^{m}S_{j}(p),\]
    где $n$ --- это общее количество покупателей, $m$ --- общее
    количество продавцов, $D_{i}(p)$ --- индивидуального функция
    спроса $i$-го потребителя, $S_{j}(p)$ ---
    функция индивидуального предложения $j$-го продавца.
    (\remrk{РИС ???????})

    Естественно под продавцом
    понимать производителя рассматриваемого продукта, а под
    покупателем --- потребителя (быть может производственного
    потребителя, использующего данный товар для производства какого-то другого
    товара). Функции спроса и предложения естественным образом
    возникают в результате решения задачи максимизации прибыли
    производителем и задачи максимизации полезности потребителем. К
    этим задачам мы перейдем чуть ниже, а сейчас скажем несколько
    слов о производственных функциях и функциях затрат.







\subsection{Однофакторная производственная функция. Функция затрат}

    Традиционным математическим аппаратом описания технологических
    возможностей фирмы или предприятия в микроэкономике являются
    \emph{производственные функции}, которые связывают затраты тех или иных
    факторов производства с выпуском продукции. В этом пункте мы
    обсудим однофакторные производственные  функции, т.е. такие, в
    которых аргументом являются затраты какого-то одного фактора.

    Предположим, что на некотором
    предприятии выпуск продукции зависит от количества имеющихся в
    его распоряжении зданий, станков и машин, а также от затрат еще
    одного фактора, например, труда. Предположим также, что количество
    зданий, станков и машин зафиксировано. В этом случае
    естественно предполагать, что объем выпускаемого продукта $q$
    (в натуральном или денежном выражении)
    зависит от затрат $x$ единственного оставшегося не зафиксированного фактора
    производства (в нашем случае --- от затрат труда). В этом случае мы можем записать:
    \[q=F(x).\]
    Функцию $F$ называют \emph{производственной функцией}. Поскольку
    в данном случае все факторы, за исключением одного,
    зафиксированы, мы ведем речь об \emph{однофакторной} производственной
    функции. Использование такой функции и интерпретация полученных
    результатов требует определенной  осторожности. Дело в том, что
    в данной ситуации мы вынуждены просто игнорировать другие факторы
    и проводить рассуждения так, как \emph{если бы} этих других
    факторов просто не было. Такого типа рассуждения в большинстве
    случаев вполне допустимы, но отдавать себе отчет в их большой
    степени условности нужно. Например, когда мы будем говорить о
    прибыли, нужно будет понимать, что затраты в рамках модели будут
    занижены, а прибыль завышена, поскольку затраты, связанные с использованием
    неучтенных факторов будут игнорироваться. Здесь сразу же надо подчеркнуть,
    что и само понятие прибыли, как разности доходов и расходов, не
    столь тривиально, как кажется на первый взгляд, ибо на вопрос
    о том, какие расходы нужно
    в явном виде вычитать из расходов, ответить совсем непросто.



    Итак, пусть нам дана однофакторная производственная функция
    $F(x)$.
    Мы <<забудем>> о существовании других факторов производства, а по
    поводу самой функции будем предполагать, что она непрерывна,
    монотонно не убывает (???????) и удовлетворяет естественному равенству
    $F(0)=0$.

    Величина $F(x)/x$
    называется \emph{средним продуктом}  ($AP$, average product), или
    средней производительностью
    рассматриваемого переменного фактора производства, а  величина
    $F\,'(x)$
    --- \emph{предельным продуктом} ($MP$, marginal product), или
    предельной производительностью этого фактора.

    В случае, когда функция $F\,'(x)$ является монотонно убывающей (?????),
    говорят об убывающей предельной производительности фактора
    (убывающем предельном продукте), а когда монотонно
    возрастающей --- о возрастающей. Возможна ситуация, когда при
    некоторых значениях $x$ производственная функция демонстрирует
    возрастающую предельную производительность, а при других ---
    убывающую.

     Если функция $F(x)$ дважды
    дифференцируема, то неравенство $F\,''(x)<0$ гарантирует убывающую
    предельная производительность (в точке $x$), а неравенство $F\,''(x)>0$
    --- возрастающую.
    (\remrk{РИС ???????}) (\remrk{РИС ???????})

\begin{exer}
    Объясните, почему производная $F\,'$ производственной функции $F$
    называется предельным продуктом. Какие доводы можно привести в пользу
    предположения об убывающей предельной производительности?
    \emph{Указание.} Используйте приближенное равенство
    \[F\,'(x)\approx\Delta q/\Delta x,\]
    где  $\Delta q=F(x+\Delta x)-F(x)$.
\end{exer}

\begin{exer}
    Постройте график производственной функции, вычислите средний и
    предельный продукт, постройте графики среднего и пердельного
    продукта, укажите, при каких значениях $x$ будет выполняться неравенства
    $F(x)/x>F\,'(x)$,
     $F(x)/x<F\,'(x)$, установите, имеет ли место убывающая предельная
     производительность для следующих производственных функций:
    \[F(x)=ax^{\alpha} , a>0, 0<\alpha<1;\]
    \[F(x)=x/(a+bx), a>0, b>0;\]
    \[F(x)=a\ln(bx+c), a>0, b>0, c\geqslant1;\]
    \[F(x)=a(b_{1}x^{\rho}+b_{2})^{-\rho}, a>0, b_{1}>0, b_{2}>0,  \rho<1;\]

    \[F(x)=\left\{
      \begin{array}{ll}
        0, & 0\leqslant x<v; \\
        (x-v)^{\alpha}, & x\geqslant v.
      \end{array}
    \right.\]

    \[F(x)=ae^{b-c/x}, a>0, b>0, c>0.\]
\end{exer}


    В некоторых случаях для описания производственных возможностей
    предприятия удобно использовать функцию
    затрат (функцию издержек) $C(q)$, которая в простейшем случае показывает, каковы у
    предприятия затраты в денежном выражении в зависимости
    от объема выпуска $q$ в натуральном выражении.
       Как обычно, мы будем предполагать, что функция затрат задана
    на некотором промежутке вида $[0,\tilde{q}), \ \tilde{q}\leqslant+\infty$,
    непрерывна, монотонно возрастает. Величина $C(0)$, называемая
    \emph{постоянными затратами} ($FC$, fixed cost), предполагается
    неотрицательной (при этом допускается, что $C(0)>0$).
    Величина $C(q)/q$ называется \emph{средними затратами} ($AC$, average cost), а величина
    $C\,'(q)$ --- \emph{предельными затратами} ($MC$, marginal cost).
    В случае, когда функция $C\,'(q)$
     являются монотонно возрастающей, говорят о
    возрастающих предельных затратах, а если монотонно убывающая, то об убывающих.

\begin{exer}
    Постройте график функции затрат, вычислите средние и предельные продукт,
    постройте графики средних и предельных затрат, укажите, при каких значениях $q$
     будет выполняться неравенства $C(q)/q>C\,'(q)$, $C(q)/q<C\,'(q)$, установите, выполняется ли закон
     возрастающих предельных затрат для следующих функций затрат:
    \[C(q)=aq^{\alpha}, a>0,  \alpha>1;\]
    \[C(q)=aq^{\alpha}+b, a>0, b>0,  \alpha>1;\]
    \[C(q)=ae^{bq}+c, c\geq0, a>0, b>0;\]
    \[C(q)=q/(a-bq)+c, a>0, b>0, c\geqslant0;\]
    \[C(q)=a/\ln(c-bq)+d, a>0, b>0, c\geqslant0, d\geqslant0;\]
    \[C(q)=q^{3}-9q^{2}+43q+6.\]
\end{exer}


    Если производственные возможности предприятия описываются
    некоторой монотонно возрастающей однофакторной производственной
    функцией $F(x)$, выпуск $q$ измеряется в натуральном выражении, а цена
    переменного фактора задана на некотором
    фиксированном уровне $\bar{p}$, то в качестве функция затрат
    может выступать функция
    \[C(q)=\bar{p}F^{-1}(q)+C(0).\]

\begin{exer}
    Объясните содержательный смысл каждого из слагаемых суммы
$pF^{-1}(q)+C(0)$.
\end{exer}
    (\remrk{РИС ???????})


\subsection{Валовой доход}

    Предположим, что фирма производит некоторый продукт в количестве
    $q$ в натуральном выражении, и
    продает его на рынке по цене $p$. В этом случае ее валовой доход
    $R(q)$ равен объему продаж в денежном выражении:
\begin{equation}
\label{val-prod}
    R=R(q)=pq.
\end{equation}
    По поводу цены $p$ на производимый продукт либо предполагается,
    что она является экзогенно заданной с точки зрения
    производителя:
\begin{equation}
\label{sov-knk}
    p=\text{const},
\end{equation}
    либо, что производитель в явном виде может влиять на цену,
    учитывая ее зависимость от объема его производства (продаж):
\begin{equation}
\label{nesov-knk}
    p=P^{D}(q).
\end{equation}
    Экономические агенты, потребители и производители, действующие
    на рынке некоторого продукта или ресурса, которые рассматривают
    цену этого продукта или ресурса как экзогенно заданную,
    называются ценополучателями (price takers). В случае, когда на
    некотором рынке все потребители и производители являются
    \emph{ценополучателями}, говорят, что имеет место
    \emph{совершенная конкуренция} (perfect competition). В противном
    случае хотя бы некоторые производители или потребители обладают некоторой рыночной
    властью (market power) и имеет место \emph{несовершенная конкуренция}
    (imperfect competition). Например, если наш производитель
    является монополистом на рынке производимого товара, то
    соотношение между объемом его производства и продаж $q$ с одной
    стороны, и ценой $p$, по которой этот товар будет продаваться,
    задается функцией спроса, т.е. равенством $q=D(p)$. В
    этом случае функция $P^{D}(q)$, которая фигурирует в
    (\ref{nesov-knk}), --- это просто обратная к $D(p)$ функция:
    \[P^{D}(q)=D^{-1}(q).\]    (\remrk{РИС ???????})
    Она называется обратной функцией спроса. Естественно, при
    использовании обратной функции спроса следует проявлять
    определенную осторожность, связанную с тем, что задать обратную
    к некоторой функции можно задать только в том случае, если прямая
    функция обратима, например
    монотонно убывает на некотором промежутке. Из контекста всегда
    будет понятно, о чем идет речь.
    Например, когда функция спроса задается равенством
    $D^{D}(p)=a-bp, \ a, b>0$ (при
    $0\leqslant p\leqslant a/b$),
    обратная функция спроса имеет вид
    $P^{D}(q)=(a-q)/b$ и определена при $0\leqslant q\leqslant a$. Конечно, ее
    можно доопределить на все множество неотрицательных чисел,
    положив $P^{D}(q)=0$ для $q>a$.




    Качественное различие между случаями совершенной и несовершенной
    конкуренции очень важно и его следует
    всегда учитывать, хотя, формально говоря, равенство (\ref{sov-knk})
    можно считать частным случаем соотношения (\ref{nesov-knk}).



    Обычно считается, что совершенная конкуренция имеет место тогда,
    когда на рынке присутствует довольно много продавцов и
    покупателей, причем каждый из них занимает незначительную долю
    рынка. Однако это не мешает простоты ради моделировать ситуацию
    совершенной  конкуренции, включая в модель совсем небольшое количество
    участников. Например, можно считать, что на рынке присутствует
    только два продавца или даже один. Важно только в явном виде сделать
    предположение, что каждый из них рассматривает цену как экзогенно заданную.







    Величина  $R\,'(q)$
    называется \emph{предельным доходом} ($MR$, marginal revenue).




\begin{exer}
Вывести функцию $R(q)$ и предельный доход в случае, когда
производитель является монополистом, а функция спроса задается
равенствами
    \[D(p)=a/p, a>0;\]
    \[D(p)=ap^{-\alpha} , a>0,  \alpha>0;\]
    \[D(p)=a-p^{1/2}, a>0;\]
    \[D(p)=(-a)\ln p, a>0;\]
    \[D(p)= ae^{-\alpha p}, a>0,  \alpha>0.\]
\end{exer}






\subsection {Максимизация прибыли}
    В простейших версиях теории фирмы предполагается, что при определении
    объема производства ее менеджеры
     пытаются максимизировать прибыль $\pi$, определяемую как  разность суммарного
      дохода фирмы и полных затрат (в дальнейшем мы обсудим и уточним это предположение).
     Предположим, что и суммарный доход и полные затраты являются
    функциями только одной переменной, а именно, объема выпуска
    $q$ в натуральном выражении.
    В этом случае и прибыль $\pi$ тоже можно рассматривать как  функцию объема
    выпуска:
    \[ \pi=\pi(q)=R(q)-C(q).\]
    Мы будем считать, что областью определения функций $R(q)$  и $C(q)$
    является $\R_{+}$, а также, что функции $R(q)$ и $C(q)$
    являются дважды дифференцируемыми при положительных $q$.

    Предположим, что при нулевом выпуске суммарный доход не больше
    полных затрат, т.е. выполняется неравенство
    \[R(0)\leqslant C(0),\]
    при этом для некоторого уровня выпуска $\bar{q}$ суммарный доход выше
    полных затрат:
    \[R(q_{1})>C(\bar{q}),\]
    а также, что при всех достаточно больших $q$ затраты превосходят
    валовой доход:
    \[C(q)>R(q).\]

    Из сделанных предположений следует, что максимум  $\pi(q)$ на $\R_{+}$
    достигаться в некоторой ненулевой точке. В нуле он не может
    достигаться, поскольку
    \[\pi(\bar{q})>\pi(0).\]
    Поэтому необходимым условием максимума прибыли является равенство
    \[ \pi\,'(q)=R\,'(q)-C\,'(q)=0,\]
    которое можно переписать в виде
    \[R\,'(q)=C\,'(q).\]
    (\remrk{РИС ???????})

\begin{exer}
    Предположим, что рассматриваемая фирма является монополией
    на рынке некоторого продукта. Функция спроса на этот продукт имеет
    следующий вид:
    \[D(p)=10-p/4,\]
    а функция затрат этой фирмы задается равенством
    \[C(q)=q^{3}-9q^{2}+43q+6.\]
    Найти максимальное значение прибыли, которое может получить фирма,
    указать уровень выпуска $q^{*}$ и цену $p^{*}$, которые обеспечивают максимум прибыли.
\end{exer}

    В случае, когда рассматриваемый производитель является ценополучателем,
    т.е. цена продукта $p$ воспринимается им
    как заданная, задача о максимизации прибыли может быть записана
    в виде
\begin{equation}
    \label{max-prib1}
    pq-C(q)\rightarrow\max, \ q\geqslant0.
\end{equation}
    Необходимые условия максимума для этой задачи имеют вид
\begin{equation}
    \label{max-prib1-uslopt}
    C\,'(q)=p.
\end{equation}                 (\remrk{РИС ???????})
    Если, кроме того, в дополнение к сделанным предположениям мы постулируем,
    что функция затрат имеет возрастающие предельные затраты, т.е.
    является строго выпуклой, то последнее равенство является не только
    необходимым, но и достаточным условием максимума прибыли.

    Запись задачи о максимизации прибыли в виде (\ref{max-prib1})
    позволяет вывести функцию предложения того продукта, который
    производит рассматриваемая фирма со стороны этой фирмы. Для
    этого достаточно рассмотреть цену продукта $p$ как параметр, а
    решение задачи (\ref{max-prib1}) --- как зависящее от этого
    параметра.
    Функция $S(p)$, которая ставит в соответствие цене $p$
    решение задачи (\ref{max-prib1}),
    и представляет собой искомую функцию предложения. Конечно, эта
    функция определена только при тех $p$, для которых решение
    существует и единственно.

    В случае, когда функция затрат $C(q)$, заданная
    на промежутке вида $[0,\tilde{q}), \ \tilde{q}\leq+\infty$, строго вогнута на нем
    функция предложения задается на промежутке
    $[C\,'(0),\lim_{Q\rightarrow+\tilde{C}}C\,'(q))$
    равенством
    \[S(p)=(C\,')^{-1}(p).\]          (\remrk{РИС ???????})
    Естественно, что $S(p)=0$ при $p<C\,'(0)$.




    Мы уже отмечали, что иногда вместо функции затрат удобно
    использовать не саму функцию предложения $S(p)$, а обратную функцию
    предложения $P^{S}(q)$. В рассматриваемом нами
    случае, когда предложение рассматриваемого продукта выводится из задачи
    максимизации прибыли (\ref{max-prib1}), такая функция устроена
    совсем просто:
    \[P^{S}(q)=C\,'(q).\]
    Более того, последнее равенство определяет обратную функцию
    спроса и тогда, когда функция затрат вогнута, но не строго
    вогнута. Важным примером такой функции является линейная функция
    затрат вида $C(q)=cq, \ c>0.$          (\remrk{РИС ???????})


\begin{exer}
    Для каких функций затрат $C(q)$ функция предложения, порожденная
    задачей (\ref{max-prib1}) является линейной?
\end{exer}





    В случае, когда мы описываем производственные возможности
    предприятия с помощью однофакторной производственной функции
    $F(x)$, прибыль $\pi$ удобно считать функцией переменной $x$,
    а задача о максимизации прибыли может быть записана в
    следующем виде:
\begin{equation}
\label{max-prib-2}
    P^{D}(F(x))F(x)-px\rightarrow\max.
\end{equation}
    Здесь предполагается, что предприятие является ценополучателем
    на рынке ресурса.
     Мы уже указывали, что
    поскольку использование однофакторной производственной функции
    означает, что в своих рассуждениях мы <<забываем>> обо всех
    факторах производства, за исключением переменного. Поэтому
    интерпретировать величину $P^{D}(F(x))F(x)-px$ как прибыль можно
    только со значительной долей условности, хотя вполне уместно с
    точки зрения тех задач, которые мы здесь перед собой ставим.





    С помощью решения задачи
    (\ref{max-prib-2}) мы можем вывести
    функцию спроса со стороны рассматриваемого предприятия на этот
    ресурс. А именно, в качестве такой функции мы можем взять
    функцию $D(p)$, которая ставит в соответствие цене ресурса
    $p$ решение задачи (\ref{max-prib-2}) (с учетом оговорок о
    существовании и единственности). В случае, когда цена выпускаемого
    продукта рассматривается производителем как экзогенно заданная
    ($P(F(x))=\tilde{p}$) и, следовательно, задача (\ref{max-prib-2})
    записывается в виде
    \begin{equation}
\label{max-prib-3}
    \tilde{p}F(x)-px\rightarrow\max,
\end{equation}
    а функция $F(x)$ дифференцируема и строго вогнута,
    причем $F'(0)=+\infty$ то необходимым и
    достаточным условием максимума прибыли является равенство
    \[\tilde{p}F\,'(x)=p,\]
    а функция спроса $D(p)$ задается равенством
    \[D(p)=(F\,')^{-1}(p/\tilde{p}),\]
    причем если выпуск измеряется в денежном выражении, т.е. $\tilde{p}=1$,
    то последнее равенство приобретает следующий вид:
    \[D(p)=(F\,')^{-1}(p).\]

\begin{exer}
    Как выглядит функция спроса, если $F'(0)<+\infty$.
\end{exer}




\subsection{Простейшая задача потребителя}
    В предыдущем пункте мы увидели, как из задачи максимизации
    прибыли можно вывести функцию предложения некоторого продукта со
    стороны отдельного производителя этого продукта и
    функцию спроса на некоторый ресурс, который выступает в качестве
    фактора производства.

    Функцию спроса на некоторое потребительское благо со стороны отдельного
    потребителя тоже можно вывести из некоторой задачи на максимум.
    Предположим, что потребление некоторого блага
    приносит рассматриваемому потребителю полезность, которую он
    оценивает с помощью функции $u(c)$, заданной на $\R_{+}$. Эта функция
    ставит  в соответствие объему потребления $c$ рассматриваемого
    блага полезность $u(c)$ от этого потребления. При этом эта полезность
    измеряется нашим потребителей в денежных единицах (безусловно,
    это предположение является очень сильным и ограничительным).
    Допуская некоторую вольность мы можем называть функцию $u(c)$
    функцией полезности (вскоре читатель поймет, в чем здесь состоит
    вольность).


    Мы будем считать, что функция $u(c)$, заданная на $\R_{+}$ или на некотором
    промежутке вида $[0,\tilde{c}], \ \tilde{c}<+\infty$, дифференцируема
    (причем
    мы допускаем, что $u\,'(0)=+\infty$), монотонно возрастает,
    строго вогнута (и, следовательно, $u\,'(c)$ монотонно убывает), а также,
    что $u\,'(\tilde{c})=0$ или $u\,'(c)\rightarrow 0$ при $c\rightarrow+\infty$. Если
    называть величину $u\,'(c)$ предельной полезностью, то строгая
    вогнутость функции $u(c)$ и убывающая предельная полезность суть
    одно и то же.

    Предположим, что нашему потребителю рассматриваемое
    благо достается не даром и его надо приобретать по некоторой
    цене $p>0$. Тем самым его расходы на приобретение $c$ единиц
    блага составят $pc$ денежных единиц. Естественно,
    тот факт, что за приобретение блага надо платить, не может
    радовать потребителя и поэтому можно считать, что факт покупки
    наносит ему <<ущерб>> в размере расходов, т.е. $pc$ денежных единиц.
    Предположим также, что потребитель является
    ценополучателем на рынке рассматриваемого блага.

    Вопрос о том, в каком количестве $c$ покупать и потреблять благо,
    наш потребитель решает на основе соизмерения полезности $u(c)$ от
    потребления и <<ущерба>> от покупки. Поскольку мы предположили,
    что полезность от потребления измеряется в денежных единицах, то
    совокупный эффект от покупки и потребления $c$ единиц блага
    составит $u(c)-pc$ денежных единиц, а спрос $D(p)$ нашего
    потребителя на рассматриваемое благо в зависимости от цены
    последнего $p$ естественно определяется как решение задачи
\begin{equation}
    \label{z-p}
    u(c)-pc\rightarrow\max, c\geqslant0.
\end{equation}
    Если $p<u\,'(0)$, то необходимыми и достаточными условиями
    максимума для этой задачи является равенство \[u\,'(c)=p.\]
    Отсюда следует, что функция спроса задается следующим образом:
\begin{equation}
    \label{f-sp-odn}
    D(p)=\left\{
             \begin{array}{ll}
               (u\,')^{-1}(p), & {p<u\,'(0);} \\
               0, & {p\geqslant u\,'(0).}
             \end{array}
           \right.
\end{equation}


\begin{exer}
    Для каких функций $u(c)$ функции спроса, построенные с помощью
    решения задачи (\ref{z-p}), является линейными?
\end{exer}

    Читатель, наверно помнит из начального курса микроэкономики, что
    стандартное предположение о поведении потребителя состоит в
    том, что он решает задачу о максимизации полезности при бюджетном
    ограничении, а совсем не задачу (\ref{z-p}) (именно поэтому называть
    $u(c)$ функцией полезности --- это некоторая вольность). В дальнейшем мы
    увидим, что решение этой последней задачи эквивалентно решению
    задачи максимизации некоторой специальной функции полезности при
    бюджетном ограничении.

    И еще одно важное замечание. Мы не делали предположения, что функция
    $u(c)$ принимает только неотрицательные значения. Мы это сделали
    совершенно сознательно, поскольку задача (\ref{z-p}) вполне
    осмысленна и без этого предположения. Кроме того, зачастую
    удобно предполагать, что функция $u(c)$ задана только на
    множестве положительных чисел причем $u(c)\rightarrow-\infty$ и
    $u\,'(c)\rightarrow+\infty$ при $c\rightarrow 0$, как, например, в случае
    \[u(c)=\ln c\]
    или
    \[u(c)=\frac{c^{\,\rho}}{\rho}, \ \rho<0\]
    (в этом последнем случае функция $u(c)$ вообще принимает только отрицательные
    значения).
    Для таких функций все проведенные рассуждения тоже верны. При этом с
     формальной точки зрения мы можем просто считать, что функция
    $u(c)$ (как и $u\,'(c)$) задана на всем множестве $\R_{+}$, но принимает значения
    на расширенной вещественной прямой $\bar{\R}=\R\cup\{-\infty\}\cup\{+\infty\}$. В
    этом случае мы просто можем считать, что $u(0)=-\infty$ и
    $u\,'(0)=+\infty$. При этом мы не потеряем ни непрерывность, ни
    выпуклость, ни дифференцируемость функции $u(c)$.












\subsection{Эластичность}
    Понятие эластичности предназначено для того, чтобы отразить степень
    чувствительности значения той или иной функции к изменению значения
    аргумента. Если, например, значение некоторой функция $f(x)$ не меняется с
    изменением аргумента, т.е.
    \[f(x)=\text{const},\]
    то про эту функцию естественно говорить как об абсолютно
    неэластичной. Если некоторая функция очень быстро возрастает или
    убывает с ростом аргумента, то такую функцию естественно назвать
    эластичной. На первый взгляд, удобно оценивать степень эластичности
    функции по величине ее производной.
    Однако, это впечатление обманчиво, поскольку величина производной
    зависит от единиц измерения в которых мы меряем величину как аргумента,
    так и значение функции. Предположим, например, что мы хотим ответить на вопрос,
    какая из функций спроса более чувствительна к изменению цен, функция
    спроса на зерно или функция спроса на чайники. Брать производные в
    данном случае просто бессмысленно, в частности, потому, что количество
     зерна, скорее всего, измеряется в тоннах, а количество чайников --- в
     штуках, да и измерение чайников в тоннах делу не поможет.


     В связи с этим в качестве показателя, отражающего степень
     чувствительности значения функции $f(x)$ к изменению
     аргумента $x$ часто используется показатель эластичности $\varepsilon_{f}$,
     который определяется следующим образом (????????):
    \[\varepsilon_{f}=\varepsilon_{f}(x)=f'(x)\left/\frac{f(x)}{x}\right. .\]
    Здесь предполагается, что функция $f(x)$ дифференцируема (иначе
    эластичность на может быть определена). Легко заметить, что
    поскольку величина $f'(x)$ имеет те же единицы измерения,
    что и величина $f(x)/x$,  эластичность является безразмерной величиной,
    т.е. не зависит от единиц измерения. Важно
    также подчеркнуть, что значение эластичности зависит от значения
    аргумента, т.е. $\varepsilon_{f}(x)$ --- это функция переменной $x$, хотя в тех
    случаях, когда используется обозначение $\varepsilon_{f}$, это может оказаться
    незамеченным. Областью определения функции $\varepsilon_{f}(x)$, вообще говоря,
    меньше области определения функции $f(x)$, поскольку в нее не входит
    $x=0$. Кроме того, обычно эластичность определяют для функций,
    принимающих только положительные значения.




\begin{exer}
        Предположим, что функции спроса и предложения
    на некоторый продукт линейны:
    \[D(p)=a-bp, \ S(p)=c+dp.\]
    Являются ли эластичности этих функций монотонно возрастающими
    или монотонно убывающими функциями цены $p$?
\end{exer}

\begin{exer}
    Вычислите эластичность производственной функции
    \[F(x)=ax^{\alpha}, \ a>0, \ \alpha>0,\]
    и объясните содержательный смысл полученного результата.
\end{exer}

\begin{exer}
    Докажите, что если для некоторой функции
    $g$, заданной на $(0,+\infty)$ и принимающей положительные значения
    эластичность постоянна:
    \[\varepsilon_{g}(x)=\gamma \ \forall x>0,\]
    то эта функция имеет вид
    \[g(x)=bx^{\gamma}, \ b>0.\]
\end{exer}

\begin{exer}
    Приведите пример функции затрат $C(q)$, для которой функция предложения, порожденная
    задачей (\ref{max-prib1}), имеет постоянную эластичность.
\end{exer}

\begin{exer}
    Приведите пример функции $u(c)$, для которой функция спроса, построенная с помощью
    решения задачи (\ref{z-p}), имеет постоянную эластичность.
\end{exer}







    Иногда говорят, что эластичность функции $f(x)$ в
    некоторой точке $\hat{x}>0$ показывает, на сколько процентов
    изменится значение функции при увеличении значения аргумента
    на один процент. Объясним это утверждение.


         Пусть $\hat{y}=f(\hat{x})$. Нам известно, что при
         <<малом>> приращении аргумента  $x$ для приращения значения
         функции  $\Delta y=f(\hat{x}+\Delta x)-\hat{y}$ выполняется следующее
         приближенное равенство:
     \[\Delta y/\Delta x\approx f'(\hat{x}).\]
    Тем самым, для эластичности $\varepsilon_{f}(\hat{x})$ выполняются следующие
    соотношения:
    \[\varepsilon_{f}(\hat{x})=f'(\hat{x})\left/\frac{\hat{y}}{\hat{x}} \right.\approx
    \frac{\Delta y}{\Delta x}\left/\frac{\hat{y}}{\hat{x}} \right.=
    \frac{\Delta y}{\hat{y}}\left/\frac{\Delta x}{\hat{x}} \right. .\]


    Если мы считаем, что приращение  $\Delta x$ <<достаточно мало>>, то
    последнее приближенное равенство даст нам достаточно точное
    значение эластичности. Если в качестве <<малого приращения>>
    аргумента мы возьмем однопроцентное приращение  $\Delta x=0.01\hat{x}$,
    тогда это приближенное равенство приобретет следующий вид:
    \[\varepsilon_{f}(\hat{x})\approx100\frac{\Delta y}{y} \]
    а величина $100\Delta y/\hat{y}$ как раз и показывает процентное изменение
    значения функции.

    Формально говоря, про функцию говорят, что она \emph{эластична} (в некоторой точке), если
    абсолютное значение показателя эластичности (в этой точке) больше
    единицы, и что \emph{неэластична}, если это значение меньше единицы.









\subsection{Оптимальное распределение выпуска между двумя предприятиями}

    Хотя задача о максимизации прибыли является важнейшей задачей
    для фирмы, некоторые другие задачи на максимум тоже могут
    оказаться для фирмы важными. Одну из задач
    такого типа мы рассмотрим здесь.

    Предположим, что некоторая фирма состоит из двух предприятий,
    производящих одинаковый продукт. Предположим далее, что нам заданы
    функции затрат этих предприятий: функция затрат первого предприятия
     $C_{1}(x)$ и функция затрат второго предприятия $C_{2}(y)$, где $x$ и $y$ --- это
     объемы выпуска на первом и втором предприятиях соответственно.
    Будем считать, что обе функции затрат дифференцируемы и строго
    выпуклы.

В случае,
     когда известен суммарный объем выпуска $q$, то самая естественная
     задача, которую необходимо решить --- это задача распределения объемов
     выпуска между предприятиями, т.е. задача об отыскании таких неотрицательных
      чисел $x\geqslant0$ и $y\geqslant0$, которые удовлетворяют равенству
\begin{equation}
    \label{sum-vip1}
    x+y=q.
\end{equation}
    Из каких соображений будут исходить руководители фирмы при решении этой задачи?
    По-видимому, в первую очередь им нужно заботиться о том, чтобы распределение
    объемов выпуска было таким, чтобы суммарные затраты фирмы оказались наименьшими.
    Именно такое распределение объемов выпуска будет оптимальным для фирмы в целом.
    Иными словами, им необходимо отыскать такие числа
    $x^{*}\geqslant0$ и $y^{*}\geqslant0$, которые
    удовлетворяют равенству
\begin{equation}
    \label{sum-vip2}
    x^{*}+y^{*}=q
\end{equation}
    и при этом обеспечивали бы наименьшее значение суммы
    \[C_{1}(x)+C_{2}(y)\]
    среди всех таких $x\geqslant0$ и $y\geqslant0$, которые
    удовлетворяют равенству (\ref{sum-vip1}).
    Если заметить, что равенство (\ref{sum-vip1}) можно переписать в виде
    \[y=q-x,\]
    то мы увидим, что необходимо решать следующую задачу:
\begin{equation}
\label{tsel-fun}
    F(x)=C_{1}(x)+C_{2}(q-x)\rightarrow\min, \ x\in[0,q].
\end{equation}
\begin{exer}
    Докажите, что из строгой выпуклости обеих функций затрат
    вытекает строгая выпуклость функции $F(x)$ (??????????).
\end{exer}

 В этом случае решением задачи (\ref{tsel-fun}) является либо решение уравнения
\begin{equation}
    \label{sum-vip3}
    F\,'(x)=0
\end{equation}
    (если оно найдется на интервале $(0,q)$), либо одна из граничных
    точек отрезка $[0,q]$: $x=0$ или $x=q$.

    Поскольку
    \[F\,'(x)=C_{1}'(x)-C_{2}'(q-x),\]
    уравнение (\ref{sum-vip3}) можно переписать в следующем виде
    \[C^{\,\prime}_{1}(x)=C^{\,\prime}_{2}(q-x).\]                 (\remrk{РИС ???????})
    Это уравнение имеет естественную экономическую интерпретацию.
    Если $x^{*}\in(0,1)$ является решением этого уравнения, то в случае когда
    объем производства на первом предприятии равен $x^{*}$, а выпуск на
    втором --- $y^{*}=q-x^{*}$, то предельные затраты на предприятиях будут совпадать.
    Итак, для того чтобы числа $x^{*}>0$ и $y^{*}>0$, удовлетворяющие равенству (\ref{sum-vip2}),
    представляли собой оптимальное распределение объемов выпуска для фирмы
    в целом, необходимо и достаточно, чтобы они удовлетворяли равенству
    \[C_{1}^{\,\prime}(x^{*})=C_{2}^{\,\prime}(y^{*}).\]
    Что касается граничных случаев, то необходимым и достаточным условием того, что пара
    чисел $x^{*}=0$, $y^{*}=q$ представляла собой оптимальное распределение объемов
    выпуска для фирмы в целом, необходимо чтобы выполнялось неравенство
    \[F\,'(0)\geqslant0,\]
    которое можно переписать в виде
    \[C_{1}^{\,\prime}(0)\geqslant C_{2}^{\,\prime}(q).\]
    Симметрично, необходимым и достаточным условием того, что пара
    $x^{*}=q$, $y^{*}=0$ задавала
    оптимальное распределение выпуска, является неравенство
    \[F\,(q)\leqslant0,\]
    которое можно переписать в виде
    \[C_{1}^{\,\prime}(q)\leqslant C_{2}^{\,\prime}(0).\]         (\remrk{РИС ???????})

\begin{exer}
    Объясните экономический смысл сформулированных условий
оптимальности для задачи распределения выпуска.
\end{exer}

\begin{exer}
    Предположим, что функция затрат первого предприятия имеет вид
    \[C_{1}(x)=x^{2}+1,\]
    а функция затрат второго предприятия ---
    \[C^{\,2}(y)=2y^{2}+3.\]
    Каким будет оптимальное распределение объемов выпуска между
    предприятиями в зависимости от объема выпуска?
\end{exer}

\begin{exer}
    Предположим, что некоторая фирма состоит из двух предприятий.
     Предположим далее, что нам заданы однофакторные производственные функции
    первого и второго предприятия: $F_{1}(x)$ и $F_{2}(y)$,
    показывающие выпуск в денежном выражении в зависимости от затрат
    некоторого вида одинакового для обоих предприятий ресурса,
     который имеется в распоряжении фирмы в некотором фиксированном количестве $b$.
     Сформулируйте задачу распределения этого ресурса в целях
    максимизации суммарного выпуска, а также сформулируйте и проинтерпретируйте
    необходимые и достаточные условия максимума в предположении, что
    производственные функции дифференцируемы, монотонно возрастают и
    строго вогнуты. Решите ее для случая, когда
    $F_{1}(x)=3\sqrt{x}$, $F_{2}(y)=4\sqrt{y}$ и $b=100$.
\end{exer}




















\subsection{Эффективная заработная плата}

    В этом пункте мы рассмотрим содержательную с экономической точки зрения
    задачу на максимум, целевая функция которой заведомо не вогнута
    (\remrk{квазивогнута????}), но
    локальный максимум является глобальным и равенство нулю
    производной является необходимым и достаточным условием
    максимума.

    Одним из важнейших
    факторов производства является рабочая сила, затраты которой которой можно
    измерять, например, в отработанных работниками человеко-часах.
    Однако, такое описание не всегда адекватно моделируемой
    ситуации, поскольку выпуск продукции зависит не только от
    количества  отработанных человеко-часов $L$, но и от усилий $\varphi$,
    предпринимаемых работниками. С учетом этого обстоятельства фактором
    производства следует считать величину \emph{эффективной рабочей силы},
    определяемой как произведения $\varphi L$ (здесь предполагается, что
    каждый работник в каждый час своей работы предпринимает одинаковые усилия
    $\varphi$).

    В некоторых моделях предполагают, что эти усилия
    зависят от размера ставки заработной платы $w$:
    \[\varphi=\varphi(w),\]
    где $\varphi(w)$ --- это некоторая непрерывная
    монотонно неубывающая функция. В этом предположении естественно
    возникает задача об отыскании \emph{эффективной заработной
    платы} (efficiency wage), которая обеспечивает наилучшее с точки
    зрения работодателя соотношение между оплатой труда и
    предпринимаемыми усилиями. Иными словами, речь идет об отыскании
    ставки заработной платы, обеспечивающей максимальное значение
    величине $\varphi(w)/w$.

    Мы рассмотрим эту задачу в предположении, что функция $\varphi(w)$
    задана на  множестве $[\tilde{v},+\infty)$, где $\tilde{v}>0$, дифференцируема,
    удовлетворяет соотношениям  $\varphi(\tilde{v})=0$,
    $\varphi\,'(w)>0 \ \forall w\geqslant \tilde{v}$, строго вогнута
    ($\varphi\,'(w)$ монотонно убывает),  и обладает тем
    свойством, что $\varphi\,'(w)\rightarrow+0$ при
    $w\rightarrow+\infty$.



\begin{exer}
    Объясните содержательный смысл сделанных предположений.
\end{exer}

    Итак, рассмотрим задачу
\begin{equation}
\label{e-w}
    \frac{\varphi(w)}{w}\rightarrow\max, \ w\geqslant \tilde{v}
\end{equation}
    и докажем следующее предложение.
\begin{prop}
\label{e-w1}
    Решение задачи (\ref{e-w}) существует и единственно. Число $w^{*}$ является
    таким решением тогда и только тогда, когда оно представляет
    собой решение уравнения
\begin{equation}
\label{e-w2}
    \varphi\,'(w)=\frac{\varphi(w)}{w}.
\end{equation}
\end{prop}                  (\remrk{РИС ???????})

    Заметим, что уравнение (\ref{e-w2}) можно переписать в следующим
    образом с использованием понятия эластичности:
    \[\varepsilon_{\varphi}(w)=1.\]

    \emph{Доказательство.} Поскольку функция $\varphi(w)$ непрерывна,
    $\varphi(\tilde{v})=0$ и $\varphi(w)>0$ при некоторых $w$, то для доказательства
    существования решения задачи (\ref{e-w}) достаточно показать,
    что
    \[\frac{\varphi(w)}{w}\rightarrow0 \ \text{при} \ w\rightarrow+\infty,\]
    т.е. что для любого $\epsilon>0$ при всех достаточно больших $x$
    выполняется неравенство
    \[\frac{\varphi(w)}{w}<\epsilon,\]
    которое можно переписать в следующем виде:
\begin{equation}
    \label{ner-e-w}
    \varphi(w)-\epsilon w<0.
\end{equation}
    Зафиксируем $\epsilon\in(0,\varphi\,'(0))$ и обозначим через
    $w_{\epsilon}$  решение уравнения
    \[\varphi\,'(w)=\frac{\epsilon}{2}.\]
    В силу строгой вогнутости функции $\varphi(w)$ для любого $w\neq w_{\epsilon}$ мы имеем
    \[\varphi(w)-\varphi(w_{\epsilon})<\varphi\,'(w_{\epsilon})(w-w_{\epsilon})
    =\frac{\epsilon}{2}(w-w_{\epsilon}),\]
    или, что то же,
    \[\varphi(w)-\epsilon w<\varphi(w_{\epsilon})
    -\frac{\epsilon}{2}w_{\epsilon}-\frac{\epsilon}{2}w.\]
    При достаточно больших $w$ правая часть последнего неравенства
    становится отрицательной и следовательно, выполняется требуемое
    неравенство (\ref{ner-e-w}).

    Мы показали, что решение $w^{*}$ задачи (\ref{e-w}) существует.
    Оно, очевидно, является решением уравнения
    \[\left(\frac{\varphi(w)}{w}\right)'=0,\]
    эквивалентного уравнению (\ref{e-w2}).
    Наоборот, если $w^{*}$ представляет собой решение уравнения
    (\ref{e-w2}), то в силу строгой вогнутости функции $\varphi(w)$ для любого
    $w\neq w^{*}$ мы имеем:
    \[\varphi(w)-\varphi(w^{*})<\varphi'(w^{*})(w-w^{*})=\frac{\varphi(w^{*})}{w^{*}}(w-w^{*})\]
    и, значит,
    \[\frac{\varphi(w^{*})}{w^{*}}>\frac{\varphi(w)}{w}. \ \blacksquare\]

\begin{exer}
    Вычислите $w^{*}$ в случае, когда
    \[\varphi(w)=\ln(\lambda w).\]
\end{exer}


\begin{exer}
    \label{e-w-primer1}
    Покажите, что в в случае, когда
    \[\varphi(w)=\kappa(w-\tilde{v})^{\eta}, \ v>0, \ \kappa>0, \ 0<\eta<1.\]
    решением задачи (\ref{e-w}) является
    \[w^{*}=\frac{\eta}{1-\eta}\tilde{v}.\]
\end{exer}





















































\section{Элементарное введение в теорию экстремальных задач}


    В предыдущем параграфе мы рассматривали экстремальные задачи для
    функций одной переменной.
      В этом параграфе мы дадим читателю некоторое первоначальное
    представление о задачах на максимум и минимум функций нескольких
    переменных.





    Наша цель в этом параграфе состоит не в том, чтобы самым
    аккуратным образом исследовать тот или иной класс задач, а в
    том, чтобы дать почувствовать читателю <<вкус>> к простейшим
    оптимизационным задачам и развить его интуицию в надежде
    на то, что к тому моменту, когда он перейдет к изучению
    соответствующих общих математических конструкций, у него не
    будет перед ними никакого страха.

    Мы проведем свое изложение на примере стилизованных, но
    вполне содержательных экономических задач и будем формулировать
    некоторые утверждения достаточно строго,
    однако приводить формальные доказательства не будем, заменяя их
    рассуждениями, позволяющими понять по-существу, почему то или
    иное утверждение верно. Подчеркнем в то же время, что эти
    рассуждения достаточно строги и читатель хоть мало-мальски
    знакомый с математическим анализом сможет превратить их во
    вполне строгие доказательства. Хочется подчеркнуть, что хотя нами
    рассматриваются задачи с несколькими переменными, для понимания
    проводимых рассуждений достаточно знакомство с дифференциальным
    исчислением функций одной переменной.

    Читателю, не знакомому с теорией экстремальных задач, мы
    рекомендуем прочитать этот параграф внимательно. Такое чтение
    очень поможет ему в дальнейшем при чтении всех остальных глав книги. Мы также
    советуем проглядеть этот параграф и хорошо знакомому с теорией
    оптимизации математику. Это поможет ему научиться думать о
    задачах на максимум и минимум с точки зрения экономиста.







\subsection{Простейшая задача распределения ресурса}


Рассмотрим фирму, включающую в себя два предприятия (подразделения), которые обозначаются с
помощью  индекса $i$, принимающего значения 1 и 2. Для производства продукции на этих
предприятиях необходим ровно один вид ресурса. Зависимость между количеством $x$ ресурса,
доставшегося предприятию $i$, и выпуском продукции на этом предприятии в денежном выражении
задается производственной функцией $f_i(x)$, заданной на  $\R_+$ и принимающей
неотрицательные значения. Мы предполагаем, что функция $f_{i}(x)$, $i=1,2$, является дважды
дифференцируемой на на множестве положительных чисел, монотонно возрастает, строго вогнутой и
даже $f_{i}''(x)<0 \ \forall x>0$, а также что $f'_{i}(0)=\infty$, $i=1,2$,. Кроме того,
естественно считать, что $f_{i}(0)=0$, $i=1,2$, но это равенство нами использоваться нигде не
будет.

Весь требуемый для производства ресурс в заданном количестве $y>0$
единиц распределяется из центрального офиса предприятия (мы будем
называть его Центром). Будем считать (это важно), что ресурс, о
котором идет речь, полностью расходуется или выбывает в течение
одного производственного цикла. Обозначим через $x_{i}$ то
количество ресурса, которое центральный офис выделяет $i$-му
подразделению, и подчеркнем. что это количество должно быть
неотрицательным, т.е. для $i=1,2$ должно выполняться неравенство
$x_{i}\geqslant0$. Поскольку распределять центр может не больше
того, что у него есть, то должно выполняться соотношение

\[x_{1}+x_{2}\leqslant y.\]

Естественно считать, что мы и будем делать, что целью Центра
является отыскание таких значений $x_{1}$ и $x_{2}$, которые
обеспечивают ему максимальное значение суммарного дохода
$f_{1}(x_{1})+f_{2}(x_{2})$, получаемого на двух предприятиях. Мы
можем записать задачу, решаемую центром, в следующем виде:

\begin{equation}
\label{a0} f_{1}(x_{1})+f_{2}(x_{2})\rightarrow\max,
\end{equation}
\begin{equation}
\label{b0}
 x_{1}+x_{2}\leqslant y,
\end{equation}
\begin{equation}
\label{c0}
 x_{1}\geqslant0, x_{2}\geqslant0.
\end{equation}

Формально говоря, можно было бы не уточнять, что $x_{1}$ и $x_{2}$
должны быть неотрицательными, поскольку функции $f_{1}(x)$ и
$f_{2}(x)$ заданы только на $\R_{+}$, но это лучше все-таки сделать.
Мы будем называть пару чисел $(x_{1},x_{2})$, удовлетворяющую
неравенствам (\ref{b0})-(\ref{c0}) допустимым распределением
ресурса.


Под \emph{решением} задачи (\ref{a0})--(\ref{c0}) мы понимаем такое
допустимое распределение $(x_{1}^{*},x_{2}^{*})$, которое обладает
тем свойством, что для любого другого допустимого распределения
$(x_{1},x_{2})$ выполняется неравенство
\[f_{1}(x_{1})+f_{2}(x_{2})\leqslant f_{1}(x^{*}_{1})+f_{2}(x^{*}_{2}).\]

Пусть пара чисел $(x_{1}^{*},x_{2}^{*})$ представляет собой решение
рассматриваемой задачи. Это решение мы будем называть
\emph{оптимальным распределением ресурса}. Попытаемся описать его
устройство. В первую очередь отметим, что при оптимальном
распределении ресурса между предприятиями весь ресурс должен быть
распределен полностью, поскольку обе производственные функции
являются монотонно возрастающими. Иными словами, должно выполняться
равенство
\begin{equation}
\label{a}
x_{1}^{*}+x_{2}^{*}=y.
\end{equation}


Теперь покажем, что должно выполняться равенство
\begin{equation}
\label{b}
f'_{1}(x_{1}^{*})=f'_{2}(x_{2}^{*}).
\end{equation}        (\remrk{РИС ???????})
Для этого сначала заметим, что ни  $x_{1}^{*}$, ни $x_{2}^{*}$ не может быть нулем. Например,
не может являться решением пара $x_{1}=0$ и $x_{2}=y$. Действительно, в силу сделанного нами
предположения, $\infty=f'_{1}(0)>f'_{2}(y)$. Поэтому при некотором небольшом $\Delta x>0$
выполняется неравенство $f_{1}(\Delta x)+f_{2}(y-\Delta x)>f_{1}(0)+f_{2}(y).$ Тем самым
$x_{1}^{*}>0$ и $x_{2}^{*}>0$.

Теперь покажем, что не может быть справедливым, например,
неравенство \[f'_{1}(x_{1}^{*})<f'_{2}(x_{2}^{*}).\] Действительно,
если бы оно выполнялось, то, поскольку

  \begin{multline*}
    f_{1}(x_{1}^{*}-\Delta x)+f_{2}(x_{2}^{*}+\Delta x)\approx \\
    \approx f_{1}(x_{1}^{*})-f'_{1}(x_{1}^{*})\Delta x
    +f_{2}(x_{2}^{*})+f'_{2}(x_{2}^{*})\Delta x= \\
    =f_{1}(x_{1}^{*})+f_{2}(x_{2}^{*})
    +(f'_{2}(x_{2}^{*})-f'_{1}(x_{1}^{*}))\Delta x > \\
    >f_{1}(x_{1}^{*})+f_{2}(x_{2}^{*}),
  \end{multline*}
нашлось бы  такое положительное число ${\Delta x<x_{1}^{*}}$, что
\[f_{1}(x_{1}^{*})+f_{2}(x_{2}^{*})<f_{1}(x_{1}^{*}-\Delta x)+f_{2}(x_{2}^{*}+\Delta x).\]
А это неравенство противоречит оптимальности
$(x_{1}^{*},x_{2}^{*})$, поскольку пара
    $(x_{1}^{*}-\Delta x,x_{2}^{*}+\Delta x)$
представляет собой допустимое распределение.

Мы видим, что равенства (\ref{a})--(\ref{b}) являются необходимыми
условиями оптимальности для задачи (\ref{a0})--(\ref{c0}). Покажем,
что эти же условия являются и достаточными. Рассмотрим пару
неотрицательных чисел $(x_{1}^{*},x_{2}^{*})$, удовлетворяющую
(\ref{a})--(\ref{b}) и покажем, что для любого другого допустимого
распределения ресурса $(x_{1}^{**},x_{2}^{**})$ выполняется
неравенство
\begin{equation}
\label{nerav}
    f_{1}(x_{1}^{**})+f_{2}(x_{2}^{**})\leqslant f_{1}(x_{1}^{*})+f_{2}(x_{2}^{*}).
\end{equation}
    Теперь мы воспользуемся вогнутостью функций $f_{1}(x)$ и
    $f_{2}(x)$, благодаря которой справедливы (?????????) следующие неравенства:
\[f_{1}(x_{1}^{**})\leqslant f_{1}(x_{1}^{*})+f\,'_{1}(x_{1}^{*})(x_{1}^{**}-x_{1}^{*})\]
и
\[f_{2}(x_{2}^{**})\leqslant f_{2}(x_{2}^{*})+f_{2}\,'(x_{2}^{*})(x_{2}^{**}-x_{2}^{*}).\]
    Сложив эти неравенства, с учетом равенства (\ref{b}) и
    очевидного неравенства
    $(x_{1}^{**}+x_{2}^{**})-(x_{1}^{*}+x_{2}^{*})\leqslant0$,
    получим (\ref{nerav}).



Мы можем подытожить наши рассуждения с помощью следующего
предложения.



\begin{prop}
\label{qq} Пара неотрицательных чисел $(x_{1}^{*},x_{2}^{*})$
является решением задачи (\ref{a0})--(\ref{c0}) тогда и только
тогда, когда она удовлетворяет соотношениям (\ref{a})--(\ref{b}).
\end{prop}
    (\remrk{РИС ???????})




\subsection{Равновесие и оптимальность}
 Данное предложение не только полезно само по себе, но и позволяет
провести важное и содержательное с экономической точки зрения рассуждение. Предположим, что
Центр решил децентрализировать процесс принятия решений и перевести подразделения на
<<хозрасчет>>. Новые правила игры будут следующими. Вместо того, чтобы централизованно
распределять ресурс посредством решения задачи (\ref{a0})--(\ref{c0}), Центр имитирует рынок.
Он просто назначает цену $p$ на ресурс в предположении, что подразделения в зависимости от
цены предъявляют на него спрос. И задачей центра является такое назначение цены, при котором
суммарный спрос на ресурс совпал бы с имеющимся его количеством $y$, которое рассматривается
как фиксированное неэластичное предложение. Здесь только нужно уточнить, каким образом
подразделения задают свой спрос на ресурс. Они делают это посредством решения задачи о
максимизации своей прибыли, т.е. разницы между доходом и затратами на приобретение ресурса и
ведет себя как ценополучатель на условном рынке ресурса. При этом, что очень важно, при
решении этой задачи подразделения совершенно не беспокоятся о том, хватит ли им ресурса или
нет.

Итак, при заданной цене ресурса $p$ предприятие $i=1,2$ \ решает
задачу
\begin{equation}
\label{max-pribili}
 f_{i}(x)-px\rightarrow\max,\ x\geqslant0.
\end{equation}


Обозначим решение этой задачи как $x_{i}(p)$ и будем рассматривать
его как спрос на ресурс со стороны подразделения $i$ в зависимости
от цены. Иными словами, $x_{i}(p)$ --- это функция спроса $i$-го
подразделения на распределяемый ресурс. Очевидно, что при каждом
заданном $p$ величина $x_{i}(p)$ представляет собой решение
уравнения
\[f'_{i}(x)=p.\]                (\remrk{РИС ???????})
Тем самым функция $x_{i}(p)$ является обратной к функции $f'_{i}(x)$:
\begin{equation}
\label{fun-sprosa}
 x_{i}(p)=(f'_{i})^{-1}(p).
\end{equation}
Поскольку функция $f'_{i}(x)$ является монотонно убывающей,
монотонно убывающей является и функция спроса $x_{i}(p)$.

Функция суммарного спроса $x(p)$ на нашем условном рынке
распределяемого ресурса определяется очевидным образом:
\[x(p)=x_{1}(p)+x_{2}(p).\]
Она тоже является монотонно убывающей. Напомним, что предложение на
этом рынке нам известно. Оно зафиксировано на уровне $y$. Поэтому мы
естественным образом определим цену равновесия $p^{*}$ как такую
цену при которой спрос и предложения выравниваются, т.е. как решение
уравнения
\[x(p)=y.\]
А пару чисел $(x_{1}^{*},x_{2}^{*})$, задаваемую  равенствами
\[x_{1}^{*}=x_{1}(p^{*}),\  x_{2}^{*}=x_{2}(p^{*}),\]
назовем \emph{равновесным} распределением ресурса.

Из предложения \ref{qq} автоматически вытекает тот факт, что
оптимальное и равновесное распределение ресурса суть одно и то же.
Нужно только указать, что если нам задано оптимальное распределение
ресурса $(x_{1}^{*},x_{2}^{*})$, то равновесная его цена равна
\[p^{*}=f'_{1}(x_{1}^{*})=f'_{2}(x_{2}^{*}).\]
Итак, мы можем сформулировать следующую теорему.

\begin{teo}
\label{teorema1} Пара неотрицательных чисел $(x_{1}^{*},x_{2}^{*})$
является решением задачи (\ref{a0})--(\ref{c0}) тогда и только
тогда, когда она представляет собой равновесное распределение
ресурса.
\end{teo}

\begin{exer}
    \label{CES1}
    Решите задачу распределения ресурса (\ref{a0})--(\ref{c0}) при
\[f_{i}(x)=\alpha_{i}\sqrt{x_{i}}, \alpha_{i}>0, \ i=1,2.\]
Выведите функции спроса $x_{i}(p)$, $i=1,2$. Найдите равновесную
цену ресурса и равновесное распределение ресурса. Сравните его с
оптимальным распределение ресурса.
\end{exer}

\begin{exer}
    Переформулируйте задачу (\ref{a0})--(\ref{c0}), определение
    равновесия и теорему \ref{teorema1}, для случая, когда распределяемый
    ресурс не расходуется в процессе производства в течение одного
    производственного цикла, а выбывает только наполовину.
\end{exer}

Читатель, знакомый с теорией экстремальных задач, по-видимому, уже
узнал в числе $p^{*}$ множитель Лагранжа, или двойственную оценку
ограничения (\ref{b0}). И этот читатель, наверно, отметит, что с
помощью множителя Лагранжа можно анализировать чувствительность
оптимального значения целевой функции к изменениям $y$. Так оно и
есть.
 Действительно, видоизменим слегка наш взгляд на задачу
(\ref{a0})--(\ref{c0}). А именно, будем считать, что количество
распределяемого ресурса является параметром, которым можно
варьировать. В этом случае числа $x_{1}^{*}$ и $x_{2}^{*}$, задающие
оптимальное распределение ресурса,  можно рассматривать как
зависящие от $y$:
\[x_{1}^{*}=x_{1}^{*}(y),\ x_{2}^{*}=x_{2}^{*}(y).\]
Соответственно, оптимальное значение задачи (\ref{a0})--(\ref{c0})
можно рассматривать как функцию от $y$. Обозначим эту функцию как
$\Phi(y)$:
\begin{equation}
\label{fi} \Phi(y)=f_{1}(x_{1}^{*}(y))+f_{2}(x_{2}^{*}(y)).
\end{equation}
Нетрудно заметить, что функция $\Phi(y)$ является монотонно
возрастающей. Можно также доказать, что она дифференцируема. В связи
с этим интересным представляется вычисление ее производной. Она
вычисляется очень несложно. А именно,
\begin{equation}
\label{dd}
\Phi'(y)=p^{*},
\end{equation}
где, как мы совсем недавно увидели,
\[p^{*}=f'_{1}(x_{1}^{*}(y))=f'_{2}(x_{2}^{*}(y)).\]
Если читатель успешно сдал экзамен по курсу математического анализа,
то он без труда докажет справедливость этого равенства (что и
оставляется ему в качестве упражнения). А мы же попытаемся только
объяснить, почему оно справедливо.

Предположим, что суммарное количество распределяемого ресурса
увеличилось с некоторого первоначального значения $y$ до значения
$y+\Delta y$. Пусть $(x_{1}^{*},x_{2}^{*})$ --- первоначальное
оптимальное распределение ресурса, а $(x_{1}^{*}+\Delta
x_{1},x_{2}^{*}+\Delta x_{2})$ --- оптимальное распределение ресурса
после увеличения его запаса. Очевидно, что $\Delta x_{1}+\Delta
x_{2}=\Delta y$. Поэтому
\[\Phi(y)-\Phi(y+\Delta y)\approx
f'_{1}(x_{1}^{*}(y))\Delta x_{1} +f'_{2}(x_{2}^{*}(y))\Delta x_{2}=
p^{*}\Delta y.\]
 А отсюда и вытекает справедливость
равенства (\ref{dd}), которое локально характеризует зависимость
оптимального значения задачи (\ref{a0})--(\ref{c0}) от имеющегося
запаса распределяемого ресурса. А именно, оно говорит о том, что
если этот запас увеличится с уровня $y$ на одну (малую) единицу, то
суммарный доход рассматриваемой нами фирмы, состоящей из двух
предприятий, увеличится примерно на $p^{*}$ денежных единиц. Мы,
конечно, исходим из предположения, что вся выпущенная продукция
будет продана.

\begin{exer}
    \label{fun-zatrat}

Рассмотрим фирму, включающую в себя два предприятия (подразделения),
которые обозначаются с помощью с помощью индекса $i$, принимающего
значения 1 и 2. Оба предприятия производят один и тот же вид
продукта. Пусть $C_{1}(y)$ --- это функция затрат (????????) первого
предприятия, показывающая, каковы будут его затраты в денежном
выражении на производство продукта в зависимости объема
производства, а $C_{2}(y)$ --- это функция затрат второго
предприятия. Мы предполагаем, что функции $C_{i}(y)$, $i=1,2$,
заданы на множестве неотрицательных чисел $\R_{+}$, непрерывны
дважды непрерывно дифференцируемы на множестве положительных чисел
$\R_{++}$ (???????). Кроме того, мы предполагаем, что они монотонно
возрастают ($C\,'_{i}(y)>0$), строго выпуклы и даже
$C\,''_{i}(y)>0$, что при нулевом выпуске затраты тоже будут равны
нулю ($C_{i}(0)=0$, $i=1,2$) и, более того, что $C\,'_{i}(0)=0$,
$i=1,2$.


    Рассмотрим следующую ситуацию. Фирма заключили контракт на производство и поставку
    $z>0$ единиц продукта и ей необходимо дать задание каждому из входящих в нее
    предприятий на производство продукта таким образом, чтобы
    обеспечить необходимые поставки и при этом сделать это с
    минимальными суммарными затратами.

    Сформулируйте задачу на отыскание минимума, которую должна решать
    в такой ситуации фирма. Сформулируйте по аналогии с предложением
    \ref{qq} необходимые и достаточные условия оптимальности.
    Объясните, каким образом можно в данной ситуации сымитировать рынок
    выпускаемого продукта, чтобы задача о минимизации суммарных затрат
    оказалась равносильной задаче об отыскании равновесия на этом
    рынке. Сформулируйте утверждение, аналогичное теореме
    \ref{teorema1}.

\end{exer}

\subsection {Агрегирование}

Соотношение (\ref{dd}) интересно само по себе. Но оно поможет нам
проанализировать ситуацию, которая отличается от только что
рассмотренной тем, что фирма, о которой идет речь, состоит не из
двух предприятий, а из трех. Будем, как и выше, считать, что каждое
из предприятий $i=1,2,3$ \ задается производственной функцией
$f_{i}(x)$, обладающей теми же свойствами, что выше. В данной
ситуации задача оптимального оптимального распределения ресурса,
имеющегося в количестве $\hat{x}>0$, записывается следующим образом:
\begin{equation}
\label{a3} f_{1}(x_{1})+f_{2}(x_{2})+f_{3}(x_{3})\rightarrow\max,
\end{equation}
\begin{equation}
\label{b3}
 x_{1}+x_{2}+x_{3}\leqslant\hat{x},
\end{equation}
\begin{equation}
\label{c3}
 x_{1}\geqslant0,\ x_{2}\geqslant0,\ x_{3}\geqslant0.
\end{equation}

Читатель без труда перенесет на данный случай понятие решения и,
повторив проведенные выше рассуждения, легко докажет, что набор
неотрицательных чисел
 $(x_{1}^{*},x_{2}^{*},x_{3}^{*})$
 представляет собой решение этой задачи тогда и только тогда, когда
 выполняются следующие соотношения:
 \[x_{1}^{*}+x_{2}^{*}+x_{3}^{*}=\hat{x},\]
 \[f'_{1}(x_{1}^{*})=f'_{2}(x_{2}^{*})=f'_{3}(x_{3}^{*}).\]

Эту задачу можно решить в два этапа, сагрегировав первые два
слагаемых. А именно, можно сначала посредством равенства (\ref{fi})
определить функцию $\Phi(y)$, а затем решить задачу


\begin{equation}
\label{a4}
 \Phi(y)+f_{3}(x_{3})\rightarrow\max,
\end{equation}
\begin{equation}
\label{b4}
 y+x_{3}\leqslant\hat{x},
\end{equation}
\begin{equation}
\label{c4}
 y\geqslant0,\ x_{3}\geqslant0.
\end{equation}
 Такую последовательность решения можно протрактовать следующим
 образом. Принимается решение об объединении первого и второго
 подразделения в некоторую объединенную структуру (с точки зрения
 внешнего наблюдателя происходит их агрегирование). После этого
 Центру
 нужно будет посредством решения задачи (\ref{a4})--(\ref{c4}) поделить
 ресурс в количестве $\hat{x}$  между этой объединенной
 структурой (ей достается $y$ единиц ресурса) и третьим
 подразделением (ему будет выделено $x_{3}$ единиц ресурса).
 Что касается структуры, объединяющей первое и второе подразделения,
 то она должна поделить между этими подразделениями то количество ресурса
 $y$, которое ему выделяется Центром. Она сделает это посредством
 решения задачи (\ref{a3})--(\ref{c3}).


Несложно проверить, что справедливо следующее предложение.
\begin{prop}
\label{qq1} Если набор $(x_{1}^{*},x_{2}^{*},x_{3}^{*})$ является
решением задачи (\ref{a3})--(\ref{c3}), то пара чисел
$(y^{*},x_{3}^{*})$ является решением задачи (\ref{a4})--(\ref{c4})
при $y^{*}=x_{1}^{*}+x_{2}^{*}$.

Если $(y^{*},x_{3}^{*})$ представляет собой решение задачи
(\ref{a4})--(\ref{c4}), а $(x_{1}^{*},x_{2}^{*})$ --- это решение
задачи (\ref{a0})--(\ref{c0}) при $y=y^{*}$, то
$(x_{1}^{*},x_{2}^{*},x_{3}^{*})$ является решением задачи
(\ref{a3})--(\ref{c3}).
\end{prop}

 Надо признать, что децентрализация, о которой идет речь в
 проведенном рассуждении, является некоторой видимостью.
 Действительно, задача (\ref{a4})--(\ref{c4}) ничуть не проще задачи
 (\ref{a3})--(\ref{c3}), поскольку она предполагает, что Центр знает
 устройство функции $\Phi(y)$, которая сама по себе получается
 посредством решения экстремальной задачи. Представляется, что
 децентрализация принятия решений посредством имитации рынка, о
 которой идет речь в теореме \ref{teorema1}, более действенна.

 Можно ожидать, что в рассматриваемом случае, когда рассматриваемая
 фирма состоит из трех предприятий, ситуация будет
 такой же, как и в случае с двумя предприятиями. Здесь, правда,
 следует в первую очередь ответить на вопрос, будут ли различаться
 равновесие в случае, когда каждое предприятие действует
 самостоятельно и в случае, когда первые два предприятия объединены.
 В первом случае функция спроса каждого предприятия $i=1,2,3$ \
 задается, как и раньше, равенством  (\ref{fun-sprosa}), а функция
 суммарного спроса получается сложением функций спроса трех
 предприятий.
 Во втором же
 случае функция спроса третьего предприятия, очевидно, останется той
 же, функция спроса $y(p)$ структуры, объединяющей первое и второе
 предприятия, выводится из решения задачи
\begin{equation}
\label{y(p)}
 \Phi(y)-py\rightarrow\max,\ y\geqslant0,
\end{equation}
 а функция суммарного спроса получается сложением $y(p)$ и $x_{3}(p)$.
 Однако, если мы взглянем на соотношение (\ref{dd}), то поймем, что
 справедливо очень важное равенство
\begin{equation}
\label{y(p)=summa}
 y(p)=x_{1}(p)+x_{2}(p).
\end{equation}
 Оно говорит о том, что спрос со стороны первого и второго
 предприятия вместе взятых не зависит от того, действуют ли они в условиях
 имитации рынка самостоятельно или объединены в некоторую единую
 структуру. А отсюда следует, что не важно, будем ли мы имитировать
 рынок для решения задачи (\ref{a3})--(\ref{c3}), или для решения
 задачи (\ref{a4})--(\ref{c4}). В обоих случаях естественным образом
 определяемая равновесная цена
 окажется одной и той же. Но этого и следовало ожидать, поскольку
 задача (\ref{a4})--(\ref{c4}) --- это та же задача
 (\ref{a3})--(\ref{c3}), только записанная в несколько иной форме.

\subsection{Репрезентативный производитель}

 До сих пор наши рассуждения касались того, как можно проводить
 посредством имитации рынка  решение задачи о распределении
 некоторого ресурса между несколькими
 предприятиями, составляющими единую фирму.
 Но практически те же рассуждения можно проводить и в обратную
 сторону.

 Рассмотрим следующую ситуацию. Пусть на рынке некоторого
  ресурса спрос предъявляют три
 предприятия, каждое из которых действует вполне самостоятельно.
 Для производства предприятию $i=1,2,3$ \ необходим только
 рассматриваемый ресурс, а зависимость между затратами этого ресурса
 и выпуском продукции в денежном выражении задается той же
 производственной функцией $f_{i}(x)$, что и выше. Целью его
 функционирования является извлечение прибыли. Это значит, что при
 заданной цене ресурса $p$ объем затрат ресурса и выпуска продукции
  на рассматриваемом предприятии определяется посредством решения
  задачи (\ref{max-pribili}). Отсюда вытекает, что его функция спроса
  на ресурс $x_{i}(p)$ задается равенством (\ref{fun-sprosa}). А
  функция суммарного спроса $x(p)$ на ресурс со стороны трех предприятий
  задается как
\begin{equation}
\label{summ spros}
  x(p)=x_{1}(p)+x_{2}(p)+x_{3}(p).
\end{equation}
  Теперь предположим, что первые два предприятия объединились в
  единую фирму, внутри которой управление централизовано в целях
  максимизации прибыли. Эта фирма будет описываться производственной
  функцией $\Phi(y)$, задаваемой равенством (\ref{fi}), а функция
  спроса этой объединенной фирмы будет определяться равенством
  (\ref{y(p)}).

  Но при этом выполняется равенство (\ref{y(p)=summa}). А оно означает,
  что с точки зрения внешнего наблюдателя, которого
  интересует только то, что происходит на рынке ресурса в целом,
  совершенно безразлично, максимизируют ли ли предприятия 1 и 2
  прибыль независимо друг от друга, или в составе единой фирмы. В
  таком рассуждении можно пойти еще дальше и рассмотреть ситуацию,
  когда все три предприятия объединены в некую централизованно
  управляемую фирму, целью которой тоже является максимизация
  прибыли. Производственная функция $\Psi(x)$ такой фирмы должна,
  очевидно, задаваться равенством
   \[\Psi(y)=f_{1}(x_{1}^{*}(y))+f_{2}(x_{2}^{*}(y))+f_{3}(x_{3}^{*}(y)),\]
   где $(x_{1}^{*}(y),x_{2}^{*}(y),x_{3}^{*}(y))$ --- это решение
    задачи


  \[f_{1}(x_{1})+f_{2}(x_{2})+f_{3}(x_{3})\rightarrow\max,\]
   \[ x_{1}+x_{2}+x_{3}\leqslant y,\]
    \[x_{i}\geqslant0,\ i=1,2,3.\]
   Если при этом предположить, что при заданной цене $p$ рассматриваемого
   нами ресурса эта объединенная фирма решает задачу
\begin{equation}
\label{Psi}
   \Psi(x)-px\rightarrow\max,\ x\geqslant0,
\end{equation}
   то ее функция спроса будет просто совпадать с функцией суммарного
   спроса $x(p)$, которая фигурирует в равенстве
   (\ref{summ spros}).

   Здесь сразу же надо сделать важное замечание.
   Предположение о том, что объединенная фирма решает
   именно задачу (\ref{Psi}), является не единственно возможным и
   даже несколько сомнительным. Дело в том, что в этой задаче цена
   ресурса $p$ рассматривается фирмой как величина для нее заданная,
    т.е. фирма ведет себя на рынке ресурса как ценополучатель.
   Однако фирма, о которой идет речь, является единственным
   покупателем на рассматриваемом нами рынке, а значит, она она
   вполне может пытаться воздействовать на цену. Такую ситуацию мы
   рассмотрим  ???????.

   Однако, для нас важно не то,
   сколь реалистичным является предположение о том, что объединенной фирмой
   решается именно задача (\ref{Psi}). Для нас важно другое. Если
   со стороны спроса на рынке некоторого ресурса представлено много
   предприятий (конечно, три --- это много), то про функцию спроса
   мы можем рассуждать так, \emph{как если бы} она была порождена одной большой
   фирмой, решающей задачу (\ref{Psi}). Такое рассуждение позволяет
   в некоторых ситуациях использовать понятие \emph{репрезентативного}
   производителя, представляющего какую-нибудь отрасль экономики или
   производственный сектор в целом. Такой репрезентативный
   производитель часто фигурирует в макроэкономических моделях, где
   весь производственный сектор описывается с помощью одной
   макроэкономической производственной функции.


    \subsection{Простейшие обобщения}
   Теперь вернемся немного назад к задаче (\ref{a0})--(\ref{c0}) и
   посмотрим, какую роль играли предположения, в рамках
   которых мы проводили свои рассуждения, в частности, предположения о том, что
   при $i=1,2$ выполняются соотношения $f'_{i}(x)>0 \ \forall x>0$, $f'_{i}(0)=\infty$ и
   $f''_{i}(x)<0 \ \forall x>0$. Первое из них гарантирует нам, что при своем оптимальном
   распределении ресурс будет распределен полностью, второе --- что
   при оптимальном распределении ресурса $(x_{1}^{*},x_{2}^{*})$
   каждому из двух подразделений достанется положительное
   его количество и что выполняется
   равенство (\ref{b}), а третье --- что функция спроса определяется равенством
   (\ref{fun-sprosa}) вполне корректно. Однако эти предположения могут
   показаться слишком жесткими. В частности, они исключают из рассмотрения
   линейные функции. Оказывается, их вполне можно до определенной степени ослабить.
   Правда, при этом нам придется несколько видоизменить формулировку
   некоторых утверждений.

   Итак, рассмотрим задачу
\begin{equation}
\label{a0-1} f_{1}(x_{1})+f_{2}(x_{2})\rightarrow\max,
\end{equation}
\begin{equation}
\label{b0-1}
 x_{1}+x_{2}\leqslant y,
\end{equation}
\begin{equation}
\label{c0-1}
 x_{1}\geqslant0, x_{2}\geqslant0.
\end{equation}
    в предположении,
   что функции $f_{i}(x)$, $i=1,2$, заданные на $\R_{+}$, дважды
   непрерывно дифференцируемы на множестве положительных чисел и вогнуты
   ($f''_{i}(x)\leqslant0$). Если мы в данных предположениях будем проводить
   рассуждения, которые привели нас к предложению \ref{qq}, то
   обнаружим, что при оптимальном распределении ресурса $(x_{1}^{*},x_{2}^{*})$,
   во-первых, не весь ресурс будет обязательно распределен
   полностью между двумя подразделениями, а что-то останется у Центра, а во-вторых,
   что какому-то подразделению (или даже обеим подразделениям) ресурс не достанется вовсе.

   Рассмотрим эти возможности подробнее. Сначала заметим, что
   ситуация, когда обоим подразделениям ресурс не будет выделен
   вовсе возможен только в том случае, когда выполняются неравенства
   $f_{i}'(0)\leqslant0$, $i=1,2$. Сделанные нами
   предположения такую возможность, формально говоря, не отвергают.
   Возможность, что какому-то подразделению ресурс выделяется, а
   другому нет, тоже допустима. В этом случае будет выполняться следующее
   соотношение между производными $f'_{1}(x_{1}^{*})$ и $f'_{2}(x_{2}^{*})$:
   \[\text{если} \ x_{1}^{*}>0\ \text{и}\  x_{2}^{*}=0,\ \text{то}\ f'_{1}(x_{1}^{*})\geqslant f'_{2}(x_{2}^{*});\]
   \[\text{если} \ x_{1}^{*}=0\ \text{и}\  x_{2}^{*}>0,\ \text{то}\ f'_{1}(x_{1}^{*})\leqslant f'_{2}(x_{2}^{*}).\]
   При этом, очевидно, если распределяется весь ресурс полностью, то
   \[\max_{i}\{f'_{i}(x_{i}^{*})\}\geqslant0.\] Ситуация, когда,
   например, $x_{1}^{*}>0$ и $x_{2}^{*}>0$ и при этом
   $x_{1}^{*}+x_{2}^{*}<y$, тоже возможна. В этом случае, очевидно,
   будут выполняться равенства
   $f'_{1}(x_{1}^{*})=f'_{2}(x_{2}^{*})=0$.
   Это рассуждение позволяет нам сформулировать следующее
   предложение.

   \begin{prop}
   \label{qq-1}
    Если пара  $(x_{1}^{*},x_{2}^{*})$, представляет собой решение задачи
    (\ref{a0-1})--(\ref{c0-1}), то найдется
    такое число $p^{*}\geqslant0$, что выполняются следующие
    условия:
    \[f'_{i}(x_{i}^{*})\leqslant p^{*},\ i=1,2,\]
    \[x_{1}^{*}+x_{2}^{*}<y \Rightarrow  p^{*}=0,\]
    \[f'_{i}(x_{i}^{*})<p^{*}\Rightarrow  x_{i}^{*}=0,\ i=1,2.\]
    \end{prop}

    Чтобы сформулированное предложение было совсем прозрачным,
    отметим, что фигурирующее в нем число $p^{*}$ просто задается равенством
    \[p^{*}=\max_{i}\{f'_{i}(x_{i}^{*})\}.\]

    Читатель уже, наверно, по аналогии с предложением \ref{qq} предположил, что условия,
    сформулированные в этом предложении, являются не только
    необходимыми, но и достаточными. Так оно и есть. Для доказательства
    следующего предложения достаточно несколько уточнить рассуждения,
    проведенные по соответствующему поводу перед формулированием предложения \ref{qq}.

    \begin{prop}
   \label{qq-2}
    Предположим, что для допустимого распределения ресурса
    $(x_{1}^{*},x_{2}^{*})$ и числа $p^{*}\geqslant0$ выполняются
    следующие условия:
    \[f'_{i}(x_{i}^{*})\leqslant p^{*},\ i=1,2,\]
    \[x_{1}^{*}+x_{2}^{*}<y \Rightarrow p^{*}=0,\]
    \[f'_{i}(x_{i}^{*})<p^{*} \Rightarrow x_{i}^{*}=0,\ i=1,2.\]
        Тогда пара $(x_{1}^{*},x_{2}^{*})$ является решением задачи
    (\ref{a0-1})--(\ref{c0-1}).
    \end{prop}

\begin{exer}
    Докажите это предложение.
\end{exer}


    Теперь сформулируем обобщение теоремы \ref{teorema1}. Для этого
    сначала заметим, что при сделанных нами здесь предположениях
    решением задачи (\ref{max-pribili}) может быть ноль. Поэтому некоторое число $x$
    является ее решение тогда и только тогда, когда выполняются
    следующее условие:
    \[f'_{i}(x)\leqslant p,\ \text{причем если}\ f'_{i}(x)<p,\ \text{то}\  x=0.\]
    Кроме того, подчеркнем, что в данном случае решение задачи
    (\ref{max-pribili}) может быть неединственным. Поэтому мы не
    можем определить функцию спроса на ресурс со стороны предприятия
    $i$ посредством (\ref{fun-sprosa}). Однако это не помешает нам
    определить \emph{состояние равновесия как набор}
    $[p^{*},(x_{1}^{*},x_{2}^{*})]$, \emph{состоящий из равновесной цены}
    $p^{*}$ \emph{и равновесного распределения ресурса}
    $(x_{1}^{*},x_{2}^{*})$,
    \emph{удовлетворяющий следующим условиям}:

    \begin{itemize}
    \item [---]
        \emph{для обоих} $i=1,2$ \emph{число} $x_{i}^{*}$
        \emph{является решением задачи} (\ref{max-pribili})
        \emph{при} $p=p^{*}$;
    \item [---]
    $x_{1}^{*}+x_{2}^{*}\leqslant y$, \emph{причем если} $x_{1}^{*}+x_{2}^{*}<y$, \emph{то}
    $p^{*}=0$.

    \end{itemize}

    Теперь читателю не составит труда доказать следующую теорему.
    \begin{teo}
    \label{teorema2}
    При сделанных
    предположениях пара $(x_{1}^{*},x_{2}^{*})$ является  решением
    задачи (\ref{a0-1})--(\ref{c0-1}) тогда и
    только тогда, когда найдется число $p^{*}\geqslant0$, такое что набор
    $[p^{*},(x_{1}^{*},x_{2}^{*})]$ представляет собой состояние
    равновесия.
    \end{teo}

\begin{exer}
    Докажите эту теорему.
\end{exer}

    Как мы уже отмечали, те предположения по поводу функций
    $f_{i}(x)$, $i=1,2$, которые сделаны в этом пункте, допускают, что эти функции
    являются линейными, т.е.
    \[f_{i}(x)=a_{i}x, \ a_{i}>0, \ i=1,2.\]
    Остановимся на этом поучительном частном случае подробнее.
    Здесь задача о максимизации прибыли (\ref{max-pribili})
    приобретает следующий вид:
    \begin{equation}
    \label{max-prib-lin}
    a_{i}x-px=(a_{i}-p)x\rightarrow\max, \ x\geqslant0.
    \end{equation}
    Ее решение зависит от соотношения между $a_{i}$ и $p$. Возможны
    три случая: 1) $a_{i}<p$, 2) $a_{i}>p$, 3) $a_{i}=p$. Рассмотрим
    их.
\begin{enumerate}[1)]
\item
    Если $a_{i}<p$, то решением задачи (\ref{max-prib-lin}) является
    $x=0$.
\item
    Если $a_{i}<p$, то задача (\ref{max-prib-lin}) не имеет решения,
    поскольку, увеличивая  $x$ до бесконечности, можно получить сколь угодно большое
    значение целевой функции $a_{i}x-px$.
\item
    Если $a_{i}=p$, то задача (\ref{max-prib-lin}) имеет бесконечно
    много решений, а именно ее решением будет любое неотрицательное
    число.
\end{enumerate}

    Может возникнуть впечатление, что из этих трех случаев третий не играет никакой
    роли, поскольку, казалось бы, совпадение $a_{i}$ и $p$ возможно
    разве что случайно, причем, вероятность этого совпадения должна
    равняться нулю. Но в рассматриваемой ситуации величина $a_{i}$
    является заданной, а цена $p$ формируется не случайным образом и
    именно третий из рассмотренных случаев играет ключевую роль.
    Дело в том, что нас интересует не любая цена, а равновесная цена
    $p^{*}$, которая, очевидно,
    формируется не случайно, а совпадает с $\max\{a_{1}, a_{2}\}$.
    Пусть для определенности $a_{1}<a_{2}$ и, следовательно, $p^{*}=a_{2}$.
    Тем самым множеством решений  задачи
    \[a_{2}x-p^{*}x=(a_{2}-p^{*})x\rightarrow\max, \ x\geqslant0,\]
    является все множество неотрицательных чисел $\R_{+}$. Однако,
    это не противоречит тому, что равновесное распределение ресурса
    $(x_{1}^{*},x_{2}^{*})$ задается однозначно: $x_{1}^{*}=0$,
    $x_{2}^{*}=y$.

\begin{exer}
    Решите задачу (\ref{a0-1})--(\ref{c0-1}) и проверьте справедливость
    теоремы \ref{teorema2} в случае, когда $f_{1}(x)=a_{1}x^{1/2}$, $f_{2}(x)=a_{2}x$.
\end{exer}


\subsection{Задача максимизации полезности при бюджетном ограничении}

    Задачи оптимального распределения ресурсов в том или ином виде
    решают не только на предприятиях. Современная экономическая
    теория исходит из того, что отдельного потребителя или отдельное
    домохозяйство тоже можно описывать в терминах оптимизационных
    задач. В частности, предполагается, что потребитель максимизирует
    свою функцию полезности на бюджетном ограничении. Именно такого
    типа задачу мы сейчас и рассмотрим, хотя и не в самом общем виде,
    а при некоторых не очень обременительных предположениях. Сразу
    же подчеркнем, что с формально-математической точки зрения та задача,
    которая сейчас будет рассмотрена, является несколько более общей,
    чем задача (\ref{a0})--(\ref{c0}). Это и не удивительно,
    поскольку она  тоже является задачей
    распределения ресурса, только ресурс этот особый --- деньги.

    Мы рассматриваем следующую ситуацию. Некоторый потребитель имеет в своем
    распоряжении некоторую сумму денег $m>0$, которую он
    предполагает потратить на на приобретение $n$ видов различных
    продуктов. От приобретения (и потребления) $x_{i}$ единиц $i$-го он извлекает
    полезность в количестве $u_{i}(x_{i})$ некоторых условных единиц
    (которые иногда даже называют утилами или ютилями). Цена $p_{i}>0$ каждого продукта
    $i=1,\ldots,n$ рассматривается нашим потребителем как экзогенно заданная
    величина.




    Для каждого $i=1,\ldots,n$ функция $u_{i}(x)$ задана на
    множестве неотрицательных чисел $\R_{+}$. Мы
    предполагаем, что она дважды непрерывно дифференцируема на
    множестве положительных чисел $\R_{++}$ и вогнута
    ($u\,''_{i}(x)\leqslant0$), но не требуем, чтобы она принимала
    только неотрицательные значения. Кроме того, мы даже допускаем,
    что в нуле она принимает значение $-\infty$. Заметим, что в этом
    случае выполняется (проверьте это) соотношение
    $\lim_{x\rightarrow0}u'_{i}(x)\rightarrow\infty$, которое часто
    записывают как $u'_{i}(0)=\infty$.

    Положим
\begin{equation}
\label{add-fup}
    U(x_{1},...,x_{n})=u_{1}(x_{1})+\ldots+u_{n}(x_{n}).
\end{equation}
    Функция $n$ переменных $U(x_{1},...,x_{n})$  показывает,
    какую полезность извлекает потребитель из потребления набора
    (вектора) продуктов $(x_{1},...,x_{n})$. Такая называется функцией
    полезности. Теорию потребителя в достаточно общем виде с более общими
    функциями полезности мы рассмотрим в главе ????, а пока рассмотрим тот
    частный случай, когда функция полезности является аддитивной, т.е. имеет вид
    (\ref{add-fup}).


    Потребитель не может тратить на приобретение продуктов больше денег,
    чем у него есть. Это значит, что тот набор продуктов $(x_{1},\ldots,x_{n})$, который он
    приобретет, должен удовлетворять неравенству
    \[p_{1}x_{1}+\ldots+p_{n}x_{n}\leqslant m.\]
    Кроме того, нужно помнить, что приобретать можно только неотрицательное количество
    каждого продукта и поэтому для всех $i=1,\ldots,n$ должно выполняться
    неравенство $x_{i}\geqslant0$.
    Цель потребителя состоит в том, чтобы таким образом потратить
    имеющиеся у него деньги, чтобы получить максимально возможную
    суммарную полезность
       от потребления всех приобретенных продуктов.

    Итак, задачу потребителя можно записать в следующем виде:
    \begin{equation}
    \label{Zadacha potrebitelia1}
    U(x_{1},...,x_{n})=\sum_{i=1}^{n}u_{i}(x_{i})\rightarrow\max,
    \end{equation}
    \begin{equation}
    \label{Zadacha potrebitelia2}
    \sum_{i=1}^{n}p_{i}x_{i}\leqslant m,
    \end{equation}
    \begin{equation}
    \label{Zadacha potrebitelia3}
    x_{i}\geqslant0,\ i=1,\ldots,n.
    \end{equation}

    Справедливо следующее предложение.
\begin{prop}
    \label{opt-potreb}
    Набор неотрицательных чисел $(x^{*}_{1},\ldots,x^{*}_{n})$
    является решением задачи
    (\ref{Zadacha potrebitelia1})--(\ref{Zadacha potrebitelia3})
    тогда и только тогда, когда найдется  число $\lambda\geqslant0$
    (множитель Лагранжа), такое что выполняются следующие условия:
\begin{enumerate}
    \item
    $\sum_{i=1}^{n}p_{i}x^{*}_{i}\leqslant M,$
    \item
    $\sum_{i=1}^{n}p_{i}x^{*}_{i}<M \Rightarrow \lambda=0,$
    \item
    $u'_{i}(x^{*}_{i})/p_{i}x^{*}_{i}\leqslant\lambda,\ i=1,\ldots,n,$
    \item
    $u'_{i}(x^{*}_{i})/p_{i}x^{*}_{i}<\lambda \Rightarrow x^{*}_{i}=0,\ i=1,\ldots,n.$
\end{enumerate}
    \end{prop}

    В этом предложении нет практически ничего нового по сравнению с
    предложениями \ref{qq-1} и \ref{qq-2}. И сама задача
    (\ref{Zadacha potrebitelia1})--(\ref{Zadacha potrebitelia3}),
    по существу, отличается от задачи (\ref{a0})--(\ref{c0})
    только тем, что в ней переменных не две, а <<много>>, а также
    в том, что вместо ограничения на сумму используемых ресурсов в натуральном
    выражении стоит ограничение на сумму продуктов в денежном
    выражении. Поэтому читатель без труда проведет все
    необходимые для доказательства рассуждения самостоятельно.
\begin{exer}
    Докажите предложение \ref{opt-potreb}.
\end{exer}



    Нужно, пожалуй, сделать пару замечаний. Во-первых, отметим, что
     решение может быть таким, что не все деньги тратятся
    на приобретение продуктов. Однако, если хотя бы для одного
    $i$ функция $u_{i}(x)$ монотонно
    возрастает ($u'_{i}(x)>0 \ \forall x>0$), то деньги
    будут истрачены полностью, а условия (1)--(2)
    (\remrk{нумерацию сделать 1)-2) здесь и в других местах}) в формулировке
    предложения сведутся к равенству
    $\sum_{i=1}^{n}p_{i}x^{*}_{i}=M$.
    Во-вторых, очевидно, что условия (3)--(4) выполняются
    тогда и только тогда, когда для всех $i=1,\ldots,n$ число
    $x^{*}_{i}$ является решением задачи

    \[u_{i}(x)-\lambda p_{i}x\rightarrow\max, \ x\geqslant0.\]

    Отметим также, что оптимальное значение целевой функции
    в задаче (\ref{Zadacha potrebitelia1})--(\ref{Zadacha potrebitelia3})
    можно рассмотреть
    как функцию величины $m$. При некоторых предположениях можно
    проверить, что производная этой функции совпадает с множителем Лагранжа
    $\lambda$, фигурирующем в этом предложении.
    Это дает повод назвать $\lambda$ предельной полезностью денег.

\begin{exer}
    \label{Kobb-Douglas1}

    Решите задачу
\[\alpha \ln x_{1}+(1-\alpha)\ln x_{2}\rightarrow\max,\]
\[p_{1}x_{1}+p_{2}x_{2}\leqslant m,\]
\[x_{1}\geqslant0,\ x_{2}\geqslant0,\]
    где $0<\alpha<1$, $p_{1}>0$, $p_{2}>0$, $m>0$.

\end{exer}




\begin{exer}
    \label{Kobb-Douglas2}

    Решите задачу
\[\sum_{i=1}^{n}\alpha_{i}\ln x_{i}\rightarrow\max,\]
\[\sum_{i=1}^{n}p_{i}x_{i}\leqslant m,\]
\[x_{i}\geqslant0,\ i=1,\ldots,n,\]
где $m>0$, $\alpha_{i}>0$, $p_{i}>0$, $i=1,\ldots,n$.

\end{exer}


\begin{exer}
    \label{CES2}

    Решите задачу
\[\sum_{i=1}^{n}\alpha_{i}(x_{i})^{\rho}\rightarrow\max,\]
\[\sum_{i=1}^{n}p_{i}x_{i}\leqslant m,\]
\[x_{i}\geqslant0,\ i=1,\ldots,n,\]
где  $m>0$, $0\neq \rho\leqslant1$, $\alpha_{i}>0$, $p_{i}>0$,
$i=1,\ldots,n$.

\end{exer}

\begin{exer}
    Предполагая, $p_{n}=1$, решите задачу
(\ref{Zadacha potrebitelia1})--(\ref{Zadacha potrebitelia3})
    в случае, когда
    \[U(x_{1},...,x_{n})=a\sum_{i=1}^{n-1}\alpha_{i}\ln x_{i}+x_{n}, \ a>0, \ \alpha_{i}>0, \
     i=1,\ldots,n-1, \ \sum_{i=1}^{n-1}\alpha_{i}=1.\]

\end{exer}


    Рассмотрим случай, когда $n=2$ и $u(x_{2})=x_{2}$, т.е.
    \[U(x_{1},x_{2})=u_{1}(x_{1})+x_{2}.\]
    Функция полезности такого вида называется \emph{квазилинейной} и часто
    используется при анализе частичного равновесия. Она имеет
    естественную
    экономическую интерпретацию в предположении, что $p_{2}=1$.
    Величина $u_{1}(x_{1})$
    показывает, какую полезность извлекает потребитель из
    потребления некоторого блага, рынок которого мы изучаем,
    в количестве $x_{1}$, а величина $x_{2}$ показывает объемы
    <<потребления>> условного $2$-го блага, представляющего собой ту
    сумму денег, которая тратится на все блага, за исключение первого. При этом здесь
    неявно предполагается, что полезность $u_{1}(x_{1})$
    измеряется в денежных единицах. Такое предположение является
    крайне сильным, но очень удобным с аналитической точки зрения,
    особенно, если отказаться от требования неотрицательности переменной $x_{2}$, т.е.
    допустить, что потребитель может <<залезать в долги>>. В этом
    случае задача потребителя приобретает следующий вид:
\begin{equation}
\label{max-pol-dop}
  \left\{
    \begin{array}{l}
      u_{1}(x_{1})+x_{2}\rightarrow\max, \\
      p_{1}x_{1}+x_{2}\leqslant m, \\
      x_{1}\geqslant0.
    \end{array}
  \right.
\end{equation}
    Ее решение $(x_{1}^{*},x_{2}^{*})$ устроено следующим образом:
    число $x_{1}^{*}$ представляет собой решение задачи

\begin{equation}
\label{max-pol-dop1}
    u_{1}(x_{1})-p_{1}x_{1}\rightarrow\max,
\end{equation}
    а число $x_{2}^{*}$ задается следующим очевидным образом:
    \[x_{2}^{*}=m-p^{*}_{1}x^{*}_{1}.\]

    Напомним, что ранее, в предыдущем параграфе, мы под задачей
    потребителя понимали именно задачу вида (\ref{max-pol-dop1}),
    сделав оговорку по поводу того, что под задачей
    потребителя обычно понимается задача о максимизации полезности
    на бюджетном ограничении. Сейчас мы видим, что задачи
    (\ref{max-pol-dop}) и (\ref{max-pol-dop1}), по-существу, эквивалентны. Тем самым
    задачу вида (\ref{max-pol-dop1}) тоже можно называть в случае квазилинейной
    функции полезности задачей потребителя.


\subsection{Репрезентативный потребитель,
репрезентативный производитель и оптимальность конкурентного равновесия}


    В заключение данного параграфа мы рассмотрим три взаимосвязанных модели
    экономического равновесия, анализ которых позволит до
    определенной степени подытожить наши предыдущие соображения о связи между
    конкурентным  равновесием и оптимальностью на рынке одного
    продукта, а также еще раз сравнить конкурентное равновесие с
    монопольным. В своем изложении мы будет очень краткими
    в надежде на то, что читатель
    уже освоился с принятым в этом параграфе способом  рассуждения.

    Начнем с модели конкурентного равновесия с несколькими
    потребителями и производителями, которую мы обозначим как
    $\mathfrak{M}_{c}$.
    Предположим, что на рынке рассматриваемого блага присутствует $m$
    потребителей, и каждый из потребителей $j=1,...,m$ формирует свою функцию
    спроса $D_{j}(p)$ на это благо
    в зависимости от его цены $p$ посредством решения задачи
    \[u_{j}(c)-pc\rightarrow\max, \ c\geqslant0.\]
    Мы считаем, что функции $u_{j}(c)$ дважды непрерывно дифференцируемы,
    а также что $u'_{j}(c)>0 \ \forall c>0$ и $u''_{j}(c)<0 \ \forall c>0$. Тем самым
    функция спроса $D_{j}(p)$ задается равенством
    \[D_{j}(p)=(u'_{j})^{-1}(p), \ j=1,...,m.\]




    Теперь предположим, что на том же рынке присутствует $n$
    производителей. Производитель $i=1,...,n$ описывается дважды непрерывно
    дифференцируемой функцией затрат $C_{i}(q)$, для которой
    $C\,'_{i}(q)>0 \ \forall q>0$ и $C\,''_{i}(q)>0 \ \forall q>0$.
    Функция предложения продукта $S_{i}(p)$ со стороны производителя
    $i$ выводится из решения задачи на максимум прибыли

    \[pq-C_{i}(q)\rightarrow\max, \ q\geqslant0\]
    Иными словами, эта функция задается равенством
    \[S_{i}(p)=(C\,'_{i})^{-1}(p).\]

    Далее мы предполагаем, что $\max_{j}u_{j}'(0)>\min_{i}C_{i}'(0)$.


    Рассмотрим \emph{состояние равновесия} в модели $\mathfrak{M}_{c}$,
    под которым мы понимаем набор
    $\{p^{*}, \ (c_{1}^{*},...,c_{m}^{*}), \
    (q_{1}^{*},...,q_{n}^{*})\},$
    где $p^{*}$ --- цена равновесия, т.е. решение уравнения
    \[\sum_{i=1}^{n}S_{i}(p)=\sum_{j=1}^{m}D_{j}(p);\]
    $c_{j}^{*}$ --- равновесный уровень потребления
    $j$-го потребителя:
    \[c_{j}^{*}=D_{j}(p^{*}), \ j=1,...,m;\]
    $q_{i}^{*}$ --- равновесный уровень выпуска $i$-го
    производителя:
    \[q_{i}^{*}=S(p^{*}), \ i=1,...,n.\]


    По аналогии с нашими предыдущими рассуждениями, можно ожидать,
    что состояние равновесия в рассматриваемой модели является,
    по-существу, решением некоторой экономически осмысленной задачи
    на максимум или минимум. Читатель без труда проверит, что
    таковой является задача
\begin{equation}
 \label{max-pol-zat1}
      \sum_{j=1}^{m}u_{j}(c_{j})-\sum_{i=1}^{n}C_{i}(q_{i})\rightarrow\max,
\end{equation}
\begin{equation}
 \label{max-pol-zat2}
      \sum_{j=1}^{m}c_{j}\leqslant\sum_{i=1}^{n}q_{i},
\end{equation}
\begin{equation}
 \label{max-pol-zat3}
      c_{j}\geqslant0, \ j=1,...,m, \ q_{i}\geqslant0, \ i=1,...n.
\end{equation}


    Итак, справедлива следующая теорема.
\begin{teo}
    Набор $\{(c_{1}^{*},...,c_{m}^{*}), \ (q_{1}^{*},...,q_{n}^{*})\}$
    является решением задачи
(\ref{max-pol-zat1})-(\ref{max-pol-zat3})
    тогда и только тогда, когда при некотором $p^{*}>0$ набор
    $\{p^{*}, \ (c_{1}^{*},...,c_{m}^{*}), \ (q_{1}^{*},...,q_{n}^{*})\}$
    является состоянием равновесия в модели $\mathfrak{M}_{c}$.
\end{teo}

    Теперь введем в рассмотрение репрезентативного потребителя и
    репрезентативного производителя. Первый из них характеризуется
    функцией <<полезнсти>> $u(x)$, которая задается равенством
    \[u(x)=\sum_{j=1}^{m}u_{j}(c_{j}(x)),\]
    где $(c_{1}(x),...,c_{m}(x))$ --- решение задачи
    \[\sum_{j=1}^{m}u_{j}(c_{j})\rightarrow\max, \ \sum_{j=1}^{m}c_{j}\leqslant x, \
    c_{j}\geqslant 0, \ j=1,...,m.\]
    Здесь нужно заметить, что сделанных по поводу функций
    $u_{j}(c)$ предположений
    достаточно для того, чтобы функция $u(x)$ была монотонно возрастающей и
    дифференцируемой, а ее производная $u'(x)$ была монотонно убывающей.





    Что касается второго, репрезентативного производителя,
    то он полностью задается функцией затрат $C(y)$,
    определяемой равенством
    \[C(y)=\sum_{i=1}^{n}C_{i}(q_{i}(y)),\]
    где $(q_{1}(y),...,q_{n}(y))$ --- решение задачи
    \[\sum_{i=1}^{n}C_{i}(q_{i})\rightarrow\min, \ \sum_{i=1}^{n}q_{i}=y,
    \ q_{i}\geqslant0.\]
    Сделанных по поводу
    функций $C_{i}(q)$ предположений достаточно
    для того, чтобы функция $C(y)$ была дифференцируемой и монотонно возрастающей,
    а ее производная $C\,'(y)$ --- монотонно убывающей.
    Легко заметить также, что $u'(0)>C'(0)$.

    В случае, когда мы используем в своих рассуждениях
    репрезентативных производителя и потребителя, то имеет смысл и
    следующая задача:

\begin{equation}
 \label{max-pol-zat4}
    u(x)-C(y)\rightarrow\max,
\end{equation}
\begin{equation}
 \label{max-pol-zat5}
    x\leqslant y,
\end{equation}
\begin{equation}
 \label{max-pol-zat6}
    x\geqslant0, \ y\geqslant0.
\end{equation}

    Связь между задачами (\ref{max-pol-zat1})-(\ref{max-pol-zat3}) и
    (\ref{max-pol-zat4})-(\ref{max-pol-zat6}) описывается следующей
    теоремой.

\begin{teo}
    Если набор $\{(c_{1}^{*},...,c_{m}^{*}), \ (q_{1}^{*},...,q_{n}^{*})\}$
    является решением задачи
    (\ref{max-pol-zat1})-(\ref{max-pol-zat3}),
    то пара $(x^{*},y^{*})$, задаваемая равенствами
    \[x^{*}=\sum_{j=1}^{m}c_{j}^{*}, \ y^{*}=\sum_{i=1}^{n}q_{i}^{*},\]
    является решением задачи
    (\ref{max-pol-zat4})-(\ref{max-pol-zat6}). При этом выполняются
    следующие соотношения:
    \[\max_{j}u'_{j}(c_{j}^{*})=\min_{i}C\,'_{i}(q_{i}^{*})=u'(x^{*})=C\,'(y^{*}).\]

    Наоборот, если пара $(x^{*},y^{*})$ представляет собой решение
    задачи (\ref{max-pol-zat4})-(\ref{max-pol-zat6}), то набор
    $\{(c_{1}(x^{*}),...,c_{m}(x^{*})), \ (q_{1}(y^{*}),...,q_{n}(y^{*}))\}$
    является решением задачи
    (\ref{max-pol-zat1})-(\ref{max-pol-zat3}) и выполняются соотношения
    \[\max_{j}u'_{j}(c_{j}(x^{*}))=\min_{i}C\,'_{i}(q_{i}(y^{*}))=u'(x^{*})=C\,'(y^{*}).\]

\end{teo}

    В свою очередь решение задачи
(\ref{max-pol-zat4})-(\ref{max-pol-zat6})
    практически совпадает с состоянием равновесия в модели
    конкурентного равновесия $\mathfrak{M}_{r}$, в которой только
    два репрезентативных агента: репрезентативный потребитель и
    репрезентативный производитель.

    В этой модели репрезентативный потребитель решает при заданной цене $p$
    на единственный присутствующий в модели продукт следующую
    задачу о максимизации своей полезности:
    \[u(x)-px\rightarrow\max.\]
    Из решения этой задачи вытекает функция спроса $D(p)$ со стороны
    репрезентативного потребителя:
    \[D(p)=(u')^{-1}(p).\]


    Что касается репрезентативного производителя, то он при заданной цене $p$ на
    продукт решает свою задачу о максимизации прибыли
    \[py-C(y)\rightarrow\max,\]
    из решения которой вытекает функция $S(p)$  его предложения:
    \[S(p)=(C\,')^{-1}(p).\]


    Под состоянием равновесием в модели $\mathfrak{M}_{r}$ понимается набор
    $\{p^{*},x^{*},y^{*}\}$, состоящий из равновесной цены $p^{*}$,
    представляющий собой решение уравнения
    \[S(p)=D(p),\]
    равновесного потребления $x^{*}=D(p^{*})$ репрезентативного
    потребителя и равновесного выпуска $y^{*}=S(p^{*})$
    репрезентативного производителя.

    Подчеркнем, что хотя в модели $\mathfrak{M}_{r}$ присутствует
    только по одному потребителю и производителю, эта модель
    представляет собой именно модель конкурентного равновесия,
    поскольку и репрезентативный потребитель, и репрезентативный
    потребитель при решении своей задачи на максимум воспринимают
    цену продукта $p$ как экзогенно заданную.

    Соотношение между решением задачи
    (\ref{max-pol-zat4})-(\ref{max-pol-zat6}) и равновесием в модели
    $\mathfrak{M}_{r}$ находит свое отражение в следующей теореме.


\begin{teo}
    Пара $(x^{*},y^{*})$ является решением задачи
    (\ref{max-pol-zat4})-(\ref{max-pol-zat6}) тогда и только тогда,
    когда найдется $p^{*}$ такое, что набор $\{p^{*},x^{*},y^{*}\}$
    представляет собой состояние равновесия в модели
    $\mathfrak{M}_{r}$.
\end{teo}


    А отсюда вытекает теорема, показывающая, что состояния
    равновесия в моделях $\mathfrak{M}_{c}$ и $\mathfrak{M}_{r}$
    суть одно и то же.


\begin{teo}
    Если набор $\{p^{*}, \ (c_{1}^{*},...,c_{m}^{*}), \ (q_{1}^{*},...,q_{n}^{*})\}$
    является равновесием в модели $\mathfrak{M}_{с}$, то набор
    $\{p^{*},x^{*},y^{*}\}$, где
    \[x^{*}=\sum_{j=1}^{m}c_{j}^{*}, \ y^{*}=\sum_{i=1}^{n}q_{i}^{*},\]
    представляет собой состояние равновесия в модели
    $\mathfrak{M}_{r}$.

    Наоборот, если набор $\{p^{*},x^{*},y^{*}\}$ представляет собой
    равновесие в модели $\mathfrak{M}_{r}$, то набор
    $\{p^{*}, \ (c_{1}(x^{*}),...,c_{m}(x^{*})), \ (q_{1}(y^{*}),...,q_{n}(y^{*}))\}$
    --- это равновесие в модели $\mathfrak{M}_{c}$.
\end{teo}


    Эта теорема станет читателю особенно прозрачной с учетом
    следующего предложения.

\begin{prop}
    \[D(p)=\sum_{j=1}^{m}D_{j}(p) \ \forall p\geqslant0,
    \ S(p)=\sum_{i=1}^{n}S_{i}(p) \ \forall p\geqslant0.\]
\end{prop}

\begin{exer}
    Докажите все утверждения, сформулированные выше в этом пункте.
\end{exer}





\


\subsection{Монопольное равновесие}
    В условиях совершенной конкуренции производитель при решении
    задачи на максимум прибыли рассматривает цену выпускаемого
    продукта как экзогенно заданную. В этом пункте мы рассмотрим ситуацию,
    когда на рынке  некоторого  продукта присутствует только один
    поставщик (он же и производитель), а потребителей на этом рынке достаточно много
    для того, чтобы каждый из них при принятии решения о
    приобретении продукта воспринимал цену последнего как экзогенно
    заданную величину. В такой ситуации производитель будет скорее
    всего вести себя как \emph{монополист}, т.е. при решении задачи
    о максимизации прибыли он в явном виде будет учитывать  связь
    между объемом производства $y$ и ценой $p$, по которой выпущенный продукт
    можно продать.

    Мы будем считать, что эта связь задается дифференцируемой и
    монотонно убывающей функцией
    спроса $D(p)$, заданной на промежутке вида $[0,\tilde{p}]$  или
     вида $[0,+\infty)$. :
    \[y=D(y).\]
    Здесь предполагается, что либо $\tilde{p}$ задается равенством
    $D(\tilde{p})=0$, либо $D(p)\rightarrow 0$ при $p\rightarrow
    +\infty$.

    Формально говоря, при решении задачи о максимизации прибыли
    монополисту необходимо определять как объем производства, так и
    цену. Однако, поскольку они однозначным образом связаны между
    собой функцией спроса, достаточно считать переменной одну из
    этих величин.
    Сейчас нам будет удобнее взять в качестве переменной объем выпуска $y$
    и использовать не саму функцию спроса, а обратную функцию спроса
    $P^{D}(y)$, задаваемую равенством
    \[P^{D}(y)=D^{-1}(y),\]
    заданную на соответствующем промежутке.
    Напомним, что обратная функция спроса ставит в соответствие
    объему продаж $y$ максимально возможную цену, по
    которой потребители готовы весь этот объем приобрести.

    Задача максимизации прибыли производителем выглядит следующим
    образом:
\begin{equation}
\label{max-prib-monop}
    P^{D}(y)y-C(y)\rightarrow\max,
\end{equation}
    где $C(y)$ --- это дифференцируемая выпуклая монотонно
    неубывающая функция затрат производителя.
    В предположении, что величина $P^{D}(y)y-C(y)$ при некоторых значениях
    положительна, но принимает
    неположительные значения при $y=0$ и при достаточно больших
    значениях $y$, то решение задачи (\ref{max-prib-monop}), которое мы
    обозначим как $y^{m}$, существует и удовлетворяет необходимому условию максимума:
    \[(P^{D}(y^{m})y^{m}-C(y^{m}))'=0.\]

    Последнее равенство можно переписать в следующем виде:
    \[\left(1+\frac{P^{D}\,'(y^{m})y^{m}}{P^{D}(y^{m})}\right)P^{D}(y^{m})=C\,'(y^{m}).\]
    Величина $\frac{P^{D}\,'(y^{m})y^{m}}{P(y^{m})}$ представляет собой эластичность
    обратной функции спроса $P^{D}(y)$  в точке $y^{m}$. Поскольку, по правилу
    дифференцирования обратных функций, $P^{D}\,'(y^{m})=1/D\,'(p^{m})$ при
    $p^{m}=P^{D}(y^{m})$, то, как читатель уже заметил,
    $\frac{P^{D}\,'(y^{m})y^{m}}{P^{D}(y^{m})}=1/\varepsilon_{D}(p^{m})$,
    где функция $\varepsilon_{D}(p)$ --- это эластичность функции спроса,
    задаваемая равенством
\[
    \varepsilon_{D}(p)=\frac{D'(p)}{D(p)/p}.
\]
    Тем самым необходимое условия максимума для задачи
    (\ref{max-prib-monop}) можно записать в следующем виде:
\begin{equation}
\label{max-prib-monop1}
    \left(1+\frac{1}{\varepsilon_{D}(p^{m})}\right)p^{m}=C\,'(y^{m}),
\end{equation}
    что обычно в учебниках микроэкономики и делают.
    Очевидно, что если уравнение
    \[\left(1+\frac{1}{\varepsilon_{D}(P(y))}\right)P^{D}(y)=C\,'(y)\]
    имеет ровно одно решение, то оно и является решением задачи
    (\ref{max-prib-monop}).

    В важном частном случае, когда функция спроса имеет постоянную
    эластичность $-\sigma$, где $\sigma>1$, равенство
    (\ref{max-prib-monop1}) выглядит очень просто:
    \[\left(1-\frac{1}{\sigma}\right)p^{m}=C\,'(y^{m}),\]
    а $y^{m}$ представляет собой решение уравнения
    \[\left(1-\frac{1}{\sigma}\right)P^{D}(y)=C\,'(y).\]
 (\remrk{РИС ???????})



    Назовем величину $y^{m}$ монопольным выпуском, величину
    $p^{m}=P^{D}(y^{m})$ --- монопольной ценой, а пару этих чисел
    $(p^{m},y^{m})$ --- \emph{монопольным равновесием}.

    Сравним монопольное равновесие с конкурентным равновесием
    $(p^{c},y^{c})$, где $y^{c}$ --- конкурентный выпуск,
    определяемый как решение уравнения
    \[P^{D}(y)=C\,'(y),\]
    а $p^{c}=P^{D}(y^{c})$ --- конкурентная цена.




\begin{exer}
    В каком смысле можно называть пару $(p^{c},y^{c})$ конкурентным равновесием?
\end{exer}

    Справедливо неравенство $C\,'(y^{m})>0$. Следовательно,
    \[\left(1+\frac{1}{\varepsilon_{D}(p^{m})}\right)p^{m}=C\,'(y^{m})>0.\]
    Тем самым
    $\left(1+\frac{1}{\varepsilon_{D}(p^{m})}\right)>0.$
    При этом,
    поскольку функция спроса монотонно убывает, мы имеем:
\begin{equation}
    \label{otr-el}
    \varepsilon_{D}(p^{m})<0 .
\end{equation}
    Отсюда вытекает неравенство
    \[|\varepsilon_{D}(p^{m})|>1,\]
    которое говорит, что максимум прибыли монополиста достигается при таком
    значении цены, где функция спроса эластична.

\begin{exer}
    Объясните содержательный смысл последнего утверждения. Поясните,
    почему оно справедливо. Как устроено решение задачи максимизации
    прибыли монополиста в случае, когда функция спроса абсолютно
    неэластична: $D(p)=\text{const}$.
\end{exer}

    Из неравенства (\ref{otr-el}) вытекает, что
\begin{equation}
    \label{ner-mon-con}
    \left(1+\frac{1}{\varepsilon_{D}(P^{D}(y^{m}))}\right)P^{D}(y^{m})<P^{D}(y^{m}).
\end{equation}
    Отсюда следует,  что монопольная цена выше конкурентной цены,
    а монопольный выпуск меньше конкурентного выпуска:
        \[ p^{m}>p^{c}, \ y^{m}<y^{c}.\]


    Очевидно, что прибыль производителя при монопольных цене и выпуске
    выше, чем при конкурентных. С другой стороны, с точки зрения
    потребителей монопольная ситуация на рынке представляется менее
    предпочтительной, чем конкурентная. Можно ли сравнить конкурентное
    ($p^{c}, \ y^{c}$) и монопольные ($p^{m}, \ y^{m}$) равновесия
    с точки зрения общественного благосостояния?


    Такое сравнение легко осуществить в предположении, что функция
    спроса $D(p)$ определяется на основе решения
    репрезентативным производителем, введенным в предыдущем пункте, задачи
    \[u(x)-px\rightarrow\max,\]
    т.е. если эта функция задается равенством
    \[D(p)=(u')^{-1}(p).\]

    В этом случае в качестве возможного показателя общественного
    благосостояния в зависимости от объема  выпуска и потребления $y$
    может выступать величина
    \[u(y)-C(y).\]
    Из рассуждений, проведенных в предыдущем пункте, мы знаем, что
    максимум этой величины достигается при конкурентном объеме
    выпуска  $y^{c}$ (и только при нем). Поскольку $y^{m}<y^{c}$, мы имеем:
    \[u(y^{m})-C(y^{m})<u(y^{c})-C(y^{c}).\]


    Величина $(u(y^{c})-C(y^{c}))-(u(y^{m})-C(y^{m}))$
    называют безвозвратными потерями (deadweight loss). С ее помощью
    оценивают общественные потери, вызванные монополизацией рынка.
    (\remrk{РИС ???????})

\begin{exer}
    Вычислите монопольные цену и выпуск, а также безвозвратные
    потери в случае, когда $D(p)=a-bp, \ a,b>0,$ и $C(y)=cy, \
    0<c<a$. Проиллюстрируйте свои вычисления с помощью графиков.
\end{exer}




\subsection{Заключительные замечания к параграфу}

    Те предположения, в рамках которых мы проводили свои рассуждения
    в этом параграфе, очень ограничительны. В частности, мы
    рассматривали задачу распределения только одного ресурса. Если
    читатель знаком с теорией дифференцирования функций нескольких
    переменных, то он без труда обобщит все проведенные рассуждения
    по поводу оптимального распределения ресурса между предприятиями
    на случай, когда распределяется не один,  а несколько различных
    видов ресурсов. Мы настоятельно рекомендуем сделать это до того,
    как будут прочитаны последующие главы.

    Однако, можно поступить и иначе, вернувшись к задаче
    оптимального распределения ресурсов после того, как будет
    освоена теория нелинейного программирования, которая излагается
    в главе ??????.
