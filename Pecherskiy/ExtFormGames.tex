
\documentclass[12pt]{article}
\newfont{\ms}{msbm10}% scaled 1440}
\sloppy
\usepackage{amssymb}
\usepackage{amsmath}
\usepackage{makeidx}
\usepackage{graphicx}
\usepackage{epsfig}

\usepackage[russian]{babel}
\pagestyle{plain} \textwidth=16.0 cm \textheight=21.5 cm

\hoffset=-1 true cm
\voffset=-1 true cm
\renewcommand{\baselinestretch}{1.50}

\begin{document}

\newcommand{\be}{\begin{equation}}
\newcommand{\ee}{\end{equation}}
\renewcommand{\theequation}{\arabic{equation}}

\newcommand {\SC}{\succcurlyeq}
\newcommand {\GT}{\gtrsim}
\newcommand {\PR}{\precsim}
\newcommand {\UC}{\succsim}

\newcommand{\R}{{\rm I\!R}}
\newcommand{\B}{\hbox{\ms B}}
\newcommand{\Z}{\hbox{\ms Z}}

\newcommand{\bc}{\begin{center}}
\newcommand{\ec}{\end{center}}

\newcounter{definition}
\newcounter{proposition}
\newcounter{corollary}
\newcounter{theorem}
\newcounter{lemma}



\newtheorem{theorem}{Теорема}
\newtheorem{lemma}{Лемма}
\newtheorem{corollary}{Следствие}
\newtheorem{definition}{Определение}
\newtheorem{proposition}{Предложение}


\makeatletter
\@addtoreset{equation}{section}
\makeatother

\renewcommand{\thesection}{\arabic{section}.}
%\renewcommand{\thetheorem}{\arabic{section}.\arabic{theorem}}
%\renewcommand{\thedefinition}{\arabic{section}.\arabic{definition}}
%\renewcommand{\thelemma}{\arabic{section}.\arabic{lemma}}
%\renewcommand{\theproposition}{\arabic{section}.\arabic{proposition}}
%\renewcommand{\thecorollary}{\arabic{section}.\arabic{corollary}}






\section{Игры в позиционной форме }

Прежде чем перейти непосредственно к теме этой главы, мы должны
сделать небольшое отступление, касающееся используемой терминологии,
поскольку мы
будем встречаться с терминами полная информация, совершенная
информация, неполная, несовершенная информация, и т.\,д. Чтобы
избежать возможных недоразумений, отметим сразу же следующее.

Информационную структуру игры можно охарактеризовать несколькими
способами. Первый подразделяет бескоалиционные игры на игры с
совершенной информацией и игры с несовершенной информацией. (Хотя
мы еще не дали строгого определения позиционной формы игры, мы
кратко описали ее в предыдущей главе).

{\it В игре с {\bf совершенной информацией}\footnote{ Perfect
information.} каждое информационное множество одноточечно.  В
противном случае игра является игрой с несовершенной
информацией\footnote{ Imperfect information.}.}

В игре с совершенной информацией каждый игрок всегда знает точно,
в какой вершине дерева игры он находится. Далее, в игре нет
одновременных ходов игроков, и все игроки наблюдают ходы Природы
(если таковые есть). Иными словами, каждый игрок в точности знает
всю предысторию развития игры: он знает все ходы, которые делались
до того, как наступила очередь его хода. Хорошим примером здесь,
как правило, служат шахматы, когда каждый игрок не только знает
позицию, в которой наступает очередь его хода, но и знает то,
каким образом эта позиция была получена.

{\it В игре с {\bf неполной информацией}\footnote{ Incomplete
information.}. Природа делает ход первой, и он ненаблюдаем по
крайней мере одним из игроков. В противном случае игра является
игрой с полной информацией\footnote{ Complete information.}.}

Игра с неполной информацией является игрой с несовершенной
информацией, так как информационные множества некоторых игроков
содержат более одной вершины.

Чтобы прояснить описанное выше различие, приведем один пример. В
покере игроки делают ставки на то, у кого окажется наилучшая
комбинация карт, причем упорядочение комбинаций известно заранее (по
возрастанию - пара, две пары, тройка и т.д.) Что можно сказать о
различных модификациях этой игры, если:

1. Все розданные карты положены лицом вверх (открыты).

2. Все розданные карты положены лицом вниз (закрыты), так что игрок
не видит даже свои карты.

3. Все розданные карты положены лицом вниз (закрыты), но игрок знает
свои карты.

4. Все розданные карты положены лицом вниз (закрыты), затем каждый
игрок, не видя своих карт, показывает их всем остальным игрокам (так
называемый индийский покер).

Здесь первый вариант - это игра с совершенной информацией, а
остальные  -- игры с неполной информацией.

Hеобходимо особо подчеркнуть, что термин неполная информация
используется в литературе часто и в другом, старом смысле, согласно
которому в игре с полной информацией все игроки знают правила игры,
а в противном случае игра называется игрой с неполной информацией.
До 1967\,г. об играх с неполной информацией (в этом смысле)
говорили, когда хотели сказать, что их невозможно анализировать.
Затем Дж.\,Харшаньи\footnote{ Джон Харшаньи (J.\,Harsanyi)~---
лауреат Hобелевской премии по экономике 1994\,г.} заметил, что любая
игра с неполной информацией может быть переформулирована как игра с
полной, но несовершенной информацией просто за счет добавления
начального хода Природы, когда Природа выбирает между различными
правилами. Мы коснемся этой ситуации кратко в конце главы, чтобы
дать представление о том, что именно понимается под игрой с неполной
информацией на примере модификации рассмотренной нами ранее дуополии
по Курно. Подробнее по этому поводу см., например,  Rasmussen, 1989.

Итак, рассмотрим теперь более подробно  позиционную форму игры.

Рассмотрим простейший пример~--- примитивную игру
"крестики-нолики"\, на поле $3\times 3$. Перенумеруем
соответствующие клетки

\begin{center}
\begin{tabular}{|c|c|c|}\hline
1&2&3\\ \hline 4&5&6\\ \hline 7&8&9\\ \hline
\end{tabular}
\end{center}

Будем обозначать игроков соответственно~--- $X$ и $0$.
Соответствующий знак у вершин дерева означает, что очередь хода
принадлежит $X$ или $0$, соответственно. Тогда дерево этой игры (в
нем информационные множества одноточечны) будет иметь вид,
изображенный на рис.\, ???? (цифры у ребер означают номера клеток,
в которых ставится соответствующий $X$ или $0,$ а в вершине,
обозначенной $N,$ ход делает Природа, равновероятно (например,
подбрасывается монетка) выбирая очередность хода. Все
информационные множества здесь одноточечны, поскольку каждый игрок
точно знает все, что произошло до его хода. Эта игра, безусловно,
является игрой с совершенной информацией При этом необходимо иметь
в виду, что дерево отображает {\it все возможные} ходы независимо
от их разумности.


Extform1


Мы не изображаем это дерево полностью, поскольку очевидно, как оно
строится. Разумеется, как только выстраивается ряд из трех крестиков
или ноликов, то игра заканчивается и победивший игрок получает,
скажем, от проигравшего 1 рубль (доллар и пр.). В случае ничьей
соответствующие выигрыши~--- это $(0,0)$, т.\,е. никто никому ничего
не платит и ничего не получает.

Легко видеть, что даже в этом очень простом случае дерево игры,
хотя оно строится очевидным образом, весьма большое. Если же говорить,
скажем, о шахматах, в которых у каждого игрока уже на первом ходу
есть 20 возможных альтернатив, то дерево становится необозримым.

Хотя в дальнейшем мы будем лишь изредка прибегать к формальному
определению позиционной формы игры, тем не менее мы приведем его,
поскольку некоторые его элементы удобны для дальнейшего изложения.

Формально позиционная форма игры описывается с помощью следующих
элементов: списка игроков; дерева игры; указания для каждой вершины
номера игрока (или Природы~--- игрок номер $0$), который должен
ходить в этой вершине; списка ходов, доступных игроку в каждой
вершине и соответствия между ходами и непосредственно следующими
вершинами; информационных множеств; указания выигрышей в каждой
терминальной (окончательной) вершине; вероятностного распределения
на множестве ходов в каждой вершине, в которой ход  делает Природа.

Таким образом, мы считаем\footnote{ Мы придерживаемся здесь
обозначений, использованных в учебнике Mas-Collel, Whinston,
Green, 1995.}, что заданы следующие элементы:

1. $I=\{1,\ldots,n\}$~--- конечное множество игроков.

2. Мы имеем дерево игры с конечным множеством вершин $X$  и конечным
множеством ходов $A$.

При этом должно быть определено отображение $p:X\to
X\cup\,\{\emptyset\}$, которое каждой вершине $x$ ставит в
соответствие единственную непосредственно предшествующую  вершину
$p(x)$, за исключением начальной вершины $x_0$, для которой
$p(x_0)=\emptyset$. Далее, непосредственно следующие за $x$ вершины
тривиально определяются по $p$: $s(x)=p^{-1}(x)$. Чтобы у нас
действительно была древесная структура, необходимо, чтобы множество
всех предшествующих и множество всех последующих  вершин не
пересекались для каждой вершины $x$ (они могут быть найдены с
помощью итераций $p$ и $s$). Множество терминальных (окончательных)
вершин $T=\{x:s(x)=\emptyset\}$.

3. Далее мы должны иметь отображение $\alpha:X\setminus\{x_o\}\to
A$, ставящее в соответствие каждой вершине $x$, кроме начальной,
ход, который из непосредственно предшествующей вершины $p(x)$
приводит к $x$ и такой, что если $x'$, $x''\in s(x)$ и $x'\ne x''$,
то $\alpha(x')\ne\alpha(x'')$. Иными словами, в разные
непосредственно последующие вершины приводят разные ходы. \clearpage

Множество возможных ходов, доступных в вершине $x$, есть
$c(x)=\{a\in A:  a=\alpha(x')$ для некоторого $x'\in s(x)\}$.

4.  Набор информационных множеств ${\cal H}$ и отображение
$H:X\setminus T\to{\cal H}$, ставящее в соответствие каждой вершине
(кроме терминальной) информационное множество  $H(x)\in{\cal H}$.
Информационные множества образуют разбиение множества $X\setminus
T$. Необходимое требование: все вершины, лежащие в одном
информационном множестве имеют одни и те же допустимые ходы, т.\,е.
формально $c(x)=c(x')$, если $H(x)=H(x')$.  Мы можем, таким образом,
определить выбор, который доступен игроку в информационном множестве
$H$:
$$
c(H)=\{a\in A:a\in c(x)\quad{\rm для}\quad x\in H\}.
$$

5. Отображение $\mu:{\cal H}\to I\cup\{0\}$, ставящее в соответствие
каждому информационному множеству $H\in {\cal H}$ игрока (или
Природу, т.\,е. игрока $i=0$), который должен ходить в вершине из
этого множества. Будем обозначать через ${\cal H}_i=\{H\in{\cal
H}:\mu(H)=i\}$ те информационные множества, в которых очередь хода
принадлежит игроку $i$.

6. Функция $\rho:{\cal H}_0\times A\to [0,1]$, ставящая в
соответствие ходам в информационных множествах Природы вероятности,
удовлетворяющие условию
$$
\rho(H,a)=0\quad{\rm для}\quad a\notin C(H)
$$
и
$$
\sum_{a\in C(H)}\rho(H,a)=1\quad\forall\quad H\in{\cal H}_0.
$$

7. Набор функций выигрышей $u=\{u_1(\cdot),\ldots,u_n(\cdot)\}$,
$u_i(\cdot):T\to\Re$.

Здесь следует заметить, что, формально говоря, мы определили все для
конечных множеств, но данные определения могут быть перенесены и на
случай бесконечных множеств (вершин, ходов, игроков). Нарисовать
дерево уже было бы, разумеется, невозможно (хотя, впрочем, как мы
видим, даже для простейшего варианта крестиков-ноликов это дерево
достаточно велико), но все формальности можно было бы соблюсти, приписывая,
скажем, выигрыши не терминальным вершинам, а путям, соответствующим
разыгрыванию игры.

Важно также отметить, что мы ограничиваемся  рассмотрением {\it
игр с полной памятью}, в которых игроки не забывают то, что они
раньше знали, включая свои собственные ходы, сделанные ранее.
Игры, изображенные на рис.\, ???? таковыми не являются. Ниже
станет понятно, почему мы ограничиваемся рассмотрением таких игр.


ExtForm2


\begin{definition}
Игра в позиционной форме называется игрой с совершенной информацией,
если каждое информационное множество состоит из единственной
вершины. В противном случае игра называется игрой с несовершенной
информацией.
\end{definition}

Теперь мы должны остановиться на центральном для бескоалиционной
теории игр понятии стратегии. В предыдущей части мы не
останавливались подробно на понятии стратегии, так как там (чистая)
стратегия просто совпадала с ходом. Здесь же понятие стратегии
(наверное, нет необходимости говорить о том, что в повседневной
жизни под стратегией обычно понимается (в отличие от тактики)
долгосрочный план действий) становится более сложным, поскольку
в игре в позиционной форме игроки ходят, вообще говоря в
определенной последовательности, а кроме того, каждый игрок
может ходить несколько раз. Неформально, {\it стратегия}~--- это
полный возможный план, который описывает то, как игрок будет
действовать в каждых потенциально возможных обстоятельствах,
когда ему, может быть, придется делать ход.

С точки зрения игрока, множества упомянутых выше возможных обстоятельств
представлены набором его информационных множеств, причем каждое
информационное множество представляет различные {\it различимые}
обстоятельства, в которых ему может потребоваться ходить. Тем самым
стратегия игрока сводится к описанию того, как он планирует ходить в
{\it каждом} из его информационных множеств. На первый взгляд может
показаться, что такое определение несколько избыточно, поэтому мы
обсудим его позднее достаточно подробно. Итак, формально стратегия
определяется следующим образом.


\begin{definition}
Пусть ${\cal H}_i$~--- семейство всех информационных множеств игрока
$i$, ${\cal A}$~--- множество всех возможных ходов (действий) в
игре, $C(H)\subset {\cal A}$~---  множество ходов, возможных в
информационном множестве $H.$ Стратегия игрока $i$~--- это
отображение $s_i:{\cal H}_i\to{\cal A}$ такое, что $s_i(H)\in C(H)$
для {\bf каждого} $H\in{\cal H}_i$.
\end{definition}



То, что стратегия~--- это {\it полный} возможный план, нельзя
недооценивать, особенно, как мы увидим, это будет важно в
дальнейшем, в частности при рассмотрении равновесия по Нэшу.
Определение игроком своей стратегии подобно написанию
перед игрой инструкции относительно того, как его представитель
может действовать, просто заглядывая в эту инструкцию. Или, иначе,
определение игроком $i$ своей стратегии можно условно трактовать следующим
образом: в каждом информационном множестве игрока $i$ находится его
агент, которому он сообщает, какой ход должен будет сделать этот
агент, если ему придется делать ход, т.\,е. если игра дойдет до
соответствующего информационного множества.

Здесь очень важно иметь в виду следующее. Как {\it полный} план,
стратегия часто определяет действия игрока в информационных
множествах, которые могут быть даже не достигнуты во время
действительного разыгрывания игры. Так, в крестиках-ноликах
стратегия игрока $0$ описывает, в частности, то, что он будет
делать, \emph{если} на 1-м ходу $X$ сыграет в центр. Но в
действительной игре $X$ может сыграть вовсе не в центр.  Более
того, стратегия игрока может включать планы действий, которые его
{\it собственная} стратегия делает неуместными. Опять же, в
крестиках-ноликах стратегия игрока $X$ включает описание того, что
он будет делать после того, что он сыграет на первом ходу в центр,
а $0$ ответит в левый нижний угол, даже если $X$ на первом ходу
играет верхний левый. Это, возможно, кажется странным, но играет
очень важную роль в динамическом случае. Ниже мы обсудим эту
кажущуюся странность подробнее. Итак, еще раз, и \emph{это крайне
важно}:

{\it стратегия}~--- это {\it полный} возможный план действий,
который говорит, \emph{что} игрок будет делать в {\it каждом его
информационном множестве}.

Рассмотрим следующую простую игру (рис.\,????).


ExtForm3



У первого игрока две стратегии: $H$ и $T$. А у игрока 2 их {\it
четыре}; поскольку у него 2 информационных множества, следовательно,
каждая стратегия должна определять ход в каждом из этих
информационных множеств. А именно:

$s_1$: $H$, если 1-й сыграл $H$; $H$, если 1-й сыграл $T$;

$s_2$: $H$, если 1-й сыграл $H$; $T$, если 1-й сыграл $T$;

$s_3$: $Т$, если 1-й сыграл $H$; $H$, если 1-й сыграл $T$;

$s_4$: $Т$, если 1-й сыграл $H$; $T$, если 1-й сыграл $T$.

Отметим здесь еще одно чрезвычайно важное обстоятельство, которое
является дополнительным оправданием того, что стратегия
определяется как отображение, ставящее в соответствие
\emph{каждому} информационному множеству некоторый ход. Имея набор
стратегий каждого игрока, мы можем построить нормальную форму
данной игры: поскольку выбор игроками своих стратегий определяет
ход в {\it каждом} информационном множестве, значит, полностью
определяет траекторию или путь, по которому будет развиваться
игра. Hормальная форма игры, изображенной на рис.\,?????, есть

\begin{center}
\begin{tabular}{cc}
&$\begin{array}{cccc} \quad s_1\quad& s_2\quad&\quad s_3\qquad& s_4\qquad \end{array}$\\
$\begin{array}{c} H\\T \end{array}$& $\left( \begin{array}{cccc}
(a_1,b_1)&(a_1,b_1)&(a_2,b_2)&(a_2,b_2)\\
(a_3,b_3)&(a_4,b_4)&(a_3,b_3)&(a_4,b_4) \end{array} \right)$\\
\end{tabular}
\end{center}

Каждый набор стратегий определяет траекторию движения по дереву и
тем самым определяет исход игры. Ясно также, что мы тем самым имеем
возможность говорить о равновесии по Нэшу. Однако здесь сразу же
необходимо отметить следующее. Простой перенос определения равновесия
по Нэшу для игр в нормальной форме на игры в позиционной форме
совершенно не учитывает того, ради чего собственно, и рассматривается
позиционная форма, а именно, последовательность ходов игроков.
Более сложная структура игры и наличие определенного порядка ходов
позволяет накладывать на равновесия по Нэшу некоторые дополнительные
требования, отражающие это усложнение структуры игры, а поэтому
выделять из множества возможных равновесий по Нэшу те, которые
обладают некоторыми дополнительными свойствами, "тоньше"\, отражая
структуру игры.

Прежде чем обратиться к более подробному рассмотрению равновесия по
Hэшу приведем важную теорему существования равновесия по Нэшу.

\begin{theorem}
{(\rm Kuhn, 1953)}\footnote{ Русский перевод статьи Куна см.:
Позиционные игры. [Сб.]~/ Под ред. H.\,H.\,Воробьева и
И.\,H.\,Врублевской.  М.: Hаука~-- Физматгиз, 1967.}.  B конечной
игре с совершенной информацией существует равновесие по Hэшу в
чистых стратегиях.
\end{theorem}

Мы начнем со следующего примера, который покажет, что, к сожалению,
просто равновесие по Нэшу не всегда дает разумное предсказание.
\smallskip

П р и м е р\,\, (Mas-Colell, Whinston, Green). Фирма $E$
(entrant)~--- новичок~--- рассматривает вопрос о том, входить ли
ей на рынок, где в текущий момент есть одна единственная
укоренившаяся фирма $I$ (incumbent). Если $E$ решается на вход, то
$I$ может ответить двумя способами: она может предоставить вход,
отдавая часть своих продаж, но не изменяя цену, либо она может
вступить в хищническую войну, которая приведет к драматическому
снижению цен. Дерево, соответствующее рассматриваемой ситуации,
изображено на рис.\,????.


ExtForm4


Нормальная форма этой игры имеет следующий вид (рис.\,5):

\begin{center}
\begin{tabular}{cccc}
&&\multicolumn{2}{c}{I}\\
&& Война &  Предоставить\\
&&(если вход)&   (если вход) \\
$\begin{array}{c} {}\\ E\\ {}\\ \end{array}$ &$\begin{array}{c} нет\\ \\
вход\end{array}$& \multicolumn{2}{c}{
$\left(\begin{array}{cccccc} &(0,2)&&&(0,2)&\\
\\
&(-3,-1)&&&(2,1)& \end{array} \right)$}\\
\multicolumn{4}{c}{}\\
\multicolumn{4}{c}{Рис. 5.}\\
\end{tabular}
\end{center}

Здесь две ситуации равновесия по Hэшу  в чистых стратегиях: (не
вх.; война) и (вход; предоставить).  Но первая из этих
ситуаций~--- это не разумное предсказание: фирма $E$ может
предвидеть, что если она выберет вход, то $I$ в действительности
выберет предоставить, т.\,е. война, если вход~--- не заслуживает
доверия.

Для того чтобы исключить ситуации типа (не вх.; война, если вход),
мы рассмотрим принцип последовательной рациональности: стратегия
игрока должна предписывать оптимальный ход в каждой вершине
дерева. То есть, если игрок находится в некоторой вершине дерева,
его стратегия должна предписывать оптимальный выбор, начиная с
этой точки, при данных стратегиях его оппонентов.  В этом смысле
стратегия "война, если вход"\, таковой не является, ибо после
входа единственная оптимальная стратегия для $I$~---
"предоставить".  В нашем примере сделать все очень просто: начнем
с того, что определим оптимальное поведение для $I$ в игре на
этапе после входа~--- это, очевидно, "предоставить".  Теперь мы
можем определить оптимальное поведение фирмы $E$ до этого момента,
с учетом предвидения того, что произойдет после входа. Это можно
сделать, рассмотрев так называемую редуцированную позиционную
форму, где пост-входное принятие решения $I$ заменено на
соответствующие выигрыши, которые возникают при оптимальном
пост-входном поведении фирмы $I$ (рис.\,????). А это уже
простейшая задача принятия индивидуального решения, причем
решение~--- "вход".


ExtForm5



Этот тип процедуры, которая начинается с нахождения оптимального
поведения в конце игры, а затем определения оптимального поведения
на более ранних шагах в предвидении того, \emph{что} будет происходить
дальше, называется \emph{обратной индукцией}.

Однако, прежде чем остановиться на обратной индукции более подробно,
мы должны отметить следующее достаточно существенное обстоятельство,
касающееся смешанных стратегий. А именно, если мы рассматриваем игры
в позиционной форме, то игроки могут рандомизировать свои чистые
стратегии способом, отличным от стандартного, в котором используются
смешанные стратегии, приписывающие каждой чистой стратегии игрока
(множество которых может быть очень большим) вероятность того, что
игрок будет ее играть. В позиционной форме появляется возможность
рандомизации раздельно в каждом информационном множестве. Такой
способ рандомизации приводит к так называемым {\it стратегиям
поведения}.

\begin{definition}
В игре в позиционной форме $\Gamma_E$ стратегия поведения игрока $i$
определяет для каждого информационного множества $H\in {\cal H}_i$ и
альтернативы $a\in C(H)$ вероятность $\lambda_i(a,H)\geq 0$, причем
$\sum_{a\in C(H)}\lambda_i(a,H)=1$ для всех $H\in {\cal H}_i$.
\end{definition}

Оказывается (Kuhn, 1953; см. также, например, Петросян, Зенкевич,
Семина, 1998), что для игр с полной памятью (в частности, поэтому мы
рассматриваем только игры с полной памятью) эти два типа
рандомизации эквивалентны. ( А именно, для любой стратегии поведения
игрока $i$ существует его смешанная стратегия, дающая в точности
такое же распределение выигрышей для любых стратегий (смешанных или
стратегий поведения), которые могут играться остальными игроками, и
наоборот. Важно подчеркнуть, что именно полная память
играет здесь ключевую роль.)

Это соответствие можно установить следующим образом. Будем, как
всегда, обозначать чистые стратегии игрока $i$ через $s_i$. Пусть
$\sigma_i$~--- некоторая его смешанная стратегия. Будем называть
некоторую вершину $x$ дерева $\Gamma_E$ возможной для $s_i$, если
существует такой набор стратегий $s=(s_i,s_{-i})$, что траектория,
определяемая этим набором, проходит через $s$. Обозначим множество
всех возможных для $s_i$ вершин через $P(s_i)$.

Информационное множество $H$ называется существенным для $s_i$, если
оно содержит некоторую возможную для $s_i$ вершину. Множество
существенных для $s_i$ информационных множеств обозначим через
$R(s_i)$.

Пусть $\sigma_i$~--- некоторая смешанная стратегия игрока $i$. Тогда
стратегия поведения $\lambda_i$, соответствующая смешанной стратегии
$\sigma_i$, определяется следующим образом. Если $H\in R(s_i)$, то
$$
\lambda_i(a,H)=\frac{\sum_{\{s_i:H\in
R(s_i),s_i(H)=a\}}\sigma_i(s_i)} {\sum_{\{s_i:H\in
R(s_i)\}}\sigma_i(s_i)}. \eqno(*)
$$
Если $H\not\in R(s_i)$, то знаменатель этой дроби обращается в ноль,
поэтому стратегию $\lambda_i$ можно определить произвольно,
например,
$$
\lambda_i(a,H)=\sum_{\{s_i:s_i(H)=a\}}\sigma_i(s_i).
$$
Если $\lambda_i$~--- стратегия поведения, то $\sigma_i$ можно
определить как
$$
\sigma_i(s_i)=\prod_{H}\lambda_i(s_i(H),H).
$$
При этом $\lambda_i$ оказывается стратегией поведения,
соответствующей $\sigma_i$. Поэтому в играх с полной памятью
безразлично, каким способом рандомизировать. Терминологически
мы всегда будем говорить о смешанных стратегиях.

В игре с неполной памятью могут существовать смешанные стратегии,
для которых эквивалентных им стратегий поведения не существует.
Действительно, рассмотрим следующий пример. Мы не указываем
выигрыши в терминальныз вершинах, поскольку это в данном случае не
существенно.
\smallskip

П р и м е р\,\, (Osborn, Rubinstein). Рассмотрим игру,
изображенную на рис.\,????.

\begin{figure}
\moveright 2cm \vbox{\special{em: graph fig2_7.pcx} \vspace{4.3cm}
         }
\centerline{Рис. 7.}
\end{figure}

Пусть смешанная стратегия игрока $\sigma$ определяется следующим
образом: с вероятностью $1/2$ играется $L$, а потом еще раз $L$, и с
вероятностью $1/2$ играется $R$, а потом еще раз $R$. Исходом,
соответствующим этой стратегии, является распределение $(1/2,0,0,1/2)$
на множестве терминальных вершин. Но такой исход
не может быть обеспечен ни одной стратегией поведения: стратегия
поведения $((p,1-p),(q,1-q))$ инициирует распределение на множестве
терминальных вершин, в которых исход, соответствующий $u_2$, имеет
вероятность $0$ в случае только, если $p=0$ или $q=1$, но тогда
вероятность $(L,L)$ или $(R,R)$ есть $0$.
