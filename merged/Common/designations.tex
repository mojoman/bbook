\chapter{Условные обозначения}

\section*{Индексы}
Обозначения для индексов:

$i=1 \ldots m$ --- индекс, принимающий значения от 1 до $m$; индексом нумеруются столбцы
\remrk{??????} матриц;

$j=1 \ldots n$ --- индекс, принимающий значения от 1 до $n$; индексом нумеруются строки
\remrk{??????} матриц;

$k=1 \ldots q$ --- индекс, принимающий значения от 1 до $q$;

$l=1 \ldots r$ --- индекс, принимающий значения от 1 до $r$.

\section*{Множества, вектора, числа}

Множества и подмножества векторов обозначаются \texttt{контурными}
заглавными латинскими буквами, например, $\st{X}$ или $\st{Y}$.

    Знаки $\mid$ и $:$ используются при описании множеств вместо словосочетания <<таких что>>.
    Например, запись
\[
    \{\vc{x}\in \st{X} \mid \vc{x}\in \st{Y}\},
\]
    как и запись
\[
    \{\vc{x}\in \st{X} : \vc{x}\in \st{Y}\},
\]
    может использоваться для обозначения пересечения $\st{X}\cap\st{Y}$ множеств $\st{X}$ и
    $\st{Y}$, а запись
\[
    \{x\in\R^{1} \mid \exists z\in\R^{1} : x=z^{2}+1\}
\]
    --- для несколько неудачного обозначения  множества $[1,+\infty)$.

Символом $\R^n$ обозначается $n$-мерное евклидово пространство, элементами которого являются
$n$-компонентные векторы (столбцы или строки).

Для обозначения векторов пространства используются
\texttt{латинские} буквы. Вектора-столбцы пространства $\R^n$
обозначаются полужирными строчными буквами, а их компоненты ---
строчными буквами с нижними индексами:

\[
\vc{x} = \left( {\begin{array}{*{20}c}
   {x_1 }  \\
   {x_2 }  \\
    \vdots   \\
   {x_n }  \\
\end{array}} \right) \in \R^n
\]

\noindent Вектора-строки также обозначаются полужирными строчными
буквами (разница между векторами-столбцами и векторами-строками
будет ясна по контексту или отдельно комментироваться):

\[
\vc{p} = (p_1 ,p_2 , \ldots ,p_n ) \in \R^n
\]

В отличие от векторов, для обозначения обычных (вещественных) чисел
используются \texttt{греческие} буквы, например, $\alpha$, $\beta$,
$\gamma$, $\theta$, $\pi$. Предполагается, что все они лежат в
множестве $\R$.


\section*{Действия над векторами}

Пусть $\vc{x},\vc{y} \in \st{X} \subset \R^n$. Будем обозначать:

\begin{itemize}
  \item $\vc{y} \geqq \vc{x}$, если $y_i \geq x_i$, $i=1 \ldots n$;
  \item $\vc{y} \geq \vc{x}$, если $y_i \geq x_i$, $i=1 \ldots n$, и хотя бы одно из
неравенств вы\-полняется как строгое;
  \item $\vc{y} \gg \vc{x}$, если $y_i > x_i$, $i=1 \ldots n$.
\end{itemize}

Пусть $\vc{p}$ --- вектор-строка из $\R^n$, а $\vc{x}$ ---
вектор-столбец из $\R^n$. Тогда

\[
\vc{p} \cdot \vc{x} = p_1 x_1 + p_2 x_2 + \ldots + p_n x_n = \sum
\limits_{i=1}^n p_i x_i,
\]

\noindent что можно трактовать как произведение строки на столбец
или \emph{скалярное произведение}\index{скалярное произведение}.


\section*{Матрицы}

Матрицы обозначаются заглавными латинскими буквами.

$A$ --- матрица коэффициентов технологических затрат; размерность
матрицы указывается нижним индексом, например, $A_{m\times n}$.

Столбцы матрицы $A_{m\times n}$ (вектора-столбцы) обозначаются
$\vc{a}^1, \vc{a}^2, \ldots, \vc{a}^n$. Строки матрицы $A_{m\times
n}$ (вектора-строки) обозначаются $\vc{a}_1, \vc{a}_2, \ldots,
\vc{a}_m$. Столбцы и строки также выделяются полужирным шрифтом.
Отдельные элементы матрицы обозначаются $a_{ij}$, где $i=1 \ldots
n$, $j=1 \ldots m$. (У МЕНЯ ВСЕ НАОБОРОТ. КАК ПРАВИЛЬНО???????)

Таким образом, матрица $A$ может быть представлена как набор
векторов-строк, векторов-столбцов или отдельных элементов, как
показано ниже:

\[
A = \left( {\begin{array}{*{20}c}
   {\vc{a}_1 }  \\
   {\vc{a}_2 }  \\
    \vdots   \\
   {\vc{a}_m }  \\
\end{array}} \right) = \left( {\vc{a}^1 ,\vc{a}^2 , \ldots ,\vc{a}^n } \right) = \left( {\begin{array}{*{20}c}
   {a_{11} } & {a_{12} } & {a_{13} } &  \ldots  & {a_{1n} }  \\
   {a_{21} } & {a_{22} } & {a_{23} } &  \ldots  & {a_{2n} }  \\
    \ldots  &  \ldots  &  \ldots  &  \ldots  &  \ldots   \\
   {a_{m1} } & {a_{m2} } & {a_{m3} } &  \ldots  & {a_{mn} }  \\
\end{array}} \right)
\]


Для квадратной матрицы размерность обозначается одним числом,
например, $B_{n}$.

Единичная матрица обозначается как $E$.
