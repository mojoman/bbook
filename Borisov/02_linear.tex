\chapter{Линейные модели экономики}

В предыдущей главе (?????) мы познакомили читателя с тем, как можно размышлять об
экономических проблемах в терминах функций и их производных (и интегралов?????????). Однако
наше рассмотрение было ограничено только функциями одной переменной, что очень сильно
уменьшало наши возможности.



Экономика --- наука <<многомерная>>. Неотъемлемым ее инструментом
являются функции многих переменных. В этой главе мы вкратце покажем,
как можно использовать для анализа экономических явлений простейшие
из функций многих переменных --- линейные. Мы будем проводить наши
рассуждения в терминах векторов и матриц, предполагая, что читатель
познакомился с действиями над ними в курсе линейной алгебры.

Сразу же обратим внимание читателя на то, что очень часто вектора называют точками, что
вполне справедливо, ибо, например, вектора в двухмерном пространстве можно изображать как
направленные отрезки, а можно и как точки на плоскости.



\section{Матрицы и векторы в анализе экономических процессов}

    Матрицы и векторы представляют собой самый естественный язык
    экономического анализа. В этом параграфе мы попытаемся
    научить экономиста им пользоваться и убедить читателя в
    том, что этот язык является очень простым и естественным. Мы не
    будем здесь доказывать  никаких теорем, а просто приведем
    несколько простых примеров, которые помогут читателю привыкнуть
    к использованию векторно-матричных обозначений и увидеть
    содержательный экономический смысл как самих матриц и векторов,
    так и операций над ними.

\subsection{Предварительные замечания}

    Векторные обозначения удобны в
    ситуации, когда рассматриваются функции многих переменных.
    Например, при рассмотрении некоторой функции $n$ переменных
    мы вместо записи $f(x_{1},\ldots,x_{n})$ можем
    использовать запись $f(\vc{x})$, где $\vc{x}$ --- это $n$-мерный
    вектор, координатами, или компонентами, которого являются числа
    $x_{1},\ldots,x_{n}$.

    Напомним, что множество всевозможных $n$-мерных векторов, т.е.
    упорядоченных наборов из $n$ вещественных чисел, обозначается как
    $\R^{n}$ и называется арифметическим или координатным пространством.
    Элементы этого множество можно складывать и умножать на
    вещественные числа по известным правилам. Это значит, что $\R^{n}$
    является линейным пространством.

    При использовании стандартных
    матричных обозначений $n$-мерный вектор $\vc{x}$ можно считать как
    вектор-строкой
    \[(x_{1},\ldots,x_{n}),\]
    так и вектор-столбцом
    \[\left(
     \begin{array}{c}
        x_{1} \\
        x_{2} \\
        \vdots \\
        x_{n}  \\
      \end{array}
    \right)=(x_{1},\ldots,x_{n})^{T},\]
    где $T$ --- это знак транспонирования.
        Как записывать элементы пространства $\R^{n}$, в виде
    вектор-строк или вектор-столбцов, это зачастую вопрос
    удобства. В случае, когда нам это не важно, из
    соображений удобства мы будем записывать $n$-мерные векторы как
    вектор-строки.
     В то же время здесь надо проявлять определенную
    осторожность. А именно, следует всегда в том или ином контексте
    точно знать, какие обозначения мы используем. В частности, это
    нужно делать для того, чтобы невзначай не начать складывать
    вектор-строки с вектор-столбцами.

    Здесь есть и еще одна важная тонкость. Дело в том, что на
    пространстве $\R^{n}$ можно задавать линейные функции. Напомним,
    что функция $f(\vc{x})$, заданная на $\R^{n}$ и принимающая
    вещественные значения, называется \emph{линейной}, если
\begin{itemize}
    \item [1)\ ] \ для любых $\vc{x}$ и $\vc{y}$ из $\R^{n}$
    выполняется равенство
    \[f(\vc{x}+\vc{y})=f(\vc{x})+f(\vc{y});\]
    \item [2)\ ] \ для любого $\vc{x}$ из $\R^{n}$ и любого
    (вещественного) числа $\alpha$ справедливо равенство
    \[f(\alpha\vc{x})=\alpha f(\vc{x}).\]
\end{itemize}
    Математики любят называть линейные функции, заданные на
    пространстве $\R^{n}$, \emph{линейными функционалами}.

    Из курса линейной алгебры мы помним, что устройство линейных
    функционалов на $\R^{n}$ очень просто. А именно, каждый линейный
    функционал $f(\vc{x})$ однозначным образом задается некоторым упорядоченным
    набором из $n$ вещественных чисел $(f_{1},\ldots,f_{n})$ в том смысле, что
\begin{equation}
  \label{l-f}
    f(\vc{x})=f_{1}x_{1}+\ldots+f_{n}x_{n} \
    \forall \vc{x}=
    \left(
     \begin{array}{c}
        x_{1} \\
                \vdots \\
        x_{n}  \\
      \end{array}
    \right)\in \R^{n}.
\end{equation}
    Поскольку упорядоченный набор из $n$ вещественных чисел представляет собой вектор из
    $\R^{n}$, мы можем заключить, что каждый линейный функционал однозначно
    задается вектором из $\R^{n}$. В дальнейшем мы будем просто отождествлять каждый линейный
    функционал $f(\vc{x})$ с тем вектором $\vc{f}=(f_{1},\ldots,f_{n})$, который
    удовлетворяет равенству (\ref{l-f}).
    Однако это не означает, что
    элемент того пространства, на котором задан линейный функционал,
    и сам линейный функционал являются математическими объектами
    одной природы. Например, далеко не всегда будет осмысленной
    сумма вектора из $\R^{n}$ и линейного функционала, заданного на
    $\R^{n}$.

    Иногда очень важно не путать элементы пространства $\R^{n}$ и
    линейные функционалы на этом пространстве, хотя и те и другие
    представляют собой именно наборы из $n$ вещественных чисел. Для
    того, чтобы такой путаницы не возникло, мы договоримся, что при
    использовании матричных обозначений в ситуации, где мы
    считаем элементами пространства $\R^{n}$ $n$-мерные
    вектор-столбцы, линейные функционалы мы будем отождествлять с
    $n$-мерными вектор-строками. И наоборот, в том случае, когда под
    элементами $\R^{n}$ мы будем понимать вектор-строки, в качестве
    линейных функционалов будут выступать вектор-столбцы.

    Уточним нашу договоренность. Предположим, что нам дана
    некоторая $n$-мерная вектор-строка
    \[\vc{y}=(y_{1},\ldots,y_{n}).\]
    В этом случае запись произведения
\begin{equation} \label{proizv-vec}
    \vc{y}\vc{x},
\end{equation}
    будет (если мы используем векторно-матричные обозначения) говорить нам о том, что
     $\vc{x}$ --- некоторый $n$-мерный вектор-столбец:
        \[\vc{x}=\left(
     \begin{array}{c}
        x_{1} \\
        x_{2} \\
        \vdots \\
        x_{n}  \\
      \end{array}
    \right),\]
    а значение величины $\vc{y}\vc{x}$ определяется согласно
    стандартным правилам векторно-матричного умножения:
    \begin{equation} \label{lin-func}
    \vc{y}\vc{x}=y_{1}x_{1}+\ldots+y_{n}x_{n}.
    \end{equation}
    Если, наоборот, мы знаем, что $\vc{x}$ --- это вектор-столбец,
    то одного взгляда на выражение (\ref{proizv-vec}) нам будет
    достаточно для того, чтобы сразу понять, что $\vc{y}$ --- это
    вектор-строка.

    Если мы не только знаем, что $\vc{y}$ --- это
    некоторая вектор-строка,    но и считаем, что она зафиксирована,
    а $\vc{x}$ рассматриваем как вектор переменных, то
    на запись (\ref{proizv-vec}) мы будем смотреть как на выражение,
    задающее по формуле (\ref{lin-func}) линейный функционал
     на пространстве $\R^{n}$, элементы которого записываются как
     вектор-столбцы. Сам этот линейный
     функционал мы зачастую будем отождествлять с вектором $\vc{y}$ и
     даже называть его линейным функционалом $\vc{y}$. Быть может, в
     этом случае было бы правильным заменить выражение (\ref{proizv-vec}) более
     точным выражением $f_{\vc{y}}(\vc{x})$, подчеркивающим, что
     речь идет о линейном функционале, задаваемом вектором
     $\vc{y}$, но это сделало бы наши обозначения слишком громоздкими.

     Наоборот, если зафиксированным мы считаем
     вектор-столбец $\vc{x}$, то выражение (\ref{proizv-vec}) будет
     свидетельствовать о том, что нам задан линейный функционал на
     пространстве $\R^{n}$, элементы которого записываются как
     вектор-строки и обозначаются с помощью $\vc{y}$. Естественно, этот
     линейный функционал мы будем отождествлять с вектором $\vc{x}$.

     В тех же ситуациях, когда мы не пользуемся матричными
     обозначениями, а все элементы пространства $\R^{n}$
     обозначаются как вектор-строки, для векторов
\[
    \vc{x}=(x_{1},\ldots,x_{n}) \ \text{и}  \ \vc{y}=(y_{1},\ldots,y_{n})
\]
     выражение $\vc{y}\vc{x}$ тоже будет использоваться для
     обозначения величины $\sum_{i=1}^{n}y_{i}x_{i}$.



    В этой главе мы будем последовательно использовать матричные
    обозначения, а в последующих главах будем при использовании
    тех или иных обозначений будем исходить из соображений
    удобства и ясности изложения.

    Прежде, чем двигаться дальше, введем еще несколько полезных обозначения. Для двух векторов
\[
    \vc{x}=(x_{1},\ldots,x_{n}) \ \text{и}  \ \vc{y}=(y_{1},\ldots,y_{n})
\]
    неравенство
\[
    \vc{x} \gg \vc{y}
\]
    будет обозначать, что вектор $\vc{x}$ превосходит вектор $\vc{y}$ по всем координатам:
\[
    x_{i}>y_{i}, \ i=1,\ldots,n,
\]
    неравенство
\[
    \vc{x} \geqq \vc{y}
\]
    будет говорить нам, что $\vc{x}$ не меньше вектора $\vc{y}$ по всем координатам:
\[
     x_{i} \geqslant y_{i}, \ i=1,\ldots,n,
\]
    а неравенство
\[
    \vc{x} \geq \vc{y}
\]
    мы будем использовать для обозначения ситуации, когда выполняются следующие соотношения:

\[
    \vc{x} \geqq \vc{y}, \ \vc{x} \neq \vc{y}.
\]
    которые говорят о том, что $\vc{x}$ не меньше вектора $\vc{y}$ по всем координатам
    и хотя бы по одной  координате превосходит последний. \remrk{РИС, РИС}

    В частности, неравенство $\vc{x}\gg \vc{0}$ означает, что все координаты вектора
    $\vc{x}$, неравенство $\vc{x}\geqq \vc{0}$ --- что они все неотрицательны, а
    неравенство $\vc{x}\geq \vc{0}$ --- что все координаты вектора $\vc{x}$ неотрицательны и
    хотя бы одна из них строго положительна. Читатель уже догадался, что $\vc{0}$ --- это
    вектор, все координаты которого раны нулю:
\[
    \vc{0}=(0,\ldots,0).
\]




\subsection{Графическое представление линейных функционалов}


\subsection{Экономическая интерпретация линейных функционалов}


    Теперь перейдем к вопросу о том, какой экономический смысл может
    иметь произведение вектор-строки на вектор-столбец, для чего рассмотрим следующий
    условный пример. Пусть нам дано предприятие, на котором производится
    $n$ различных видов
    продукции. Предположим, что в течение какого-то года выпуск
    1-го продукта составил $x_{1}$ единиц, выпуск 2-го продукта ---
    $x_{2}$ единиц, ..., выпуск $n$-го продукта --- $x_{n}$. В этом
    случае мы можем записать набор чисел, показывающих выпуск
    различных продуктов как вектор
        \[\vc{x}=\left(
     \begin{array}{c}
        x_{1} \\
        x_{2} \\
        \vdots \\
        x_{n}  \\
      \end{array}
    \right)\]
    и назвать его вектором выпуска.

    Предположим далее, что цена 1-го продукта равна $p_{1}$
    руб./ед., цена 2-го продукта --- $p_{2}$ руб./ед., ...,
    цена $n$-го продукта --- $p_{n}$ руб./ед. Мы можем записать набор цен на
    выпускаемую продукцию как вектор
    \[\vc{p}=(p_{1}, p_{2},\ldots,p_{n}).\]
    Сразу же подчеркнем, что вектор выпуска удобно записывать именно как
    вектор-столбец, а вектор цен --- как вектор строку.
        Если нам известен вектор цен $\vc{p}$ и вектор выпуска $\vc{x}$,
    то величина
\begin{equation} \label{vipusk}
    \vc{p}\vc{x}=p_{1}x_{1}+\ldots+p_{n}x_{n}
\end{equation}
     представляет собой суммарный выпуск продукции рассматриваемого
     предприятия в денежном выражении в течение рассматриваемого
     года.

     Вспомнив наш недавний разговор о линейных функционалах,
     отметим, что если считать заданным вектор цен $\vc{p}$, а
     вектор выпуска $\vc{x}$ рассматривать как вектор переменных, то
     выражение (\ref{vipusk}) задает линейный функционал,
     показывающий зависимость выпуска  в денежном
     выражении от того, что производится
     на предприятии в натуральном выражении. Если же считать заданным
     вектор выпуска $\vc{x}$, и взглянуть на вектор цен $\vc{p}$ как
     на вектор переменных, то речь будет идти о линейном
     функционале, показывающем зависимость номинального выпуска от
     цен.

     Если мы рассматриваем не завод, а экономику в целом, производящую $n$
     видов конечной продукции, обозначим через $\vc{y}$ и $\vc{p}$
     векторы, компоненты которых показывают объем выпуска различных
     видов конечной продукции и цены цен на них соответственно, то величина
     $\vc{p}\vc{y}$ показывает размер выпуска валового внутреннего продукта
     (ВВП) в номинальном выражении.

     Рассмотрим еще один пример. Допустим, что для
     производства некоторого продукта требуется использование
     $m$ различных видов ресурсов. Будем считать, что технология
     производства этого продукта такова, что для выпуска
     одной его единицы необходимо затратить $a_{1}$ единиц первого
     ресурса, $a_{2}$ единиц второго ресурса, ..., $a_{m}$ единиц $m$-го
     ресурса. Запишем все эти числа в виде вектор-столбца
    \[\vc{a}=\left(
     \begin{array}{c}
        a_{1} \\
        a_{2} \\
        \vdots \\
        a_{m}  \\
      \end{array}
    \right).\]




     Такой вектор-столбец, который называют \emph{технологическим
     способом} производства рассматриваемого продукта, читатель,
     безусловно, встречал в поваренной
     книге, поскольку рецепты, которые в ней можно найти и
     представляют собой именно технологические способы. Например,
     если для производства одной порции борща необходимо положить в
     кастрюлю 1 свеклу, 2 морковки, половину кочана капусты и 0.1 кг.
     мяса, то технологический способ (рецепт) производству борща --- это
     вектор-столбец
     \[\left(
     \begin{array}{c}
        1 \\
        2 \\
        0.5 \\
        0.1  \\
      \end{array}
    \right).\]

    Если нам известен технологический способ $\vc{a}$ изготовления
    продукта и объем $x$ его выпуска, то суммарные затраты ресурсов
    на производство этого объема задаются вектором
    \[x\vc{a}=x\left(
     \begin{array}{c}
        a_{1} \\
        a_{2} \\
        \vdots \\
        a_{m}  \\
      \end{array}
    \right)=
    \left(
     \begin{array}{c}
        xa_{1} \\
        xa_{2} \\
        \vdots \\
        xa_{m}  \\
      \end{array}
    \right).\]
    Например, количество различных ингредиентов,  необходимых
    для производства 5 порций борща, полностью задается вектором
    \[5\left(
     \begin{array}{c}
        1 \\
        2 \\
        0.5 \\
        0.1  \\
      \end{array}
    \right)=
    \left(
     \begin{array}{c}
        5 \\
        10 \\
        2.5 \\
        0.5  \\
      \end{array}
    \right).\]

    Если наряду с технологическим способом $\vc{a}$
    производства некоторого продукта нам задан вектор цен
    \[\vc{p}=(p_{1},p_{2},\ldots,p_{m})\]
    на ресурсы, где, очевидно, $p_{1}$ --- цена первого ресурса,
    $p_{2}$ --- цена второго ресурса, ..., $p_{m}$ --- цена $m$-го
    ресурса, то величина
    \[\vc{p}\vc{a}=p_{1}a_{1}+\ldots+p_{m}a_{m}\]
    показывает себестоимость производства (одной единицы)
    рассматриваемого продукта.


\subsection{Экономическая интерпретация умножения матрицы на вектор-столбец}
    Теперь рассмотрим более общую ситуацию, когда на некотором
    предприятии может производиться  $n$
    различных видов продукции, для производства которых требуются затраты
    $m$ различных видов ресурсов. При этом допускается, что для
    производства каждого отдельного продукта какие-то ресурсы могут и не
    понадобиться. Будем считать, что каждый из продуктов
    производится с помощью ровно одного технологического способа. Под технологическим
    способом  производства мы, как и выше, понимаем $m$-мерный
    вектор-столбец. Пусть векторы
    \[\vc{a_{1}}=\left(
     \begin{array}{c}
        a_{11} \\
        a_{21} \\
        \vdots \\
        a_{m1}  \\
      \end{array}
    \right),\
    \vc{a_{2}}=\left(
     \begin{array}{c}
        a_{12} \\
        a_{22} \\
        \vdots \\
        a_{m2}  \\
      \end{array}
    \right),\ \ldots,\
    \vc{a_{n}}=\left(
     \begin{array}{c}
        a_{1n} \\
        a_{2n} \\
        \vdots \\
        a_{mn}  \\
      \end{array}
    \right)\]
    представляют собой технологические способы производства
    соответственно $1$-го,  $2$-го, ..., $n$-го продуктов, а числа
    $x_{1}$, $x_{2}$, ..., $x_{n}$ показывают размер выпуска этих
    продуктов. В этом случае суммарные затраты различных видов
    ресурсов можно записать в виде вектора-столбца
    \[\vc{y}=\left(
     \begin{array}{c}
        y_{1} \\
        y_{2} \\
        \vdots \\
        y_{m}  \\
      \end{array}
    \right),\]
    где $y_{1}$ --- затраты первого ресурса, $y_{2}$ ---
    затраты второго ресурса, ...,  $y_{m}$ --- затраты $m$-го
    ресурса, который определяется следующим образом:
    \[\vc{y}=x_{1}\vc{a_{1}}+\ldots+x_{n}\vc{a_{n}},\]
    или, более подробно,
    \[\left(
     \begin{array}{c}
        y_{1} \\
        y_{2} \\
        \vdots \\
        y_{m}  \\
      \end{array}
    \right)=
    x_{1}\left(
     \begin{array}{c}
        a_{11} \\
        a_{21} \\
        \vdots \\
        a_{m1}  \\
      \end{array}
    \right)+\ldots+
    x_{n}\left(
     \begin{array}{c}
        a_{1n} \\
        a_{2n} \\
        \vdots \\
        a_{mn}  \\
      \end{array}
    \right).\]
    Если мы составим из технологических способов $m \times n$ матрицу
    \[\vc{A}=\left(
\begin{array}{cccc}
   a_{11} & a_{12} & \ldots & a_{1n} \\
   a_{21} & a_{22} & \ldots & a_{2n} \\
   \ldots& \ldots &\ldots &\ldots \\
   a_{m1} & a_{m2} & \ldots & a_{mn}
\end{array}
\right),\]
    которую мы будем называть технологической или производственной
    матрицей, а из чисел
    $x_{1}$, $x_{2}$, ..., $x_{n}$ --- вектор-столбец
    \[\vc{x}=\left(
     \begin{array}{c}
        x_{1} \\
        x_{2} \\
        \vdots \\
        x_{n}  \\
      \end{array}
    \right),\]
    который будем назвать вектором выпуска, то вектор суммарных
    затрат ресурсов $\vc{y}$ можно получить как
    произведение матрицы $\vc{A}$ на вектор $\vc{x}$:
\begin{equation}
\label{lin-rav}
    \left(
     \begin{array}{c}
        y_{1} \\
        y_{2} \\
        \vdots \\
        y_{m}  \\
      \end{array}
    \right)=\left(
\begin{array}{cccc}
   a_{11} & a_{12} & \ldots & a_{1n} \\
   a_{21} & a_{22} & \ldots & a_{2n} \\
   \ldots& \ldots &\ldots &\ldots \\
   a_{m1} & a_{m2} & \ldots & a_{mn}
\end{array}
\right)\left(
     \begin{array}{c}
        x_{1} \\
        x_{2} \\
        \vdots \\
        x_{n}  \\
      \end{array}
    \right).
\end{equation}
    Последнее равенство можно записать в следующей совсем компактной
    форме:
\begin{equation}
    \label{lin-rav-vec}
    \vc{y}=\vc{A}\vc{x}.
\end{equation}
    Именно такого типа компактные и обозримые выражения делают
    матрично-векторные обозначения очень полезными. Если хорошо
    понимать, что они означают, то работать с ними
    намного проще, чем с теми или иными громоздкими конструкциями,
    в которых векторно-матричные обозначения не используются.

    Матрицу $\vc{A}$ можно рассмотреть не только как совокупность
    столбцов, но и как набор сток,  что
    позволит лучше понять равенства  (\ref{lin-rav}) и (\ref{lin-rav-vec}),
    взглянув на
    них с еще одной точки зрения. Пусть $\vc{a^{1}}=(a_{11},a_{12},\ldots,a_{1n})$ --- это
    первая строка матрицы $\vc{A}$,  $\vc{a^{2}}=(a_{21},a_{22},\ldots,a_{2n})$ --- вторая ее строка,
    ..., $\vc{a^{m}}=(a_{m1},a_{m2},\ldots,a_{mn})$ --- $m$-я строка. С помощью таких
    обозначений соотношения (\ref{lin-rav}) и (\ref{lin-rav-vec})
    можно эквивалентным образом записать как следующий набор
    равенств:
\[\left\{
\begin{array}{ccc}
y_{1}&=&\vc{a^{1}}\vc{x},   \\
y_{2}&=&\vc{a^{2}}\vc{x}, \\
                      & \ldots &\\
y_{m}&=&\vc{a^{m}}\vc{x} , \\
\end{array} \right.\]
    содержательная интерпретация которых не составляет особого
    труда. Первая строка $\vc{a^{1}}=(a_{11},\ldots,a_{1n})$ матрицы
    $\vc{A}$ состоит из чисел, показывающих, сколько единиц
    первого ресурса необходимо израсходовать для
    производства одной единицы первого продукта, одной единицы второго продукта
    и т.д. Поэтому, если нам известен вектор выпуска $\vc{x}$, то мы
    можем вычислить суммарные затраты первого ресурса, которые равны
    \[\vc{a^{1}}\vc{x}= a_{11} x_1 + a_{12} x_2 + \ldots + a_{1n} x_n\]
    единиц. Очевидно, что суммарные затраты второго ресурса равны
    \[\vc{a^{2}}\vc{x} = a_{21} x_1 +  a_{22} x_2 + \ldots + a_{2n} x_n,\]
    единиц и т.д.


\subsection{Экономическая интерпретация умножения вектор-строки на матрицу}

   Пусть нам дана технологическая матрица $\vc{A}$ и вектор цен на ресурсы
    \[\vc{p}=(p_{1},p_{2},\ldots,p_{m}).\] Чтобы понять
    содержательный смысл вектора $\vc{q}=\vc{p}\vc{A}$, достаточно
    заметить, что величина
    \[q_{1}=\vc{p}\vc{a_{1}}=p_{1}a_{11}+\ldots+p_{m}a_{m1}\]
    показывает себестоимость производства одной единицы первого
    продукта, величина
    \[q_{2}=\vc{p}\vc{a_{2}}=p_{1}a_{12}+\ldots+p_{m}a_{m2}\]
    --- себестоимость производства одной единицы второго продукта,
    ...,
    \[q_{n}=\vc{p}\vc{a_{n}}=p_{1}a_{1n}+\ldots+p_{m}a_{mn}\]
    --- себестоимость производства (одной единицы) $n$-го продукта.
    Тем самым, в рамках наших построений, $\vc{q}$ --- это вектор
    себестоимостей выпускаемых продуктов.

    Если в дополнение к технологической матрице $\vc{A}$ и вектору
    цен цен на ресурсы $\vc{p}$ нам задан вектор выпуска $\vc{x}$,
    имеет смысл скалярная величина
    \[\vc{p}\vc{A}\vc{x}.\]
    Эта величина представляет собой полные затраты предприятия на
    производство продуктов в количествах, которые показывает вектор
    выпуска, поскольку
    \[\vc{p}\vc{A}\vc{x}=\vc{p}\vc{y},\]
    где $\vc{y}=\vc{A}\vc{x}$ --- вектор затрат ресурсов. Заметим,
    что, в то же время,
    \[\vc{p}\vc{A}\vc{x}=\vc{q}\vc{x},\]
    где $\vc{q}=\vc{p}\vc{A}$ --- вектор себестоимостей. Но это
    равенство не противоречит данной нами интерпретации величины
    $\vc{p}\vc{A}\vc{x}$, а просто напоминает нам о том, что, с
    одной стороны, величину $\vc{p}\vc{A}\vc{x}$ можно определять
    посредством равенства
    \[\vc{p}\vc{A}\vc{x}=\vc{p}(\vc{A}\vc{x}),\]
    а с другой --- посредством равенства
    \[\vc{p}\vc{A}\vc{x}=(\vc{p}\vc{A})\vc{x}.\]
    При этом, очевидно, в обоих случаях получаем одно и то
    же число, что и позволяет нам не расставлять скобки
    в выражении $\vc{p}\vc{A}\vc{x}$.


\subsection{Экономическая интерпретация умножения матриц}

    В этом параграфе мы рассмотрим несколько более сложную
    ситуацию. А именно, мы будем предполагать, что с помощью
    ресурсов, поступающих на предприятие извне, производятся не конечные
    продукты, а промежуточные. А эти последние необходимы для
    производства конечной продукции. Как и выше, будем предполагать,
    что существует $n$ видов конечной продукции и $m$ видов
    поступающих извне ресурсов. Что касается промежуточных
    продуктов, то будем считать, что имеется $r$ их  различных видов.

    Предположим, что производственные возможности предприятия
    по производству промежуточных продуктов описываются с помощью
    технологической $r \times m$ матрицы
    \[\vc{B}=\left(
\begin{array}{cccc}
   b_{11} & b_{12} & \ldots & b_{1r} \\
   b_{21} & b_{22} & \ldots & b_{2r} \\
   \ldots& \ldots &\ldots &\ldots \\
   b_{m1} & b_{m2} & \ldots & b_{mr}
\end{array}
\right),\]
    а производственные возможности
    по производству конечных продуктов --- с помощью
    технологической $r \times n$ матрицы
    \[\vc{C}=\left(
\begin{array}{cccc}
   c_{11} & c_{12} & \ldots & c_{1n} \\
   c_{21} & c_{22} & \ldots & c_{2n} \\
   \ldots& \ldots &\ldots &\ldots \\
   c_{r1} & c_{r2} & \ldots & c_{rn}
\end{array}
\right).\]

    Пусть нам задан вектор выпуска $\vc{x}$. Для того, чтобы
    обеспечить выпуск конечных продуктов в количествах, задаваемых
    этим вектором, необходимо затратить те или иные количества
    различных промежуточных продуктов. Эти количества полностью
    задаются вектором затрат промежуточных продуктов
    $\vc{z}=\vc{C}\vc{x}$. В свою очередь, для того чтобы произвести
    промежуточные продукты в требуемых количествах, необходимо
    затратить ресурсы в количествах, задаваемых вектором
    $\vc{y}=\vc{B}\vc{z}=\vc{B}\vc{C}\vc{x}$.

    Отсюда следует, что соотношение между вектором выпуска $\vc{x}$
    и вектором затрат ресурсов $\vc{y}$ можно записать в виде
    равенства
    \[\vc{y}=\vc{A}\vc{x},\]
    где $\vc{A}$ --- произведение матрицы $\vc{B}$ на матрицу
    $\vc{C}$:
    \[\vc{A}=\vc{B}\vc{C}.\]
    Тем самым произведение матриц $\vc{B}$ и $\vc{C}$,
    матрица $\vc{A}$, представляет собой
    технологическую матрицу, устанавливающую связь между вектором
    выпуска $\vc{x}$ и непосредственно вектором затрат ресурсов
    $\vc{y}$ без привлечения вектора производимых и тут же
    затрачиваемых промежуточных продуктов. Например, второй столбец
    матрицы $\vc{A}$, столбец
        \[\vc{a_{2}}=\left(
     \begin{array}{c}
        a_{12} \\
        a_{22} \\
        \vdots \\
        a_{m2}  \\
      \end{array}
    \right),\]
    состоит из чисел, показывающих, сколько единиц всех видов
    ресурсов, поступающих на предприятие извне, необходимо затратить
    при производстве одной единицы второго вида конечной продукции.

    Здесь следует сделать одно важное замечание, которое лишний раз
    подчеркивает важность и плодотворность матрично-векторных
    обозначений. Проводя рассуждение, поясняющее содержательный
    смысл матрицы $\vc{A}$ мы даже не упомянули формулу, по которой
    вычисляются ее элементы, но это не помешало нам объяснить их
    экономический смысл.

\begin{exer}
    Пусть $\vc{p}$ --- вектор цен на ресурсы. Какую экономическую
    интерпретацию можно дать векторам $\vc{p}\vc{B}$ и $\vc{p}\vc{B}\vc{C}$,
    а также числам $\vc{p}\vc{B}\vc{z}$ и
    $\vc{p}\vc{B}\vc{C}\vc{x}$?
\end{exer}
\begin{exer}
    Что с экономической точки зрения
    означают следующие равенства:
    \[(\vc{p}\vc{B})\vc{C}=\vc{p}(\vc{B}\vc{C}),\]
    \[(\vc{B}\vc{C})\vc{x}=\vc{B}(\vc{C}\vc{x}),\]
    \[(\vc{p}\vc{B})(\vc{C}\vc{x})=(\vc{p}\vc{B})(\vc{C}\vc{x}).\]
\end{exer}

\




\section{Модель Леонтьева}
    В этом параграфе мы вкратце расскажем о модели Леонтьева (модели
    межотраслевого баланса),
    представляющая собой одну из первых современных линейных моделей
    экономики. Она описывает экономические процессы на народно-хозяйственном
    уровне. Эта модель интересна сама по себе, а также является
    хорошей иллюстрацией применения теории матриц в экономической
    науке.



\subsection{Описание модели}
    Представим себе экономику страны в целом как совокупность  из $n$ отраслей,
    каждая из которых производит только один вид продукции.
    Обозначим через $x_{i}\geqslant0$ суммарное количество
    (валовой выпуск) продукции, выпущенное отраслью $i=1,\ldots,n$,
    т.е. $i$-го продукта,
    в течение года или другого единичного промежутка времени.
    Произведенная продукция частично выступает в качестве конечного продукта,
    идущего на непроизводственное потребление и капитальные
    вложения, а частично расходуется в самом процессе
    производства как этого продукта, так и продуктов, производимых в других отраслях.
    Тем самым производимые и воспроизводимые
    в моделируемой экономике продукты
    могут играть двоякую роль --- роль конечных продуктов и роль
    ресурсов.

    При этом надо учитывать то, что некоторые виды ресурсов,
    например, рабочая сила, не производятся в рамках моделируемой
    экономики, они являются невоспроизводимыми. Однако сначала мы
    этот факт будем игнорировать.

    Пусть $x_{ij}\geqslant0$ --- количество продукта $i$, затрачиваемого на
    производства продукта $j$. Основное предположение модели
    Леонтьева состоит в том, что для всех $i,j=1,\ldots,n$
    затраты $x_{ij}$ $i$-го продукта
    на производство  $j$-го продукта пропорциональны валовому
    выпуску последнего, $x_{j}$. Это значит, что заданы неотрицательные числа
    $a_{ij}$, $i,j=1,\ldots,n$, такие что
    \[x_{ij}=a_{ij}x_{j},\ i,j=1,\ldots,n.\]
    Числа $a_{ij}$ называются коэффициентами прямых затрат.
    Обозначим через $y_{j}\geqslant0$ конечный выпуск продукта
    $j=1,\ldots,n$.
    Основные соотношения, задающие модель, выглядят следующим
    образом:
\begin{equation}
     \label{Leontief}
     x_{i}=\sum_{j=1}^{n}a_{ij}x_{j}+y_{i},\ i=1,\ldots,n.
\end{equation}
    Они как раз и говорят о том, что всех $i=1,\ldots,n$ валовой
    выпуск $i$-го продукта в размере $x_{i}$ делится на две части.
    Некоторая часть, в размере $\sum_{j=1}^{n}a_{ij}x_{j}$,
    затрачивается в процессе производства, а оставшаяся часть
    $y_{i}$ представляет собой конечную продукцию.
    Соотношения (\ref{Leontief}) удобно запиывать в матричной форме:
\begin{equation} \label{Leontief-vect}
    \vc{x}=\vc{A}\vc{x}+\vc{y},
\end{equation}
    где
    \[\vc{A}=\left(
\begin{array}{cccc}
   a_{11} & a_{12} & \ldots & a_{1n} \\
   a_{21} & a_{22} & \ldots & a_{2n} \\
   \ldots& \ldots &\ldots &\ldots \\
   a_{n1} & a_{n2} & \ldots & a_{nn}
\end{array}
\right), \ \vc{x}=\left(
     \begin{array}{c}
        x_{1} \\
        x_{2} \\
        \vdots \\
        x_{n}  \\
      \end{array}
    \right),\ \vc{y}=\left(
     \begin{array}{c}
        y_{1} \\
        y_{2} \\
        \vdots \\
        y_{m}  \\
      \end{array}
    \right).\]
    Матрицу $\vc{A}$, состоящую из коэффициентов прямых затрат,
    мы будем называть \emph{матрицей Леонтьева} или \emph{леонтьевской},
    вектор $\vc{x}$ --- вектором валового
    выпуска, а вектор   $\vc{y}$ ---вектором конечного выпуска.
    Следует обратить внимание, что леонтьевская  матрица $\vc{A}$ является
    квадратной. Количество ее строк совпадает с количеством столбцов
    и равно количеству выпускаемых в моделируемой экономике продуктов.


\subsection{Продуктивность}
    Соотношение (\ref{Leontief-vect}) связывают объемы валового выпуска с
    объемами выпуска конечной продукции и может быть использовано
    для согласованного расчета этих величин. Если нам известен
    вектор валового выпуска $\vc{x}$, то по нему однозначным образом
    определяется вектор конечного выпуска $\vc{y}$. Если же
    первоначально задан желаемый вектор конечного выпуска $\vc{y}$,
    то (\ref{Leontief-vect}) можно рассмотреть как уравнение
    относительно вектора переменных $\vc{x}$. Решение этого уравнения
    позволит определить необходимые объемы валовых выпусков по отраслям.

    По экономическому смыслу коэффициенты прямых затрат являются
    неотрицательными, что мы и предположили. В то же время
    неотрицательными должны быть также вектор валового выпуска
    $\vc{x}$ и вектор конечного выпуска $\vc{y}$. Однако, очевидно, эти
    векторы нельзя рассматривать как параметры модели. Они
    являются векторами переменных и просто предположить их
    неотрицательность нельзя. Конечно, пара векторов $\vc{x}=0$ и
    $\vc{y}=0$ удовлетворяет равенству (\ref{Leontief-vect}), но
    случай, когда просто ничего не производится, интереса не
    представляет. В связи с этим становится понятным
    следующее определение.

    Матрица коэффициентов прямых затрат $\vc{A}$ называется
    \emph{продуктивной}, если найдется вектор $\vc{\tilde{x}}\gg0$,
    такой что
    \[\vc{\tilde{x}}-\vc{A}\vc{\tilde{x}}\gg0.\]

\begin{exer} \label{monot-neotr-matr}
    Пусть $\vc{D}$ --- неотрицательная, т.е. состоящая
    из неотрицательных элементов, $m\times n$ матрица, а $\vc{u}$ и
    $\vc{v}$ --- два вектора-столбца размерности $n$.
       Докажите, что если $\vc{u}\geqq\vc{v}$, то
    $\vc{D}\vc{u}\geqq\vc{D}\vc{v}$.
    Докажите, что если матрица $\vc{D}$ строго положительна, т.е.
    состоит только  из положительных элементов, то из
    неравенство $\vc{u}\gg\vc{v}$ вытекает неравенство
    $\vc{D}\vc{u}\gg\vc{D}\vc{v}$. Покажите, что если матрица
    $\vc{D}$ неотрицательна, но не является строго положительной, то
    из неравенства $\vc{u}\gg\vc{v}$, вообще говоря, не следует
    соотношение $\vc{D}\vc{u}\gg\vc{D}\vc{v}$.

\end{exer}


    Продуктивность леонтьевской матрицы означает, что
    экономика может обеспечить некоторый положительный конечный выпуск всех
    продуктов. Здесь, правда нужно подчеркнуть, матрица $\vc{A}$
    не включает в себя коэффициенты затрат тех необходимых для
    производства невоспроизводимых
    ресурсов, которые сами по себе не являются выпускаемыми одной из
    $n$ отраслей продуктами, например, коэффициенты затрат труда. Тем
    самым в данном определении речь идет только о технологической
    возможности производить все продукты в положительных
    количествах. Обеспечена ли эта возможность невоспроизводимыми
    ресурсами --- это отдельный вопрос.

    Заметим также, что в случае, когда у экономики имеется принципиальная
    технологическая возможность выпуска хотя бы одного строго
    положительный вектора конечных продукта, а именно вектора
    \[\vc{\tilde{y}}=\vc{\tilde{x}}-\vc{A}\vc{\tilde{x}}\gg0,\]
     то имеется
    возможность выпуска и любого другого неотрицательного вектора
    конечного продукта. Действительно, для любого неотрицательного
    вектора конечного конечного выпуска $\vc{y}$ найдется такое
    число $\lambda>0$, что
\begin{equation}
    \label{NER_LAMB}
    \vc{y}\leqq\lambda\vc{\tilde{y}}.
\end{equation}
    Существование такого числа гарантируется тем, что вектор
    $\vc{\tilde{y}}$ является положительным по всем координатам.
\begin{exer}
    Чему равно наименьшее из тех $\lambda$, которые удовлетворяют неравенству
    (\ref{NER_LAMB})?
\end{exer}
    Очевидно, что для $\lambda$, при котором выполняется
    неравенство (\ref{NER_LAMB}), и вектора
    $\vc{x}=\lambda\vc{\tilde{x}}$ выполняется неравенство
    \[\vc{x}-\vc{A}\vc{x}\geqq\vc{y},\]
    которое и говорит о том, что выпуск вектора конечного выпуска
    $\vc{y}$ является технологически возможным. Здесь, правда,
    возникает естественный вопрос, можно ли отыскать неотрицательный
    вектор $\vc{x}$, для которого это неравенство выполняется как
    равенство. Как мы увидим ниже, если матрица $\vc{A}$
    продуктивна, то ответ на этот вопрос является положительным.

    До сих пор мы рассматривали модель Леонтьева с
    натурально-вещественной стороны. Теперь посмотрим на нее с
    ценностно-стоимостной точки зрения. Предположим, что нам задан
    неотрицательный вектор цен $\vc{p}=(p_{1},\ldots,p_{n})$.
    В этом случае величина
    \[\vc{p}\vc{a_{j}}=\sum_{i=1}^{n}p_{i}a_{ij}\]
    показывает, чему в денежном выражении равны те затраты на
    производство одной единицы продукта $j$, которые связаны с
    расходом воспроизводимых продуктов,  (но не включают те
    затраты, которые вызваны
    использованием ресурсов, не производимых в рамках последней).
    Это позволяет назвать величину
    \[p_{j}-\vc{p}\vc{a_{j}}\]
    добавленной стоимостью (одной единицы) продукта $j$. Добавленная
    стоимость сосем необязательно окажется положительной или хотя бы
    неотрицательной при любом неотрицательном векторе цен. При этом
    если она отрицательна, то это значит, что с экономической точки
    зрения производство данного продукта заведомо невыгодно. Оно
    будет невыгодно даже в том случае, если добавленная стоимость
    окажется равной нулю, поскольку не будут покрыты затраты
    на невоспроизводимые ресурсы, например, на оплату труда.

    Для того чтобы производство каждого из $n$ было экономически
    выгодным, необходимо (но не обязательно достаточно), чтобы цены
    были такими, что добавленная стоимость каждого продукта была
    положительной. Возникает вопрос о том, существуют ли такие цены.
    Ответ на этот вопрос зависит от устройства матрицы $\vc{A}$.
    Дадим в связи с этим вопросом следующее определение.

    Леонтьевская Матрица $\vc{A}$ называется \emph{прибыльной},
    если существует неотрицательный
    вектор цен $\vc{\tilde{p}}=(\tilde{p}_{1},\ldots,\tilde{p}_{n})$, такой что
    \[\vc{\tilde{p}}-\vc{\tilde{p}}\vc{A}\gg\vc{0}.\]

    Данное определение является довольно условным, потому что
    величина добавленной стоимости и прибыль --- это не одно и то
    же, а вектор $\vc{p}-\vc{p}\vc{A}$ можно интерпретировать как
    вектор добавленных стоимостей при векторе цен $p$, но не как вектор прибылей.

    Определения прибыльности и продуктивности в некотором смысле
    симметричны. Естественно ожидать, что продуктивность и
    прибыльность матрицы как-то связаны. Так оно и есть. Оказывается, матрица
    продуктивна тогда, и только тогда, когда она прибыльна. Об этом,
    в частности (но далеко не только об этом) говорит важная
    теорема, которую мы сформулируем чуть ниже, напомнив несколько определений.

    Предположим, что нам дана квадратная матрица $\vc{B}$. Эта
    матрица называется \emph{обратимой}, если у нее существует \emph{обратная
    матрица}, т.е. такая матрица $\vc{B^{-1}}$, для которой
    выполняется хотя бы одно из двух следующих равенств:
    \[\vc{B^{-1}}\vc{B}=\vc{E} \ \text{или} \ \vc{B}\vc{B^{-1}}=\vc{E}.\]
    Заметим, что каждое из этих равенство выполняется в том, и
    только том случае, когда выполняется другое. Напомним также, что
     матрица обратима тогда, и только тогда, когда ее
    определитель не равен нулю.
    Квадратную матрицу $\vc{B}$ можно возводить в квадрат:
    \[\vc{B^{2}}=\vc{B}\vc{B},\]
    в куб:
    \[\vc{B^{3}}=\vc{B}\vc{B}\vc{B},\]
    и в любую целую положительную степень $s=1,2,\ldots$ по следующему
    естественному правилу:
    \[\vc{B^{s}}=\vc{B}\vc{B^{s-1}}, \ s=1,2,\ldots,\]
    где предполагается, что
    \[\vc{B^{0}}=\vc{E}.\]


    Теперь предположим, что нам дана бесконечная последовательность
    $\{\vc{B}(s)\}_{s=0}^{\infty}=\{\vc{B}(0),\vc{B}(1),\vc{B}(2),\ldots\}$
    матриц одинакового размера $m\times n$:
    \[\vc{B}(s)=\left(
\begin{array}{cccc}
   b_{11}(s) & b_{12}(s) & \ldots & b_{1n}(s) \\
   b_{21}(s) & b_{22}(s) & \ldots & b_{2n}(s) \\
   \ldots& \ldots &\ldots &\ldots \\
   b_{n1}(s) & b_{n2}(s) & \ldots & b_{mn}(s)
\end{array}
\right), \ s=0,1,\ldots \,.\]
    В этом случае можно задаться вопросом о ее сходимости.
    По определению последовательность $\{\vc{B}(s)\}_{s=0}^{\infty}$
    сходится при $s\rightarrow\infty$ к матрице
    \[\vc{B}=\left(
\begin{array}{cccc}
   b_{11} & b_{12} & \ldots & b_{1n} \\
   b_{21} & b_{22} & \ldots & b_{2n} \\
   \ldots& \ldots &\ldots &\ldots \\
   b_{n1} & b_{n2} & \ldots & b_{mn}
\end{array}
\right), \]
    если для всех $i=1,\ldots,m$ и $j=1,\ldots,n$ выполняется
    соотношение
    \[b_{ij}(s)\rightarrow_{s\rightarrow\infty}b_{ij},\]
    т.е. если имеет место поэлементная сходимость.

    Можно также поставить вопрос о сходимости матричного ряда
    \[\sum_{s=0}^{\infty}\vc{B}(s)=\vc{B}(0)+\vc{B}(1)+\vc{B}(2)+\ldots .\]
    Этот \emph{матричный ряд сходится}, если сходится
    последовательность его частичных сумм
    $\{\vc{C}(r)\}_{r=0}^{\infty}$, задаваемых как
    $\vc{C}(q)=\sum_{s=0}^{r}\vc{B}(s), \ r=0,1,\ldots,$
     или, что то же, если для каждого $i=1,\ldots,m$ и
    $j=1,\ldots,n$ сходится числовой ряд
\begin{equation} \label{summa-riada}
    \sum_{s=0}^{\infty}b_{ij}(s)=b_{ij}(0)+b_{ij}(1)+b_{ij}(2)+\ldots\, .
\end{equation}
    Если матричный ряд $\sum_{s=0}^{\infty}\vc{B}(s)$ сходится, то
    его суммой является матрица
    $\vc{C}=\sum_{s=0}^{\infty}\vc{B}(s)$, представляющая собой
    предел последовательности частичных сумм. Каждый из
    элементов $c_{ij}$ матрицы
    \[\vc{C}=\left(
    \begin{array}{cccc}
   c_{11} & c_{12} & \ldots & c_{1n} \\
   c_{21} & c_{22} & \ldots & c_{2n} \\
   \ldots& \ldots &\ldots &\ldots \\
   c_{n1} & c_{n2} & \ldots & c_{mn}
\end{array}
\right), \]
    представляет собой при каждом $i=1,\ldots,m$ и
    $j=1,\ldots,n$ сумму ряда (\ref{summa-riada}).

    Напомним, что вектор-столбцы и вектор-строки можно
    рассматривать, как матрицы, состоящие из одного столбца и одной
    строки соответственно. Поэтому у читателя не должно возникнуть
    вопроса о том, что понимать под сходимостью векторной последовательности
    или векторного ряда.

\begin{exer}
    Предположим, что нам дана последовательность вектор-столбцов (или
    вектор-строк) $\{\vc{a}(s)\}_{s=0}^{\infty}$, такая что
    \[\vc{a}(0)\leqq\vc{a}(1)\leqq\vc{a}(2)\leqq\ldots\]
    и для некоторого вектора $\vc{\bar{a}}$ при всех $s=0,1,\ldots$
    выполняется неравенство
    \[\vc{a}(s)\leqq\vc{\bar{a}}.\]
    Докажите, что у последовательности
    $\{\vc{a}(s)\}_{s=0}^{\infty}$ имеется предел.
\end{exer}

\begin{exer} \label{riad-shoditsia}
    Предположим, что нам дана последовательность неотрицательных вектор-столбцов (или
    вектор-строк) $\{\vc{a}(r)\}_{r=0}^{\infty}$, такая что
    и для некоторого вектора $\vc{\bar{a}}$ при всех $s=0,1,\ldots$
    выполняется неравенство
    \[\sum_{r=0}^{s}\vc{a}(s)\leqq\vc{\bar{a}}.\]
    Докажите, что ряд
    $\sum_{r=0}^{\infty}\vc{a}(r)$ сходится.
\end{exer}

\begin{exer} \label{predel-proizverd}
    Предположим, что последовательности матриц
    $\{\vc{B}(s)\}_{s=0}^{\infty}$ и $\{\vc{D}(s)\}_{s=0}^{\infty}$
    сходятся к соответственно матрице $\vc{B}$ и матрице $\vc{D}$.
    Докажите, что матричная последовательность
    $\{\vc{B}(s)\vc{D}(s)\}_{s=0}^{\infty}$ сходится к матрице
    $\vc{B}(s)\vc{D}(s)$ (здесь, естественно, предполагается, что
    размерности матриц таковы, что соответствующие произведения имеют
    смысл).
\end{exer}






    Теперь мы готовы сформулировать уже обещанную теорему,
    достаточно полно характеризующую продуктивные матрицы.

\begin{teo}
\label{Teor-Leontief}
    Для леонтьевской матрицы  $\vc{A}$ следующие условия эквивалентны:
\begin{enumerate}[1)]
    \item
    матрица $\vc{A}$ продуктивна;
    \item
    для любого вектора $\vc{y}\geqq0$ найдется вектор
    $\vc{x}\geqq0$, такой что
\begin{equation}
\label{Leontief-uravn}
    \vc{y}=\vc{x}-\vc{A}\vc{x}.
\end{equation}
    \item
    матрица $\vc{A}$ прибыльна;
    \item
    матрица $(\vc{E}-\vc{A})$ обратима, причем матрица
    $(\vc{E}-\vc{A})^{-1}$ неотрицательна (состоит из неотрицательных
    элементов);
    \item
    матричный ряд
\begin{equation} \label{Leontief-riad}
    \vc{E}+\vc{A}+\vc{A^{2}}+\vc{A^{3}}+\ldots.
\end{equation}
    сходится и, более того,
    \begin{equation} \label{Leontief-summa-riada}
    \vc{E}+\vc{A}+\vc{A^{2}}+\vc{A^{3}}+\ldots =(\vc{E}-\vc{A})^{-1}.
\end{equation}
    \end{enumerate}
\end{teo}

    Доказательство этой теоремы мы приведем ниже, а сейчас
    неформально прокомментируем ее в надежде на то, что наши
    комментарии сделают ее для читателя достаточно правдоподобной.

    Рассмотрим упрощенную ситуацию, когда в моделируемой экономике
    производится только один продукт, т.е. когда $n=1$, а
     матрица $\vc{A}$ состоит из одного числа $a$. В этом случае
     продуктивность матрицы $\vc{A}$ просто означает неравенство
     $a<1$ и, следовательно, эквивалентна прибыльности. Поскольку число $a$ является
     неотрицательным по предположению, неравенство $a<1$ выполняется
     тогда, и только тогда, когда
     \[(1-a)^{-1}=\frac{1}{1-a}>0.\]
    А последнее неравенство справедливо в том, и только том случае,
    когда ряд
    \[1+a+a^{2}+\ldots\]
    сходится, а сумма этого ряда совпадает с $\frac{1}{1-a}$.

    Теперь вернемся к общему случаю, когда $n>1$, и постараемся
    понять, в чем состоит содержательный смысл ряда (\ref{Leontief-riad})
    в предположении, что матрица $\vc{A}$ продуктивна и, тем самым
    выполняются все условия, упомянутые в теореме \ref{Teor-Leontief}.
    Перепишем равенство (\ref{Leontief-uravn}) в виде
    \[\vc{y}=(\vc{E}-\vc{A})\vc{x}\]
    предполагая, что нам задан неотрицательный вектор выпуска конечной
    продукции $\vc{y}$, и рассмотрим его как уравнение относительно
    $\vc{x}$. Поскольку матрица $(\vc{E}-\vc{A})$ обратима, это
    уравнение имеет решение, причем единственное. Иными словами,
    вектор $\vc{x}$ однозначным образом определяется вектором
    $\vc{y}$:
    \[\vc{x}=(\vc{E}-\vc{A})^{-1}\vc{y}.\]
    Тот факт, что матрица $(\vc{E}-\vc{A})^{-1}$ неотрицательна,
    гарантирует неотрицательность вектора валового выпуска. Более
    того, в силу (\ref{Leontief-summa-riada}) мы имеем:
    \begin{equation} \label{Leontief-dop}
    \vc{x}=\vc{y}+\vc{A}\vc{y}+\vc{A}^{2}\vc{y}+\ldots.
\end{equation}



    Последнее равенство показывает, какова зависимость вектора валового
    выпуска от вектора конечного выпуска.
    Для того чтобы обеспечить конечный
    выпуск в количествах, определяемых вектором
    \[\vc{y}=\left(
     \begin{array}{c}
        y_{1} \\
        y_{2} \\
        \vdots \\
        y_{n}\\
      \end{array}\right),\]
    совершенно недостаточно произвести $y_{1}$ единиц первого
    продукта, $y_{2}$ единиц второго продукта и т.д., поскольку этот
    выпуск требует затрат. Размеры этих затрат
    задаются вектором $\vc{A}\vc{y}$. Но тратить можно только то,
    что произведено. Значит, в дополнение к вектору $\vc{y}$
    необходимо осуществить выпуск в количествах, определяемых
    вектором  $\vc{A}\vc{y}$. Этот выпуск тоже сопряжен с
    соответствующими затратами. Их задает вектор $\vc{A}^{2}\vc{y}$.
    Но и эти затраты невозможно осуществить без производства того, что
    затрачивается. И так далее. Если мы продолжим это рассуждение и дальше,
    то в конце концов придем к заключению, что для обеспечения конечного выпуска в
    количествах $y_{1},y_{2},\ldots, y_{n}$  необходимо, чтобы вектором валового
    выпуска был именно вектор $\vc{x}$, определяемый равенством (\ref{Leontief-dop}).

    \textbf{Доказательство} теоремы \ref{Teor-Leontief}. Логическая
    цепочка нашего доказательства будет следующей:
    \[2)\Rightarrow 1)\Rightarrow 5)\Rightarrow 4)\Rightarrow 2).\]
    Иными словами, мы докажем, что из свойства 2) вытекает свойство
    1), которое, в свою очередь, влечет свойство 5), и т.д. В приведенной
    цепочке отсутсвует условие 3), которое очень похоже на условие 1). Мы
    оставляем логическую цепочку
    \[3)\Rightarrow 5)\Rightarrow 4)\Rightarrow 3)\]
    читателю в качестве самостоятельного \emph{упражнения}.

    $2)\Rightarrow 1)$. Тот факт, что из
    свойства 2) вытекает свойство 1) не требует доказательства, он
    очевиден.

    $1)\Rightarrow 5)$ (это основной и самый утомительный этап доказательства всей теоремы).
    Докажем, что из свойства 1) вытекает свойство 5).
        Пусть \[\vc{\tilde{x}}=\left(
     \begin{array}{c}
        \tilde{x}_{1} \\
        \tilde{x}_{2} \\
        \vdots \\
        \tilde{x}_{n}\\
      \end{array}\right)\geqq0\] --- это вектор обладающий тем свойством, что
        \[\vc{\tilde{x}}-\vc{A}\vc{\tilde{x}}\gg0.\]



        Перепишем   последнее неравенство в следующем виде:
\begin{equation} \label{strogoe-nerav}
        \vc{A}\vc{\tilde{x}}\ll\vc{\tilde{x}}
\end{equation}
        Это неравенство является строгим. Значит
        можно найти такое положительное число $\tilde{\lambda}<1$,
        что $\vc{A}\vc{\tilde{x}}\ll\tilde{\lambda}\vc{\tilde{x}}$.

\begin{exer}
    Чему равно наименьшее из тех $\lambda$, которые удовлетворяют неравенству
    $\vc{A}\vc{\tilde{x}}\leqq\lambda\vc{\tilde{x}}$?
\end{exer}

    Применив одно из утверждений упражнения \ref{monot-neotr-matr} к
    неравенству (\ref{strogoe-nerav})
    мы получаем неравенство
    \[\vc{A}^{2}\vc{\tilde{x}}\leqq \vc{A}(\tilde{\lambda}\vc{\tilde{x}}).\]
    Используя очевидное равенство
    $\vc{A}(\tilde{\lambda}\vc{\tilde{x}})=\tilde{\lambda}\vc{A}\vc{\tilde{x}}$
    и еще раз вспоминая (\ref{strogoe-nerav}), получаем следующую
    цепочку неравенств:
    \[\vc{A}^{2}\vc{\tilde{x}}\leqq\tilde{\lambda}\vc{A}\vc{\tilde{x}}
    \ll\tilde{\lambda}^{2}\vc{\tilde{x}}.\]

    Повторив проведенное рассуждение нужное число раз приходим у выводу,
    что для всех $k=0,1,2,\ldots$ справедливо неравенство
\begin{equation} \label{Leont-dop1}
    \vc{A}^{k}\,\vc{\tilde{x}}\leqq\tilde{\lambda}^{k}\vc{\tilde{x}}.
\end{equation}
    Поскольку для любого $s=1,2,\ldots$ мы имеем
    \[\sum_{k=0}^{s}\tilde{\lambda}^{k}<
    \sum_{k=0}^{\infty}\tilde{\lambda}^{k}=\frac{1}{1-\tilde{\lambda}},\]
    сложив неравенство (\ref{Leont-dop1}) по всем $k=0,1,\ldots$, получаем следующее
    неравенство:
\begin{equation} \label{Leont-dop2}
    \sum_{k=0}^{s}\vc{A}^{k}\,\vc{\tilde{x}}
    \leqq\frac{1}{1-\tilde{\lambda}}\vc{\tilde{x}}, \ s=0,1,\ldots\,.
\end{equation}
    Вспомнив упражнение \ref{riad-shoditsia}, приходим к выводу, что
    векторный ряд $\sum_{k=0}^{\infty}\vc{A}^{k}\,\vc{\tilde{x}}$
    сходится.

    Покажем, что отсюда вытекает сходимость матричного ряда
    $\sum_{k=0}^{\infty}\vc{A}^{k}$. Для этого рассмотрим
    последовательность $\{\vc{C}(s)\}_{s=0}^{\infty}$ частичных сумм этого
    ряда, состоящей из матриц
    \[\vc{C}(s)=
\left(
\begin{array}{cccc}
   c_{11}(s) & c_{12}(s) & \ldots & c_{1n}(s) \\
   c_{21}(s) & c_{22}(s) & \ldots & c_{2n}(s) \\
   \ldots& \ldots &\ldots &\ldots \\
   c_{n1}(s) & c_{n2}(s) & \ldots & c_{nn}(s)
\end{array}
\right), \ s=0,1,\ldots,\]
    задаваемых равенством
    \[\vc{C}(s)=\sum_{k=0}^{s}\vc{A}^{k},  \ s=0,1,\ldots,\]
    и покажем, что для всех $i,j=1,\ldots,n$ числовая
    последовательность $\{c_{ij}(s)\}_{s=0}^{\infty}$ сходится.
    Это и будет означать сходимость
    матричной последовательности $\{\vc{C}(s)\}_{s=0}^{\infty}$, т.е.
    сходимость ряда $\sum_{k=0}^{\infty}\vc{A}^{k}$.

    Перепишем (\ref{Leont-dop2}) в следующем виде:
    \[\vc{C}(s)\vc{\tilde{x}}
    \leqq\frac{1}{1-\tilde{\lambda}}\vc{\tilde{x}}, \ s=0,1,\ldots\,.\]
    Это неравенство означает, что для всех $s=0,1,\ldots$ выполняются следующие
    неравенства:
\begin{equation} \label{Leont-dop3}
    \left\{
\begin{array}{ccccc}
   c_{11}(s)\tilde{x}_{1} +& c_{12}(s)\tilde{x}_{2} +& \ldots & +c_{1n}(s)\tilde{x}_{n}
   & \leq \frac{1}{1-\tilde{\lambda}}\tilde{x}_{1}, \\
   c_{21}(s)\tilde{x}_{1} +& c_{22}(s)\tilde{x}_{2} +& \ldots & +c_{2n}(s)\tilde{x}_{n}
   & \leq \frac{1}{1-\tilde{\lambda}}\tilde{x}_{2},\\
   \ldots& \ldots &\ldots &\ldots &\ldots \\
   c_{n1}(s)\tilde{x}_{1}  +& c_{n2}(s)\tilde{x}_{1}  +& \ldots & +c_{nn}(s)\tilde{x}_{1}
   & \leq \frac{1}{1-\tilde{\lambda}}\tilde{x}_{n}.
\end{array}
\right.
\end{equation}
    Положим
    \[\underline{x}=\min\{\tilde{x}_{1}, \tilde{x}_{2},\ldots,
    \tilde{x}_{n}\}, \
    \bar{x}=\max\{\tilde{x}_{1}, \tilde{x}_{2},\ldots, \tilde{x}_{n}\}.\]
    Из (\ref{Leont-dop3}) вытекают следующие неравенства:
    \begin{equation} \label{Leont-dop4}
    \left\{
\begin{array}{ccccc}
   c_{11}(s)\underline{x} +& c_{12}(s)\underline{x} +& \ldots & +c_{1n}(s)\underline{x}
   & \leq \frac{1}{1-\tilde{\lambda}}\bar{x}, \\
   c_{21}(s)\underline{x} +& c_{22}(s)\underline{x} +& \ldots & +c_{2n}(s)\underline{x}
   & \leq \frac{1}{1-\tilde{\lambda}}\bar{x},\\
   \ldots& \ldots &\ldots &\ldots &\ldots \\
   c_{n1}(s)\underline{x}  +& c_{n2}(s)\underline{x} +& \ldots & +c_{nn}(s)\underline{x}
   & \leq \frac{1}{1-\tilde{\lambda}}\bar{x}.
\end{array}
\right.
\end{equation}

    Здесь надо вспомнить, что вектор
    $\vc{\tilde{x}}$ строго положителен. Следовательно
    $\underline{x}>0$ и поэтому мы можем переписать
    (\ref{Leont-dop4}) в следующем виде:
    \[\left\{
\begin{array}{ccccc}
   c_{11}(s) +& c_{12}(s) +& \ldots & +c_{1n}(s)
   & \leq \frac{\bar{x}}{(1-\tilde{\lambda})\underline{x}}, \\
   c_{21}(s) +& c_{22}(s) +& \ldots & +c_{2n}(s)
   & \leq \frac{\bar{x}}{(1-\tilde{\lambda})\underline{x}},\\
   \ldots& \ldots &\ldots &\ldots &\ldots \\
   c_{n1}(s) +& c_{n2}(s) +& \ldots & +c_{nn}(s)
   & \leq \frac{\bar{x}}{(1-\tilde{\lambda})\underline{x}}.
\end{array}
\right.\]

    Мы видим, что сумма элементов каждой строки матрицы $\vc{C}(s)$ не
    превосходит числа
    $\frac{\bar{x}}{(1-\tilde{\lambda})\underline{x}}$. Но поскольку
    все элементы этой матрицы неотрицательны, отсюда следует, что
    каждый отдельный ее элемент $c_{ij}(s)$
    тоже не превосходит $\frac{\bar{x}}{(1-\tilde{\lambda})\underline{x}}$.

    Для доказательства сходимости последовательностей
    $\{c_{ij}(s)\}_{s=0}^{\infty}$ при всех $i,j=1,\ldots,n$ нам
    осталось только заметить, что они являются монотонно
    неубывающими, ибо для всех $k=1,2,\ldots$ матрица $\vc{A}^{k}$
    состоит из неотрицательных элементов.

    Итак, мы доказали, что ряд $\sum_{k=0}^{\infty}\vc{A}^{k}$
    сходится. Покажем, что
    \[\sum_{k=0}^{\infty}\vc{A}^{k}=(\vc{E}-\vc{A})^{-1},\]
    т.е. что последовательность $\{\vc{C}(s)\}_{s=0}^{\infty}$ частичных
    сумм рассматриваемого ряда сходится именно к матрице $(\vc{E}-\vc{A})^{-1}$.
    Пусть матрица $\vc{C}$ является пределом последовательности
    $\{\vc{C}(s)\}_{s=0}^{\infty}$, т.е. $\vc{C}=\sum_{k=0}^{\infty}\vc{A}^{k}$.
    Мы имеем:
    \[\vc{C}(s)(\vc{E}-\vc{A})=(\vc{E}+\vc{A}+\ldots+\vc{A}^{s})(\vc{E}-\vc{A})
    =\vc{E}-\vc{A}^{s+1}.\]
    Поскольку, очевидно, сходимость ряда
    $\sum_{k=0}^{\infty}\vc{A}^{k}$ означает, в частности, что
    матричная последовательность $\{\vc{A}^{k}\}_{k=0}^{\infty}$
    сходится к нулевой матрице, последовательность
    $\{(\vc{E}-\vc{A}^{s+1})\}_{s=0}^{\infty}$ сходится к единичной
    матрице $\vc{E}$. Вспомнив упражнение \ref{predel-proizverd}, делаем заключение,
    что $\vc{C}(\vc{E}-\vc{A})=\vc{E}$, т.е. что
    $\vc{C}=(\vc{E}-\vc{A})^{-1}$, что и требуется.

    $5)\Rightarrow 4)$. Здесь и доказывать нечего.

    $4)\Rightarrow 2)$. Если матрица $(\vc{E}-\vc{A})$ обратима,
    причем матрица $(\vc{E}-\vc{A})^{-1}$ неотрицательна, то при
    любом неотрицательном векторе конечного выпуска $\vc{y}$
    уравнение (\ref{Leontief-uravn}) относительно $\vc{x}$ имеет
    единственное решение, причем неотрицательное. Этим решением
    является вектор $(\vc{E}-\vc{A})^{-1}\vc{y}$. $\Box$


\subsection{Полные затраты труда}

    До сих пор в своем анализе мы ограничивались анализом
    потенциальных технологических возможностей моделируемой экономики, обсуждая
    понятие продуктивности леонтьевской матрицы. Этот анализ
    совершенного не учитывал того факта, что для производства нужно
    затрачивать не только те продукты, которые производятся одной из
    $n$ отраслей, составляющих экономику, но и некоторые невоспроизводимые ресурсы.
    В этом параграфе мы введем в рассмотрение один из такого типа
    ресурсов --- труд, или рабочую силу.


    Итак, предположим, что наряду с продуктивной леонтьевской  матрицей $\vc{A}$
    нам задана вектор-строка $\vc{l}=(l_{1},\ldots,l_{n})$
    коэффициентов \emph{прямых затрат труда}, компоненты которой
    показывают, что на производство одной единицы первого продукта в
    валовом исчислении
    необходимо затратить $l_{1}$ единиц рабочей силы, одной
    единицы второго продукта --- $l_{2}$ единиц рабочей силы и т.д.
    Тем самым, если в экономике осуществляется выпуск валовой
    продукции в количествах, задаваемых вектором
    \[\vc{x}=\left(
     \begin{array}{c}
        x_{1} \\
        x_{2} \\
        \vdots \\
        x_{n}  \\
      \end{array}
    \right),\]
    то соответствующие затраты труда составляют
    \[\vc{l}\,\vc{x}=l_{1}x_{1}+\ldots+l_{n}x_{n}\]
    единиц рабочей силы. Мы предполагаем, что без затрат труда
    никакой продукт произведен быть не может, т.е. что $\vc{l}\gg0$.


    Для того, чтобы произвести одну единицу продукта $i$, необходимо
    затратить $l_{i}$ единиц рабочей силы. Но кроме труда необходимо
    также осуществить затраты продуктов в количествах, задаваемых
    $i$-м столбцом матрицы $\vc{A}$, вектор-столбцом $\vc{a_{i}}$.
    До того, как эти продукты затратить, их надо произвести, а это
    производство тоже требует затрат труда. Эти затраты труда
    составляют $l_{i}(1)=\vc{l}\vc{a_{i}}$ единиц рабочей силы.
    Назовем эту величину косвенными затратами труда первого порядка
    на производство (одной единицы) продукта $i$.

    Продолжим наше рассуждение. Для производства набора
    продуктов $\vc{a_{i}}$ тоже необходимо затратить не только труд,
    но и продукты, которые опять-таки прежде надо произвести. В
    данном случае речь идет о количествах, которые задаются вектором
    $\vc{A}\vc{a_{i}}$ и требуют затрат труда в размере
    $l_{i}(2)=\vc{l}\vc{A}\vc{a_{i}}$. Как уже догадался читатель,
    эту величину мы назовем косвенными затратами труда второго
    порядка на производство (одной единицы) продукта $i$.

    Таким образом мы можем рассуждать до бесконечности и для всех
    $s=1,2,\ldots$ определить величину \emph{косвенных затрат труда}
    $s$-го порядка на производство продукта $i$ как
    \[l_{i}(s)=\vc{l}\vc{A}^{s-1}\vc{a_{i}}.\]
    Сумму
    \[t_{i}=l_{i}+l_{i}(1)+l_{i}(2)+\ldots\]
    прямых  затрат труда и косвенных  затрат труда всех
    порядков на производство одной единицы продукта $i$ мы назовем \emph{полными
     затратами труда} на его производство, а вектор
    \[\vc{t}=(t_{1},\ldots,t_{n})\]
    --- \emph{вектором полных затрат труда}. (\remrk{удельных????})
    Очевидно, что
\begin{equation}
 \label{polnie zatrati truda}
    \vc{t}=\vc{l}(\vc{E}+\vc{A}+\vc{A}^{2}+\ldots)
    =\vc{l}(\vc{E}-\vc{A})^{-1}.
\end{equation}

    На первый взгляд не совсем понятно, зачем нам может понадобиться
    понятие полных (\remrk{удельных????}) затрат труда. Хотя, быть может, читатель
    слышал о трудовой теории стоимости, согласно которой в основе
    цены товара лежат так называемые общественно
    необходимые затраты труда, требующиеся для производства этого товара.
    В рамках нашей модели под этими затратами можно понимать именно
    полные затраты труда.

    Чтобы объяснить, в чем тут дело, предположим,
    что цена каждого продукта в моделируемой экономике совпадает с
    его себестоимостью, т.е. затратами  на производство одной единицы
    этого продукта в денежном выражении. Эти последние равны
    материальным затратам в денежном выражении (по тем ценам, о которых идет
    речь) плюс затраты на заработную плату. Вектор цен
    $\vc{p}=(p_{1},\ldots,p_{n})$, формирующихся по этому правилу,
     задается задается следующими соотношениями:
\begin{equation} \label{ceni-po-sebestoimosti}
 \left\{
\begin{array}{ccccccccccc}
  p_{1} & = & p_{1}a_{11} & + & p_{2}a_{21} & + & \ldots & + & p_{n}a_{n1} & + & wl_{1} \\
  p_{2} & = & p_{1}a_{12} & + & p_{2}a_{22} & + & \ldots & + & p_{n}a_{n2} & + & wl_{2}\\
  \ldots &  & \ldots &  & \ldots &  & \ldots &  & \ldots  &  & \ldots\\
  p_{n} & = & p_{1}a_{1n} & + & p_{2}a_{2n} & + & \ldots & + & p_{n}a_{nn} & + &  wl_{n},
\end{array}
\right.
\end{equation}
    где $w$ --- ставка номинальной заработной платы. В
    векторно-матричной форме эти соотношения записываются совсем
    просто:
    \[\vc{p}=\vc{p}\vc{A}+w\vc{l}.\]
    Перепишем последнее равенство в следующем виде:
    \[\vc{p}(\vc{E}-\vc{A})=w\vc{l}.\]
    Следовательно
    \[\vc{p}=w\vc{l}(\vc{E}-\vc{A})^{-1}.\]
    Вспомнив (\ref{polnie zatrati truda}), получаем:
    \[\vc{p}=w\vc{t}.\]

    Итак, если  вектор цен формируется по правилу, согласно которому
    цена каждого отдельного продукта совпадает с его себестоимостью,
    то этот вектор цен с точностью до множителя $w$
    (который может быть выбран самым произвольным образом)
    совпадает с вектором полных затрат труда.

    Вектор полных затрат труда позволяет в рамках нашей модели
    достаточно полно описывать производственные возможности
    экономики с учетом того факта, что рабочая сила представляет
    собой ограниченный невоспроизводимый ресурс в предположении, что
    рабочая сила является единственным   ресурсом такого типа.
    Итак, пусть количество  рабочей силы $\bar{l}>0$, которым располагает наша
    экономика, является  экзогенно заданной величиной. Это значит,
    что экономика может произвести только тот набор продуктов,
    который потребует затрат труда, не превосходящих этого значения.
    Если речь идет о векторе валового выпуска
    \[\vc{x}=\left(
     \begin{array}{c}
        x_{1} \\
        x_{2} \\
        \vdots \\
        x_{n}  \\
      \end{array}
    \right),\]
    то он должен удовлетворять ограничению
\begin{equation}
 \label{ogranich-na-trud}
    \vc{l}\,\vc{x}=l_{1}x_{1}+\ldots+l_{n}x_{n}\leqslant\bar{l}.
    \end{equation}

    Однако, когда идет речь о производственных возможностях
    экономики, то в первую очередь интерес представляет не вектор
    валового выпуска, а вектор конечного выпуска. Однако в рамках
    нашей модели соотношение между вектором валового выпуска
    $\vc{x}$ и соответствующем ему вектором конечного выпуска
    \[\vc{y}=\left(
     \begin{array}{c}
        y_{1} \\
        y_{2} \\
        \vdots \\
        y_{n}  \\
      \end{array}
    \right),\]
    устроено очень просто:
    \[\vc{y}=(\vc{E}-\vc{A})\vc{x} \Leftrightarrow
    \vc{x}=(\vc{E}-\vc{A})^{-1}\vc{y}.\]
    Это значит, что текущие затраты руда на производство вектора валового
    выпуска $\vc{x}$ совпадают с полными затратами труда на
    производство соответствующего ему вектора конечного выпуска
    $\vc{y}$:
    \[\vc{l}\vc{x}=\vc{t}\vc{y}.\]
    Следовательно, ограничение (\ref{ogranich-na-trud}) можно записать в
    следующем виде:
    \[\vc{t}\vc{y}=t_{1}y_{1}+\ldots+t_{n}y_{n}\leqslant\bar{l}.\]

    Множество $\mathbb{Y}$ тех векторов конечного выпуска, которые являются
    технологически возможными и обеспечены ресурсами, можно назвать
    множеством \emph{достижимых} (??????) выпусков
    (или множеством производственных возможностей?????). Проведенное
    рассуждение показывает, что в рамках нашей модели это
    множество задается самым простым образом:
\[
    \mathbb{Y}=\{\vc{y}\in\R_{+}^{n} \,|\, \vc{t}\vc{y}\leqslant\bar{l}\}.
\]
    \remrk{РИС}


\subsection{Цены производства}
    Предположение о том, что цена каждого продукта в моделируемой экономике совпадает с
    его себестоимостью, означает, что производство этого продукта
    хотя и не является убыточным, но и не приносит прибыли. Поэтому
    оно не представляется вполне реалистичным. Однако, его несложно
    несколько видоизменить, в явном виде предположив, что
    производство каждого из продуктов обеспечивает некоторую единую
    норму прибыли $r\geq0$. В этом случае соотношения (\ref{ceni-po-sebestoimosti}),
    преобразуются к следующему виду:
    \[\left\{
\begin{array}{ccccccccccc}
  p_{1} & = & (1+r)(p_{1}a_{11} & + & p_{2}a_{21} & + & \ldots & + & p_{n}a_{n1}) & + & wl_{1} \\
  p_{2} & = & (1+r)(p_{1}a_{12} & + & p_{2}a_{22} & + & \ldots & + & p_{n}a_{n2}) & + & wl_{2}\\
  \ldots &  & \ldots &  & \ldots &  & \ldots &  & \ldots  &  & \ldots\\
  p_{n} & = & (1+r)(p_{1}a_{1n} & + & p_{2}a_{2n} & + & \ldots & + & p_{n}a_{nn}) & + &  wl_{n},
\end{array}
\right.\]
    или, иначе,
    \[\vc{p}=(1+r)\vc{p}\vc{A}+w\vc{l}.\]
    Цены $p_{1},\ldots,p_{n}$, которые задаются этими соотношениями,
    называются \emph{ценами производства}, а вектор
    $\vc{p}=(p_{1},\ldots,p_{n})$ --- вектором цен производства.

    Здесь следует пояснить, в каком смысле цены производства, которые задаются этими
    соотношениями, обеспечивают при производстве каждого продукта
    $i=1,\ldots,n$ норму прибыли $r$. Мы предполагаем, что в рамках нашей модели
    длина условного производственного цикла равна некоторому единичному промежутку
    времени, а сам процесс
    производства  развивается во времени следующим
    образом. Затраты на производство осуществляются в начале
    производственного цикла, а выпуск получается в его
    конце. <<Внутри>> производственного цикла происходит только сам процесс
    производства и больше ничего не происходит.

    При этом, что очень важно, оплата
    расходуемых продуктов осуществляет в начале производственного
    цикла, а вот оплата труда --- в его конце (в этом случае говорят,
    что заработная палата выплачивается \emph{ex post}). Поэтому для того,
    чтобы запустить процесс производства одной единицы $i$-го продукта,
    в начале производственного цикла
    необходимо  осуществить финансовые  вложения в размере
    \[p_{1}a_{1i}  +  p_{2}a_{2i}  +  \ldots  +  p_{n}a_{ni}\]
    денежных единиц. После того, как процесс производства
    заканчивается, производители этого продукта получает $p_{i}$ денежных
    единиц (мы предполагаем, что нет
    проблем со сбытом продукции),  из которых сразу же нужно выплатить заработную плату в
    размере $wl_{i}$. Это значит, что в конце
    производственного цикла он получает доход в размере
    $p_{1}-wl_{i}$ денежных единиц.
     Следовательно, производство одной единицы
    продукта $i$ приносит в течение единичного промежутка времени
    прибыль в размере
    \[(p_{1}-wl_{i})-(p_{1}a_{1i}  +  p_{2}a_{2i}  +  \ldots  +  p_{n}a_{ni})\]
    денежных единиц, а норма прибыли при производстве этого продукта
    равна
    \[\frac{p_{1}-wl_{i}}{p_{1}a_{1i}  +  p_{2}a_{2i}  +  \ldots  +  p_{n}a_{ni}}-1.\]
    Если мы самого начала знаем, что эта норма прибыли равна $r$, то
    и получаем то соотношение
    \[p_{i}=(1+r)(p_{1}a_{1i}+p_{2}a_{2i}+\ldots+p_{n}a_{ni})+wl_{i},\]
    выписав которое для всех $i=1,\ldots,n$, мы и определили вектор
    цен производства.

    Безусловно, вектор цен производства зависит от нормы прибыли. Для того,
    чтобы подчеркнуть эту зависимость, предположим, что номинальная заработная
    плата $w>0$ зафиксирована, обозначим вектор цен
    производства при при норме прибыли  $r$ через
    \[\vc{p}(r)=(p_{1}(r),\ldots,p_{n}(r))\]
    и сразу же заметим, что он корректно определен только при тех
    значениях нормы прибыли $r$, для которых является продуктивной
    матрица $(1+r)\vc{A}$. В этом случае
\begin{equation}
\label{tseni-proizvodstva}
    \vc{p}(r)=w\vc{l}(\vc{E}+(1+r)\vc{A}+(1+r)^{2}\vc{A}^{2}+\ldots) \
    =  w\vc{l}(\vc{E}-(1+r)\vc{A})^{-1}.
\end{equation}
    Очевидно, что в случае, когда $r=0$, вектор цен производства
    совпадает с вектором полных затрат труда:
    \[\vc{p}(0)=\vc{t}.\]
\begin{exer}
    Дайте экономическую интерпретацию соотношениям
    (\ref{tseni-proizvodstva}).
\end{exer}
    При определении цен производства мы исходили из предположения о
    том, что норма прибыли является заданной величиной. Безусловно,
    это предположение не является очень правдоподобным, но
    вопрос о том, как формируется норма прибыли, выходит за
    рамки нашего неглубокого анализа модели Леонтьева. Однако мы можем
    рассмотреть норму прибыли как параметр и обратить внимание на
    одно важное свойство вектора цен производства, а именно, на то, что
    рост нормы прибыли приведет к падению (или, по крайней мере, к невозрастанию)
     реальной заработной платы.



    Подчеркнем, что при определении вектора цен производства $\vc{p}(r)$, мы
    считали, что ставка номинальной заработной платы является заданной
    величиной. Но это не означает, что заданной величиной является
    реальная заработная плата. Дело в том, что если при неизменной
    номинальной заработной плате меняются цены, реальная заработная
    плата тоже, скорее всего, будет меняться.

     Правда, мы еще не уточнили, как
    измерять реальную заработную плату. Пора это сделать. Предположим, что реальная
    заработная плата в зависимости от нормы прибыли в экономике,
    которую мы обозначим через $\omega(r)$,
    измеряется с помощью некоторого фиксированного
    набора продуктов, задаваемого некоторым ненулевым неотрицательным вектором
    \[\vc{\bar{c}}=\left(
     \begin{array}{c}
        \bar{c}_{1} \\
        \bar{c}_{2} \\
        \vdots \\
        \bar{c}_{n}  \\
      \end{array}
    \right),\]
    по следующей формуле:
    \[\omega(r)=\frac{w}{\vc{p}(r)\vc{\bar{c}}}
    =\frac{w}{p_{1}(r)\bar{c}_{1}+\ldots+p_{n}(r)\bar{c}_{n}}.\]
    Эта формула означает, что реальная заработная плата показывает, сколько
    наборов продукции, задаваемых вектором $\vc{\bar{c}}$, можно
    приобрести за номинальную заработную плату $w$ при ценах
    $p_{1}(r),\ldots,p_{n}(r)$.


    Поскольку для всех $k=1,2,\ldots$ матрица $\vc{A}^{k}$
    неотрицательна, для всех $i=1,\ldots,n$ цена $i$-го продукта
     $p_{i}(r)$ не убывает с ростом $r$, а для тех продуктов $i$,
     косвенные затраты труда хоть какого-нибудь порядка
     положительны, цена $p_{i}(r)$ увеличивается с ростом $r$.

     Из этого мы заключаем, что справедливо следующее предложение.
     \begin{prop}
     Вне зависимости от того, с
     помощью какого набора продуктов $\vc{\bar{c}}$ мы измеряем
     реальную заработную плату $\omega(r)$, с увеличением нормы прибыли она не
     будет увеличиваться. При этом, если для производства набора
     $\vc{\bar{c}}$ необходимы косвенные затраты труда хоть
     какого-нибудь порядка, то с ростом нормы прибыли реальная
     заработная плата будет убывать.
     \end{prop}


\begin{exer}
    Следует ли из равенства нулю $i$-го столбца матрицы $\vc{A}$ (вектор-столбца
    $\vc{a_{i}}$), что косвенные косвенные затраты труда первого и
    всех остальных порядков на производство продукта $i$ являются
    нулевыми?
\end{exer}


\begin{exer}
    Какими соотношениями задаются цены,
    обеспечивающие одинаковую норму прибыли $r$ при производстве
    каждого из продуктов в случае, когда заработная плата
    выплачивается \emph{ex ante}, т.е. не в конце производственного
    цикла, как мы предположили выше, а в начале. Какова зависимость
    реальной заработной платы от нормы прибыли в этом случае?
\end{exer}























































\











\


\


\section{Неформальное введение в теорию линейного программирования}

    Безусловно, среди линейных моделей экономики самыми популярными
    являются модели линейного программирования, точнее, модели,
    которые сводятся к задачам линейного программирования. Численные
    методы линейного программирования представляют собой полезный
    прикладной инструмент экономического анализа. Однако нас
    будут интересовать не прикладные аспекты теории линейного
    программирования, а только те ее элементы, которые
    позволяют лучше понимать содержательный экономический смысл
    оптимальных решений и их основные свойства.


\subsection{Простейшая линейная задача распределения ресурса}

    Свое изложение теории линейного программирования мы начнем с
    рассмотрения совсем простой стилизованной ситуации, для изучение
    которой и теории никакой не требуется. Однако анализ этой
    ситуации позволит читателю понять в дальнейшем содержательный
    смысл многих теорем линейного программирования.


Пусть на некотором предприятии производятся продукты $n$ различных
видов, для производства которых необходим только один вид ресурса,
который имеется в распоряжении предприятия в количестве $b$ единиц.
Предположим, что цена  $i$-го продукта  равна $q_{i}>0$ руб./ед., и
что
     для производства 1 единицы этого продукта
требуется $a_{i}>0$ единиц ресурса. Это значит, что если объем его
производства равен $x_{i}\geqslant0$
    единиц (этот объем, безусловно, должен быть неотрицательным),
    то на это производство потребуется $a_{i}x_{i}$ единиц ресурса,
    а выручка составит $q_{i}x_{i}$ руб. И если план
    выпуска различных продуктов на предприятии задается неотрицательным
    вектор-столбцом
\[\vc{x}=\left(
     \begin{array}{c}
        x_{1} \\
        x_{2} \\
        \vdots \\
        x_{n}  \\
      \end{array}
\right),\]
    то суммарная выручка составит
    \[\vc{q}\vc{x}=q_{1}x_{1}+\ldots+q_{n}x_{n}\]
    рублей, а затраты ресурса ---
    \[\vc{a}\vc{x}=a_{1}x_{1}+\ldots+a_{n}x_{n}\]
    единиц, где, как уже догадался читатель,
\[\vc{a}=(a_{1},\ldots,a_{n}), \, \ \vc{q}=(q_{1},\ldots,q_{n}).\]

    Задача, которую необходимо решить руководству предприятия, состоит в том,
чтобы  имеющийся в количестве $b$ единиц ресурс распределить на
производство различных продуктов таким образом, чтобы
максимизировать суммарную выручку от их продажи по заданным ценам.
Иными словами, среди всех \emph{допустимых планов}, т.е.
неотрицательных векторов $\vc{x}=(x_{1},\ldots,x_{n})^{T}$, для
которых $\vc{a}\vc{x}=a_{1}x_{1}+\ldots+a_{n}x_{n}\leqslant b,$
 нужно отыскать \emph{оптимальный план}, т.е. такой, который доставляет максимальное
 значение величине
 $\vc{q}\vc{x}=q_{1}x_{1}+\ldots+q_{n}x_{n}.$


     Эту задачу можно записывать следующим образом:
\begin{equation}
    \label{LP0-1}
    q_{1}x_{1}+\ldots+q_{n}x_{n}\rightarrow\max,
\end{equation}
\begin{equation}
    \label{LP0-2}
    a_{1}x_{1}+\ldots+a_{n}x_{n}\leqslant b,
\end{equation}
\begin{equation}
    \label{LP0-3}
    x_{1}\geqslant0, \ldots, x_{n}\geqslant0,
\end{equation}
    или совсем коротко:
    \[\vc{q}\vc{x}\rightarrow\max,\] \[ \vc{a}\vc{x}\leqslant b, \] \[ \vc{x}\geqq0.\]


    Для того, чтобы решить эту задачу, не надо применять
    никакой сложный численный метод, а достаточно найти максимальное среди чисел
    \[q_{i}/a_{i}, \ i=1,\ldots,n,\]
    взять $i_{0}$, для которого
    \[q_{i_{0}}/a_{i_{0}}=\max_{i=1,\ldots,n}\{q_{i}/a_{i}\}\]
     и задать вектор
    \[\vc{\hat{x}}=\left(
     \begin{array}{c}
        \hat{x}_{1} \\
        \hat{x}_{2} \\
        \vdots \\
        \hat{x}_{n}  \\
      \end{array}
\right)\]
    посредством  следующих соотношений:
\begin{equation}
    \label{LP0-edinstv-resh}
    \hat{x}_{i_{0}}=b/a_{i_{0}},\ \hat{x}_{i}=0,\ i\neq i_{0}.
\end{equation}
        Очевидно, что этот вектор является решением рассматриваемой
        задачи, т.е. оптимальным планом. \remrk{РИС для n=2}

        Иными словами, для решения задачи
    нужно найти самый доходный продукт (тот продукт, который
    обеспечивает максимальный доход на единицу используемого ресурса)
    и весь ресурс потратить на производство именно этого продукта.
    Если для всех $i\neq i_{0}$ выполняется неравенство
    $q_{i}/a_{i}<\max_{i}\{q_{i}/a_{i}\}$, то указанный вектор $\vc{\hat{x}}$ является
    единственным решением задачи(\ref{LP0-1})--(\ref{LP0-3}).
    В противном случае, очевидно, существуют и другие решения (какие?).

    В любом случае можно утверждать, что вектор
    \[\vc{x^{*}}=\left(
     \begin{array}{c}
        x_{1}^{*} \\
        x_{2}^{*} \\
        \vdots \\
        x_{n}^{*}  \\
      \end{array}
\right)\geqq\vc{0}\]
    является решением рассматриваемой задачи тогда и только тогда,
    когда выполняются следующие условия:
\begin{equation}
    \label{LP0-ravenstvo}
    a_{1}x_{1}^{*}+\ldots+a_{n}x_{n}^{*}=b.
\end{equation}
    \[q_{i}/a_{i}<\max_{i}\{q_{i}/a_{i}\}\Rightarrow x_{i}^{*}=0,
     \ i=1,\ldots n.\]
    Эти условия говорят о том, что ресурс должен быть распределен
    полностью, при этом выделяться он будет только на производство тех продуктов,
    которые обеспечивают максимальную доходность. В следующем
    предложении эти условия переформулированы в удобной для
    интерпретирования виде.

    \begin{prop}
    \label{LP0-usloviya-optimalnosti}
    Вектор  $\vc{x^{*}}\geqq\vc{0}$
    является решением задачи (\ref{LP0-1})--(\ref{LP0-3})
    тогда и только тогда, когда выполняется равенство (\ref{LP0-ravenstvo})
    и найдется число $p^{*}>0$, такое что
    \[p^{*}a_{i}\leqslant q_{i},\ i=1,\ldots, n,\]
    \[p^{*}a_{i}< q_{i} \Rightarrow x_{i}^{*}=0, \ i=1,\ldots, n.\]
    \end{prop}

\begin{exer}
    Покажите, что это утверждение можно сформулировать следующим образом:
    вектор  $\vc{x^{*}}\geqq0$
    является решением задачи (\ref{LP0-1})--(\ref{LP0-3})
    тогда и только тогда, когда выполняется равенство
    $\vc{a}\vc{x^{*}}=b$ и найдется число $p^{*}>0$, такое что
    выполняются следующие соотношения:
    \[p^{*}\vc{a}\leqq\vc{q},\]
    \[p^{*}\vc{a}\vc{x^{*}}=\vc{q}\vc{x^{*}}.\]
\end{exer}

    Число $p^{*}$, о котором идет речь в этом предложении,
    определяется очень просто:
    \[p^{*}=\max_{i}\{q_{i}/a_{i}\}.\]
    Читатель, который знаком с теорией линейного программирования,
    видимо узнал в этом числе двойственную оценку оценку
    единственного содержательного ограничения в
    нашей задаче, ограничения (\ref{LP0-2}). По-видимому, он также
    слышал, что в некоторых случаях двойственные
    оценки называют теневыми ценами.

    Мы не будем называть их <<теневыми>>. Нам кажется, правильнее
    интерпретировать величину $p^{*}$ как равновесную цену ресурса, т.е.
    такую цену, которая делает для руководства предприятия
    реализацию
    оптимального плана самым выгодным действием планом действий c точки
    зрения максимизации прибыли. Поясним эту
    интерпретацию, которая в дальнейшем будет уточнена в более общем
    контексте.

    При решении задачи (\ref{LP0-1})--(\ref{LP0-3}) мы исходили из
    того, что количество $b$ распределяемого ресурса является заданной
    величиной. Именно поэтому естественной является задача о
    максимизации суммарной выручки от продажи выпускаемой продукции.
    Если допустить, что какое-то количество ресурса можно докупить
    или, наоборот, продать, то в своих действиях руководство
    предприятия должно ориентироваться не на доход, а на прибыль,
    т.е. разность между доходами от продажи
    произведенного продукта и затратами на его производство,
    измеренными в денежных единицах (рублях).

    Доход, который будет получен предприятием при выпуске одной
    единицы продукта $i$ составляет $q_{i}$ рублей.
    А для того, чтобы
    рассчитать расходы в денежном исчислении, нужно знать цену
    используемого при производстве ресурса. Пусть
    эта цена равна $p$ руб./ед. В этом случае прибыль, которую
    получит производитель при производстве одной единицы $i$-го
    продукта окажется равной $q_{i}-pa_{i}$ рублей. Если эта
    величина отрицательна, то с экономической точки зрения
    производство данного продукта будет невыгодным. Более того, если
    величина $q_{i}-pa_{i}$ отрицательна для всех $i=1,\ldots,n$,
    можно сказать, что цена на ресурс столь высока, что на
    рассматриваемом предприятии любое производство является
    невыгодным; предприятию выгоднее продать весь запас имеющегося у
    него ресурса, а не терпеть убытки от производства.

    Если же цена ресурса не столь высока, чтобы делать убыточным
    любое производство на предприятии, то имеющийся ресурс можно и
    не продавать, а использовать для производства безубыточных видов
    продукции. При этом, если производство хотя бы одного продукта
    $i$ является по-настоящему прибыльным, т.е. $q_{i}-pa_{i}>0$, то
    предприятию будет выгодно не ограничиваться использованием
    ресурса в имеющемся количестве $b$ единиц, а докупить по возможности
    еще какое-то (точнее, как можно большее) количества ресурса  и направить
    его на производство продуктов, приносящих положительную прибыль.

    И только в том случае, когда цена ресурса равна $p^{*}$ руб./ед.,
    ни приобретение дополнительного количества ресурса, ни
    его продажа не принесут предприятию дополнительной прибыли по
    сравнению с полным использованием имеющегося его количества для
    производства безубыточных продуктов.

    Это позволяет нам  интерпретировать предложение
    \ref{LP0-usloviya-optimalnosti} следующим образом.
    Если $\vc{x^{*}}$ --- это оптимальный план задачи
    (\ref{LP0-1})--(\ref{LP0-3}), то существует такая цена
    ресурса $p^{*}$, что с точки зрения максимизации прибыли
    наилучшим способом действия является реализация именно плана
    $\vc{x^{*}}$, даже с учетом возможности продажи или приобретения
    ресурса по цене $p^{*}$. Наоборот, если для некоторого
    допустимого плана $\vc{x^{*}}$, предполагающего полное использование имеющегося ресурса,
    найдется такая цена $p^{*}$, что
    наилучшим с точки зрения максимизации прибыли способом действия
    является реализация этого плана, то этот план является
    решением задачи (\ref{LP0-1})--(\ref{LP0-3}).
    Именно эта интерпретация и дает нам возможность рассматривать величину $p^{*}$
    как равновесную для рассматриваемого предприятия цену ресурса.



    Теперь посмотрим на задачу определения цены ресурса с точки
    зрения внешнего наблюдателя. Предположим, что для какого-то
    стороннего предпринимателя тот ресурс, который используется на
    рассматриваемом нами предприятии оказался нужен для своих целей и он
    вознамерился приобрести его в полном объеме $b$ единиц.

    На какую цену может рассчитывать потенциальный покупатель?
    Естественно, хочется приобрести ресурс как можно дешевле. Однако,
    если предложить слишком маленькую цену, его владелец может
    отказаться от продажи. Чтобы был шанс договориться о покупке,
    покупатель должен предложить такую цену, которая делает продажу
    ресурса не менее выгодным делом, чем использование этого ресурса
    для производства какого-либо из $n$ продуктов. Это значит, что цена ресурса
    $p\geqslant0$, предлагаемая его владельцу,  должна удовлетворять
    следующим неравенствам:
    \[pa_{i}\leqslant q_{i}, \ i=1,\ldots,n.\]
    А задача, которую следует решать покупателю, состоит в том,
    чтобы просто найти минимальную среди тех цен, которые этим
    неравенствам удовлетворяют. Эта задача выглядит следующим
    образом:
    \[p\rightarrow\min,\]
   \[pa_{i}\geqslant q_{i}, \ i=1,\ldots,n,\]
   \[p\geqslant0.\]
   В векторно-матричных обозначениях ее можно переписать следующим
   образом:
   \[p\rightarrow\min,\]
   \[p\vc{a}\geqq\vc{q},\]
   \[p\geqslant0.\]
   Если читатель уже прошел курс линейного программирования, он,
   видимо, уже заметил, что это не что иное, как задача, двойственная к
   задаче (\ref{LP0-1})--(\ref{LP0-3}). С теорией двойственности в
   линейном программировании мы познакомимся вкратце  ниже
   ??????, а сейчас просто заметим, что решением
   сформулированной задачи о выборе цены ресурса является именно
   число $p^{*}$, которое мы проинтерпретировали чуть выше как
   равновесную цену.

    Более того, выполняется равенство
    \[p^{*}b=q_{1}x_{1}^{*}+\ldots+q_{n}x_{n}^{*}.\]
    А это значит, что владельцам предприятия
   безразлично с точки зрения извлечения максимального дохода,
   продавать ли весь имеющийся ресурс по цене $p^{*}$, или
   оптимальным образом организовать производство.

Быть может, тот факт, что при равновесной цене
    положительная прибыль не может быть получена в
    принципе, а в оптимальном плане производства она окажется равной только нулю,
    вызывал у читателя некоторое недоумение. Это недоумение вполне
    обоснованно, однако сейчас мы не будем его развеивать.
    В несколько более широком контексте
    мы коснемся данного вопроса чуть позднее в параграфе (????????????),
    а сейчас только заметим, что прибыль о которой идет речь
    называется в экономической теории \emph{экономической прибылью} и ее равенство нулю считается
    в экономической теории нормальным явлением.

\begin{exer}
    Рассмотрим задачу
    \[\begin{array}{ccccccccc}
        q_{1}x_{1} & + & q_{2}x_{2}  & + & \ldots & + & q_{n}x_{n}  & \rightarrow & \max \\
        a_{1}x_{1} & + & a_{2}x_{2}  & + & \ldots & + & a_{n}x_{n}& \leqslant & b \\
        x_{1} &  &  &  &  &  &  & \leqslant & \bar{x}_{1} \\
         &  & x_{2} &  &  &  &  & \leqslant & \bar{x}_{2} \\
         &  & \ldots &  &  &  &  &  &  \\
         &  &  &  &  &  & x_{n} & \leqslant & \bar{x}_{n} \\
        x_{1}\geqslant0 &  & x_{2}\geqslant0 &  & \ldots &  & x_{n}\geqslant0 &  &
      \end{array}
    \]
    где \[q_{j}>0, \ a_{j}>0, \ \bar{x}_{n}>0,  \ j=1,\ldots,n.\]
    Объясните содержательный смысл этой задачи и опишите устройство
    ее решения в предположении, что выполняются следующие неравенства:
    \[\frac{q_{1}}{a_{1}}>\frac{q_{2}}{a_{2}}>\ldots>\frac{q_{n}}{a_{n}}.\]
\end{exer}











\subsection{Линейная задача распределения двух видов ресурсов}
    В этом пункте мы попытаемся обобщить те рассуждения, которые
    провели выше, на случай, когда распределяются два вида ресурсов.
    При этом в своем рассуждении мы не будем стремиться
    к математической строгости, а будем основываться на здравом
    экономическом смысле. Можно было бы рассмотреть общий случай с
    любым количеством различных видов ресурсов, однако на задаче
    именно с двумя видами ресурсов можно увидеть все
    основные проблемы, которые появляются в случае, когда количество
    ограничений по ресурсам больше одного.


    Итак, предположим, что для производства $n$ различных видов
    продуктов требуется не один, а два различных вида
    ресурсов. Пусть в распоряжении
    предприятия имеется $b_{1}>0$ единиц первого ресурса и $b_{2}>0$
    единиц второго ресурса. Мы будем
    предполагать, что для производства одной единицы $i$-го
    продукта, цена которого равна $q_{i}$ руб./ед.,
    требуется $a_{1i}\geqslant0$ единиц первого ресурса и
    $a_{2i}\geqslant0$ единиц второго ресурса. При этом хотя бы одно из
    этих двух чисел является положительным.
    Тем самым, если объем выпуска этого продукта равен
    $x_{i}\geqslant0$ единиц, то доход составит $q_{i}x_{i}$ рублей,
    затраты первого ресурса --- $a_{1i}x_{i}$ единиц, а затраты
    второго ресурса --- $a_{2i}x_{i}$ единиц.

    Как и выше, под планом (производства) мы будем понимать любой
    неотрицательный вектор
\[\vc{x}=\left(
     \begin{array}{c}
        x_{1} \\
        x_{2} \\
        \vdots \\
        x_{n}  \\
      \end{array}
\right),\]
    под допустимым планом --- план, который обеспечен ресурсами,
    т.е. такой вектор $\vc{x}=(x_{1},\ldots,x_{n})^{T}\geqq\vc{0}$, который удовлетворяет неравенствам
    \[a_{11}x_{1}+\ldots+a_{1n}x_{n}\leqslant b_{1}, \]
    \[a_{21}x_{1}+\ldots+a_{2n}x_{n}\leqslant b_{2},\]
    а под оптимальным планом --- такой допустимый план, который
    среди всех допустимых планов обеспечивает максимальный доход.




    Задачу об отыскании оптимального плана естественно записать так:

\begin{equation}
    \label{LP1}
\left\{
\begin{array}{l}
  q_{1}x_{1}+\ldots+q_{n}x_{n}\rightarrow\max, \\
  a_{11}x_{1}+\ldots+a_{1n}x_{n}\leqslant b_{1}, \\
  a_{21}x_{1}+\ldots+a_{2n}x_{n}\leqslant b_{2},\\
  x_{1}\geqslant0, \ldots, x_{n}\geqslant0.
\end{array}
\right.
\end{equation}

    Чтобы решить эту задачу, тех совсем простых соображений, которые
    мы использовали применительно к задаче распределения одного ресурса,
    уже недостаточно. Но мы даже и не будем
    пытаться ее решать. Простых рецептов отыскания решений в задачах
    линейного программирования не существует. Для этого существует
    довольно много численных методов, среди
    которых наиболее известными являются симплекс-метод и различные
    его модификации. Такие методы не являются предметом нашего
    рассмотрения в этой книге. Мы ограничимся качественным
    анализом устройства оптимальных планов.

    Попытаемся сформулировать применительно к задаче \ref{LP1}
    утверждение, аналогичное предложению
    \ref{LP0-usloviya-optimalnosti}, причем делать это будем,
    опираясь на экономическую интерпретацию последнего.

    Пусть вектор
    \[\vc{x^{*}}=\left(
     \begin{array}{c}
        x_{1}^{*} \\
        x_{2}^{*} \\
        \vdots \\
        x_{n}^{*}  \\
      \end{array}
\right)\]
     представляет собой
    решение задачи (\ref{LP1}). Зададимся вопросом о том, можно ли найти
    такой вектор равновесных цен на распределяемые ресурсы
    $\vc{p^{*}}=(p_{1}^{*},p_{2}^{*})$,
    который делал бы реализацию оптимального плана $\vc{x^{*}}$
    самым выгодным действием для владельцев предприятия с точки
    зрения извлечения максимальной прибыли даже при условии, что
    возможными вариантами действий являются продажа или покупка
    ресурсов по этим равновесным ценам.

    Сформулируем условия, которым должны удовлетворять равновесные цены:
  \begin{itemize}

    \item [1)] приобретение какого-то количества ресурсов и
    соответствующее расширение производства не принесет предприятию дополнительной
    прибыли по сравнению с ситуацией, когда выпуск задается вектором $\vc{x^{*}}$;
    \item [2)] продажа какого-то количества ресурсов, сопровождаемая или
    не сопровождаемая уменьшением объемов выпуска некоторых
    продуктов,  не принесет предприятию дополнительной
    прибыли по сравнению с ситуацией, когда выпуск задается вектором $\vc{x^{*}}$;
    \item [3)] план $\vc{x^{*}}$ является самым прибыльным среди
    всех возможных допустимых или недопустимых планов.
  \end{itemize}

    Запишем эти условия в виде математических
    соотношений. Первое из этих условий записать несложно, ибо оно просто
    означает, что производство ни одного из продуктов
    не может приносить положительную прибыль:
\begin{equation}\label{bezpribilnost}
    p_{1}^{*}a_{1i}+p_{2}^{*}a_{2i}\geqslant q_{i}, \ i=1,\ldots,n,
\end{equation}
    ибо в противном случае выгодно было бы приобретать как можно
    большее количество ресурсов, необходимых для производства
    прибыльных продуктов, и заниматься производством этих продуктов.
    Заметим, что это соотношение гарантирует нам, что ни один план
    (будь он допустимый или недопустимый)
    не может дать положительной прибыли, ибо для любого
    $\vc{x}=(x_{1},\ldots,x_{n})^{T}\geqq\vc{0}$
     выполняется неравенство
    \[
     (p_{1}^{*}a_{11}+p_{2}^{*}a_{21})x_{1}+\ldots
     +(p_{1}^{*}a_{1n}+p_{2}^{*}a_{2n})x_{n}
     \geqslant q_{1}x_{1}+\ldots+q_{n}x_{n}.
    \]
    Справедливость этого неравенства основана на том, что под
    планами мы понимаем только неотрицательные векторы.
    В его правой части стоит доход, который принесет план
    $\vc{x}$ при заданных ценах на продукты,
    а в левой --- полные расходы на его реализацию при ценах на
    ресурсы $\vc{p^{*}}$ .



    Условие 2) вместе с  условием 3) говорят, что в оптимальном плане
    производятся только безубыточные продукты:
\begin{equation}\label{dop-nejost1}
        p_{1}^{*}a_{1i}+p_{2}^{*}a_{2i}>q_{i} \Rightarrow
    x_{i}^{*}=0, \ i=1,\ldots,n.
\end{equation}
    Действительно, в противном случае те ресурсы, которые
    затрачиваются на производство убыточных продуктов, было бы
    выгодно просто продать.
    Поскольку ни один план не может принести положительную прибыль,
    это условие эквивалентно условию безубыточности оптимального
    плана $\vc{x^{*}}$ в целом. Это последнее записывается
    следующим образом:
\begin{equation}\label{dop-nejost2}
    (p_{1}^{*}a_{11}+p_{2}^{*}a_{21})x_{1}^{*}+\ldots
     +(p_{1}^{*}a_{1n}+p_{2}^{*}a_{2n})x_{n}^{*}
    = q_{1}x_{1}^{*}+\ldots+q_{n}x_{n}^{*}.
\end{equation}



    Кроме того, условие 2) содержит в себе еще одно важное
    требование. Для того, чтобы артикулировать это требование,
    вспомним, что  в предложении \ref{LP0-usloviya-optimalnosti},
    посвященном задаче распределения одного ресурса, важную роль играет условие
    (\ref{LP0-ravenstvo}), говорящее, что распределяемый ресурс
    используется полностью. В ситуации, когда распределяются два и
    более видов ресурсов, в оптимальном плане
    совсем необязательно полностью будут исчерпаны
    ресурсы всех видов, поскольку не исключена ситуация, когда какой-то вид
    ресурса находится в избытке. Например, в качестве ресурса,
    необходимого для производства, может выступать воздух. Можно
    надеяться, что в оптимальном плане воздух, имеющийся в
    распоряжении нашего предприятия, не будет исчерпан полностью.

    Хотя нельзя гарантировать, что в оптимальном плане полностью
    расходуются оба ресурса, можно с уверенностью сказать, что хотя
    бы один из них будет потрачен полностью.

\begin{exer}
    Докажите, что в оптимальном плане хотя бы один ресурс будет
    потрачен полностью. Приведите числовой пример, показывающий, что возможна
    ситуация, когда полностью будет потрачен только один из двух
    ресурсов.
\end{exer}

    Конечно, мы не знаем априорно, до решения задачи,
     какой из двух ресурсов будет использован в оптимальном плане полностью.
    Однако, нам этого и не надо знать, чтобы формально записать
    требования, составляющие суть условия 2). Одно из этих
    требований, мы уже записали в виде (\ref{dop-nejost1}). Оно
    говорит, что уменьшение выпуска продукции и продажа
    высвободившихся ресурсов не увеличит прибыль, поскольку в
    оптимальном плане производятся только безубыточные продукты. Но
    условие 2) говорит еще и о том, что и продажа недоиспользованных
    в оптимальном плане ресурсов тоже не может принести предприятию
    дополнительной прибыли. А это означает, что если какой-то ресурс
    используется в оптимальном плане не полностью, то его
    равновесная цена должна быть равной нулю. Это требование можно
    записать следующим образом:
    \[a_{j1}x_{1}^{*}+\ldots+a_{jn}x_{n}^{*}<b_{j}\Rightarrow p_{j}^{*}=0, \ j=1,2.\]
    Поскольку обе цены $p_{1}^{*}$ и $p_{2}^{*}$ должны быть
    неотрицательными, причем они одновременно не могут равняться
    нулю (если выполняется условие 1)), а оптимальный план является
    допустимым, то это требование можно переписать в виде следующего равенства:
\begin{equation}\label{dop-nejost3}
    p_{1}^{*}(a_{11}x_{1}^{*}+\ldots+a_{1n}x_{n}^{*})
      +p_{2}^{*}(a_{21}x_{1}^{*}+\ldots+a_{2n}x_{n}^{*})
    =p_{1}^{*}b_{1}+p_{2}^{*}b_{2}.
\end{equation}

    Теперь мы можем сформулировать вопрос о существовании
    равновесных цен следующим образом. Если  вектор $\vc{x^{*}}\geqq0$
    является решением задачи (\ref{LP1}), то существует ли такой вектор
    равновесных цен $\vc{p^{*}}=(p_{1}^{*},p_{2}^{*})\geqq0$, что
    выполняются соотношения (\ref{bezpribilnost}),
    (\ref{dop-nejost2}) и (\ref{dop-nejost3})?
    Ответ на этот вопрос положительный. А именно, справедливо
    следующее предложение.


  \begin{prop}
    \label{LP1-usloviya-optimalnosti}
    Допустимый план $\vc{x^{*}}\geqq0$ задачи (\ref{LP1}) является
    ее решением тогда и только тогда, когда найдется вектор
    $\vc{p^{*}}=(p_{1}^{*},p_{2}^{*})\geqq0$,
    такой что выполняются соотношения (\ref{bezpribilnost}),
    (\ref{dop-nejost2}) и (\ref{dop-nejost3})
  \end{prop}

    В отличие от предложения \ref{LP0-usloviya-optimalnosti}, это
    предложение не является самоочевидным и нуждается в аккуратном
    математическом доказательстве. Мы на некоторое время отложим доказательство
    и проведем его для несколько более общей задачи в следующем ????? параграфе.
    А сейчас только отметим, что условия оптимальности, о которых идет речь в данном
    предложении, являются не только необходимыми, но и достаточными.
    Это значит, что всякий допустимый план $\vc{x^{*}}=(x_{1}^{*},\ldots,x_{n}^{*})\geqq0$,
    при котором найдется вектор
    $\vc{p^{*}}=(p_{1}^{*},p_{2}^{*})\geqq0$, для которого выполняются
    соотношения (\ref{bezpribilnost}),
    (\ref{dop-nejost2}) и (\ref{dop-nejost3}), является оптимальным
    планом.



    Задачу (\ref{LP1}) можно записать в матричной форме:
    \[\vc{q}\vc{x}\rightarrow\max,\]
    \[ \vc{A}\vc{x}\leqq \vc{b}, \]
    \[ \vc{x}\geqq0.\]
    где
    \[\vc{q}=(q_{1},\ldots,q_{n}), \
    \vc{A}=
    \left(
      \begin{array}{cccc}
        a_{11} & a_{12} & \ldots & a_{1n} \\
        a_{21} & a_{22} & \ldots & a_{2n} \\
      \end{array}
    \right), \
    \vc{b}=\left(
               \begin{array}{c}
                 b_{1} \\
                 b_{2} \\
               \end{array}
             \right).\]

    \begin{exer}\label{LP1-usloviya-optimalnosti-matr}
    Докажите, что предложение \ref{LP1-usloviya-optimalnosti} можно
    сформулировать следующим образом:
    вектор  $\vc{x^{*}}\geqq0$
    является решением задачи (\ref{LP1})
    тогда и только тогда, когда он является допустимым планом, т.е.
    удовлетворяет неравенству  $\vc{A}\vc{x^{*}}\leqq \vc{b}$,
    и найдется вектор $\vc{p^{*}}\geqq0$, для которого выполняются
    следующие соотношения:
    \[\vc{p^{*}}\vc{A}\geqq\vc{q},\]
    \[\vc{p^{*}}\vc{b}=\vc{p^{*}}\vc{A}\vc{x^{*}}=\vc{q}\vc{x^{*}}.\]
\end{exer}

    Теперь предположим, что на вектор ресурсов $\vc{b}$ нашелся покупатель,
    который хочет убедить владельцев рассматриваемого нами
    предприятия не заниматься производством, а просто продать ему
    вектор ресурсов целиком. Для этого предполагаемому покупателю
    нужно предложить владельцам такие
    цены $p_{1}\geqslant0$ и $p_{2}\geqslant0$ на ресурс первого и второго
    вида соответственно, которые, с одной стороны, были
    бы выгодными для полследних, а с другой --- минимизировали бы
    его затраты на приобретение ресурсов.


    Какими должен быть вектор цен $\vc{p}=(p_{1},p_{2})\geqq0$, чтобы он
    оказался приемлемым для владельцев? Он просто должен
    удовлетворять неравенствам
\begin{equation}\label{nevigod}
    p_{1}a_{1i}+p_{2}a_{2i}\geqslant q_{i}, \ i=1,\ldots,n,
\end{equation}
    которые в матрично-векторном виде можно коротко записать как
    \[\vc{p}\vc{A}\geqq\vc{q}.\]
    Действительно, если бы для некоторого продукта $i$  выполнялось бы
    соотношение
    \[p_{1}a_{1i}+p_{2}a_{2i}<q_{i},\]
    то это свидетельствовало бы о том, что хотя бы какую-то часть
    имеющихся
    ресурсов предприятию было бы более выгодно не продавать, а потратить
    на производство этого продукта. В том же случае, когда
    соотношения (\ref{nevigod}) верны, никакое производство не
    окажется более выгодным делом, чем продажа ресурсов.

    Поскольку при ценах $\vc{p}=(p_{1},p_{2})$
    затраты на приобретения вектора ресурсов $\vc{b}=\left(
               \begin{array}{c}
                 b_{1} \\
                 b_{2} \\
               \end{array}
             \right)$,
    составляют
\begin{equation} \label{zatr-na-priobr}
    \vc{p}\vc{b}=p_{1}b_{1}+p_{2}b_{2}
\end{equation}
    денежных единиц,
    задача потенциального покупателя состоит в том, чтобы
    предложить такой вектор цен $\vc{p}\geqq0$, который
    минимизировал бы его затраты на приобретение вектора ресурсов
    $\vc{b}$, т.е. величину (\ref{zatr-na-priobr}) при условии, что
    эти цены приемлемы для владельца, т.е. удовлетворяют
    неравенствам (\ref{nevigod}). Запишем эту задачу:
    \begin{equation}\label{dvoistv}
    \begin{array}{ccccc}
         p_{1}b_{1} & + & p_{2}b_{2} & \rightarrow & \min \\
         p_{1}a_{11} & + & p_{2}a_{21} & \geqslant & q_{1} \\
         p_{1}a_{12} & + & p_{2}a_{22} & \geqslant & q_{2} \\
         \ldots & \ldots & \ldots & \ldots & \ldots \\
         p_{1}a_{1n} & + & p_{2}a_{2n} & \geqslant & q_{n} \\
         p_{1}\geqslant0 &   & p_{1}\geqslant0. &  &
      \end{array}
\end{equation}
    В векторно-матричной форме она выглядит совсем коротко:
        \[\begin{array}{lcl}
        \vc{p}\vc{b} & \rightarrow & \min \\
        \vc{p}\vc{A}  & \geqq & \vc{b} \\
        \vc{p} & \geqq & \vc{0}
      \end{array}
    \]
    и называется двойственной к задаче (\ref{LP1}).

    Читатель,
    скорее всего, в одном из предыдущих учебных курсов уже
    встречался с двойственностью в линейном программировании. Но
    если это не так, то он познакомится с ней ниже(??????).


    Но в любом случае он,
    видимо, уже заметил, что для любого вектора $\vc{x}\geqq\vc{0}$,
    удовлетворяющего неравенству $\vc{A}\vc{x}\leqq\vc{b}$  и любого
    вектора $\vc{p}\geqq\vc{0}$, удовлетворяющего неравенству
    $\vc{p}\vc{A}\geqq\vc{q}$ выполняется неравенство
    \[\vc{p}\vc{b}\geqslant\vc{q}\vc{x}.\]
    Иными словами, на любом допустимом векторе задачи (\ref{LP1})
    значение ее целевой функции не превосходит значения
    целевой функции задачи (\ref{dvoistv}) на любом ее допустимом
    векторе.




    При этом для вектора $\vc{x^{*}}$, являющегося решением задачи
    (\ref{LP1}), и вектора $\vc{p^{*}}=(p_{1}^{*},p_{2}^{*})$, о котором идет речь в
    предложении \ref{LP1-usloviya-optimalnosti},  вектором справедливо
    равенство $\vc{p^{*}}\vc{b}=\vc{q}\vc{x^{*}}$
    (см. упражнение \ref{LP1-usloviya-optimalnosti-matr}). Заметив, что
    векторы $\vc{x^{*}}$ и $\vc{p^{*}}$ является допустимыми для  задач
    (\ref{LP1}) и (\ref{dvoistv}) соответственно, заключаем, что
    справедливо следующее предложение.

\begin{prop}
    \label{LP1-priamaya-dvoistvennaya}
    Оптимальное значение задачи (\ref{LP1}) совпадает с оптимальным значением двойственной
    к ней задаче (\ref{dvoistv}).
\end{prop}

    С точки зрения экономического содержания это предложение выглядит
    вполне естественным. Оно говорит, что вместо того, чтобы
    решать задачу (\ref{dvoistv}), потенциальному покупателю вектора
    ресурсов $\vc{b}$ просто нужно предложить владельцам предприятия ту
    сумму, которую те могут получить
    в случае реализации на предприятии оптимального плана $\vc{x^{*}}$
    задачи (\ref{LP1}).



\

\section{Формальное введение в теорию линейного программирования}


    Мы познакомили читателя со
    стилизованной задачей распределения ресурсов,
    представляющую собой хороший пример задачи линейного
    программирования, а также провели некоторые содержательные
    рассуждения, которые позволили нам сформулировать
    условия оптимальности, имеющие естественную экономическую
    интерпретацию. Сейчас мы переходим к формальному
    изложению теории линейного программирования.









\subsection{Различные формы задачи линейного программирования}

    Задачей линейного программирования называется задача
нахождения наибольшего (задачи на максимум) или наименьшего значения (задачи на минимум)
некоторой линейной целевой функции
\[
    \vc{c}\vc{x}=c_{1}x_{1}+\ldots+c_{n}x_{n}
\]
на некотором множестве $\st{D}\subset\R^{n}$, которое задается системой линейных неравенств и
уравнений.

Любой вектор $\vc{x}$, принадлежащий множеству $\st{D}$ называется \emph{допустимым}
вектором, допустимым планом или допустимой точкой. Решением задачи линейного программирования
называется такой допустимый вектор $\vc{x^{*}}$, что для любого другого допустимого вектора
$\vc{x}$ выполняется неравенство $\vc{c}\vc{x^{*}}\geqslant\vc{c}\vc{x}$ (если идет речь о
задаче на максимум) или неравенство $\vc{c}\vc{x^{*}}\leqslant\vc{c}\vc{x}$ (если идет речь о
задаче на минимум). Если решение у задачи линейного программирования существует, то
\emph{значением} этой задачи называется значение целевой функции в точке решения, т.е.
величина $\vc{c}\vc{x^{*}}$.

Далее мы увидим, что если решения у задачи линейного программирования нет, то для любого
сколь угодно большого (в случае задачи на максимум) или малого  (в случае задачи на минимум)
числа найдется такой допустимый план, для которого значение целевой функции равно этому
числу. В этом случае мы говорим, что значением задачи является $+\infty$ для задачи на
максимум или $-\infty$ для задачи на минимум.

Любая задача линейного программирования допускают несколько
различных форм. При этом путем несложных преобразований можно всегда
перейти от одной формы к другой таким образом, что значение задачи
при переходе не изменится, а решение задачи в одной из форм
однозначно задает решение задачи в другой. В этом смысле различные
формы являются эквивалентными. В этом пункте мы вкратце расскажем о
различных формах задач линейного программирования и том, как
переходить от одной формы к другой. Наши рассуждения, не умаляя
общности, мы будем проводить применительно к задачам на максимум.

Положим
\[\vc{c}=(c_{1},\ldots,c_{n}),\]
\[\vc{A}=\left(
\begin{array}{cccc}
   a_{11} & a_{12} & \ldots & a_{1n} \\
   a_{21} & a_{22} & \ldots & a_{2n} \\
   \ldots& \ldots &\ldots &\ldots \\
   a_{m1} & a_{m2} & \ldots & a_{mn}
\end{array}
\right), \ \vc{b}=\left(
     \begin{array}{c}
        b_{1} \\
        b_{2} \\
        \vdots \\
        b_{m}  \\
      \end{array}
    \right),\ \vc{x}=\left(
     \begin{array}{c}
        x_{1} \\
        x_{2} \\
        \vdots \\
        x_{n}  \\
      \end{array}
    \right).\]
    Здесь мы предполагаем, что векторы $\vc{c}$ и $\vc{b}$, а также
    матрица $\vc{A}$ являются заданными, а $\vc{x}$ --- это вектор
    переменных.


Начнем  с так называемой \emph{основной задачи} линейного
программирования, которая записывается как
\begin{equation} \label{OSZLP}
\left\{
\begin{array}{rrrrllll}
     c_1 x_1 + & c_2 x_2 +    & \ldots +& c_n x_n &\to & \max\\
     a_{11} x_1 + & a_{12} x_2 + &\ldots +& a_{1n} x_n &\leqslant& b_1 \\
     a_{21} x_1 + & a_{22} x_2 + &\ldots +& a_{2n} x_n &\leqslant& b_2\\
                      & \ldots &&&&\\
     a_{m1} x_1 + & a_{m2} x_2 +& \ldots +& a_{mn} x_n &\leqslant& b_m\\
    \end{array} \right.
\end{equation}
    или, в векторно-матричном виде, как
\[
    \left\{
    \begin{array}{rl}
     \vc{c} \, \vc{x} & \to \max  \\
    \vc{A} \vc{x} &\leqq \vc{b} \\
        \end{array} \right.
\]
    Такая задача удобна для математического анализа, точнее,
    для доказательства необходимых и достаточных условий
    оптимальности, которые мы сформулируем ниже.

    Основная задача линейного программирования обладает простой геометрической интерпретацией.
    \remrk{РИС (решение существует) РИС (не существует)}





\emph{Стандартная задача} линейного программирования записывается
следующим образом:
\begin{equation} \label{SZLP}
\left\{
\begin{array}{rrrrllll}
     c_1 x_1 + & c_2 x_2 +    & \ldots +& c_n x_n &\to & \max\\
     a_{11} x_1 + & a_{12} x_2 + &\ldots +& a_{1n} x_n &\leqslant& b_1 \\
     a_{21} x_1 + & a_{22} x_2 + &\ldots +& a_{2n} x_n &\leqslant& b_2\\
                      & \ldots &&&&\\
     a_{m1} x_1 + & a_{m2} x_2 +& \ldots +& a_{mn} x_n &\leqslant& b_m\\
     x_1 \geqslant 0,   & x_2 \geqslant 0,  & \ldots,&  x_n \geqslant 0\\
\end{array} \right.
\end{equation}
или, коротко,
\begin{equation}\label{SZLPV}
\left\{
\begin{array}{rl}
 \vc{c} \, \vc{x} & \to \max  \\
 \vc{A} \vc{x} &\leqq \vc{b} \\
 \vc{x} &\geqq \vc{0}\\
\end{array} \right.
\end{equation}
Такая задача характеризуется тем, что все ограничения задачи
представлены в виде неравенств, и, в отличие от основной,  на все
переменные в явном виде наложены ограничения неотрицательности.
Популярность стандартной задачи линейного программирования
обусловлена тем, что математическая постановка многих содержательных
экономико-математических моделей сводится именно к задаче такого
рода. Примером могут служить задачи распределения одного или двух
ресурсов, с которых мы начали свой рассказ о линейном
программировании.





Большинство методов решения задач линейного программирования
разработаны для \emph{канонической задачи} линейного
программирования. В этой форме записи задачи линейного
программирования на все переменные наложены условия
неотрицательности а все ограничения, за исключением ограничений на
неотрицательность переменных, задаются в виде уравнений:


\begin{equation}\label{KZLP}
 \left\{
\begin{array}{rrrrllll}
     c_1 x_1 + & c_2 x_2 +    & \ldots +& c_n x_n &\to & \max\\
     a_{11} x_1 + & a_{12} x_2 + &\ldots +& a_{1n} x_n &=& b_1 \\
     a_{21} x_1 + & a_{22} x_2 + &\ldots +& a_{2n} x_n &=& b_2\\
                      & \ldots &&&&\\
     a_{m1} x_1 + & a_{m2} x_2 +& \ldots +& a_{mn} x_n &=& b_m\\
     x_1 \geqslant 0,   & x_2 \geqslant 0,  & \ldots,&  x_n \geqslant 0\\
\end{array} \right.
\end{equation}
Каноническая задача линейного программирования в векторно-матричном
виде выглядит так:
\begin{equation}\label{KZLPV}
\left\{
\begin{array}{rl}
 \vc{c} \, \vc{x} & \to \max  \\
 \vc{A} \vc{x} &= \vc{b} \\
 \vc{x} &\geqq \vc{0}\\
\end{array} \right.
\end{equation}

Конечно, при заданной матрице $\vc{A}$ и заданных векторах $\vc{c}$
и $\vc{b}$ три выписанные задачи не эквивалентны. В то же время
каждая из них легко преобразуется в другую форму с помощью несложных
преобразований. Стандартную и каноническую задачи линейного
программирования совсем несложно свести к  основной.


    Действительно, стандартную задачу линейного
    программирования (\ref{SZLP}) можно записать как основную следующим образом:
\begin{equation}\label{SZLPSV}
\left\{
\begin{array}{rrrrrrlc}
            & c_1 x_1     & + & c_2 x_2       &+ \ldots + & c_n x_n    &\to  & \max\\
            & a_{11} x_1  & + & a_{12} x_2    &+ \ldots + & a_{1n} x_n &\leqslant & b_1 \\
            & a_{21} x_1  & + & a_{22} x_2    &+ \ldots + & a_{2n} x_n &\leqslant & b_2\\
            &             &   &  \ldots \\
            & a_{m1} x_1  & + & a_{m2} x_2    &+ \ldots + & a_{mn} x_n &\leqslant& b_m\\
            &(-1)\cdot x_1& + & 0\cdot x_2    &+ \ldots + & 0\cdot x_n &\leqslant& 0 \\
            & 0 \cdot x_1 & + & (-1)\cdot x_2 &+ \ldots + & 0\cdot x_n &\leqslant& 0 \\
            &             &   &  \ldots \\
            & 0 \cdot x_1 & + & 0\cdot x_2    &+ \ldots + &(-1)\cdot x_n &\leqslant& 0 \\
\end{array} \right.
\end{equation}
    В векторно-матричном виде ее можно записать в виде
\begin{equation}\label{SZLPVSV}
\left\{
\begin{array}{rl}
 \vc{c} \, \vc{x} & \to \max  \\
 \vc{A} \vc{x} &\leqq \vc{b} \\
 \vc{-E} \vc{x}&\leqq \vc{0}
\end{array} \right.
\end{equation}

\noindent где $\vc{E}$ --- единичная $n\times n$ матрица, а $\vc{0}$
--- $n$-мерный вектор-столбец, состоящий из нулей.





    Что касается канонической задачи (\ref{KZLP}), то она очевидным
    образом сводится к стандартной форме, ибо всякое
равенство эквивалентно системе из двух неравенств:
\[
a = b \Leftrightarrow \left\{ \begin{array}{l}
 a \leqslant b \\
 a \geqslant b \\
 \end{array} \right.
\]
    Отсюда следует, что задачу (\ref{KZLP}) можно записать в форме стандартной задачи:
    \begin{equation}\label{KZLP-osn}
\left\{
\begin{array}{rllllllll}
     c_1 x_1 & + & c_2 x_2 &+ & \ldots &+& c_n x_n &\to & \max\\
        a_{11} x_1 &+& a_{12} x_2  &+ &\ldots &+& a_{1n} x_n & \leqslant & b_1 \\
        a_{21} x_1 &+& a_{22} x_2  &+&\ldots &+& a_{2n} x_n & \leqslant & b_2\\
&&\ldots &&&&\\
        a_{m1} x_1 &+& a_{m2} x_2 &+& \ldots &+& a_{mn} x_n & \leqslant & b_m\\
        (-a_{11}) x_1 &+& (-a_{12}) x_2  &+ &\ldots &+& (-a_{1n}) x_n & \leqslant & -b_1 \\
        (-a_{21}) x_1 &+& (-a_{22}) x_2  &+&\ldots &+& (-a_{2n}) x_n & \leqslant & -b_2\\
&&\ldots &&&&\\
        (-a_{m1}0 x_1 &+& (-a_{m2}) x_2 &+& \ldots &+& (-a_{mn}) x_n & \leqslant & -b_m\\
        x_1 \geqslant 0, & & x_2 \geqslant 0, & & \ldots,& & x_n \geqslant 0\\
\end{array} \right.
\end{equation}
    или, коротко, как
\begin{equation}\label{KZLP-osn-vec}
\left\{
\begin{array}{rl}
 \vc{c} \, \vc{x} & \to \max  \\
 \vc{A} \vc{x} &\leqq \vc{b} \\
 \vc{-A} \vc{x} &\leqq \vc{-b} \\
 \vc{x} &\geqq \vc{0}\\
\end{array} \right.
\end{equation}


\begin{exer}
     Как задачу (\ref{SZLPVSV}), так и задачу (\ref{KZLP-osn-vec}),
     можно записать в следующем виде:
    \begin{equation*}\label{SZLPVSV-2}
\left\{
\begin{array}{rl}
 \vc{c} \, \vc{x} & \to \max  \\
 \vc{\tilde{A}} \vc{x} &\leqq \vc{\tilde{b}} \\
 \end{array} \right.
\end{equation*}
    Как устроены матрица $\vc{\tilde{A}}$ и вектор $\vc{\tilde{b}}$ для этих задач?
\end{exer}


    Для того, чтобы преобразовать основную задачу в стандартную, напомним, что любое
    число $x$ можно представить как разность двух неотрицательных
    чисел $x'$ и $x''$:
\[
    \forall x\in\mathbb{R} \ \exists \ x'\geqslant0, \ x''\geqslant0 \ : \ x=x'-x''.
\]
    Тем самым если при рассмотрении основной задачи (\ref{OSZLP}) мы
    каждую из переменных $x_{j}, \ j=1,\ldots,n,$ перепишем в виде
\[
    x_{j}=x'_{j}-x''_{j}, \ x'_{j}\geqslant0, \ x''_{j}\geqslant0,
\]
    то задача приобретет форму стандартной задачи:
    \begin{equation} \label{OKSZLP}
\left\{
\begin{array}{ccccccccccc}
c_1 x'_1 &+ & (-1)c_1 x''_1 &+    & \ldots &+& c_n x'_n &+&(-1)c_n x''_n&\to & \max\\
         a_{11}x'_1 &+ & (-1)a_{11}x''_1& + &\ldots &+& a_{1n} x'_n &+&(-1)a_{1n} x''_n &\leqslant& b_1 \\
                      & \ldots & & & & &\\
         a_{m1} x'_1 &+ & (-1)a_{m1} x''_1 &+& \ldots &+& a_{mn} x'_n &+&(-1)a_{mn} x''_n&\leqslant& b_m\\
            x'_1\geqslant0, && x''_1\geqslant0, & & \ldots & & x'_n\geqslant0, && x''_n\geqslant0.\\
         \end{array} \right.
\end{equation}

\begin{exer}
    Запишите задачу (\ref{OKSZLP}) в векторно-матричном виде.
\end{exer}





Для преобразования стандартной задачи линейного программирования
(\ref{SZLPV}) в каноническую поступают следующим образом.
 Ограничения-неравенства преобразуются в равенства за счет
добавления дополнительных неотрицательных переменных. А именно, для
каждого $i=1,\ldots,m$ ограничение
\[  a_{i1} x_1 + a_{i2} x_2  +\ldots + a_{in} x_n \leqslant b_i\]
за счет добавления новой переменной $x_{n+i} \geqslant 0$ записывается как равенство
\[  a_{i1} x_1 + a_{i2} x_2  +\ldots + a_{in} x_n + 1 \cdot x_{n+i} = b_i;\]
Одновременно новые  переменные $x_{n+i}$, $i=1,\ldots,m$, добавляются в целевую функцию с
нулевыми коэффициентами, которая приобретает следующий вид:
\[
    c_1 x_1  +  c_2 x_2 +  \ldots + c_n x_n + 0 \cdot x_{n+1}+\ldots+0 \cdot x_{n+m}.
\]

\begin{exer}
    Запишите задачу (\ref{SZLPV}) после ее преобразования в каноническую форму
    в векторно-матричном виде.
\end{exer}


Определенным обобщением описанных выше форм задачи линейного
программирования является следующая \emph{общая задача} линейного
программирования, которая характерна тем, что
  часть ограничений задачи представлены в форме равенств, а
  часть --- в форме неравенств,
   на некоторые переменные накладывается требование
  неотрицательности, а на остальные --- нет. С помощью, если необходимо,
  перестановки строк и столбцов любая общая задача записывается
  следующим образом:

\begin{equation}\label{OZLP}
\left\{
\begin{array}{ccrrrcc}
 c_1 x_1  & +  \ldots + & c_r x_r+     & c_{r+1}x_{r+1}    +\ldots +  &c_n x_n    &\to  &\max \\
 a_{11} x_1 & +  \ldots + & a_{1r} x_r+  & a_{1,r+1} x_{r+1} +\ldots +  &a_{1n} x_n &\leqslant & b_1 \\
 a_{21} x_1 & +  \ldots + & a_{2r} x_r+  & a_{2,r+1} x_{r+1} +\ldots +  &a_{2n} x_n &\leqslant & b_2 \\
                            & \ldots \\
 a_{q1} x_1 & +  \ldots + & a_{qr} x_r+  & a_{q,r+1} x_{r+1} +\ldots +  &a_{qn} x_n &\leqslant & b_q \\
 a_{q+1,1} x_1 & + \ldots + & a_{q+1,r} x_r+ & a_{q+1,r+1} x_{r+1} +\ldots + &a_{q+1,n} x_n &= & b_{q+1} \\
                            && \ldots \\
 a_{m1} x_1 & + \ldots + & a_{mr} x_r+ & a_{m,r+1} x_{r+1} +\ldots + &a_{mn} x_n &= & b_m \\
             x_1 \geqslant 0,& \ldots, \,  &x_r \geqslant 0\\
\end{array} \right.
\end{equation}



Очевидно, что, с одной стороны, стандартная и каноническая задача
линейного программирования являются частными случаями общей задачи.
С другой стороны, общую задачу линейного программирования можно
свести к любому из этих частных случаев.

\begin{exer}
Опишите процедуру сведения общей задачи линейного программирования к
основной, канонической и стандартной форме. Сведите к основной,
канонической и стандартной форме следующую задачу линейного
программирования:
\end{exer}
\begin{equation*}\label{OZLPsimple_ex}
\left\{
\begin{array}{llllllllll}
    & 8 x_1      & + 10 x_2  & +  x_3  &\to & \max\\
    & 5 x_1      & -  7 x_2  & + 9 x_3 & \leqslant  & 6 \\
    & 11 x_1     & + 12 x_2  & - 3 x_3 & =  & 11\\
    & x_1 \geqslant 0, & x_2 \leqslant 0\\
\end{array} \right.
\end{equation*}

Отметим, что хотя приведенная выше общая задача линейного
программирования в виде~(\ref{OZLP}) выглядит довольно громоздко, ее
можно весьма компактно представить в векторно-матричном виде:

\begin{equation}\label{OZLPV}
\left\{
\begin{array}{rlcc}
    \vc{c_1} \, \vc{x_1} &+ \vc{c_2} \, \vc{x_2} &\to & \max \\
   \vc{A_{11}} \vc{x_1}          &+ \vc{A_{12}} \vc{x_2} & \leqq &\vc{b_1}\\
  \vc{A_{21}}  \vc{x_1}          &+ \vc{A_{22}} \vc{x_2} & = &\vc{b_2}\\
  \vc{x_1} \geqq \vc{0}&\\
\end{array} \right.
\end{equation}
Здесь $\vc{b_1}$ --- это $q$-мерный вектор-столбец, соответствующий
ограничениям, записанным в виде неравенство,  $\vc{b_2}$
--- $(m-q)$-мерный мерный-столбец, соответствующий ограничениям,
записанным в виде равенств, $\vc{x_1}$ --- это $r$-мерный
вектор-столбец, состоящий из  переменных, на которые наложено
условие неотрицательности, а $\vc{x_2}$ --- $(n-r)$-мерный
вектор-столбец, состоящий из переменных, на которые условие
неотрицательности не наложено. Что касается матриц $\vc{A_{11}}$, \
$\vc{A_{12}}$, \  $\vc{A_{21}}$ и $\vc{A_{22}}$, то они имеют
соответственно размерности $q \times r$, \ $q \times (n-r)$, \
$(m-q) \times r$ и $(m-q) \times (n-r)$.

\begin{exer}
Выпишите подробно векторы $\vc{b_1}$, \ $\vc{b_2}$, \ $\vc{x_1}$, \
$\vc{x_2}$ и матрицы $\vc{A_{11}}$, \ $\vc{A_{12}}$, \
$\vc{A_{21}}$, \ $\vc{A_{22}}$.
\end{exer}







\subsection{Условия оптимальности для задач линейного
программирования. Предварительные соображения.}

Отыскание решения задачи линейного программирования --- задача, с
которой, скорее всего, успешно справится тот или иной численный
метод, например, симплекс-метод или какая-нибудь его модификация.
Экономисту, будь он практик или теоретик, вникать в то, как работают
численные методы, в общем-то, необязательно. Зато экономисту очень
полезно знать, какими свойствами обладают оптимальные планы, и как
их отличить от неоптимальных. Для этого ему нужно быть хорошо
знакомым с необходимыми и достаточными условиями оптимальности,
особенно с учетом того, что, как мы увидели выше (мы имеем в виду
предложения \ref{LP0-usloviya-optimalnosti} и
\ref{LP1-usloviya-optimalnosti}), эти условия имеют очень
естественную экономическую интерпретацию.


Наша цель состоит в том, чтобы сформулировать необходимые и достаточные условия оптимальности
для различных форм задачи линейного программирования в различных формах примерно в том духе,
как они сформулированы в предложении \ref{LP1-usloviya-optimalnosti}. Мы могли бы это сделать
в достаточно сжатой форме. Однако, если бы мы именно так и поступили, у читателя, скорее
всего, возникло бы впечатление, что проделан некоторый удачный математический фокус, но не
помогло бы ему понять, в чем же состоит существо дела. По нашему мнению, глубокое понимание
всех математических конструкций, приведенных в этом параграфе, является просто необходимым
каждому экономисту. Более того, понимание этих конструкций может очень помочь читателю при
изучении теории нелинейного программирования, изложению которой посвящена глава ?????.
Поэтому наше изложение мы начнем с изучения нескольких простых частных случаев, неспешный
анализ которых позволит читателю лучше понять те рассуждения, которые мы проведем в
дальнейшем.

    Сначала мы рассмотрим задачу, которая может показаться читателю
    совсем простой и даже несколько странной:
\begin{equation}
   \label{prim-zad-lp}
    \begin{array}{c}
      \vc{c}\vc{x}=c_{1}x_{1}+\ldots+c_{n}x_{n}\rightarrow\max, \\
      \vc{a}\vc{x}=a_{1}x_{1}+\ldots+a_{n}x_{n}=b,
    \end{array}
\end{equation}
    где $\vc{c}=(c_{1},\ldots,c_{n})$ и $\vc{a}=(a_{1},\ldots,a_{n})$ --- это некоторые
    заданные ненулевые вектора, а $\vc{x}=(x_{1},\ldots,x_{n})^{T}$ --- вектор
    переменных, на который не накладывается даже требования неотрицательности.

    Ответить на вопрос о том, существует ли решение этой задачи, и
    какими свойствами оно обладает, совсем несложно. Следующее очевидное утверждение
     проиллюстрировано \remrk{рисунками ???а и ???б}, где предполагается, что $\vc{c}\neq0$.

\begin{prop}
    Если существует такое число $u$, что
    $\vc{c}=u\vc{a}$, то решением задачи (\ref{prim-zad-lp}) является любой допустимый вектор,
    а ее значением является число $ub$. В противном случае решения
    у рассматриваемой задачи не
    существует, а значением задачи является $+\infty$.
\end{prop}

    Теперь рассмотрим несколько более общую задачу.
Пусть, как и выше,
\[\vc{c}=(c_{1},\ldots,c_{n}),\]
\[\vc{A}=\left(
\begin{array}{cccc}
   a_{11} & a_{12} & \ldots & a_{1n} \\
   a_{21} & a_{22} & \ldots & a_{2n} \\
   \ldots& \ldots &\ldots &\ldots \\
   a_{m1} & a_{m2} & \ldots & a_{mn}
\end{array}
\right), \ \vc{b}=\left(
     \begin{array}{c}
        b_{1} \\
        b_{2} \\
        \vdots \\
        b_{n}  \\
      \end{array}
    \right),\ \vc{x}=\left(
     \begin{array}{c}
        x_{1} \\
        x_{2} \\
        \vdots \\
        x_{m}  \\
      \end{array}
    \right).\]
    Мы будем рассматривать векторы $\vc{c}$ и $\vc{b}$, а также
    матрица $\vc{A}$ как заданные, а $\vc{x}$ --- как вектор
    переменных.
    Напомним, что через $\vc{a_{j}}$ мы обозначаем $j$-й столбец
    матрицы $\vc{A}$, а через $\vc{a^{i}}$ --- ее $i$-ю строку. Мы
    предполагаем, что каждая из строк является ненулевой.

Рассмотрим задачу
\begin{equation}\label{LP-RAV}
\left\{
\begin{array}{rrrrllll}
c_1 x_1 + & c_2 x_2 +    & \ldots +& c_n x_n &\to & \max\\
a_{11} x_1 + & a_{12} x_2 + &\ldots +& a_{1n} x_n &=& b_1 \\
a_{21} x_1 + & a_{22} x_2 + &\ldots +& a_{2n} x_n &=& b_2\\
                       \ldots &&&\\
a_{m1} x_1 + & a_{m2} x_2 +& \ldots +& a_{mn} x_n &=& b_m\\
\end{array} \right.
\end{equation}
которую удобно записать в виде
\begin{equation}\label{LP-RAV}
\left\{
\begin{array}{rll}
\vc{c}\vc{x}  &\to & \max\\
\vc{a^{1}}\vc{x} &=& b_1 \\
\vc{a^{2}}\vc{x}  &=& b_2\\
                      & \ldots &\\
\vc{a^{m}}\vc{x}  &=& b_m\\
\end{array} \right.
\end{equation}
или совсем коротко:
\begin{equation}\label{LP-RAV-VEC}
\left\{
\begin{array}{rcl}
 \vc{c} \, \vc{x} & \to & \max  \\
 \vc{A} \vc{x} & = & \vc{b} \\
 \end{array} \right.
\end{equation}

Эта задача тоже несложна. В ней нет ограничений на неотрицательность
переменных и она \emph{не является} канонической задачей линейного
программирования. Ее анализ мы разобьем на два этапа, рассмотрев
сначала совсем простой случай, когда вектор $\vc{b}$ является
нулевым:
\begin{equation}\label{LP-RAV-VEC-NIL}
\left\{
\begin{array}{rcl}
 \vc{c} \, \vc{x} & \to &\max  \\
 \vc{A} \vc{x}& =& \vc{0} \\
 \end{array} \right.
\end{equation}
Сразу же заметим, что множество допустимых векторов этой задачи является подпространством
пространства $\mathbb{R}^{n}$, причем непустым (вектор $\vc{x}=\vc{0}$ является допустимым).
Что касается существования и устройства решений, то все зависит от того, найдется ли хотя бы
один такой вектор $\vc{\hat{x}}$, для которого выполняется равенство $\vc{A} \vc{\hat{x}} =
\vc{0}$ и, в то же время, $\vc{c} \,\vc{\hat{x}}\neq0$.

Если такого вектора не найдется, т.е. для любого допустимого плана
$\vc{x}$  выполняется равенство $\vc{c} \, \vc{x}=0$, то решение у
задачи (\ref{LP-RAV-VEC-NIL}) существует. Более того, в этом случае
решением является любой допустимый вектор, а значение задачи равно
нулю.

Если же найдется хотя бы один такой допустимый вектор
$\vc{\hat{x}}$, для которого $\vc{c} \,\vc{\hat{x}}\neq0$, то
решения у задачи (\ref{LP-RAV-VEC-NIL}) не существует, а ее
значением является $+\infty$. Действительно, в данной ситуации при
любом числе $\lambda$ вектор $\lambda\vc{\hat{x}}$ тоже является
допустимым, а значение целевой функции для этого вектора равно
$\lambda\vc{c} \,\vc{\hat{x}}$. Значит, подобрав соответствующее
$\lambda$, мы можем сделать эту величину сколь угодно большой.

Мы можем заключить, что решение у задачи (\ref{LP-RAV-VEC-NIL})
существует тогда, и только тогда, когда справедлива следующая
импликация
\begin{equation}
    \label{LP-RAV-VEC-NIL-usl0}
    \vc{A} \vc{x} = \vc{0}\ \Rightarrow\ \vc{c} \, \vc{x}=0,
\end{equation}
т.е. для любого такого вектора $\vc{x}$, что $\vc{A} \vc{x} =
\vc{0}$, справедливо равенство $\vc{c} \, \vc{x}=0$.
    Если читатель достаточно хорошо знаком с линейной алгеброй, то
    он, быть может, знает, что это свойство, в свою очередь,
    справедливо тогда, и только тогда, когда найдется такой вектор
    $\vc{u}=(u_{1},\ldots,u_{m})\in\R^{m}$, что
\begin{equation}
    \label{LP-RAV-VEC-NIL-fredgolm}
    \vc{c}=u_{1}\vc{a^{1}}+\ldots+u_{m}\vc{a^{m}} \ (=\vc{u}\vc{A})\,
\end{equation}
    или, что то же,
    \[c_{j}=\vc{u}\vc{a_{j}}, \ j=1,\ldots,n.\]
        Этот факт является одним из утверждений так называемой
    теоремы Фредгольма (см., например, ????).
    Тем самым справедливо следующая лемма, говорящая о том,
    что задача (\ref{LP-RAV-VEC-NIL}) имеет решение тогда и только
    тогда когда вектор $\vc{c}$, задающий целевую функцию,
    можно представить как линейную комбинацию векторов $\vc{a^{i}}$,
    $i=1,\ldots,m$, задающих ограничения.
\begin{lem}
    \label{fredgolm}
    Задача (\ref{LP-RAV-VEC-NIL}) имеет решение тогда, и только
    тогда, когда найдется вектор
    $\vc{u}=(u_{1},\ldots,u_{m})$, для которого выполняется
    (\ref{LP-RAV-VEC-NIL-fredgolm}). При этом в случае, когда решение
    существует, любой допустимый вектор является решением, а
    значением задачи является ноль.
\end{lem}

Несмотря на то, что эта лемма не нуждается в доказательстве, ибо мы
уже сделали ссылку на теорему Фредгольма, мы все же приведем ее
доказательство, поскольку оно поможет читателю в дальнейшем понять
существо метода множителей Лагранжа для нелинейных задач на условный
максимум или минимум (который, строго говоря, методом не является, а
представляет собой необходимые условия экстремума).

\textbf{Доказательство} леммы \ref{fredgolm}. Если вектор
$\vc{u}=(u_{1},\ldots,u_{m})$, для которого выполняется
(\ref{LP-RAV-VEC-NIL-fredgolm}), существует, то выполняется свойство
\ref{LP-RAV-VEC-NIL-usl0}. Это означает что для любого допустимого
плана значение целевой функции равно нулю. Поэтому любой допустимый
план и является решением.

Теперь предположим, что вектора $\vc{u}=(u_{1},\ldots,u_{m})$, для
которого выполняется (\ref{LP-RAV-VEC-NIL-fredgolm}), не существует.
Покажем, что в этом случае обязательно найдется допустимый план
 $\vc{\hat{x}}$, для которого $\vc{c} \,\vc{\hat{x}}\neq0$, и,
 значит, решения у задачи (\ref{LP-RAV-VEC-NIL}) не существует.

Пусть $r$ --- это ранг матрицы $\vc{A}$. Не умаляя общности будем
считать, ее первые $r$ строк линейно независимы. Поскольку вектор
$\vc{c}$ нельзя представить линейной комбинацией этих $r$ строк,
ранг матрицы
\[\left(
\begin{array}{ccc}
  c_{1} & \ldots & c_{n} \\
  a_{11} & \ldots & a_{1n} \\
   \ldots& \ldots &\ldots  \\
  a_{r1} & \ldots & a_{rn}
\end{array}
\right)\] равен $r+1$. Будем считать не умаляя общности, что линейно
независимыми являются первые $r+1$ столбцов этой матрицы.
Следовательно, квадратная матрица
\[\left(
\begin{array}{ccc}
  c_{1} & \ldots & c_{r+1} \\
  a_{11} & \ldots & a_{1r+1} \\
   \ldots& \ldots &\ldots  \\
  a_{r1} & \ldots & a_{rr+1}
\end{array}
\right)\]
    является обратимой. Отсюда вытекает, что хотя бы при одном ненулевом
     (а точнее, при любом) $\varepsilon$  существует
    решение у следующей системы уравнений:
\[\begin{array}{ccccccc}
    c_{1}x_{1}& + & \ldots & + & c_{r+1}x_{r+1} & = & \varepsilon \\
    a_{11}x_{1} & + & \ldots & + & a_{1r+1}x_{r+1} & = & 0 \\
     &  \ldots&\ldots  & \ldots &  &  &  \\
    a_{r1}x_{1} & + & \ldots & + & a_{1r+1}x_{r+1} & = & 0
  \end{array}\]
    Если вектор $(x_{1}^{*},\ldots,x_{r+1}^{*})$ является этим
    решением, то, очевидно, вектор
    \[\vc{x^{*}}=(x_{1}^{*},\ldots,x_{r+1}^{*},0,\ldots,0)\]
    представляет собой решение системы
    \[\left\{
    \begin{array}{ccc}
      \vc{c}\,\vc{x} & = & \varepsilon \\
      \vc{A}\vc{x} & = & \vc{0}
    \end{array}\right. \]
    Но отсюда вытекает, что у задачи (\ref{LP-RAV-VEC-NIL}) решения
    нет, а ее значением является $+\infty$. $\Box$

    Теперь вернемся к рассмотрению задачи (\ref{LP-RAV-VEC}) в общем
    случае, когда вектор $\vc{b}$ не обязательно равен нулю. В
    первую очередь сразу же заметим, что, вообще говоря у этой
    задачи допустимое множество может быть пустым, т.е. система
    уравнений
\begin{equation} \label{neodnorodnaya systema}
    \vc{A}\vc{x}=\vc{b}
\end{equation}
    может не иметь решений. Но мы будем предполагать, что эта
    система разрешима, что можно гарантировать при любом $\vc{b}$ в случае,
    когда строки матрицы $\vc{A}$ являются линейно независимыми, т.е. когда ее ранг
    равен числу ее строк.

    Из курса линейной алгебры читатель помнит, что множество решений
    $\mathbb{S}_{0}$ однородной системы уравнений
    \[\vc{A}\vc{x}=\vc{0}\]
    представляет собой линейное подпространство  пространства
    ${\R}^{n}$, а множество решений $\mathbb{S}_{b}$
    неоднородной системы (\ref{neodnorodnaya systema})
    является аффинным множеством, представляющим собой сдвиг
    подпространства $\mathbb{S}_{0}$ на произвольный вектор
    $\vc{\hat{x}}$ из $\mathbb{S}_{b}$:
    \[\mathbb{S}_{b}=\{\vc{x}=\vc{y}+\vc{\hat{x}} \mid \vc{y}\in\mathbb{S}_{0}\}\]
    (см. Рис. ?????). Отсюда вытекает, что, зафиксировав какой-либо
    вектор $\vc{\hat{x}}\in\mathbb{S}_{b}$, мы можем переписать задачу
    (\ref{LP-RAV-VEC}) в следующем виде:
\[
    \left\{
    \begin{array}{rcl}
     \vc{c} \vc{y}+\vc{c} \vc{\hat{x}} & \to & \max  \\
    \vc{A} \vc{y}+\vc{A} \vc{\hat{x}} & = & \vc{b} \\
     \end{array} \right.
\]
    где вектором переменных является $\vc{y}$.

    А поскольку $\vc{A}\vc{\hat{x}}=\vc{b}$, эта задача сводится к задаче
\begin{equation} \label{LP-RAV-VEC-NIL-1}
    \left\{
    \begin{array}{rcl}
     \vc{c} \vc{y} & \to & \max  \\
    \vc{A} \vc{y} & = & \vc{0} \\
     \end{array} \right.
\end{equation}
    которую мы уже проанализировали. А именно, вектор $\vc{y^{*}}$
    является решением задачи (\ref{LP-RAV-VEC-NIL-1}) тогда и только
    тогда, когда вектор $\vc{x^{*}}=\vc{y^{*}}+\vc{\hat{x}}$
    является решением задачи (\ref{LP-RAV-VEC}).




    Отсюда следует, что решение задачи (\ref{LP-RAV-VEC}) существует
    тогда, и только тогда, когда существует решение задачи
    (\ref{LP-RAV-VEC-NIL-1}), т.е. тогда и только тогда, когда найдется вектор
    $\vc{u}=(u_{1},\ldots,u_{m})$, для которого выполняется
    (\ref{LP-RAV-VEC-NIL-fredgolm}).
     При этом, если решение задачи (\ref{LP-RAV-VEC})
    существует, то
    решением этой задачи является любой допустимый вектор, а ее
    значение, как легко легко проверит читатель, равно
    $\vc{u}\vc{b}$. Итак, справедливо следующее предложение.

\begin{prop}
    \label{fredgolm-1}
    Если допустимое множество задачи
\begin{equation*}
    \left\{
    \begin{array}{rcl}
     \vc{c} \, \vc{x} & \to & \max  \\
    \vc{A} \vc{x}& =& \vc{b} \\
     \end{array} \right.
\end{equation*}
    непусто, то
    она имеет решение тогда, и только тогда, когда найдется вектор
    $\vc{u}=(u_{1},\ldots,u_{m})$, для которого
\begin{equation*} \label{razlojenie vektora c}
    \vc{c}=u_{1}\vc{a^{1}}+\ldots+u_{m}\vc{a^{m}}=\vc{u}\vc{A}.
\end{equation*}
    При этом в случае, когда решение задачи
    существует, любой допустимый вектор является решением, а ее
    значение равно $\vc{u}\vc{b}$.
\end{prop}


    Вообще говоря, вектор $\vc{u}=(u_{1},\ldots,u_{m})$, о котором идет речь в
    сформулированном предложении, может быть не единственным.
    Однако, если строки матрицы $\vc{A}$ линейно независимы, то этот
    вектор задается единственным образом. Кроме того, в этом случае, как мы уже
    отмечали, система уравнений $\vc{A}\vc{x}=\vc{b}$ разрешима при
    любом $\vc{b}$.

    В линейном (и нелинейном) программировании важную роль
    играет так называемый анализ чувствительности, т.е. анализ
    зависимости оптимального плана и значения задачи, т.е. значения целевой функции
    в точке оптимума, от параметров задачи. В первую очередь
    обычно анализируют зависимость значения задачи от правой части
    ограничений. Для задачи (\ref{LP-RAV-VEC}) этот анализ является
    является тривиальным, но очень поучительным.

    Предположим, что строки матрицы $\vc{A}$ линейно независимы,  и
    при некотором $\vc{u}=(u_{1},\ldots,u_{m})$ выполняется
    (\ref{razlojenie vektora c}). В этом случае задача
    (\ref{LP-RAV-VEC}) имеет решение при любом векторе $\vc{b}$.
    Рассмотрим это значения как функцию, аргументом которой является
    именно $\vc{b}$. Обозначим эту функцию как $\Phi(\vc{b})$.


    В рамках сделанных нами предположений эта функция устроена очень
    просто. Дело в том, что поскольку вектор
    $\vc{u}=(u_{1},\ldots,u_{m})$, который фигурирует в
    (\ref{razlojenie vektora c}), задается однозначным образом и не
    зависит от $\vc{b}$, она является просто линейной:
    \[\Phi(\vc{b})=\vc{u}\vc{b}.\]

        На всякий случай напомним, как в данном случае следует понимать
    выражение
\begin{equation}\label{lin-fun}
    \vc{u}\vc{b}=u_{1}b_{1}+\ldots+u_{m}b_{m}.
\end{equation}
    Если бы как заданный мы рассматривали вектор
    $\vc{b}=(b_{1},\ldots,b_{m})$, то эта запись задавала бы линейную функцию
    вектора переменных $\vc{u}=(u_{1},\ldots,u_{m})$. В рассматриваемой нами ситуации
    заданным является вектор $\vc{u}=(u_{1},\ldots,u_{m})$, а запись
    (\ref{lin-fun}) задает линейную функцию вектора переменных
    $\vc{b}=(b_{1},\ldots,b_{m})$.

    Как и всякая линейная функция, функция $\Phi(\vc{b})$ является
    дифференцируемой. Ее частные производные вычисляются очень
    просто:
\begin{equation}\label{proizvodnaya-lin-fun}
    \frac{\partial\Phi(\vc{b})}{\partial b_{i}}=u_{i},\ i=1,...,m.
\end{equation}

    Мы очень рекомендуем читателю приглядеться к этим совсем простым
    равенствам. В дальнейшем такого типа соотношения нам в различных
    контекстах еще встретятся.


\subsection{Условия оптимальности для основной задачи линейного
программирования}

Рассмотренная нами задача (\ref{LP-RAV-VEC}) является не совсем, так
сказать, полноценной задачей линейного программирования, поскольку к
ней нельзя свести задачу линейного программирования в канонической
или стандартной форме. В этом смысле основная задача линейного
программирования (\ref{OSZLP}) является вполне полноценной, ибо, как
мы знаем, любую задачу линейного программирования можно переписать в
форме основной задачи. Нам удобно записывать основную задачу
(\ref{OSZLP}) в виде
\begin{equation}\label{PSZLP}
\left\{
\begin{array}{rll}
\vc{c}\vc{x} &\to & \max\\
\vc{a^{1}}\vc{x} &\leqslant& b_1 \\
\vc{a^{2}}\vc{x} &\leqslant& b_2\\
                      & \ldots &\\
\vc{a^{m}}\vc{x}  &\leqslant& b_m\\
\end{array} \right.
\end{equation}
и совсем коротко в матрично-векторном виде
\begin{equation}\label{PSZLPV}
\left\{
\begin{array}{rcl}
 \vc{c} \, \vc{x} & \to & \max,  \\
 \vc{A} \vc{x} &\leqq & \vc{b}. \\
 \end{array} \right.
\end{equation}

Основная задача отличается от задачи (\ref{LP-RAV-VEC}) только тем,
что в ней ограничения записаны в виде неравенств, но является
обобщением последней, поскольку, как уже отмечалось, любое равенство
можно просто записать систему из двух два неравенств.

Чтобы сделать изложение прозрачным, как и в случае задачи
(\ref{LP-RAV-VEC}), мы проанализируем задачу (\ref{PSZLPV}) в два
этапа. Сначала мы рассмотрим тот частный случай, когда
$\vc{b}=\vc{0}$:
\begin{equation}\label{PSZLPV-nil}
\left\{
\begin{array}{rcl}
 \vc{c} \, \vc{x} & \to& \max,  \\
 \vc{A} \vc{x} &\leqq &\vc{0}. \\
 \end{array} \right.
\end{equation}
Допустимое множество этой задачи непусто, потому что вектор
$\vc{x}=\vc{0}$ является допустимым. При этом если хотя бы одно
решение задачи существует, то одним из ее решений обязательно
является $\vc{x}=\vc{0}$, а значение задачи рано нулю.
Действительно, с одной стороны $\vc{c}\vc{0}=0$. С другой стороны,
допустимое множество нашей задачи является конусом (??????), т.е.
если некоторый вектор $\vc{x}$ является допустимым, то при любом
$\lambda\geqslant0$ допустимым является и вектор $\lambda\vc{x}$.
Тем самым, если хоть для какого-то вектора $\hat{\vc{x}}$
выполняется неравенство $\vc{c}\hat{\vc{x}}>0$, рассматриваемая
задача не имеет решения, поскольку, умножая $\hat{\vc{x}}$ на
достаточно большое $\lambda$, мы получим допустимый вектор, значение
целевой функции на котором сколь угодно велико.

    Для дальнейшего рассуждения нам понадобится так называемая лемма
    Фаркаша, доказательство  которой в несколько иной, но эквивалентной формулировке
    будет приведено в следующей главе (??????, см. лемму \ref{lemma-F3}).

\begin{lem}
    Предположим, что нам даны $m \times n$ матрица $\vc{A}$ и $n$-мерный
    вектор-строка $\vc{c}$. В этом случае импликация
    \[\vc{A} \vc{x} \leqq \vc{0} \Rightarrow \vc{c}\vc{x}\leqslant0\]
    справедлива тогда, и только тогда, когда когда существует вектор
    $\vc{u}=(u_{1},\ldots,u_{m})\geqq\vc{0}$, для которого
    \[\vc{c}=u_{1}\vc{a^{1}}+\ldots+u_{m}\vc{a^{m}}=\vc{u}\vc{A}.\]
\end{lem}




    Из леммы Фаркаша (эта ссылка является ключевым моментом всего нашего рассуждения)
    вытекает следующая лемма
\begin{lem}
    \label{fredgolm-0-neravenstvo}
    Задача (\ref{PSZLPV-nil}) имеет решение тогда, и только тогда, когда найдется вектор
    $\vc{u}=(u_{1},\ldots,u_{m})\geqq\vc{0}$, для которого
    \[\vc{c}=u_{1}\vc{a^{1}}+\ldots+u_{m}\vc{a^{m}}=\vc{u}\vc{A}.\]
       При этом в случае, когда решение задачи
    существует, ее значение равно нулю, а хотя бы одним решений ---
    вектор $\vc{x^{*}}=\vc{0}$.
\end{lem}

    Эта лемма \remrk{РИС, РИС} является обобщением леммы \ref{fredgolm}.
    Ее отличие от последней состоит в том, что в задаче, которая
    рассматривается здесь, ограничения
    являются неравенствами, а вектор $\vc{u}$ является
    неотрицательным. Чтобы читатель достаточно хорошо освоился с
    ситуацией, мы предлагаем ему следующее упражнение.
\begin{exer}
    Рассмотрим задачу
\begin{equation}
\left\{\begin{array}{rcl}
          \vc{c} \, \vc{x} & \to & \max, \\
         \vc{A_{1}}\vc{x} & \leqq &\vc{0_{1}},\\
         \vc{A_{2}}\vc{x} & = &\vc{0_{2}},
       \end{array}
\right.
\end{equation}
где $\vc{A_{1}}$, $\vc{A_{2}}$ --- матрицы, имеющие по $n$ столбцов,
$\vc{0_{1}}$ и $\vc{0_{2}}$ --- нулевые векторы соответствующих
размерностей, а $\vc{x}$
--- это $n$-мерный вектор-столбец переменных. Докажите, что эта
задача имеет решение тогда, и только тогда, когда найдутся векторы
соответствующих размерностей
    $\vc{u_{1}}\geqq\vc{0}$ и $\vc{u_{2}}$, для которых выполняется
    соотношение
    \[\vc{c}=\vc{u_{1}}\vc{A_{1}}+\vc{u_{2}}\vc{A_{2}}.\]
    При этом в случае, когда решение этой задачи
    существует, ее значение равно нулю.
\end{exer}

    Теперь рассмотрим задачу (\ref{PSZLP}) без
    предположения о том, что $\vc{b}=\vc{0}$, сразу же заметив, что
    допустимое множество этой задачи может, вообще говоря, быть
    пустым. Но далее в этом пункте мы будем предполагать,
    что допустимое множество непусто.
    В этой ситуации для любого допустимого вектора $\vc{x}$
    некоторые из ограничений $j=1,\ldots,m$ выполняются как
    равенства (их мы будем называть \emph{активными}), а
    некоторые --- как строгие неравенства (такие ограничения называют
    \emph{пассивными}). Множество активных
    ограничений для вектора $\vc{x}$ мы будем обозначать как
    $I(\vc{x})$. А именно, положим
    \[I(\vc{x})=\{i=1,\ldots,m\ \mid \vc{a^{i}}\vc{x}=b_{i}\}.\]


\begin{exer}
    Докажите, что если
    $\vc{c}\neq\vc{0}$ и $\vc{x^{*}}$ --- это решение задачи (\ref{PSZLP}),
    то $I(\vc{x^{*}})\neq\emptyset$.
\end{exer}

     Все дальнейшие рассуждения мы будем проводить в предположении, что
    $\vc{c}\neq\vc{0}$, хотя, как легко проверит читатель, все формулируемые
    ниже предложения справедливы и для случая, когда
    $\vc{c}=\vc{0}$.

    Нам понадобится следующая лемма. \remrk{РИС}
\begin{lem}
  \label{vspom-lemma}
    Допустимый вектор $\vc{x}^{*}$  задачи
    (\ref{PSZLP}) является ее решением тогда и только тогда он является
    решением задачи
    \begin{equation}\label{PSZLP-dop}
\begin{array}{c}
    \vc{c}\vc{x}  \to  \max\\
    \vc{a^{j}}\vc{x}  \leqslant b_j, \ j\in I(\vc{x^{*}}). \\
\end{array}
\end{equation}
\end{lem}
    \textbf{Доказательство.}

    \emph{Достаточность.} Пусть $\vc{x}^{*}$ представляет собой
    решение задачи (\ref{PSZLP-dop}). Поскольку допустимое множество
    этой задачи включает в себя допустимое множество задачи
    (\ref{PSZLP}), вектор $\vc{x}^{*}$ является и решение последней.

    \emph{Необходимость.} Предположим, что $\vc{x}^{*}$ является
    решением задачи (\ref{PSZLP}), не являясь решением задачи (\ref{PSZLP-dop}).
    В этом случае существует такой вектор
    $\vc{x^{**}}$, для которого одновременно выполняется строгое неравенство
    $\vc{c}\vc{x^{**}}>\vc{c}\vc{x^{*}}$ и неравенства
\[
    \vc{a^{j}}\vc{x^{**}}\leqslant b_j, \ j\in I(\vc{x^{*}}).
\]
    При этом возможно, что для каких-то $j\notin I(\vc{x^{*}})$
    неравенство $\vc{a^{j}}\vc{x^{**}}\leqslant b_j$ не выполняется,
    а выполняется неравенство $\vc{a^{j}}\vc{x^{**}}>b_j$.

    Очевидно, что для любого числа $\lambda\in(0,1)$ справедливы как
    соотношения
\[
    \vc{c}((1-\lambda)\vc{x^{**}}+\lambda\vc{x^{*}})=
    (1-\lambda)\vc{c}\vc{x^{**}}+\lambda\vc{c}\vc{x^{*}}>\vc{c}\vc{x^{*}},
\]
    так и соотношения
\[
    \vc{a^{j}}((1-\lambda)\vc{x^{**}}+\lambda\vc{x^{*}})=
    (1-\lambda)\vc{a^{j}}\vc{x^{**}}+\lambda\vc{a^{j}}\vc{x^{*}}\leqslant b_j, \ j\in I(\vc{x^{*}}).
\]
    При этом, поскольку
\[
    \vc{a^{j}}\vc{x^{*}}<b_j, \ j\notin I(\vc{x^{*}}),
\]
    найдется такое $\tilde{\lambda}\in(0,1)$, что
\[
    \vc{a^{j}}((1-\tilde{\lambda})\vc{x^{**}}+\tilde{\lambda}\vc{x^{*}})=
    (1-\tilde{\lambda})\vc{a^{j}}\vc{x^{**}}+\tilde{\lambda}\vc{a^{j}}\vc{x^{*}}
    \leqslant b_j, \ j\notin I(\vc{x^{*}}).
\]

\begin{exer}
    Чему равно $\tilde{\lambda}$?
\end{exer}

    Следовательно, для вектора
    $\vc{x^{***}}=(1-\tilde{\lambda})\vc{x^{**}}+\tilde{\lambda}\vc{x^{*}}$
    мы имеем:
\[
    \vc{c}\vc{x^{***}}>\vc{c}\vc{x^{*}},
\]
\[
    \vc{a^{j}}\vc{x^{***}}\leqslant b_j, \ j=1\ldots,m.
\]
    Но эти неравенства говорят нам о том, что вектор $\vc{x^{*}}$ не
    является решением задачи (\ref{PSZLP}).

    Полученное противоречие
    доказывает, что этот $\vc{x^{*}}$ является решением задачи
    (\ref{PSZLP-dop}). $\Box$

   Теперь мы готовы к тому, чтобы доказать следующую важную теорему,
   содержащую необходимые и достаточные условия оптимальности.


\begin{teo}
    \label{PSZLPV-usl-opt}
    Допустимый план $\vc{x}^{*}$ является решением задачи
    (\ref{PSZLP}) тогда и только тогда, когда найдется такой вектор
    $\vc{u}^{*}=(u_{1}^{*},\ldots,u_{m}^{*})\geqq0$, что выполняются
    следующие соотношения:
\begin{itemize}
    \item [1)\ ]
    $\vc{c}=u_{1}^{*}\vc{a^{1}}+\ldots+u_{m}^{*}\vc{a^{m}}=\vc{u}^{*}\vc{A}$

    $($или, что то же,
    $c_{j}=\vc{u^{*}}\vc{a_{j}}, \ j=1,\ldots,n$$);$
    \item [2)\ ]
    $u_{i}^{*}(b_{i}-\vc{a^{i}}\vc{x}^{*})=0, \ i=1,\ldots,m.$
\end{itemize}
\end{teo}

    Для тех, кто с теорией линейного программирования еще не
    встречался, укажем, что
    числа $u_{1}^{*},\ldots,u_{m}^{*}$, о которых здесь идет речь, называются
    двойственными оценками ограничений рассматриваемой задачи (эти
    числа также можно назвать множителями Лагранжа, но в линейном
    программировании так их называют нечасто), и
    сделаем несколько замечаний по поводу набора соотношений 2), которые
    называют \emph{условиями дополняющей нежесткости}. Подчеркнем, что речь в
    данном предложении идет о допустимом плане $\vc{x}^{*}$,
    т.е. таком, что
    \[b_{i}-\vc{a^{i}}\vc{x}^{*}\geqslant0, \ i=1,\ldots,m.\]
    Поэтому, с учетом того, что вектор $\vc{u}^{*}$ является неотрицательным,
    набор соотношений 2) говорит, что если $i\notin I(\vc{x}^{*})$, то
    двойственная оценка ограничения $j$ должна равняться нулю. Тем самым
    это соотношения можно переписать следующим образом:
    \[\vc{a^{i}}\vc{x}^{*}<b_{i}\Rightarrow  u_{i}^{*}=0.\]
    А поскольку вектор $\vc{x}^{*}$ является допустимым планом, т.е.
    удовлетворяет неравенству $\vc{b}-\vc{A}\vc{x}^{*}\geqq\vc{0}$,
    то набор соотношений 2) можно переписать в матрично-векторном виде следующим образом:
    \[\vc{u}^{*}(\vc{b}-\vc{A}\vc{x}^{*})=0.\]






    \textbf{Доказательство.} Пусть вектор $\vc{x}^{*}$
    является решением задачи (\ref{PSZLP}). Будем считать, что
    что для этого решения множество активных ограничений состоит из
    $r$ ограничений и, не умаляя общности, что
    \[I(\vc{x^{*}})=\{1,\ldots,r\}.\]

    По лемме \ref{vspom-lemma} вектор $\vc{x}^{*}$ представляет
    собой решение задачи
\begin{equation}\label{PSZLP-dop-dop-0}
    \left\{
    \begin{array}{cll}
    \vc{c}\vc{x}  &\to & \max\\
    \vc{a^{1}}\vc{x}  &\leqslant& b_1 \\
    \vc{a^{2}}\vc{x}&\leqslant& b_2\\
                      & \ldots &\\
    \vc{a^{r}}\vc{x} &\leqslant& b_r\\
\end{array} \right.
\end{equation}



а следовательно и задачи \remrk{РИС}
\begin{equation}\label{PSZLP-dop-dop}
\left\{
\begin{array}{cll}
\vc{c}\vc{x}-\vc{c}\vc{x^{*}}  &\to & \max\\
\vc{a^{1}}\vc{x}-\vc{a^{1}}\vc{x^{*}}  &\leqslant& b_1 -\vc{a^{1}}\vc{x^{*}}\\
\vc{a^{2}}\vc{x}-\vc{a^{2}}\vc{x^{*}}  &\leqslant& b_2-\vc{a^{1}}\vc{x^{*}}  \\
                      & \ldots &\\
\vc{a^{r}}\vc{x}-\vc{a^{r}}\vc{x^{*}}  &\leqslant& b_r-\vc{a^{r}}\vc{x^{*}} \\
\end{array} \right.
\end{equation}

\begin{exer}
    Докажите, что вектор $\vc{x}^{*}$ является
    решением задачи (\ref{PSZLP-dop-dop}).
\end{exer}


Положив
\[\vc{y}=\vc{x}-\vc{x^{*}},\]
перепишем последнюю задачу в следующем виде:
\begin{equation}\label{PSZLP-dop-0}
\left\{
\begin{array}{rll}
\vc{c}\vc{y} &\to & \max\\
\vc{a^{1}}\vc{y} &\leqslant& 0 \\
\vc{a^{2}}\vc{y} &\leqslant& 0\\
                      & \ldots &\\
\vc{a^{r}}\vc{y}  &\leqslant& 0\\
\end{array} \right.
\end{equation}
Решением этой задачи является вектор $\vc{y^{*}}=\vc{0}$. В силу
леммы \ref{fredgolm-0-neravenstvo} существует вектор
$(u_{1}^{*},\ldots,u_{r}^{*})\geqq0$, такой что
\[\vc{c}=u_{1}^{*}\vc{a^{1}}+\ldots+u_{r}^{*}\vc{a^{r}}.\]
Отсюда следует, что вектор
$\vc{u^{*}}=(u_{1}^{*},\ldots,u_{r}^{*},0,\ldots,0)\geqq0$
удовлетворяет соотношениям 1)-2), фигурирующим в формулировке
предложения.

Теперь предположим, что вектор $\vc{x}^{*}$  является допустимым
планом задачи (\ref{PSZLP}) и существует такой вектор
    $\vc{u}^{*}=(u_{1}^{*},\ldots,u_{m}^{*})\geqq0$, что выполняются
соотношениям 1)-2), фигурирующим в формулировке предложения. Будем
считать, что множество активных ограничений для данного допустимого
плана состоит из $r$ ограничений и что
    \[I(\vc{x^{*}})=\{1,\ldots,r\}.\]
По лемме \ref{fredgolm-0-neravenstvo} задача (\ref{PSZLP-dop-0})
имеет хотя бы одно решение. Одним из ее решений является вектор
$\vc{y^{*}}=\vc{0}$. Поскольку
\[\vc{a^{i}}\vc{x^{*}}=b_{i},\ i=1,\ldots r,\]
отсюда следует, что вектор $\vc{x^{*}}$ является решением задачи
(\ref{PSZLP-dop-dop}). По лемме \ref{vspom-lemma} этот же вектор
является и решением задачи (\ref{PSZLP-dop}).

    Теорема доказана.
$\Box$


Здесь следует сделать следующее важное замечание. Если мы знаем,
каков вектор $\vc{u}^{*}=(u_{1}^{*},\ldots,u_{m}^{*})$, фигурирующий
в доказанном предложении, то нам, по-существу, известно значение
$\vc{c}\vc{x^{*}}$ задачи (\ref{PSZLP-dop}). Это значение, очевидно,
совпадает с
\[\vc{u}^{*}\vc{b}=u_{1}^{*}b_{1}+\ldots+u_{m}^{*}b_{m}.\]


\subsection{Условия оптимальности для задачи линейного
программирования в стандартной и канонической формах}

Теперь перейдем к формулировке и доказательству условий
оптимальности для задач линейного программирования в стандартной и
канонической формах. Сразу же подчеркнем, что соответствующие
утверждения выводятся из теоремы \ref{PSZLPV-usl-opt} с помощью
очень несложных преобразований, ибо как мы помним, любая задача
линейного программирования легко преобразуется к основной.

    Рассмотрим стандартную задачу линейного программирования, записав ее в следующем виде:
\begin{equation} \label{SZLP-qq}
\left\{
\begin{array}{rll}
     \vc{c} \, \vc{x} &\to & \max\\
     \vc{a^{1}}\vc{x} &\leqslant& b_1 \\
     \vc{a^{2}}\vc{x} &\leqslant& b_2\\
                      & \ldots &\\
     \vc{a^{m}}\vc{x} &\leqslant& b_m\\
     \vc{x}&\geqq&0  \\
\end{array} \right.
\end{equation}

    Следующее предложение представляет собой незначительное обобщение предложения
    \ref{LP1-usloviya-optimalnosti}, которое мы сформулировали, основываясь на содержательных
    экономических соображениях.

\begin{prop}
    \label{SZLP-1-usl-opt}
    Допустимый план $\vc{x^{*}}=(x_{1}^{*},\ldots,x_{n}^{*})^{T}$ задачи
    (\ref{SZLP-qq}) является ее решением тогда и только тогда, когда
    найдется вектор двойственных  оценок
    $\vc{u}^{*}=(u_{1}^{*},\ldots,u_{m}^{*})\geqq0$,  такой что
    справедливы следующие соотношения
\begin{itemize}
\item [1)\ ] \
    $c_{j}\leqslant\vc{u}^{*}\vc{a_{j}}, \ j=1,\ldots,n;$
\item [2)\ ]
    $(c_{j}-\vc{u}^{*}\vc{a_{j}})x_{j}^{*}=0, \ j=1,\ldots,n;$
\item [3)\ ]
    $u_{i}^{*}(b_{i}-\vc{a^{i}}\vc{x}^{*})=0, \ i=1,\ldots,m.$
\end{itemize}
\end{prop}

    Прокомментируем это предложение, сравнив его с предложением
    \ref{PSZLPV-usl-opt}. В формулировке этого предложения
    соотношения 3) (условия дополняющей нежесткости) по
    своей формулировке в точности совпадают с соотношениями 2) в
    предложении \ref{PSZLPV-usl-opt}, а соотношения 1)-2)
    соответсвуют соотношениям 1) в предложении \ref{PSZLPV-usl-opt}.
    Отличие в формулировках связано с тем, что если в
    основной задаче нет отдельных ограничений на неотрицательность
    переменных, то здесь они выписаны в явном виде. Этим и вызвано
    то, что если в случае основной задачи для всех $j=1,\ldots,n$
    величины $c_{j}$  и $\vc{u}^{*}\vc{a_{j}}$ совпадают, то здесь
    первая из этих двух величин может быть строго меньше второй. При этом, правда,
    соотношения 2), тоже называемые условиями дополняющей
    нежесткости, говорят, что строгое неравенство возможно только при тех
    $j$, для которых $x_{j}^{*}=0$, ибо, очевидно, эти соотношения можно
    переписать следующим образом:
\[
    \vc{u}^{*}\vc{a_{j}}>c_{j}\Rightarrow x_{j}^{*}=0.
\]
    Заметим также, что соотношения 1) можно записать в данном случае в
    виде
    \[\vc{c}\leqq u_{1}^{*}\vc{a^{1}}+\ldots+u_{m}^{*}\vc{a^{m}}=\vc{u}^{*}\vc{A},\]
    а соотношения 2) и 3) в виде
    \[\vc{c}\vc{x}^{*}=\vc{u}^{*}\vc{A}\vc{x}^{*}=\vc{u}^{*}\vc{b}.\]

    \textbf{Доказательство} предложения \ref{SZLP-1-usl-opt}.
     Перепишем задачу (\ref{SZLP-1}) в
    форме (\ref{SZLPSV}), т.е. форме основной задачи, и применим к
    ней предложение \ref{PSZLPV-usl-opt}. Оно говорит о том,
    допустимый план $\vc{x}^{*}=(x_{1}^{*},\ldots,x_{n}^{*})^{T}$ задачи
    (\ref{SZLP-1}) является ее решением тогда и только тогда, когда найдутся такие
    векторы $\vc{u}^{*}=(u_{1}^{*},\ldots,u_{m}^{*})\geqq0$
    и $\vc{v}^{*}=(v_{1}^{*},\ldots,v_{n}^{*})\geqq0$, что выполняются
    следующие соотношения:
\begin{itemize}
    \item [1')\ ]
    $c_{j}=\vc{u}^{*}\vc{a_{j}}-v_{j}, \ j=1,\ldots,n,$
    \item [2')\ ]
    $v^{*}_{j}x_{j}^{*}=0, \ j=1,\ldots,n;$
    \item [3')\ ]
    $u_{i}^{*}(b_{i}-\vc{a^{i}}\vc{x}^{*})=0, \ i=1,\ldots,m.$
\end{itemize}

    В этой формулировке можно опустить упоминание вектора
    $\vc{v}^{*}=(v_{1}^{*},\ldots,v_{n}^{*})\geqq0$. Для этого
    только  заметить, что, в силу его неотрицательности,
    соотношения 1') эквивалентны соотношениям 1) в формулировке предложения,
    соотношения 2') --- соотношениям 2), а соотношения 3') --- соотношениям 3).
    Но именно этого замечания нам вполне достаточно для
    доказательства предложения. $\Box$

    Теперь рассмотрим каноническую задачу, записав ее следующим образом:
\begin{equation}\label{KZLP-1}
 \left\{
\begin{array}{cll}
     \vc{c} \, \vc{x} &\to & \max\\
     \vc{a^{1}}\vc{x} &=& b_1 \\
     \vc{a^{2}}\vc{x} &=& b_2\\
                      & \ldots &\\
     \vc{a^{m}}\vc{x}&=& b_m\\
     \vc{x} & \geqq & \vc{0}\\
\end{array} \right.
\end{equation}

\begin{prop}
    \label{KZLP-1-usl-opt}
    Допустимый план $\vc{x}^{*}=(x_{1}^{*},\ldots,x_{n}^{*})^{T}$ задачи
     (\ref{KZLP-1}) является ее решением
     тогда и только тогда, когда найдется такой вектор двойственных
     оценок
    $\vc{u}^{*}=(u_{1}^{*},\ldots,u_{m}^{*})$, что выполняются
    следующие соотношения:
\begin{itemize}
    \item [1)\ ] \
    $c_{j}\leq\vc{u}^{*}\vc{a^{j}}, \ j=1,\ldots,n;$
    \item [2)\ ]
    $(c_{j}-\vc{u}^{*}\vc{a^{j}})x_{j}^{*}=0, \ j=1,\ldots,n.$
\end{itemize}
\end{prop}

    В этом предложении соотношения 1) можно переписать в
    матрично-векторном виде
    \[\vc{c}\leqq u_{1}^{*}\vc{a^{1}}+\ldots+u_{m}^{*}\vc{a^{m}} =\vc{u}^{*}\vc{A},\]
    а соотношения 2) --- в виде
    \[(\vc{c}-\vc{u}^{*}\vc{A})\vc{x}^{*}=0\]
    или  в виде
    \[\vc{u}^{*}\vc{a_{j}}>c_{j}\Rightarrow x_{j}^{*}=0.\]

        \textbf{Доказательство.} Напомним, что рассматриваемую каноническую
     задачу можно переписать в виде стандартной задачи
     (\ref{KZLP-osn}), применив к которой предложение
     \ref{SZLP-1-usl-opt}, заключаем, что допустимый план
     $\vc{x}^{*}$ задачи
    (\ref{KZLP-1}) является  ее решением
    тогда и только тогда, когда найдутся такие векторы
    $\vc{u'}=(u'_{1},\ldots,u'_{m})$ и $\vc{u''}=(u''_{1},\ldots,u''_{m})$,
    что выполняются следующие соотношения:
\begin{itemize}
    \item [1')\ ] \
    $c_{j}\leq\vc{u'}\vc{a^{j}}-\vc{u''}\vc{a^{j}}=(\vc{u'}-\vc{u''})\vc{a^{j}}, \
    j=1,\ldots,n;$
    \item [2')\ ]
    $[c_{j}-(\vc{u'}\vc{a^{j}}-\vc{u''}\vc{a^{j}})]x_{1}^{*}=0;$

    \item [3')\ ]
    $(b_{i}-\vc{a^{i}}\vc{x^{*}})u'_{i}=0,  \ i=1,\ldots,n;$
    \item [3'')\ ]
    $(b_{i}-\vc{a^{i}}\vc{x^{*}})u''_{i}=0,  \ i=1,\ldots,n.$
\end{itemize}
    Поскольку речь идет о допустимом плане
    $\vc{x}^{*}$, для него заведомо
    выполняются равенства
    \[\vc{a^{i}}\vc{x^{*}} = b_{i}, \ i=1,\ldots,n.\]

    Отсюда вытекает, что условия 3') и 3'') являются бессодержательными,
    поскольку выполняются автоматически. Это значит, что допустимый план
     $\vc{x}^{*}$ является  решением задачи (\ref{KZLP-1})
    тогда и только тогда, когда найдутся такие векторы
    $\vc{u'}=(u'_{1},\ldots,u'_{m})\geqq\vc{0}$ и
    $\vc{u''}=(u''_{1},\ldots,u''_{m})\geqq\vc{0}$,
    что выполняются свойства 1')-2').

    Предположим, что вектор $\vc{x}^{*}$ является решением задачи
    (\ref{KZLP-1}). Определим вектор $\vc{u}^{*}=(u_{1}^{*},\ldots,u_{m}^{*})$
    следующим образом:
    \[\vc{u}^{*}=\vc{u'}-\vc{u''},\]
    заметив, что компоненты вектора $\vc{u}^{*}$ могут
    оказаться как отрицательными, так и положительными, а может
    быть и нулевыми. Очевидно, что для такого вектора выполняются соотношения 1)-2).

    Теперь предположим, что допустимый вектор $\vc{x}^{*}$ и
    вектор $\vc{u}^{*}=(u_{1}^{*},\ldots,u_{m}^{*})$ удовлетворяют
    соотношениям 1)-2), фигурирующим в формулировке предложения.
    Определим векторы $\vc{u'}=(u'_{1},\ldots,u'_{m})\geqq\vc{0}$ и
    $\vc{u''}=(u''_{1},\ldots,u''_{m})\geqq\vc{0}$ посредством
    следующих соотношений:
    \[u'_{i}=\max\{0,u^{*}_{i}\}, u''_{i}=\max\{0,-u^{*}_{i}\}, i=1,\ldots,n.\]
    Легко проверить, что для вектора $\vc{x}^{*}$ и векторов $\vc{u'}$ и
    $\vc{u''}$ выполняются соотношения 1')-2'). А это означает, что
    $\vc{x}^{*}$ является решением рассматриваемой задачи. $\Box$


    Очевидно, что все сформулированные для различных форм задачи
    линейного программирования условия оптимальности можно обобщить,
    сформулировав их применительно к общей задаче


\begin{equation}\label{OZLPV-1}
\left\{
\begin{array}{rlcc}
    \vc{c_1} \, \vc{x_1} &+ \vc{c_2} \, \vc{x_2} &\to & \max \\
   \vc{A_{11}} \vc{x_1}          &+ \vc{A_{12}} \vc{x_2} & \leqq &\vc{b_1}\\
  \vc{A_{21}}  \vc{x_1}          &+ \vc{A_{22}} \vc{x_2} & = &\vc{b_2}\\
  \vc{x_1} \geqq \vc{0}&\\
\end{array} \right.
\end{equation}

\begin{teo}\label{usl-opt-obschaya}
    Допустимый план $\vc{x^{*}}=\left(
                                 \begin{array}{c}
                                   \vc{x_{1}^{*}} \\
                                   \vc{x_{2}^{*}} \\
                                 \end{array}
                               \right)$
    задачи (\ref{OZLPV-1}) является ее решением тогда и только
    тогда, когда найдется вектор
    $\vc{u^{*}}=(\vc{u_{1}^{*}},\vc{u_{2}^{*}})$, такой что
    $\vc{u_{1}^{*}}\geqq\vc{0}$ и выполняются следующие соотношения:
\begin{itemize}
    \item [$1')$\ ] \
    $\vc{c_{1}}\leqq\vc{u_{1}^{*}}\vc{A_{11}}+\vc{u_{2}^{*}}\vc{A_{21}};$
    \item [$1'')$\ ] \
    $\vc{c_{2}}=\vc{u_{1}^{*}}\vc{A_{12}}+\vc{u_{2}^{*}}\vc{A_{22}};$
    \item [$2)$\ ] \
    $(\vc{c_{1}}-\vc{u_{1}^{*}}\vc{A_{11}}+\vc{u_{2}^{*}}\vc{A_{21}})\vc{x_{1}^{*}}=\vc{0};$
    \item [$3)$\ ] \
    $\vc{u_{1}^{*}}[\vc{b_{1}}-(\vc{A_{11}}\vc{x_{1}^{*}}+\vc{A_{12}}\vc{x_{2}^{*}})]=\vc{0}.$
\end{itemize}
\end{teo}
    Как и в аналогичной ситуации выше, в этой теореме условия 2) и
    3) называются условиями дополняющей нежесткости.


\begin{exer}
    Докажите теорему \ref{usl-opt-obschaya}.
\end{exer}

\


\subsection{Существование решений в задачах линейного программирования}

    В предыдущих пунктах мы сформулировали и доказали важные для
    экономической теории и практики условия оптимальности для задач
    линейного программирования. Однако, мы еще не обсудили
    проблему существования решения для задач линейного
    программирования. Это мы и сделаем в это пункте.

    Решение может существовать не у каждой задачи линейного
    программирования. В частности, если допустимое множество такой
    задачи пусто, то у нее нет и решения. В любой задаче линейного
    программирования допустимое множество задается с помощью набора
    равенств и нестрогих неравенств, т.е. неравенств типа <<меньше
    или равно>> или <<больше или равно>>. Поскольку линейные функции непрерывны,
    отсюда вытекает, что
    допустимое множество в такой задаче является замкнутым. Если при
    этом оно еще и ограничено, то по теореме Вейерштрасса она имеет
    решение. В случае же, когда допустимое множество является
    неограниченным, общие условия существования решения
    сформулировать, видимо, нельзя. В каких-то задачах линейного
    программирования с неограниченным допустимым множеством решение
    существует, в каких-то --- нет. Зачастую на вопрос о существовании
    решения можно ответить, если хорошо понимать содержательную
    экономическую подоплеку задачи.
    Некоторая полезная информация содержится в следующем предложении.

\begin{prop} \label{sushestv-lp} Решение задачи линейного программирования с непустым
допустимым существует тогда и только тогда, когда ее значение
является конечным, т.е. не равным $\pm\infty$.
\end{prop}

    Это предложение говорит, что если в задачи линейного
    программирования
     с непустым допустимым множеством решения не существует, то найдутся
     такие допустимые векторы, которые доставляют целевой функции
     сколь угодно большие (в задаче на максимум) или сколь угодно малые
     (в задаче на минимум) значения.


\textbf{Доказательство} предложения \ref{sushestv-lp}. Если решение
задачи существует, то, очевидно, ее значение конечно. Докажем, что
если значение задачи конечно, то у нее существует решение. Сначала
мы проведем доказательство для канонической задачи
\begin{equation}\label{KZLPV-1}
\left\{
\begin{array}{rl}
 \vc{c} \, \vc{x} & \to \max  \\
 \vc{A} \vc{x} &= \vc{b} \\
 \vc{x} &\geqq \vc{0}\\
\end{array} \right.
\end{equation}

Обозначим через $v$ значение этой задачи. Нам достаточно доказать,
что существует вектор $\vc{x^{*}} \geqq \vc{0}$, для которого
выполняются равенства
\begin{equation}\label{KZLPV-2}
\left\{
\begin{array}{rl}
 \vc{c} \, \vc{x^{*}} & = v  \\
 \vc{A} \vc{x^{*}} &= \vc{b} \\
 \end{array} \right.
\end{equation}
    Рассмотрим матрицу
\[\vc{\tilde{A}}=\left(
                   \begin{array}{c}
                     \vc{-c} \\
                     \vc{A} \\
                   \end{array}
                 \right)
=\left(
                   \begin{array}{cccc}
                     -c_{1} & -c_{2} & \ldots & -c_{n} \\
                     a_{11} & a_{12} & \ldots & a_{1n} \\
                     \ldots & \ldots & \ldots & \ldots \\
                     a_{m1} & a_{m2} & \ldots & a_{mn}\\
                   \end{array}
                 \right)
\]
    и вектор
\[\vc{\tilde{b}}=\left(
                   \begin{array}{c}
                     -v \\
                     \vc{b} \\
                   \end{array}
                 \right)
=\left(
   \begin{array}{c}
     -v \\
     b_{1} \\
     \vdots \\
     b_{m} \\
   \end{array}
 \right)
\]
    По одной из версий леммы Фаркаша (лемме \ref{lemma-F2}), одну из версий которой
    мы уже использовали, выполняется ровно одна из двух следующих альтернатив:
\begin{itemize}
    \item [1)\ ] либо существует вектор $\vc{x} \geqq \vc{0}$, для
    которого $\vc{\tilde{A}}\vc{x}=\vc{\tilde{b}};$
    \item [2)\ ] либо найдется вектор
    $\vc{\tilde{u}}=(u_{0},\vc{u})=(u_{0},u_{1},\ldots,u_{m})$,
    такой что одновременно выполняются следующие неравенства:
    \[\vc{\tilde{u}}\vc{\tilde{A}}=-u_{0}\vc{c}+\vc{u}\vc{A}\geqq\vc{0},\]
    \[\vc{\tilde{u}}\vc{\tilde{b}}=-u_{0}v+\vc{u}\vc{b}<0.\]
\end{itemize}

    Первая из этих альтернатив эквивалентна существованию вектора
    $\vc{x}\geqq\vc{0}$, являющегося решением системы уравнений
    (\ref{KZLPV-2}), что нам и требуется.

    Докажем от противного, что вторая альтернатива выполняться не
    может. Предположим, что она выполняется. Это означает, что для некоторого
    $\epsilon>0$ существует такой вектор
    $\vc{\tilde{u}}=(u_{0},\vc{u})=(u_{0},u_{1},\ldots,u_{m})$, что
    одновременно выполняются следующие соотношения:
    \[u_{0}\vc{c}\leqq\vc{u}\vc{A}, \ \vc{u}\vc{b}<u_{0}v-\epsilon.\]
    Отсюда вытекает, что для любого допустимого вектора $\vc{x}$ задачи (\ref{KZLPV-1}),
    который, во-первых, является неотрицательным, а во-вторых,
    удовлетворяет равенству $\vc{A} \vc{x} = \vc{b}$, справедлива
    цепочка неравенств
    \[u_{0}\vc{c}\vc{x}\leqq\vc{u}\vc{A}\vc{x}=\vc{u}\vc{b}<u_{0}v-\epsilon.\]

    Из справедливости неравенства
    \[u_{0}\vc{c}\vc{x}<u_{0}v-\epsilon\]
    для любого допустимого $\vc{x}$ следует, что $u_{0}>0$.
    Действительно, $u_{0}$ не может равняться нулю, ибо это
    означало бы, что $0<-\epsilon$. Число $u_{0}$ не может быть и отрицательным,
    поскольку в противном случае значения целевой функции на допустимых векторах задачи
    оказались бы больше значения самой задачи.

    Поскольку
    $u_{0}>0$, для любого допустимого вектора $\vc{x}$ должно
    выполняться неравенство
    \[\vc{c}\vc{x}<v-\frac{\epsilon}{u_{0}}.\]
    Но это неравенство противоречит тому, что $v$ --- это значение
    рассматриваемой задачи, ибо непосредственно по определению значения задачи
         для любого сколь угодно малого
    $\delta>0$, в частности, для $\delta=\frac{\epsilon}{u_{0}}$,
    должен найтись допустимый вектор $\vc{x}$, такой что
    $\vc{c}\vc{x}>v-\delta$. Полученное противоречие показывает, что
    вторая из двух альтернатив, на которые указывает лемма Фаркаша,
    в рассматриваемом нами случае не возможна.

    Мы доказали предложение для канонической задачи линейного
    программирования. Нам осталось только вспомнить, что любая
    задача линейного программирования сводится к канонической.
    $\Box$





\subsection{Двойственность в линейном программировании}

    При анализе задач линейного программирования их удобно
    рассматривать одновременно с так называемыми двойственными задачами.
    Содержательные экономические рассуждения, ведущие к построению
    двойственных задач, мы уже проводили ранее при обсуждении задач
    распределения одного или двух видов ресурсов ресурсов.
    В этом пункте мы коротко расскажем о том, как формально определяются
    двойственные задачи в общем случае и об
    основных теоремах двойственности в линейном программировании.

    Начнем с задачи линейного программирования в стандартной форме.
    Рассмотрим задачу
    \begin{equation} \label{SZLP-1}
\left\{
\begin{array}{rcrrrrllll}
     c_1 x_1 + & c_2 x_2 +    & \ldots +& c_n x_n &\to & \max\\
     a_{11} x_1 + & a_{12} x_2 + &\ldots +& a_{1n} x_n &\leqslant& b_1 \\
     a_{21} x_1 + & a_{22} x_2 + &\ldots +& a_{2n} x_n &\leqslant& b_2\\
                      && \ldots &&&&\\
     a_{m1} x_1 + & a_{m2} x_2 +& \ldots +& a_{mn} x_n &\leqslant& b_m\\
     x_1 \geqslant 0,   & x_2 \geqslant 0,  & \ldots,&  x_n \geqslant 0\\
\end{array} \right.
\end{equation}
    \emph{Двойственной} к ней называется следующая задача:
\begin{equation} \label{SZLP-1-dual}
\left\{
\begin{array}{rcrrrrllll}
     b_1 u_1 + & b_2 u_2 +    & \ldots +& b_n u_n &\to & \min\\
     a_{11} u_1 + & a_{21} u_2 + &\ldots +& a_{m1} u_n &\geqslant& c_1 \\
     a_{12} u_1 + & a_{22} u_2 + &\ldots +& a_{m2} u_n &\geqslant& c_2\\
                      && \ldots &&&&\\
     a_{n1} u_1 + & a_{n2} u_2 +& \ldots +& a_{mn} u_n &\geqslant& c_n\\
     u_1 \geqslant 0,   & u_2 \geqslant 0,  & \ldots,&  u_n \geqslant 0\\
\end{array} \right.
\end{equation}


    Когда задачу (\ref{SZLP-1}) рассматривают в паре с двойственной,
    ее иногда называют прямой задачей. Переменные двойственной задачи называют двойственными
    переменными.




    Подчеркнем, что мы определили двойственную задачу к задаче
    на максимум, в которой все ограничения, кроме ограничений
    на неотрицательность переменных, имеют вида <<меньше или равно>>. А
    двойственной к ней является задача на минимум при ограничениях
    вида <<больше или равно>>. При этом каждому ограничению прямой
    задачи, за исключением ограничений на неотрицательность
    переменных, соответствует своя переменная двойственной задачи,
    на которую накладывается ограничение на неотрицательность.
    Ограничению
    \[a_{11} x_1 +  a_{12} x_2 + \ldots + a_{1n} x_n \leqslant b_1  \]
    соответствует переменная $u_1$, ограничению
    \[a_{21} x_1 +  a_{22} x_2 + \ldots + a_{2n} x_n \leqslant b_2\]
    --- переменная $u_2$ и т.д. А каждой переменной прямой задачи
    соответствует ограничение двойственной задачи. Переменной
    $x_{1}$ соответствует ограничение
    \[a_{11} u_1 +  a_{21} u_2 + \ldots + a_{m1} u_n \geqslant c_1,\]
    переменной $x_{2}$ --- ограничение
    \[a_{12} u_1 +  a_{22} u_2 + \ldots + a_{m2} u_n \geqslant c_2\]
    и т.д.




\begin{exer}
    Запишите задачу (\ref{SZLP-1-dual}) в виде задачи на максимум
    с ограничениями вида <<меньше или
    равно>> и постройте для нее двойственную. Покажите, что эта
    двойственная задача переписывается в виде (\ref{SZLP-1}).
\end{exer}

    Это упражнение говорит что при совместном рассмотрении задач (\ref{SZLP-1}) и
    (\ref{SZLP-1-dual}) необязательно считать задачу (\ref{SZLP-1}) прямой,
    а (\ref{SZLP-1-dual}) --- двойственной к ней. Можно и наоборот, считать
    задачу (\ref{SZLP-1-dual}) прямой, а задачу (\ref{SZLP-1})
    двойственной. Можно даже не
    уточнять, какая задача является прямой, а какая ---
    двойственной. Достаточно просто говорить о паре двойственных задач.

    Чтобы сформулировать теоремы двойственности, нам удобно записать
    прямую и двойственную задачи в векторно-матричном виде. Задача
    (\ref{SZLP-1}) запишется как
    \begin{equation}\label{SZLPD}
\left\{
\begin{array}{rl}
 \vc{c} \, \vc{x} & \to \max  \\
 \vc{A} \vc{x} &\leqq \vc{b} \\
 \vc{x} &\geqq \vc{0}\\
\end{array} \right.
\end{equation}
    а задача (\ref{SZLP-1-dual}) --- как

\begin{equation}\label{DualSZLP}
\left\{
\begin{array}{rl}
 \vc{u} \, \vc{b} & \to \min  \\
 \vc{u} \vc{A} & \geqq \vc{c} \\
 \vc{u} & \geqq \vc{0} \\
\end{array} \right.
\end{equation}


    Если допустимые множества в обеих задачах существуют, то, очевидно, для
    любого допустимого вектора $\vc{x}$ задачи (\ref{SZLPD}) и
    любого допустимого вектора $\vc{u}$ задачи (\ref{DualSZLP})
    справедливы следующие соотношения:
    \[\vc{c}\vc{x}\leqslant\vc{u}\vc{A}\vc{x}\leqslant\vc{u}\vc{b}.\]
    Отсюда следует, что значения обеих задач конечны. В силу
    предложения \ref{sushestv-lp} они имеют решения.

    Теперь вспомним, что согласно предложению \ref{SZLP-1-usl-opt}
    если вектор $\vc{x^{*}}$ является решением задачи (\ref{SZLPD}),
    то найдется вектор $\vc{u^{*}}$, такой что выполняются следующие
    соотношения:
    \[\vc{c}\leqq \vc{u}^{*}\vc{A},\]
    \[\vc{c}\vc{x}^{*}=\vc{u}^{*}\vc{A}\vc{x}^{*}=\vc{u}^{*}\vc{b}.\]
    Очевидно, что вектор $\vc{u^{*}}$ является допустимым вектором
    задачи (\ref{DualSZLP}) и, более того, ее решением.

    Проведенные рассуждения мы можем подытожить в следующем предложении.

\begin{prop} \label{teor-dv-st} (теорема двойственности для
стандартной задачи линейного программирования)

    Рассмотрим пару двойственных задач (\ref{SZLPD}) и
    (\ref{DualSZLP}).
\begin{itemize}
    \item [$1)$\ ] \
    Если допустимое множество прямой задачи непусто, то у этой
    задачи решение существует тогда и только тогда, когда непусто
    допустимое множество двойственной задачи.
    \item [$2)$\ ] \
    Для любых допустимых векторов $\vc{x}$ и $\vc{u}$
    задач (\ref{SZLPD})и
    (\ref{DualSZLP}) соответственно выполняется неравенство
    \[\vc{c}\vc{x}\leqq\vc{u}\vc{b}.\]
    \item [$3)$\ ] \
    Если допустимые множества прямой и двойственной задач непусто,
    то у обеих задач существуют решения $\vc{x^{*}}$ и $\vc{u^{*}}$,
    а значения этих задач совпадают:
    \[\vc{c}\vc{x}^{*}=\vc{u}^{*}\vc{b}.\]

\end{itemize}
\end{prop}

\begin{exer}
    Докажите, что из утверждения этого предложения вытекает
    предложение \ref{SZLP-1-usl-opt}.
\end{exer}


    Конечно, теория двойственности касается задач линейного
    программирования вне зависимости от того, в какой форме они записаны.
    Мы не будем останавливаться на основной,
    и канонической  задачах, поскольку и та, и другая являются
    частным случаем общей задачи линейного программирования

\begin{equation*}
\left\{
\begin{array}{rrrrll}
    c_1 x_1  +  \ldots + & c_r x_r+     & c_{r+1}x_{r+1}    +\ldots +  &c_n x_n    &\to  & \max \\
    a_{11} x_1 +  \ldots + & a_{1r} x_r+  & a_{1,r+1} x_{r+1} +\ldots +  &a_{1n} x_n &\leqslant & b_1  \\
    a_{21} x_1 +  \ldots + & a_{2r} x_r+  & a_{2,r+1} x_{r+1} +\ldots +  &a_{2n} x_n &\leqslant & b_2 \\
                                  & \ldots &&&&\\
    a_{q1} x_1 +  \ldots + & a_{qr} x_r+  & a_{q,r+1} x_{r+1} +\ldots +  &a_{qn} x_n &\leqslant & b_q  \\
    a_{q+1,1} x_1 +  \ldots + & a_{q+1,r} x_r+ & a_{q+1,r+1} x_{r+1} +\ldots + &a_{q+1,n} x_n &= & b_{q+1}  \\
                                  & \ldots &&&&\\
    a_{m1} x_1 +  \ldots + & a_{mr} x_r+ & a_{m,r+1} x_{r+1} +\ldots + &a_{mn} x_n &= & b_m  \\
             x_1 \ge 0&,  \ldots, \quad  &x_r \ge 0&&&\\
\end{array} \right.
\end{equation*}
    которая в векторно-матричной записи выглядит так:
    \begin{equation}\label{OZLPV-2}
\left\{
\begin{array}{rlcc}
    \vc{c_1} \, \vc{x_1} &+ \vc{c_2} \, \vc{x_2} &\to & \max \\
   \vc{A_{11}} \vc{x_1}          &+ \vc{A_{12}} \vc{x_2} & \leqq &\vc{b_1}\\
  \vc{A_{21}}  \vc{x_1}          &+ \vc{A_{22}} \vc{x_2} & = &\vc{b_2}\\
  \vc{x_1} \geqq \vc{0}&\\
\end{array} \right.
\end{equation}

    Двойственной к ней называется задача

\begin{equation*}\label{DualOZLP}
\left\{
\begin{array}{rrrrll}
     b_1 u_1  +  \ldots + & b_q u_q+     & b_{l+1}u_{l+1}    +\ldots +  &b_m u_m    &\to  &\min \\
           a_{11} u_1 +  \ldots + & a_{q1} u_q+  & a_{l+1,1} u_{l+1} +\ldots +  &a_{m1} u_m &\geqslant & c_1 \\
           a_{12} u_1 +  \ldots + & a_{q2} u_q+  & a_{l+1,2} u_{l+1} +\ldots +  &a_{m2} u_m &\geqslant & c_2 \\
                                  & \ldots &&&&\\
           a_{1r} u_1+   \ldots + & a_{qr} u_q+  & a_{l+1,r} u_{l+1} +\ldots +  &a_{mr} u_m &\geqslant & c_r \\
        a_{1,r+1} u_1+   \ldots + & a_{q,r+1} u_q+  & a_{l+1,r+1} u_{l+1} +\ldots +  &a_{m,r+1} u_m &= & c_{r+1} \\
                                  & \ldots &&&&\\
           a_{1n} u_1+   \ldots + & a_{qn} u_q+  & a_{l+1,n} u_{l+1} +\ldots +  &a_{mn} u_m &= & c_n \\
           u_1 \ge 0, \ldots,  \;\;  &u_q \ge 0&&&&\\
\end{array} \right.
\end{equation*}
    имеющая в векторно-матричной записи следующий вид:

\begin{equation}\label{OZLPV-2-dv}
\left\{
\begin{array}{rlcc}
    \vc{u_1} \, \vc{b_1} &+ \vc{u_2} \, \vc{b_2} &\to & \min \\
   \vc{u_1}\vc{A_{11}} & + \vc{u_2}\vc{A_{21}} & \geqq &\vc{c_1}\\
  \vc{u_1}\vc{A_{12}} & + \vc{u_2}\vc{A_{22}} & = &\vc{c_2}\\
  \vc{u_1} \geqq \vc{0}&\\
\end{array} \right.
\end{equation}





    Как и в случае основной задачи, здесь тоже каждому ограничению прямой задачи,
    не считая ограничений на неотрицательность, соответствует
    переменная двойственной задачи (двойственная переменная).
    При этом на
    двойственные переменные, которые в прямой задаче соответствуют
    ограничениям-неравенствам, накладывается ограничение на
    неотрицательность, а на двойственные переменные, соответствующие
    ограничениям-равенствам, таких ограничений не накладывается.
    Симметрично, каждой переменной прямой задачи соответствует
    некоторое ограничение двойственной. При этом
    ограничения двойственной задачи, которым
    соответствуют неотрицательные переменные прямой задачи,
    выглядят как неравенства, а ограничения, соответствующие
    переменным прямой задачи, на которые не наложены ограничения на
    неотрицательность, записаны как равенства.

    \begin{exer}
    Покажите, что с точностью до эквивалентной формы записи,
    двойственной к задаче (\ref{OZLPV-2-dv}) является задача
    (\ref{OZLPV-2}). Проверьте это на примере задачи
    \begin{equation*}
\left\{
\begin{array}{lllllc}
f(\vc{x}) = & 8 x_1      & + 10 x_2  & +  x_3  &\to    & \max\\
            & 5 x_1      & -  7 x_2  & + 9 x_3 & \leqslant  & 6   \\
            & 11 x_1     & + 12 x_2  & - 3 x_3 & =     & 11  \\
            & x_1 \geqslant 0, & x_2 \geqslant 0 &&\\
\end{array} \right.
\end{equation*}
\end{exer}

    Следующая теорема является простым обобщением предложения
    \ref{teor-dv-st}.

    \begin{teop} \label{teorema-dvoistv} (теорема двойственности для общей
    задачи линейного программирования)

    Рассмотрим пару двойственных задач (\ref{OZLPV-2}) и
    (\ref{OZLPV-2-dv}).
\begin{itemize}
    \item [$1)$\ ] \
    Если допустимое множество прямой задачи непусто, то у этой
    задачи решение существует тогда и только тогда, когда непусто
    допустимое множество двойственной задачи.
    \item [$2)$\ ] \
    Для любых допустимых векторов $\vc{x}=\left(
                                            \begin{array}{c}
                                              \vc{x_{1}} \\
                                              \vc{x_{2}} \\
                                            \end{array}
                                          \right)$
    и $\vc{u}=(\vc{u_{1}},\vc{u_{2}})$
    задач (\ref{SZLPD})и
    (\ref{DualSZLP}) соответственно выполняется неравенство
    \[\vc{c_{1}}\vc{x_{1}}+\vc{c_{2}}\vc{x_{2}}\leqq\vc{u_{1}}\vc{b_{1}}+\vc{u_{2}}\vc{b_{2}}.\]
    \item [$3)$\ ] \
    Если допустимые множества прямой и двойственной задач непусто,
    то у обеих задач существуют решения $\vc{x^{*}}=\left(
                                            \begin{array}{c}
                                              \vc{x_{1}^{*}} \\
                                              \vc{x_{2}^{*}} \\
                                            \end{array}
                                          \right)$
    и $\vc{u^{*}}=(\vc{u_{1}^{*}},\vc{u_{2}}^{*})$, а значения этих задач совпадают:
    \[\vc{c_{1}^{*}}\vc{x^{*}_{1}}+\vc{c_{2}^{*}}\vc{x^{*}_{2}}
    =\vc{u^{*}_{1}}\vc{b_{1}}+\vc{u^{*}_{2}}\vc{b_{2}}.\]
\end{itemize}
\end{teop}

\begin{exer}
    Сформулируйте теорему двойственности для канонической и основной
    задач линейного программирования.
\end{exer}
\begin{exer}
    Докажите теорему \ref{teorema-dvoistv}.
\end{exer}







\
